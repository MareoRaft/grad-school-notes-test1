b\documentclass[master.tex]{subfiles}

\setcounter{section}{1}

\begin{document}
\section{Ring Theory}
\subsection{Basic Definitions}
\newcommand{\F}{\mathbb{F}} \newtheorem*{notation}{Notation}
% Lecture 10/27/2016

\begin{defn*}
  A \emph{ring} is set \(R\) together with two binary operations \(+\) and \(\cdot\) satisfying
  \begin{enumerate}[label=(\roman*)]
  \item \((R,+)\) is an abelian group (denote the additive identity by \(0\))
  \item \(\cdot\) is associative, \((xy)z=x(yz)\) for all \(x,y \in R\)
  \item There exists a multiplicative identity (denoted \(1 \in R\)).
  \item Distribution laws hold:
    \begin{align*}
      x(y+z) &= xy + xz\\
      (x+y)z &= xz+yz
    \end{align*}
  \end{enumerate}
\end{defn*}

\begin{defn*}
  A \emph{commutative ring} is a ring with the additional property that \(xy=yx\) for all \(x,y \in R\).
\end{defn*}
Notice that if multiplication is commutative either distribution law implies the other. One of the most basic
observations one can make is that \(R\) is the trivial ring \(\iff 1=0\).

\begin{example*}
  A few familiar rings: \[\Z,\Z_n,\Q,\R,\C\] A non-commutative ring:
  \[M_n(\F), n \ge 2\] or more generally take any ring \(R\).
\end{example*}

\begin{defn*}
  A \emph{skew field (or division ring)} is a ring \(R\) such that \(R \neq \{0\}\) and \(R \setminus \{0\}\) is a group
  under multiplication.
\end{defn*}

\begin{defn*}
  A \emph{field} is a commutative skew field.
\end{defn*}

\begin{example*}
  An example of a skew field is
  \[\H = \R + \R i + \R j + \R k\]
\end{example*}

One may form direct products of rings in the usual manner.

\begin{defn*}
  A \emph{ring homomorphism} is map \(\varphi \colon R \to S\) between rings \(R\) and \(S\) satisfying
  \begin{enumerate}[label=(\arabic*)]
  \item \(\varphi(x+y)=\varphi(x)+\varphi(y)\)
  \item \(\varphi(xy)=\varphi(x)\varphi(y)\)
  \item \(\varphi(1_R)=1_S\)
  \end{enumerate}
\end{defn*}

In keeping with our demand that all of rings have unity, we also force our homomorphisms to respect the unital
structure. In particular our ring homomorphisms are morphisms in the category of commutative rings.

\begin{example*}
  For us a map \funcdeclaration{\varphi}{R}{R \times R}{r}{(r,0)} is not a ring homomorphism as
  \(1_{R \times R} = (1,1)\) while \(\varphi(1)=(1,0)\).
\end{example*}

\begin{defn*}
  A subset \(S\) of a ring \(R\) is a \emph{subring} of \(R\) if
  \begin{enumerate}[label=(\arabic*)]
  \item \(S\) is closed under both operations
  \item \(1_R \in S\)
  \end{enumerate}
\end{defn*}

\begin{example*}
  Under this definition \(R \times \{0\}\) is not a subring of \(R \times R\) as \(1_R \not \in R \times R\) as long as
  \(R\neq\{0\}\).
\end{example*}

\begin{defn*}
  An element \(x \in R\) is said to be a \emph{zero divisor} of \(R\) if there exists an element \(y \in R\) such that
  \(y \neq 0\) and
  \[xy = 0 \qquad \text{ or } \qquad yx = 0.\]
\end{defn*}

\begin{defn*}
  The commutative ring \(R\) is an \emph{integral domain} if its only zero divisor is \(0\).
\end{defn*}

\begin{example*}
  \(\Z_n\) integral domain \(\iff n\) is prime
\end{example*}

\begin{prop*}
  \(R\) finite integral domain \(\implies\) \(R\) is field.
\end{prop*}

\begin{defn*}
  A \(x \in R\) is a \emph{unit} if \(xy=1\) and \(yx=1\).
\end{defn*}

\begin{defn*}
  The group of units \(R^\times\) of a ring is
  \[R^\times = \{x \in R \mid x\text{ is a unit}\}.\]
\end{defn*}

\begin{example*}
  \begin{align*}
    \Z^\times &= \{-1,1\}\\
    \Z_n^\times &= \{\bar{a} \mid (a,n)=1\}
  \end{align*}
\end{example*}

\begin{prop*}
  If \(\varphi \colon R \to S\) is a homomorphism and \(x \in R^{\times}\) then \(\varphi(x) \in S^\times\). In
  particular this means \(\varphi(R^\times) \le S^\times\).
\end{prop*}

Even if \(\varphi \colon R \twoheadrightarrow S\) the image \(\varphi(R^\times)\) might not equal \(S^\times\).

\begin{example*}
  Consider the surjective function \funcdeclaration{\varphi}{\Z}{\Z_n}{a}{\bar{a}} but \(|\Z_n^\times|=\varphi(n)>2\) if
  \(n \ge 7\)
\end{example*}

\begin{notation}
  From this point onwards \(R\) denotes a \textbf{commutative ring}.
\end{notation}

\begin{defn*}
  A non-empty subset \(I \subset R\) is an \emph{ideal} if
  \begin{enumerate}[label=(\roman*)]
  \item \(x,y \in I \implies x+y \in I \)
  \item \(x \in I, r \in R \implies rx \in I\)
  \end{enumerate}
\end{defn*}
Condition one may be rephrased as \((I,+)\) forms an abelian group.
\begin{prop*}
  An ideal \(I=R \iff 1 \in R \iff I \cap R^\times \neq \emptyset\)
\end{prop*}

\begin{notation}
  If \(I\) is an ideal of \(R\) we denote it as \(I \unlhd R\).
\end{notation}

\begin{prop}
  A ring homomorphism \(\varphi \colon R \to S\) if \(I \unlhd R\), \(J \unlhd S\) then
  \[\varphi(I) \unlhd S \iff \varphi \text{ is surjective }\]
  while
  \[\varphi^{-1} \unlhd R \text{ is always true.}\]
\end{prop}

\begin{defn*}
  Let \(I,J \unlhd R\) then we define
  \begin{align*}
    I+J &:= \{x+y \mid x \in I, y \in J\} \unlhd R\\
    IJ  &:= \left\{\sum_{l=0}^nx_l y_l \mid n \in \N, x_l \in I, y_l \in J\right\}.
  \end{align*}
\end{defn*}
We have defined \(IJ\) in the above manner to force \(IJ\) to be an abelian group, and hence an ideal.

\begin{defn*}
  Given an ideal \(I \unlhd R\) we may form the \emph{quotient ring} denote \(R/I\) as follows
  \[R/I = \{a+I\mid a \in R\}.\] This forms a ring under addition and multiplication by representatives.
\end{defn*}

\begin{thm}[Isomorphism Theorems For Rings] Given a ring homomorphism \(\varphi \colon R \to S\). Then
  \begin{enumerate}[label=(\alph*)]
  \item \(\varphi(R)\) is a subring of \(S\) and \(\ker(\varphi)=\{r \in R \mid \varphi(r)=0\} \unlhd R\)
  \item \(R/\ker \varphi \isom \varphi(R)\)
  \item \(I,J \unlhd R\) with \(I \subset J \implies\)
    \[J/I\unlhd R/I\] and
    \[(R/I)/(J/I) \isom R/S.\]
  \end{enumerate}
\end{thm}

\begin{prop}{Correspondence Theorem}
  Let \(I \unlhd R\) define \funcdeclaration{\pi}{R}{R/I}{a}{\bar{a}} Then there is a bijection
  \[\{J \unlhd R \mid I \subset J\} \longleftrightarrow \{\text{Ideals of } R/I \}\] given by
  \[J \longmapsto J/I\] and
  \[L \longmapsto \pi^{-1}(L).\]
\end{prop}

\begin{proof}
  First \(\bar{J}=J/I=\pi(J)\unlhd \bar{R}\) since \(\pi\) is surjective \(\pi^{-1}(L) \unlhd R\) and contains
  \(I=\pi^{-1}(O)\). Claim: \(gf(J)=\pi^{-1}(J/I)=J\). Assume \(x \in R\) with \(x \in \pi^{-1}(J/I)\) then
  \(\pi(x) \in J/I\). Then \(x+I=y+I\) for some \(y \in J\). Thus \(x \in y + I \subset J\). Lastly
  \(fg(L)=\pi(\pi^{-1}(L))\) since \(\pi\) is surjective.
\end{proof}

\begin{defn}
  A proper ideal \(I \lhd R\) is called
  \begin{enumerate}
  \item a \emph{prime ideal} if
    \[xy \in I \implies x \in I \text{ or } y \in I.\]
  \item a \emph{maximal ideal} if
    \[\forall J \lhd R,\ I \subseteq J \implies I=J.\]
  \end{enumerate}
\end{defn}

\begin{prop}
  For every \(I \lhd R\) there exists a maximal ideal \(M\) such that \(I \subseteq M\) and \(M \lhd R\).
\end{prop}

\begin{proof}[Short Proof]
  This follows from Zorn's Lemma.
\end{proof}

Here we will be pedantic and write out the proof in its entirety as it sets the flavor of proof for the existence of
other maximal entities.

\begin{lem*}[Zorn's Lemma]
  If \(A\) is a nonempty partially ordered set such that every chain in \(A\) has an upper bound in \(A\), then \(A\)
  contains a maximal element.
\end{lem*}

\begin{proof}[Detailed Proof]
  Define set \(S=\{J \lhd R \mid I \subset J\} \neq \emptyset\). This set is never empty as \(I \in S\) by
  construction. The set \(S\) is partially ordered \(j \le J' \iff J \subseteq J'\). Now check that every chain in \(S\)
  has an upper bound. Let \((I_\alpha)_{\alpha in A}\) be a chain in \(S\) where \(\alpha, \beta \in A\) then
  \(J_\alpha \subseteq J_\beta\) or \(J_\beta \subseteq J_\alpha\). Here we can compare any two elements. The candidate
  for an upper bound is
  \[J=\bigcup_{\alpha \in A} J_\alpha.\] We must check the following requirements:
  \begin{enumerate}[label=(\arabic*)]
  \item \(I \subset J\)
  \item \(J \unlhd R\)
  \item \(J \neq R\).
  \end{enumerate}
  The condition (1) is immediate as \(I \subset J_\alpha\) for each \(\alpha \in A\). To see (2) let \(x,y \in J\) and
  \(r \in R\). Then there exists \(\alpha, \beta \in A\) such for \(x \in I\) and \(y \in J_\beta\) without loss of
  generality \(J_\alpha \subseteq J_\beta\). Then \(x+y \in J_\beta \subseteq J\). Since \(J_\alpha\) is an ideal for
  \(x \in J_\alpha\) we have \(rx \in J_\alpha \subset J\). Thus \(J\) is an ideal of \(R\). Lastly to see (3) notice
  \(1 \not \in J\) since \(1 \not \in J_\alpha\) for all \(\alpha \in A\) as each of these ideals is proper. Recall an
  ideal is the whole ring if and only if it contains unity, thus we have \(J \neq R\). We have shown that the conditions
  of Zorn's Lemma hold, hence \(S\) has a maximal element.
\end{proof}

\begin{prop}
  \(I \unlhd R\)
  \begin{enumerate}[label=(\alph*)]
  \item \(I\) is a prime ideal \(\iff\) \(R/I\) integral domain.
  \item \(I\) is a maximal ideal \(\iff\) \(R/I\) is a field.
  \end{enumerate}
\end{prop}

\begin{proof}
  \begin{enumerate}[label=(\alph*)]
  \item For \(x,y \in R\), the following are equivalent:
    \[xy \in I \implies x \in \text{ or } y \in I\\\] \[\iff\]
    \[\bar{x}\bar{y}=\bar{0} \implies \bar{x}=\bar{0} \text{ or } \bar{y}=\bar{0}\] \[\iff\] \[\text{R/I is an
        integral domain}\]
  \item \(I\) is maximal \(\iff I + Rx = \{a+rx \mid r \in R\} \unlhd R\)
    \[\iff\]
    \[\bar{R}\bar{x}=\bar{R} \ \forall \bar{x} \neq \bar{0} \in \bar{R}\]
    \[\iff\]
    \[\text{All non-zero elements of \(\bar{R}\) are units.}\]
  \end{enumerate}

  \begin{cor*}
    Every maximal ideal is prime.
  \end{cor*}
\end{proof}

\begin{rmk}
  Recall from the Correspondence Theorem that there is a bijection between ideals of \(R\) that contain an ideal \(I\)
  and the ideals of \(R/I\). This bijection may be restricted to either the prime ideals or the maximal ideals. Yielding
  a bijection between the prime/maximal ideals of \(R\) containing \(I\) and the prime/maximal ideals of \(R/I\).
\end{rmk}

\begin{defn*}
  Given \(S \subseteq R\) the \emph{ideal generated by \(S\) in \(R\)} denoted \((S)\) is the smallest ideal of \(R\)
  containing \(S\). More specifically it is defined to be
  \[(S)=\bigcap_{S \subset I \unlhd R} I\]
\end{defn*}

\begin{notation}
  Given a finite set \(S=\{x_1,\ldots,x_n\} \subset R\) we write \((x_1,\ldots,x_n)\) for
  \(\left(\{x_1,\ldots,x_n\}\right)\) which is exactly the set of all linear combinations of the elements of \(S\) with
  coefficients from \(R\).
\end{notation}

\begin{defn*}
  For \(x \in R\) the ideal \((x)=Rx=\{rx \mid r \in R\}\) is called the principal ideal generated by the element \(x\).
\end{defn*}

\begin{defn}
  An integral domain \(R\) is called a \emph{principal ideal domain (PID)} if every ideal of \(R\) is principle. 
\end{defn}

\begin{examples}
  \(Z\), \(\F\), \(\F[x]\)
\end{examples}

\begin{notation}
  \(a,b \in R\) we say \emph{\(a\) divides \(b\)} or \(b\)is a multiple of \(R\) if \(\exists c \in R\) such that
  \(ac=b\) denoted \(a \mid b\).
\end{notation}

\begin{rmk} Let \(a \in R\)
  \begin{enumerate}[label=(\alph*)]
  \item \(\{b \in R \mid a \mid b\}=(a)\)
  \item \(a \mid 0\) since \(0=a \cdot 0 \) 
  \item  \(u \in R^\times \implies u \mid a\) since \(u(u^{-1}a)=a\)
  \item \(ac=b\) should not be written as \(c=\frac{b}{a}\)since in general \(\frac{b}{a}\) is ambiguous. Ex in \(\Z_6\)
    \(2 \times 2 = 4\) but \(2 \times 5 = 4\) also so is \(\frac{4}{2}=2\) or 5.
  \end{enumerate}
\end{rmk}

\begin{defn}
  Two ideals \(I, J \unlhd R\) are called \emph{comaximal} if \(I+J=R\) (\(\iff \exists a \in I, b \in J\) such that \(a+b=1\))
\end{defn}

\end{document}

%%% Local Variables:
%%% mode: latex
%%% TeX-master: "master.tex"
%%% End:
