\documentclass[11pt,leqno,oneside]{amsart}
%\usepackage{amsmath,amsthm}
\usepackage{amssymb,mathrsfs}
%\usepackage{ytableau}
%\ytableausetup{smalltableaux,centertableaux}
\usepackage{tikz}
\usepackage{upgreek}
\usepackage{enumitem}
%\usepackage[nohead,nofoot,centering]{geometry}

\usepackage{color} 
%% Some user-defined colors
      \definecolor{mydefi}{cmyk}{1,0,0,.5}
      \definecolor{myred}{rgb}{.7,.1,.1}
      \definecolor{myblue}{rgb}{.1,.1,.6}
      \definecolor{mygreen}{rgb}{.1,.6,.1}

\usepackage[urlbordercolor={1 1 1}, pdfborder={0 0 0}, bookmarks=true,
  colorlinks=true, linkcolor=myblue, citecolor=myblue,
  urlcolor=myblue, hyperfootnotes=false]{hyperref}

\usepackage[alphabetic,abbrev]{amsrefs} % use AMS ref scheme

\addtolength{\footskip}{2\baselineskip} % to lower the page numbers


%%%%%%%%%%%%%%%%%%%%%%%%%%%%%%%%%%%%%%%%%%%%%%%%%%%%%%%%%%%%%%%%%%%
%%  MACRO DEFINITIONS:  Co-authors -- PLEASE use these! 
%%%%%%%%%%%%%%%%%%%%%%%%%%%%%%%%%%%%%%%%%%%%%%%%%%%%%%%%%%%%%%%%%%%
\newcommand{\N}{{\mathbb N}} % natural numbers
\newcommand{\Z}{{\mathbb Z}} % integers
\newcommand{\Q}{{\mathbb Q}} % rational numbers
\newcommand{\R}{{\mathbb R}} % real numbers
\newcommand{\C}{{\mathbb C}} % complex numbers
\newcommand{\End}{\operatorname{End}} % endomorphisms
\newcommand{\Hom}{\operatorname{Hom}} % homomorphisms
\newcommand{\GL}{\operatorname{GL}} % general linear group
\newcommand{\B}{\mathfrak{B}} % use for the Brauer algebra
\newcommand{\Sym}{\mathfrak{S}} % symmetic group
\newcommand{\sgn}{\operatorname{sgn}} % sign
\newcommand{\T}{\mathsf{T}} % use for tableaux
\newcommand{\U}{\mathsf{U}} % use for tableaux
\newcommand{\V}{\mathsf{V}} % use for tableaux
\newcommand{\TA}{\mathsf{A}} % use for tableaux
\newcommand{\TB}{\mathsf{B}} % use for tableaux
\newcommand{\TC}{\mathsf{C}} % use for tableaux
\newcommand{\TS}{\mathsf{S}} % use for tableaux
\newcommand{\shape}{\operatorname{shape}} % shape of a tableau
\newcommand{\col}[2]{\genfrac{}{}{0pt}{1}{#1}{#2}} % column of bitableau
\newcommand{\ov}{\overline} % shorthand for a bar on a symbol
\newcommand{\dd}{\partial} % use for diagram basis; e.g. d(V_l)
\newcommand{\X}{\mathcal{X}} % use for the Gelfand-Tsetlin subalgebra
\newcommand{\JM}{\mathcal{J}} % a subalgebra of the GT-subalgebra
\newcommand{\Std}{\operatorname{Std}} % set of standard tableaux
\newcommand{\StdB}{\operatorname{StdB}} % set of standard bitableaux
\newcommand{\Orb}{\mathcal{O}} % use for orbits
\newcommand{\OS}{\ov{\Orb}} % use for orbit sums
\newcommand{\OSS}{\ov{\OS}} % use for double bar orbit sums
\newcommand{\OR}{\mathscr{R}} % orbit representatives
\newcommand{\Stab}{\operatorname{Stab}} % stabilizer
\newcommand{\rev}{\operatorname{rev}} % reverse of a cycle
\newcommand{\A}{\mathcal{A}} % the algebra
\newcommand{\fraka}{\mathfrak{a}} % Young symmetrizer
\newcommand{\frakb}{\mathfrak{b}} % Young symmetrizer
\newcommand{\frakc}{\varphi} % canonical basis
\newcommand{\yy}{\mathsf{y}} % Young symmetrizer; scaled
\newcommand{\idem}{\varepsilon} % primitive central idem in symm gp
\newcommand{\cA}{\mathcal{A}} % group algebra of symmetric group
\newcommand{\Tab}{\operatorname{Tab}} % trails in branching graph from source
\newcommand{\BG}{\mathbf{B}} % branching graph
\newcommand{\bb}{\varnothing} % the unique element of \Irr(0) 
\newcommand{\res}{\operatorname{res}} % restriction
\newcommand{\Irr}{\operatorname{Irr}} % irreps
\newcommand{\Wt}{\operatorname{Wt}} % possible content vectors for an irrep
\newcommand{\trace}{\operatorname{trace}} % the trace
\newcommand{\type}{\operatorname{type}} % type = generalized shape
\newcommand{\gen}[1]{\langle #1 \rangle} % use for generating sets
\newcommand{\parm}{\updelta} % Brauer algebra parameter
\newcommand{\sep}{\,|\,} % separator for two partitions - used in tables 
\newcommand{\covered}{\lessdot}
\newcommand{\qand}{\quad\hbox{and}\quad}
\newcommand{\subgroup}{\mathrel{<}}
\newcommand{\normsubgroup}{\mathrel{\unlhd}}
\newcommand{\isom}{\mathrel{\cong}}
\swapnumbers %% put numbers in front of proclamations
\newtheorem{thm}{Theorem}[subsection]
\newtheorem*{thm*}{Theorem}
\newtheorem{lem}[thm]{Lemma}
\newtheorem*{lem*}{Lemma}
\newtheorem{prop}[thm]{Proposition}
\newtheorem*{prop*}{Proposition}
\newtheorem{cor}[thm]{Corollary}
\newtheorem*{cor*}{Corollary}
\newtheorem{conj}[thm]{Conjecture}
\newtheorem*{conj*}{Conjecture}

\theoremstyle{definition}
\newtheorem{defn}[thm]{Definition}
\newtheorem*{defn*}{Definition}
\newtheorem{example}[thm]{Example}
\newtheorem*{example*}{Example}
\newtheorem{examples}[thm]{Examples}
\newtheorem*{examples*}{Examples}
\newtheorem{alg}[thm]{Algorithm}
\newtheorem*{alg*}{Algorithm}
%\theoremstyle{remark}
\newtheorem{rmk}[thm]{Remark}
\newtheorem*{rmk*}{Remark}
\newtheorem{rmks}[thm]{Remarks}
\newtheorem*{rmks*}{Remarks}

%%%%%%%%%%%%%%%%%%%%%%%%%%%%%%%%%%%%%%%%%%%%%%%%%%%%%%%%%%%%%%%%%%%
\numberwithin{equation}{section} 
%% The following avoids conflict between numbers of proclamations 
%% and numbers of equations
\renewcommand{\theequation}{\thesection\alph{equation}} 
%%%%%%%%%%%%%%%%%%%%%%%%%%%%%%%%%%%%%%%%%%%%%%%%%%%%%%%%%%%%%%%%%%%
\parskip = 2pt
\allowdisplaybreaks
\renewcommand{\labelenumi}{(\theenumi)} % use round brackets
\renewcommand{\theenumi}{\alph{enumi}} % use alphabetic enumerations


\pagestyle{plain} % suppress the running head - for working document

\title[Abstract Algebra]{Abstract Algebra}
\author{George H. Seelinger (inspired from class by Abramenko)}
\date{Fall 2016}
\begin{document}
\maketitle
\section{Group Theory}
\subsection{Basic Facts}
\subsection*{Lecture 1: Normalizers and Centralizers}
\begin{defn}
    For $A \subset G$, we set $N_G(A) := \{g \in G | gAg^{-1} = A\}$ and
    $C_G(A) := \{g \in G | gag^{-1} = a, \forall a \in A\}$. 
\end{defn}
Note that $C_G(A) \subset N_G(A)$.
\begin{rmk*}
    If $A \subgroup G$, then $A \normsubgroup N_G(A)$. In fact, $N_G(A)$ is the
    largest subgroup of $G$ in which $A$ is normal.
\end{rmk*}
    The first part of this remark follows from the definition of $N_G(A)$. The
    second part follows by assuming that $N_G(A)$ is not the largest and then
    showing the ``largest'' is contained in $N_G(A)$.
\begin{rmk*}
    If $A \subgroup G$, then $A \normsubgroup G$ if $N_G(A) = G$.
\end{rmk*}
    This follows from the second part of the remark above.
    Now for some remakrs about centralizers.
    \begin{rmk*}
        If $A \subgroup G$, then $A$ is abelian if and only if $A \subset
        C_G(A)$. Furthermore, if $A \subgroup Z(G)$, then $A \normsubgroup G$.
        This follows using basic commutativity arguments.
    \end{rmk*}
    \begin{example}
        $Z(S_n) = \{ id \}$ if $n \geq 3$. To prove this, prove $\sigma \in S_n, \sigma \neq id \implies \sigma \notin Z(S_n)$. 

        $\sigma \neq id \implies \exists i,j \in \{1, \ldots, n\}$ where $i
        \neq j$ such that $\sigma(i) = j$. Now, choose $k \in \{1, \ldots, n\}$
        such that $k \neq i,j$. Now, let $\pi = \sigma(i,k)\sigma^{-1} =
        (\sigma(i), \sigma(j)) = (j, \sigma(k))$. So now, $\pi(j) = \sigma(k)
        \neq \sigma(i) = j$. However, $(ik)(j)=j$ since $j \neq i,k$. This
        means $\pi \neq (ik)$, but if $\sigma$ were in the center, then it
        would have. So, we have that $\sigma \notin C_{S_n}( (ik) ) \implies
        \sigma \notin Z(S_n)$.
    \end{example}
    \begin{lem}
        Let $G$ be a group and $H \subgroup Z(G)$. Note this means $H
        \normsubgroup G$. If $G/H$ is cyclic, then $G$ is abelian.
    \end{lem}
    \begin{proof}
        Since $G/H$ is cyclic, this means $\exists g \in G$ such that
        $G/H=\langle gH \rangle$. Now, let $x,y \in G$. Then, there are $n,m
        \in \Z$ such that $xH = (gH)^n = g^nH$ and, similarly, $yH = (gH)^m =
        g^mH$. This means $\exists h,h' \in H$ such that $x=g^nh$ and
        $y=g^mh'$. Thus, we get $xy = g^nhg^mh' = g^ng^mhh' = g^{n+m}hh'$ and
        $yx = g^mh'g^nh = g^mg^nh'h = g^{n+m}h'h = g^{n+m}hh'$. Thus, we have
        our commutativity, thereby showing $G$ is abelian.
    \end{proof}
    \begin{rmk*}
        Note, simply having $H \subgroup Z(G)$ and $G/H$ abelian does not imply
        $G$ is abelian. For a counter-example, look at $Q_8 = \{\pm 1, \pm i,
        \pm j, \pm k\}$, the standard quaternion group. It is easy to compute
        that $Z(Q_8) = \{\pm 1\}$. Now, $Q_8/Z(Q_8) \cong \Z_2 \times \Z_2$,
        the Klien-4 group, which is abelian. However, $Q_8$ is not abelian!
    \end{rmk*}
    \begin{defn}
        Let $A,B \subgroup G$, then \begin{itemize}
            \item $A^{-1} = \{a^{-1} | a \in A\}$.
            \item $AB = \{ab | a \in A, b \in B\}$.
        \end{itemize}
    \end{defn}
    \begin{lem}
        Let $G$ be a group with $A,B \subgroup G$. Then,
        \begin{enumerate}
            \item $AB \subgroup G$ if and only if $AB = BA$.
            \item If $A \subgroup N_G(B)$ or $B \subgroup N_G(A)$, then $AB =
                BA$ and hence $AB \subgroup G$ by 1. 
        \end{enumerate}
    \end{lem}
    \begin{proof}
        ($\Rightarrow$) Let $AB \subgroup G$. Then, $AB = (AB)^{-1} =
        B^{-1}A^{-1} = BA$.

        ($\Leftarrow$) Let $AB = BA$. Then, we simply must show that $AB$ has
        group properties.\begin{enumerate}
            \item $e \in A, e \in B \implies e = ee \in AB$.
            \item This is the same as above. $(AB)^{-1} = B^{-1}A^{-1} = B! = AB$.
            \item $(AB)(AB) = A(BA)B = A(AB)B = (AA)(BB) = AB$.
        \end{enumerate}. Thus, $AB \subgroup G$.

        Now, for the second part, let $A \subgroup N_G(B)$. Then, let $a \in A,
        b \in B$. We know $a,a^{-1} \in A \subset N_G(B)$. Then, we have $ab =
        aba^{-1}a = (aba^{-1})a$. However, we know $aba^{-1} \in B$ from the
        definition of $N_G(B)$. So, $AB \subset BA$. Similar logic shows $BA
        \subset AB$ and thus $AB = BA$.
    \end{proof}

    \subsection*{Lecture 2}
    \begin{rmk}
        \begin{enumerate}
            \item If $A,B \subgroup G$ and if $A \normsubgroup G$ or $B \normsubgroup G$, then $AB \subgroup G$.
            \item If $A,B \normsubgroup G$, then $AB \normsubgroup G$.
        \end{enumerate}
    \end{rmk}
    \begin{proof}
        The first remark can be done by $(ab)(a'b') = aba'b^{-1}bb' = aa'b' = (aba'b^{-1})b' \in AB$ because $a$ and $ba'b^{-1}$ are in $A$ and $b \in B$. Also, $(ab)^{-1} = b^{-1}a^{-1} = (b^{1}a^{-1}b)b^{-1} \in AB$. The identity inclusion is obvious.
        The second remark is proven by, for $g \in G$, $gABg^{-1} = gAg^{-1}gBg^{-1} = AB$. 
    \end{proof}

    \subsubsection*{Index Computations}

    \begin{lem}
        Let $A,B \subgroup G$ and $|A|, |B| < \infty$. Then, $|AB| = \frac{|A| \cdot |B|}{A \cap B}$.
    \end{lem}
    \begin{proof}
        Consider $A/(A \cap B)$. Take a complete set of coset representatives $A' = \{a_1, \ldots, a_n\}$. Then, $n = |A/(A \cap B)| = [A : A \cap B] = \frac{|A|}{|A \cap B|}$ by Lagrange's Theorem.

        Now, consider $f: A' \times B \to AB$ defined by $(a_i,b) \to a_ib$. If
        we show that $f$ is bijective, then we will have that $|A'| \cdot |B| =
        |A' \times B| = |AB|$, which will allow us to finish the proof.
        To show $f$ is injective, let us take \begin{align*}
            a_ib = a_jb' (1 \leq i,j \leq n; b,b' \in B) & \ \implies \ a_j^{-1} a_i = b'b^{-1} \in A\cap B\\
            \ & \ \implies \ a_i(A \cap B) = a_j(A \cap B) \\
            \ & \ \implies \ a_i = a_j \\
            a_ib = a_jb' & \ \implies b = b' \\
            \ & \ \implies (a_i, b) = (a_j, b')
        \end{align*}

        To show $f$ is surjective, we let $a \in A, b \in B$. Then $\exists i
        (1 \leq i \leq n)$ such that $a=a_ix$ with $x \in A \cup B$. This
        implies that $xb \in B$ so $ab = a_ixb = f( (a_i,xb) )$.

        Thus, we have that $f$ is bijective, and so we conclude that $n \cdot
        |B| = \frac{|A|}{|A \cap B|} \cdot |B|$ giving us our result when we
        divide by $|B|$.
    \end{proof}
    \begin{example}
        (a) $G = \S_3, A = \langle (12) \rangle, B = \langle (13) \rangle
        \subgroup G$. Then, $A \cap B = \{ id \} \implies |AB| = 2 \cdot 2 = 4$
        which does not divide $6 = |\S_3|$. Thus, $AB$ is not a subgroup of
        $\S_3$. 
        (b) Let $G=\S_4, A = S_3 \to S_4, B = \langle (1234) \rangle$. Then, $A \cap B = \{id\} \implies |AB| = |A||B| = 6 \cdot 4 = 24 = |\S_4|$. This means that $AB = \S_4$ but note that $A \not\subset N_G(B)$ and $B \not\subset N_G(A)$. 
    \end{example}
    \begin{lem}
        (Lemma 1.1.9) Let $G$ be a group and $A,B \subgroup G$. Then
        \begin{enumerate}[label=\Alph*]
            \item If $A \subgroup B$, then $[G:A] = [G:B][B:A]$
            \item $[A: A \cup B] \leq [G:B]$
            \item $[G: A \cup B] \leq [G:A][G:B]$
        \end{enumerate}
    \end{lem}
    \begin{proof}
       For the first part, fix a set of coset representatives (using the Axiom of Choice in the infinite case) for $G/A$. Then, given a coset $gA$, we can TODO 
\end{proof}
\subsection*{The Isomorphism Theorems}
\begin{thm}
    (First Isomorphism Theorem) If $\phi: G \to H$ is a group homomorphism, then $G/\ker \phi \cong \phi(G) \subgroup H$.
\end{thm}
\begin{proof}
    Set $N = \ker \phi \normsubgroup G$. Then, consider $\widetilde{\phi}: G/N
    \to \phi(G)$ defined by $\widetilde{\phi}(gN) = \phi(g)$. First, we check
    that $\widetilde{\phi}$ is well-defined.

    Assume $gN = g'N, g,g' \in G$. Then, there exists $n \in N$ such that $g'
    = gn$. This tells us that $\phi(g') = \phi(gn) = \phi(g)\phi(n) = \phi(g)e
    = \phi(g)$. Now, we show that $\widetilde{\phi}$ is a group homomorphism.

    Check $\widetilde{\phi}( (gN)(g'N) ) = \widetilde{\phi}(gg'N) = \phi(gg') =
    \phi(g)\phi(g') = \widetilde{\phi}(gN) \widetilde{\phi}(g'N)$.

    We know $\widetilde{\phi}$ is surjective by definition. 

    To show $\widetilde{\phi}$ is injective, it suffices to show that $\ker
    \widetilde{\phi} = \{\widetilde{e} = eN\}$. If $\widetilde{\phi}(gN) =
    \phi(g) = e \in H$, then $g \in \ker \phi = N$ and therefore $gN = N =
    \widetilde{e}$
\end{proof}
\begin{cor}
    (Second Isomorphism Theorem) If $A,B \subgroup G$ and $A \subgroup N_G(B)$,
    then $AB/B \cong A/A\cap B$.
\end{cor}
\begin{proof}
    First, we know since $A \subgroup N_G(B)$ that $AB \subgroup N_G(B)$. Then,
    by definition of normalizer, $B \normsubgroup AB$. Now, define a group
    homomorphism $\phi: A \to AB/B$ by $\phi(a) = aB$. Then, we can use the
    first isomorphism theorem to get that $A/\ker \phi \cong \phi(A)$. Now,
    $\ker \phi$ is the set of all $a \in A$ that map to an element in $B$. This
    clearly only happens when $a \in B$. So, $\ker \phi = A \cap B$.
    Furthermore, it is clear $\phi(A) = AB/B$ (as opposed to $A/B$) because $B$
    need not be a subset of $A$. Thus, elements of the form $aB$ where $a \in A$
    are in $AB/B$. (This part is still unclear to me.) Thus, we get $A/A \cap B
\cong AB/B$. 
\end{proof}
The second isomorphism theorem is sometimes referred to as the ``diamond
theorem'' because if you draw a diagram of what is happening, it is a diamond
with $A, B, AB$, and $A \cap B$. 
\begin{cor}
    (Third Isomorphism Theorem) If $A,B \normsubgroup G$ and $A \subgroup B$, then $G/B \isom (G/A)/(A/B)$.
\end{cor}
\begin{proof}
    Let us define a group homomorphism $\phi: G/A \to G/B$ by $\phi(gA) = gB$.
    Then, let us examine $\ker \phi$. We note that if $g \in B \subgroup A$,
    then $\phi(gA) = B$. Thus, $\ker \phi = A/B$. Now, applying the first
    isomorphism theorem with $\phi$, we get that $(G/A)/(A/B) \isom \phi(gA)$.
    $\phi$ is clearly surjective because $B \subgroup A$, so we conclude that
    $(G/A)/(G/B) \isom B/A$.
\end{proof}

These isomorphism theorems are useful in many proof techniques and there are
analogues for many different algebraic structures. Next, we consider the
Correspondance Theorem which is sometimes referred to as the fourth isomorphism
theorem. The fundamental idea of the correspondance theorem is that the
structure of subgroups of $G/N$ for $N \normsubgroup G$ is the same as the
structore of the subgroups of $G$ containing $N$ where $N$ is ``collapsed'' as
the identity element.

\begin{thm}
    Let $G$ be a group, $N \subgroup G$, and $\overline{G} = G/N$. Next, let
    $\phi_N = \{ A \subgroup G | N \subgroup A \}$ and $\overline{\phi}$ be the
    set of all subgroups of $\overline{G}$. Then, all the following are true for $A,B \in \phi_N$.
    \begin{enumerate}[label=(\alph*)]
        \item $A \subgroup B$ if and only if $\overline{A} \subgroup \overline{B}$.
        \item $A \normsubgroup B$ if and only if $\overline{A} \normsubgroup
            \overline{B}$.
        \item If $A \subgroup B$, then $[B:A] = [\overline{B}:\overline{A}]$.
        \item $\overline{A \cap B} = \overline{A} \cap \overline{B}$.
        \item The map $\rho: \phi_N \to \overline{\phi}$ where $\rho(A) =
            \overline{A}$ is bijective.
        \item A map from $\{ A \in \phi_N | A \normsubgroup G\}$ to the set of
            normal subgroups of $\overline{G}$ defined by $A \mapsto
            \overline{A}$ is bijective.
    \end{enumerate}
\end{thm}
\begin{proof}
    Proofs for (a) -- (d) were left as an exercise (to be filled in later). 
    For (e), let $\pi: G \twoheadrightarrow \overline{G}$ be the surjective group homomorphism defined by $\pi(g) = gN$. Now, consider $\psi: \overline{\phi} \to \phi_N$ defined by $\psi(H) = \pi^{-1}(H) \subgroup G$. Then, we want to show that $\rho \circ \psi = id_{\overline{\phi}}$ and $\psi \circ \rho = id_{\phi_N}$ because then we will know $\rho$ is invertible and thus bijective. 

    Examine $\rho \circ \psi$. We know that $(\rho \circ \psi)(H) = \rho(\pi^{-1}(H))$. Now, note that $\rho(A)$ is the image of $\pi(A)$, so we get $\rho(\pi^{-1}(H)) = \pi(\pi^{-1}(H)) = H$ because $\pi$ is surjective. 

    Next, examine $\psi \circ \rho$. We then get that $(\psi \circ \rho)(A) = \pi^{-1}(\pi(A))$ for $A \subgroup G$. We would like for this to be equal $A$. It is clear that $A \subset \pi^{-1}(\pi(A))$ because the pre-image of $A/N$ under $\pi$ must at least contain $A$. So, we still need to show that $\pi^{-1}(\pi(A)) \subset A$. To do this, take $x \in \pi^{-1}(A/N)$. Then, $xN = \pi(x) \in A/B$, so $xN = aN$ for some $a \in A$. This means that there exists an $n \in N$ such that $x = an$ since $N \subgroup A$. So, $\pi^{-1}(A/N) \subset A$.

    Thus, we have shown that $\rho$ and $\psi$ are bijective maps that are inverse to each other and we have proven (e). \\

    (f) follows from the same proof technique as (e), I believe.
\end{proof}

\subsubsection*{Commutators}

\begin{defn}
    The commutator of two elements $x,y \in G$ is defined as \[
        [x,y] = xyx^{-1}y^{-1}
    \]
\end{defn}

Note that mathematicians have not completely agreed on the definition and thus some references will say $[x,y] = x^{-1}y^{-1}xy$. In essence, the commutator of two elements indicates how ``close'' elements are to commuting. This can be seen in the following lemma.

\begin{lem}
    For $x,y,z \in G$, we obtain
    \begin{enumerate}[label=(\alph*)]
        \item $[x,y] = e$ if and only if $xy=yx$ if and only if $[x^{-1},y^{-1}] = e$.
        \item $[x,y]^{-1} = [y,x]$.
        \item $z[x,y]z^{-1} = [zxz^{-1},zyz^{-1}]$.
        \item If $\phi: G \to H$ is a group homoromorphism, then $[\phi(g),\phi(h)] = \phi([g,h])$.
        \item $xy = [x,y]yx = yx[x^{-1},y^{-1}]$.
        \item $[xy,z] = x[y,z]x^{-1}[x,z]$
        \item $[x,yz] = [x,y]y[x,z]z^{-1}$.
    \end{enumerate}
\end{lem}
\begin{proof}
    All proofs are straightforward computations.
\end{proof}
\subsection*{Lecture 3: Continuing with Commutators}
\begin{defn}
    Let $A, B \subgroup G$. Then, the \emph{commutator} of $A$ and $B$ is the subgroup of $G$ denoted $[A,B] = \langle \{[a,b] | a \in A, b \in B\} \rangle$. Also note that $[G,G]$ is called the \emph{commutator subgroup of $G$}.
\end{defn}
\begin{rmk}
    Let $A,B \subgroup G$. Then
    \begin{enumerate}
        \item $[A,B] = \{e\}$ if and only if $ab = ba \forall a \in A, b \in B$ if and only if $A \subgroup C_G(B)$ if and only if $B \subgroup C_G(A)$. In this case, we say that $A$ and $B$ commute.
        \item $[G,G] = \{e\}$ if and only if $G$ is abelian if and only if $Z(G) = G$.
        \item $[A,B] = [B,A]$ since $[a,b]^{-1} = [b,a]$.
    \end{enumerate}
\end{rmk}
\begin{example}
    \begin{enumerate}
        \item $[Q_8,Q_8] = \{1,-1\} = [\langle i \rangle, \langle j \rangle]$. This can be seen because $[i,j] = iji^{-1}j^{-1} = ij(-i)(-j) = ij(-i)(-j) = iij(-j) = -i^2j^2 = -1$. 
        \item $[A_n, \langle (12) \rangle] = A_n$ for $n geq 2$. This follows because, for $i \geq 3$, we get $[(1i2),(12)] = (1i2)(12)(2i1)(12) = (12i)$. This tells us that $A_n = \langle \{(12i) | 3 \leq i \leq n\} \rangle \subset [A_n, \langle (12)\rangle$.
        \item $[S_n,S_n] = A_n$ for $n \geq 2$. To show this, for ($\subset$) we have \begin{align*}
                \sgn( [\sigma,\tau] ) & = \sgn(\sigma \tau \sigma^{-1} \tau^{-1}) \\
                \ & = \sgn(\sigma) \sgn(\tau) \sgn(\sigma)^{-1} \sgn(\tau)^{-1} \\
                \ & = 1 \forall \sigma, \tau \in S_n
            \end{align*}
            For ($\supset$), see the previous item.
        \item $[A_n,A_n] = A_n$ for $n \geq 5$. This is because every 3 cycle is a commutator. For instance $[(ijk),(klm)] = (ilk)$ for 5 distinct elements $i,j,k,l,m \in \{1, \ldots, n\}$. Also note that $[A_3,A_3] = \{e\}$ and $[A_4,A_4] = \Z_2 \times \Z_2$, the set of all elements of the form $(ij)(kl)$. 
    \end{enumerate}
\end{example}
\begin{lem}
    Let $A,B \subgroup G$. Then,
    \begin{enumerate}[label=(\alph*)]
        \item $A \subgroup N_G(B)$ if and only if $[A,B] \subgroup B$.
        \item $B \subgroup N_G(A)$ if and only if $[A,B] \subgroup A$.
        \item If $A,B \normsubgroup G$ then $[A,B] \normsubgroup G$ and $[A,B] \subgroup A \cap B$
    \end{enumerate}
\end{lem}
\begin{proof}
    The forward direction of $A$ is given by the fact that $[a,b] =
    aba^{-1}b^{-1}$ but $aba^{-1} \in B$ since $A$ normalizes $B$. Thus, $[a,b]
    \in B \forall a \in A, b \in B$ so $[A,B] \subgroup B$. To do the other
    direction, we have that, we have that $aba^{-1}b^{-1} \in B, \forall a \in A,
    \forall b \in B$. Well, $(aba^{-1}b^{-1})b = aba^{-1} \in B$, so $A$
    normalizes $B$, so $A \subgroup N_G(B)$. \\

    (b) is similar to (a). \\

    Now, given that $A,B \normsubgroup G$, then $g[a,b]g^{-1} = gaba^{-1}b^{-1}g^{-1} = gag^{-1}gbg^{-1}ga^{-1}g^{-1}gb^{-1}g^{-1}$. Now note that $gag^{-1} \in A$ since $A \normsubgroup G$ and $(gag^{-1})^{-1} = ga^{-1}g^{-1}$. Similarly for $b$. So, we have that $g[a,b]g^{-1} \in [A,B] \implies [A,B] \normsubgroup G$. Finally, an element of the form $aba^{-1}b^{-1} \in A$ because $ba^{-1}b^{-1} \in A$ since $A$ is normal in $G$ and similarly $aba^{-1} \in B$. So, $[A,B] \subgroup A \cap B$. This final fact also follows from the first two items in the lemma. 
\end{proof}

This lemma also gives us the important fact that $[G,G] \normsubgroup G$. This follows easily from the third item.

\begin{prop}
    If $H \subgroup G$, then $[G,G] \subgroup H$ if and only if $H \normsubgroup G$ and $G/H$ is abelian.
\end{prop}
\begin{proof}
    Let $[G,G] \subgroup H$. Then, if $a,b \in G$, $aba^{-1}b^{-1} \in H$. Now, in particular, if $b \in H$, then $aba^{-1}b^{-1} \cdot b = aba^{-1} \in H$ so $H \normsubgroup G$. More concisely, we have that $[G,H] \subgroup [G,G] \subgroup H \implies G \subgroup N_G(H)$ so $H \normsubgroup G$. To show $G/H$ is abelian, take a map $[a,b] \mapsto [aH,bH] = (aH)(bH)(aH)^{-1}(bH)^{-1} = aba^{-1}b^{-1}H = [x,y]H = H = e_{G/H}$. So, then $[G/H,G/H] = \{e\}$. 

    To prove the other direction, let $G/H$ be abelian and $H \normsubgroup G$. Since $G/H$ is abelian, we have that $[aH,bH] = [a,b]H = H$. This means that $[a,b] \in H$ which implies that $[G,G] \in H$. (Where is the normalcy of $H$ used?!)
\end{proof}

\subsubsection*{Direct Products}

\end{document}
