\documentclass[11pt,leqno,oneside]{amsart}

\usepackage{../notes}
\usepackage{enumitem}
\newcommand{\Fp}{{\mathbb{F}_p}} % finite field of order p
\newcommand{\subgroup}{\mathrel{<}}
\newcommand{\normsubgroup}{\mathrel{\unlhd}}
\newcommand{\Aut}{\text{Aut}}  % automorphism group
\newcommand{\primedecomposition}[3]{#1_1^{#2_1} \cdots {#1_{#3}}^{#2_{#3}}}
%%%%%%%%%%%%%% BEGIN CONTENT: %%%%%%%%%%%%%%

\title[Abstract Algebra]{Abstract Algebra}
\author{George H. Seelinger, Chris Lloyd\\ (inspired from class by Abramenko)}
\date{Fall 2016}
\begin{document}
\maketitle
\section{Group Theory}
\subsection{Basic Facts}
\subsection*{Lecture 1: Normalizers and Centralizers}
\begin{defn}
    For $A \subset G$, we set $N_G(A) := \{g \in G | gAg^{-1} = A\}$ and
    $C_G(A) := \{g \in G | gag^{-1} = a, \forall a \in A\}$.
\end{defn}
Note that $C_G(A) \subset N_G(A)$.
\begin{rmk*}
    If $A \subgroup G$, then $A \normsubgroup N_G(A)$. In fact, $N_G(A)$ is the
    largest subgroup of $G$ in which $A$ is normal.
\end{rmk*}
    The first part of this remark follows from the definition of $N_G(A)$. The
    second part follows by assuming that $N_G(A)$ is not the largest and then
    showing the ``largest'' is contained in $N_G(A)$.
\begin{rmk*}
    If $A \subgroup G$, then $A \normsubgroup G$ if $N_G(A) = G$.
\end{rmk*}
    This follows from the second part of the remark above.
    Now for some remakrs about centralizers.
    \begin{rmk*}
        If $A \subgroup G$, then $A$ is abelian if and only if $A \subset
        C_G(A)$. Furthermore, if $A \subgroup Z(G)$, then $A \normsubgroup G$.
        This follows using basic commutativity arguments.
    \end{rmk*}
    \begin{example}
        $Z(S_n) = \{ id \}$ if $n \geq 3$. To prove this, prove $\sigma \in S_n, \sigma \neq id \implies \sigma \notin Z(S_n)$.

        $\sigma \neq id \implies \exists i,j \in \{1, \ldots, n\}$ where $i
        \neq j$ such that $\sigma(i) = j$. Now, choose $k \in \{1, \ldots, n\}$
        such that $k \neq i,j$. Now, let $\pi = \sigma(i,k)\sigma^{-1} =
        (\sigma(i), \sigma(j)) = (j, \sigma(k))$. So now, $\pi(j) = \sigma(k)
        \neq \sigma(i) = j$. However, $(ik)(j)=j$ since $j \neq i,k$. This
        means $\pi \neq (ik)$, but if $\sigma$ were in the center, then it
        would have. So, we have that $\sigma \notin C_{S_n}( (ik) ) \implies
        \sigma \notin Z(S_n)$.
    \end{example}
    \begin{lem}
        Let $G$ be a group and $H \subgroup Z(G)$. (Note this implies $H
        \normsubgroup G$.) If $G/H$ is cyclic, then $G$ is abelian.
    \end{lem}
    \begin{proof}
        Since $G/H$ is cyclic, this means $\exists g \in G$ such that
        $G/H=\langle gH \rangle$. Now, let $x,y \in G$. Then, there are $n,m
        \in \Z$ such that $xH = (gH)^n = g^nH$ and, similarly, $yH = (gH)^m =
        g^mH$. This means $\exists h,h' \in H$ such that $x=g^nh$ and
        $y=g^mh'$. Thus, we get $xy = g^nhg^mh' = g^ng^mhh' = g^{n+m}hh'$ and
        $yx = g^mh'g^nh = g^mg^nh'h = g^{n+m}h'h = g^{n+m}hh'$. Thus, we have
        our commutativity, thereby showing $G$ is abelian.
    \end{proof}
    \begin{rmk*}
        Note, simply having $H \subgroup Z(G)$ and $G/H$ abelian does not imply
        $G$ is abelian. For a counter-example, look at $Q_8 = \{\pm 1, \pm i,
        \pm j, \pm k\}$, the standard quaternion group. It is easy to compute
        that $Z(Q_8) = \{\pm 1\}$. Now, $Q_8/Z(Q_8) \isom \Z_2 \times \Z_2$,
        the Klien-4 group, which is abelian. However, $Q_8$ is not abelian!
    \end{rmk*}
    \begin{defn}
        Let $A,B \subgroup G$, then \begin{itemize}
            \item $A^{-1} = \{a^{-1} | a \in A\}$.
            \item $AB = \{ab | a \in A, b \in B\}$.
        \end{itemize}
    \end{defn}
    \begin{lem}
        Let $G$ be a group with $A,B \subgroup G$. Then,
        \begin{enumerate}
            \item $AB \subgroup G$ if and only if $AB = BA$.
            \item If $A \subgroup N_G(B)$ or $B \subgroup N_G(A)$, then $AB =
                BA$ and hence $AB \subgroup G$ by 1.
        \end{enumerate}
    \end{lem}
    \begin{proof}
        ($\Rightarrow$) Let $AB \subgroup G$. Then, $AB = (AB)^{-1} =
        B^{-1}A^{-1} = BA$.

        ($\Leftarrow$) Let $AB = BA$. Then, we simply must show that $AB$ has
        group properties.\begin{enumerate}
            \item $e \in A, e \in B \implies e = ee \in AB$.
            \item This is the same as above. $(AB)^{-1} = B^{-1}A^{-1} = B! = AB$.
            \item $(AB)(AB) = A(BA)B = A(AB)B = (AA)(BB) = AB$.
        \end{enumerate}. Thus, $AB \subgroup G$.

        Now, for the second part, let $A \subgroup N_G(B)$. Then, let $a \in A,
        b \in B$. We know $a,a^{-1} \in A \subset N_G(B)$. Then, we have $ab =
        aba^{-1}a = (aba^{-1})a$. However, we know $aba^{-1} \in B$ from the
        definition of $N_G(B)$. So, $AB \subset BA$. Similar logic shows $BA
        \subset AB$ and thus $AB = BA$.
    \end{proof}

    \subsection*{Lecture 2}
    \begin{rmk}
        \begin{enumerate}
            \item If $A,B \subgroup G$ and if $A \normsubgroup G$ or $B \normsubgroup G$, then $AB \subgroup G$.
            \item If $A,B \normsubgroup G$, then $AB \normsubgroup G$.
        \end{enumerate}
    \end{rmk}
    \begin{proof}
        The first remark can be done by $(ab)(a'b') = aba'b^{-1}bb' = aa'b' = (aba'b^{-1})b' \in AB$ because $a$ and $ba'b^{-1}$ are in $A$ and $b \in B$. Also, $(ab)^{-1} = b^{-1}a^{-1} = (b^{1}a^{-1}b)b^{-1} \in AB$. The identity inclusion is obvious.
        The second remark is proven by, for $g \in G$, $gABg^{-1} = gAg^{-1}gBg^{-1} = AB$.
    \end{proof}

    \subsubsection*{Index Computations}

    \begin{lem}
        Let $A,B \subgroup G$ and $|A|, |B| < \infty$. Then, $|AB| = \frac{|A| \cdot |B|}{|A \cap B|}$.
    \end{lem}
    \begin{proof}
        Consider $A/(A \cap B)$. Take a complete set of coset representatives $A' = \{a_1, \ldots, a_n\}$. Then, $n = |A/(A \cap B)| = [A : A \cap B] = \frac{|A|}{|A \cap B|}$ by Lagrange's Theorem.

        Now, consider $f: A' \times B \to AB$ defined by $(a_i,b) \to a_ib$. If
        we show that $f$ is bijective, then we will have that $|A'| \cdot |B| =
        |A' \times B| = |AB|$, which will allow us to finish the proof.
        To show $f$ is injective, let us take \begin{align*}
            a_ib = a_jb' (1 \leq i,j \leq n; b,b' \in B) & \ \implies \ a_j^{-1} a_i = b'b^{-1} \in A\cap B\\
            \ & \ \implies \ a_i(A \cap B) = a_j(A \cap B) \\
            \ & \ \implies \ a_i = a_j \\
            a_ib = a_jb' & \ \implies b = b' \\
            \ & \ \implies (a_i, b) = (a_j, b')
        \end{align*}

        To show $f$ is surjective, we let $a \in A, b \in B$. Then $\exists i
        (1 \leq i \leq n)$ such that $a=a_ix$ with $x \in A \cup B$. This
        implies that $xb \in B$ so $ab = a_ixb = f( (a_i,xb) )$.

        Thus, we have that $f$ is bijective, and so we conclude that $n \cdot
        |B| = \frac{|A|}{|A \cap B|} \cdot |B|$ giving us our result when we
        divide by $|B|$.
    \end{proof}
    \begin{example}
        (a) $G = \S_3, A = \langle (12) \rangle, B = \langle (13) \rangle
        \subgroup G$. Then, $A \cap B = \{ id \} \implies |AB| = 2 \cdot 2 = 4$
        which does not divide $6 = |\S_3|$. Thus, $AB$ is not a subgroup of
        $\S_3$.
        (b) Let $G=\S_4, A = S_3 \to S_4, B = \langle (1234) \rangle$. Then, $A \cap B = \{id\} \implies |AB| = |A||B| = 6 \cdot 4 = 24 = |\S_4|$. This means that $AB = \S_4$ but note that $A \not\subset N_G(B)$ and $B \not\subset N_G(A)$.
    \end{example}
    \begin{lem}
        Let $G$ be a group and $A,B \subgroup G$. Then
        \begin{enumerate}
            \item If $A \subgroup B$, then $[G:A] = [G:B][B:A]$
            \item $[A: A \cap B] \leq [G:B]$
            \item $[G: A \cap B] \leq [G:A][G:B]$
        \end{enumerate}
    \end{lem}
    \begin{proof}
       For the first part, fix a set of coset representatives (using the Axiom of Choice in the infinite case) for $G/A$. Then, given a coset $gA$, we can TODO
\end{proof}
\subsection*{The Isomorphism Theorems}
\begin{thm}
    (First Isomorphism Theorem) If $\phi: G \to H$ is a group homomorphism, then $G/\ker \phi \isom \phi(G) \subgroup H$.
\end{thm}
\begin{proof}
    Set $N = \ker \phi \normsubgroup G$. Then, consider $\widetilde{\phi}: G/N
    \to \phi(G)$ defined by $\widetilde{\phi}(gN) = \phi(g)$. First, we check
    that $\widetilde{\phi}$ is well-defined.

    Assume $gN = g'N, g,g' \in G$. Then, there exists $n \in N$ such that $g'
    = gn$. This tells us that $\phi(g') = \phi(gn) = \phi(g)\phi(n) = \phi(g)e
    = \phi(g)$. Now, we show that $\widetilde{\phi}$ is a group homomorphism.

    Check $\widetilde{\phi}( (gN)(g'N) ) = \widetilde{\phi}(gg'N) = \phi(gg') =
    \phi(g)\phi(g') = \widetilde{\phi}(gN) \widetilde{\phi}(g'N)$.

    We know $\widetilde{\phi}$ is surjective by definition.

    To show $\widetilde{\phi}$ is injective, it suffices to show that $\ker
    \widetilde{\phi} = \{\widetilde{e} = eN\}$. If $\widetilde{\phi}(gN) =
    \phi(g) = e \in H$, then $g \in \ker \phi = N$ and therefore $gN = N =
    \widetilde{e}$
\end{proof}
\begin{cor}
    (Second Isomorphism Theorem) If $A,B \subgroup G$ and $A \subgroup N_G(B)$,
    then $AB/B \isom A/A\cap B$.
\end{cor}
\begin{proof}
    First, we know since $A \subgroup N_G(B)$ that $AB \subgroup N_G(B)$. Then,
    by definition of normalizer, $B \normsubgroup AB$. Now, define a group
    homomorphism $\phi: A \to AB/B$ by $\phi(a) = aB$. Then, we can use the
    first isomorphism theorem to get that $A/\ker \phi \isom \phi(A)$. Now,
    $\ker \phi$ is the set of all $a \in A$ that map to an element in $B$. This
    clearly only happens when $a \in B$. So, $\ker \phi = A \cap B$.
    Furthermore, it is clear $\phi(A) = AB/B$ (as opposed to $A/B$) because $B$
    need not be a subset of $A$. Thus, elements of the form $aB$ where $a \in A$
    are in $AB/B$. (This part is still unclear to me.) Thus, we get $A/A \cap B
    \isom AB/B$.
\end{proof}
The second isomorphism theorem is sometimes referred to as the ``diamond
theorem'' because if you draw a diagram of what is happening, it is a diamond
with $A, B, AB$, and $A \cap B$.
\begin{cor}
    (Third Isomorphism Theorem) If $A,B \normsubgroup G$ and $A \subgroup B$, then $(G/A)/(B/A) \isom G/B$.
\end{cor}
\begin{proof}
    Let us define a group homomorphism $\phi: G/A \to G/B$ by $\phi(gA) = gB$.
    Then, let us examine $\ker \phi$. We note that if $t \in B$,
    then $\phi(tA) = B$. Thus, $\ker \phi = \{tA | t \in B\} = B/A$. Now, applying the first
    isomorphism theorem with $\phi$, we get that $(G/A)/(B/A) \isom \phi(G/A)$.
    $\phi$ is clearly surjective because $A \subgroup B$, so we conclude that
    $(G/A)/(B/A) \isom G/B$.
\end{proof}

These isomorphism theorems are useful in many proof techniques and there are
analogues for many different algebraic structures. Next, we consider the
Correspondance Theorem which is sometimes referred to as the fourth isomorphism
theorem. The fundamental idea of the correspondance theorem is that the
structure of subgroups of $G/N$ for $N \normsubgroup G$ is the same as the
structore of the subgroups of $G$ containing $N$ where $N$ is ``collapsed'' as
the identity element.

\begin{thm}[Lattice Isomorphism Theorem]
    Let $G$ be a group, $N \normsubgroup G$, and $\overline{G} = G/N$. Next, let
    $\phi_N = \{ A \subgroup G | N \subgroup A \}$ and $\overline{\phi}$ be the
    set of all subgroups of $\overline{G}$. Then, all the following are true for $A,B \in \phi_N$.
    \begin{enumerate}[label=(\alph*)]
        \item $A \subgroup B$ if and only if $\overline{A} \subgroup \overline{B}$..
        \item $A \normsubgroup B$ if and only if $\overline{A} \normsubgroup
            \overline{B}$.
        \item If $A \subgroup B$, then $[B:A] = [\overline{B}:\overline{A}]$.
        \item $\overline{\langle A, B \rangle} = \langle \overline{A}, \overline{B} \rangle$
        \item $\overline{A \cap B} = \overline{A} \cap \overline{B}$.
        \item The map $\rho: \phi_N \to \overline{\phi}$ where $\rho(A) =
            \overline{A}$ is bijective.
        \item A map from $\{ A \in \phi_N | A \normsubgroup G\}$ to the set of
            normal subgroups of $\overline{G}$ defined by $A \mapsto
            \overline{A}$ is bijective.
    \end{enumerate}
\end{thm}
\begin{proof}
    Proofs for (a) -- (d) were left as an exercise (to be filled in later).
    For (e), let $\pi: G \twoheadrightarrow \overline{G}$ be the surjective
    group homomorphism defined by $\pi(g) = gN$. Now, consider $\psi:
    \overline{\phi} \to \phi_N$ defined by $\psi(H) = \pi^{-1}(H) \subgroup G$.
    Then, we want to show that $\rho \circ \psi = id_{\overline{\phi}}$ and
    $\psi \circ \rho = id_{\phi_N}$ because then we will know $\rho$ is
    invertible and thus bijective.

    Examine $\rho \circ \psi$. We know that $(\rho \circ \psi)(H) =
    \rho(\pi^{-1}(H))$. Now, note that $\rho(A)$ is the image of $\pi(A)$, so
    we get $\rho(\pi^{-1}(H)) = \pi(\pi^{-1}(H)) = H$ because $\pi$ is
    surjective.

    Next, examine $\psi \circ \rho$. We then get that $(\psi \circ \rho)(A) =
    \pi^{-1}(\pi(A))$ for $A \subgroup G$. We would like for this to be equal
    $A$. It is clear that $A \subset \pi^{-1}(\pi(A))$ because the pre-image of
    $A/N$ under $\pi$ must at least contain $A$. So, we still need to show that
    $\pi^{-1}(\pi(A)) \subset A$. To do this, take $x \in \pi^{-1}(A/N)$. Then,
    $xN = \pi(x) \in A/B$, so $xN = aN$ for some $a \in A$. This means that
    there exists an $n \in N$ such that $x = an$ since $N \subgroup A$. So,
    $\pi^{-1}(A/N) \subset A$.

    Thus, we have shown that $\rho$ and $\psi$ are bijective maps that are
    inverse to each other and we have proven (e). \\

    (f) follows from the same proof technique as (e), I believe.
\end{proof}

\subsubsection*{Commutators}

\begin{defn}
    The commutator of two elements $x,y \in G$ is defined as \[
        [x,y] = xyx^{-1}y^{-1}
    \]
\end{defn}

Note that mathematicians have not completely agreed on the definition and thus
some references will say $[x,y] = x^{-1}y^{-1}xy$. In essence, the commutator
of two elements indicates how ``close'' elements are to commuting. This can be
seen in the following lemma.

\begin{lem}
    For $x,y,z \in G$, we obtain
    \begin{enumerate}[label=(\alph*)]
        \item $[x,y] = e$ if and only if $xy=yx$ if and only if $[x^{-1},y^{-1}] = e$.
        \item $[x,y]^{-1} = [y,x]$.
        \item $z[x,y]z^{-1} = [zxz^{-1},zyz^{-1}]$.
        \item If $\phi: G \to H$ is a group homoromorphism, then $[\phi(g),\phi(h)] = \phi([g,h])$.
        \item $xy = [x,y]yx = yx[x^{-1},y^{-1}]$.
        \item $[xy,z] = x[y,z]x^{-1}[x,z]$
        \item $[x,yz] = [x,y]y[x,z]z^{-1}$.
    \end{enumerate}
\end{lem}
\begin{proof}
    All proofs are straightforward computations.
\end{proof}
\subsection*{Lecture 3: Continuing with Commutators}
\begin{defn}
    Let $A, B \subgroup G$. Then, the \emph{commutator} of $A$ and $B$ is the
    subgroup of $G$ denoted $[A,B] = \langle \{[a,b] | a \in A, b \in B\}
    \rangle$. Also note that $[G,G]$ is called the \emph{commutator subgroup of
    $G$}.
\end{defn}
\begin{rmk}
    Let $A,B \subgroup G$. Then
    \begin{enumerate}
        \item $[A,B] = \{e\}$ if and only if $ab = ba \forall a \in A, b \in B$
            if and only if $A \subgroup C_G(B)$ if and only if $B \subgroup
            C_G(A)$. In this case, we say that $A$ and $B$ commute.
        \item $[G,G] = \{e\}$ if and only if $G$ is abelian if and only if $Z(G) = G$.
        \item $[A,B] = [B,A]$ since $[a,b]^{-1} = [b,a]$.
    \end{enumerate}
\end{rmk}
\begin{example}
    \begin{enumerate}
        \item $[Q_8,Q_8] = \{1,-1\} = [\langle i \rangle, \langle j \rangle]$.
            This can be seen because $[i,j] = iji^{-1}j^{-1} = ij(-i)(-j) =
            ij(-i)(-j) = iij(-j) = -i^2j^2 = -1$.
        \item $[A_n, \langle (12) \rangle] = A_n$ for $n geq 2$. This follows
            because, for $i \geq 3$, we get $[(1i2),(12)] = (1i2)(12)(2i1)(12)
            = (12i)$. This tells us that $A_n = \langle \{(12i) | 3 \leq i \leq
            n\} \rangle \subset [A_n, \langle (12)\rangle$.
        \item $[S_n,S_n] = A_n$ for $n \geq 2$. To show this, for ($\subset$)
            we have \begin{align*}
                \sgn( [\sigma,\tau] ) & = \sgn(\sigma \tau \sigma^{-1} \tau^{-1}) \\
                \ & = \sgn(\sigma) \sgn(\tau) \sgn(\sigma)^{-1} \sgn(\tau)^{-1} \\
                \ & = 1 \forall \sigma, \tau \in S_n
            \end{align*}
            For ($\supset$), see the previous item.
        \item $[A_n,A_n] = A_n$ for $n \geq 5$. This is because every 3 cycle
            is a commutator. For instance $[(ijk),(klm)] = (ilk)$ for 5
            distinct elements $i,j,k,l,m \in \{1, \ldots, n\}$. Also note that
            $[A_3,A_3] = \{e\}$ and $[A_4,A_4] = \Z_2 \times \Z_2$, the set of
            all elements of the form $(ij)(kl)$.
    \end{enumerate}
\end{example}
\begin{lem}
    Let $A,B \subgroup G$. Then,
    \begin{enumerate}[label=(\alph*)]
        \item $A \subgroup N_G(B)$ if and only if $[A,B] \subgroup B$.
        \item $B \subgroup N_G(A)$ if and only if $[A,B] \subgroup A$.
        \item If $A,B \normsubgroup G$ then $[A,B] \normsubgroup G$ and $[A,B]
            \subgroup A \cap B$
    \end{enumerate}
\end{lem}
\begin{proof}
    The forward direction of $(a)$ is given by the fact that $[a,b] =
    aba^{-1}b^{-1}$ but $aba^{-1} \in B$ since $A$ normalizes $B$. Thus, $[a,b]
    \in B \forall a \in A, b \in B$ so $[A,B] \subgroup B$. To do the other
    direction, we have that, we have that $aba^{-1}b^{-1} \in B, \forall a \in A,
    \forall b \in B$. Well, $(aba^{-1}b^{-1})b = aba^{-1} \in B$, so $A$
    normalizes $B$, so $A \subgroup N_G(B)$. \\

    (b) is similar to (a). \\

    Now, given that $A,B \normsubgroup G$, then $g[a,b]g^{-1} =
    gaba^{-1}b^{-1}g^{-1} = gag^{-1}gbg^{-1}ga^{-1}g^{-1}gb^{-1}g^{-1}$. Now
    note that $gag^{-1} \in A$ since $A \normsubgroup G$ and $(gag^{-1})^{-1} =
    ga^{-1}g^{-1}$. Similarly for $b$. So, we have that $g[a,b]g^{-1} \in [A,B]
    \implies [A,B] \normsubgroup G$. Finally, an element of the form
    $aba^{-1}b^{-1} \in A$ because $ba^{-1}b^{-1} \in A$ since $A$ is normal in
    $G$ and similarly $aba^{-1} \in B$. So, $[A,B] \subgroup A \cap B$. This
    final fact also follows from the first two items in the lemma.
\end{proof}

This lemma also gives us the important fact that $[G,G] \normsubgroup G$. This
follows easily from the third item.

\begin{prop}
    If $H \subgroup G$, then $[G,G] \subgroup H$ if and only if $H
    \normsubgroup G$ and $G/H$ is abelian.
\end{prop}
\begin{proof}
    Let $[G,G] \subgroup H$. Then, if $a,b \in G$, $aba^{-1}b^{-1} \in H$. Now,
    in particular, if $b \in H$, then $aba^{-1}b^{-1} \cdot b = aba^{-1} \in H$
    so $H \normsubgroup G$. More concisely, we have that $[G,H] \subgroup [G,G]
    \subgroup H \implies G \subgroup N_G(H)$ so $H \normsubgroup G$. To show
    $G/H$ is abelian, take a map $[a,b] \mapsto [aH,bH] =
    (aH)(bH)(aH)^{-1}(bH)^{-1} = aba^{-1}b^{-1}H = [x,y]H = H = e_{G/H}$. So,
    then $[G/H,G/H] = \{e\}$.

    To prove the other direction, let $G/H$ be abelian and $H \normsubgroup G$.
    Since $G/H$ is abelian, we have that $[aH,bH] = [a,b]H = H$. This means
    that $[a,b] \in H$ which implies that $[G,G] \in H$. (Where is the normalcy
    of $H$ used?!)
\end{proof}

\subsubsection*{Direct Products}

\begin{defn}
    The (external) direct product of two groups $A$ and $B$ is defined as the set (Cartesian product) \[
        A \times B = \{(a,b) | a \in A, b \in B \}
    \] with multiplication $(a_1,b_1) \cdot (a_2,b_2) = (a_1a_2,b_1b_2)$. With this multiplication ,it is clear that $A \times B$ is a group by direct computation.
\end{defn}

Direct products provide us with a way to build new groups from known groups.

\begin{lem}
    Let $G_1, G_2$ be groups and let $H_1 \subgroup G_1$ and $H_2 \subgroup G_2$. Then
    \begin{enumerate}[label=(\alph*)]
        \item $H_1 \times H_2 = \{(h_1,h_2) | h_1 \in H_1, h_2 \in H_2\} \subgroup G_1 \times G_2$.
        \item $H_1 \times H_2 \normsubgroup G_1 \times G_2$ if and only if $H_1 \normsubgroup G_1$ and $H_2 \normsubgroup G_2$. In this case, $(G_1 \times G_2)/(H_1 \times H_2) \isom (G_1/H_1) \times (G_2/H_2)$.
    \end{enumerate}
\end{lem}
\begin{proof}
    For the first item, the proof is a clear from the definition of subgroup. \\

    For the second item, examine the fact that
    $(g_1,g_2)(h_1,h_2)(g_1,g_2)^{-1} = (g_1h_1g_1^{-1},g_2h_2g_2^{-1})$. Then,
    if $H_1 \normsubgroup G_1$ and $H_2 \normsubgroup G_2$, we get
    $(g_1,g_2)(h_1,h_2)(g_1,g_2)^{-1} = (h_1,h_2)$, so $H_1 \times H_2
    \normsubgroup G_1 \times G_2$. Going the other direction is similar, but
    reducing the left hand side instead.

    Finally, we can show the final equivalence using $\phi: G_1 \times G_2 \to
    G_1/H_1 \times G_2/H_2$ where $\phi( (g_1,g_2) ) = (g_1H_1, g_2H_2)$. It
    is clear that $\phi$ is surjective and a group homomorphism, and we note
    that $\ker \phi = H_1 \times H_2$. Thus, we apply the first isomorphism
    theorem to get our equivalence.
\end{proof}

A note of warning is that not every subgroup of a direct product needs to be
of the form $H_1 \times H_2$ where $H_1 \subgroup G_1$ and $H_2 \subgroup G_2$.

\begin{example*}
    Examine $\Z_2 = \{0,1\}$. Then, $\Z_2 \times \Z_2 =
    \{(0,0),(0,1),(1,0),(1,1)\}$.  However, $H = \{(0,0),(1,1)\} \subgroup \Z_2
    \times \Z_2$ but is not of the form $H_1 \times H_2$.
\end{example*}

\begin{rmk}
    (Universal Property) Let $A \times B$ be the direct product of two groups
    $A$, $B$. Then, there are natural projections (surjective group
    homomorphisms) defined by $\pi_A: A \times B \to A$ by $\pi( (a,b) ) = a$
    and similarly for $\pi_B$. They satisfy the following universal property. \\

    For any group $G$, and any pairs of homomorphisms $\phi_A: G \to A$ and
    $\phi_B: G \to B$, there exists a \emph{unique} homomorphism $\phi: G \to A
    \times B$ such that $\pi_A \circ \phi = \phi_A$ and $\pi_b \circ \phi =
    \phi_B$. Indeed, if $\phi(g) = (a,b)$ for $g \in G$, then $a = \pi_A( (a,b)
    ) = \pi_A(\phi(g)) = \phi_A(g)$ and similarly for $b = \phi_B(g)$. \\

    The only possible map with these properties is given by $\phi: G \to A
    \times B$ where $\phi(g) = (\phi_A(g), \phi_B(g))$.

    We now check that $\phi$ so defined is a homomorphism with the properties
    $\pi_A \circ \phi = \phi_A$ and $\pi_B \circ \phi = \phi_B$.
\end{rmk}
\begin{proof}
    Let $\phi$ be defined as above. Then, $\phi( (a_1,b_1)(a_2,b_2) ) =
    (\phi_A(a_1a_2), \phi_B(b_1b_2)) =
    (\phi_A(a_1),\phi_B(b_1))(\phi_A(a_2),\phi_B(b_2)) = \phi( (a_1,b_1) )\phi(
    (a_2,b_2) )$. The other homomorphism properties clearly follow. \\

    $\pi_A \circ \phi( (a,b) ) = \pi_A( (\phi_A(a), \phi_B(b)) ) = \phi_A(a)$.
    Similarly for $\pi_B \circ \phi( (a,b))$. \\
\end{proof}
It is an exercise to generalize this for any $I$ index set with $(A_i)_{i \in
I}$ a family of groups to then generalize the unversal property to the direct
product of these groups, say $\times_{i \in I} A_i = \{(a_i) | a_i \in A_i
\forall i \in I\}$. \\

Now, let $A \times B$ be the external direct product of groups $A$ and $B$.
Then, we know $A \isom A' = \{(a,e_B) | a \in A\} \subgroup A \times B$ and
similarly for $B \isom B'$. Then, $A',B'$ have the following properties.
\begin{itemize}
    \item $A'B' = A \times B$.
    \item $A' \cap B' = \{e\}$.
    \item $A', B' \normsubgroup A \times B$.
\end{itemize}

Let us next discuss internal direct products.

\begin{defn}
    A group $G$ is the \emph{internal direct product} of two subgroups $A,B
    \subgroup G$ if the following conditions are satisfied:
    \begin{enumerate}[label=\roman*)]
        \item $AB = G$
        \item $A \cap B = \{e\}$
        \item $A,B \normsubgroup G$
    \end{enumerate}
\end{defn}
\begin{rmk*}
    In which case, $AB \isom A \times B$.
\end{rmk*}
\begin{example}
    \begin{enumerate}[label=(\alph*)]
        \item Let $G = \Z_6, A = \{\overline{0}, \overline{3}\} \subgroup G, B
            = \{\overline{0}, \overline{2}, \overline{4}\} \subgroup G$. Then,
            we have that $A \cap B = \{ \overline{0} \} \implies |AB|=|A||B| =
            6 = |G|$. Thus, $AB = G$. In abelian groups, all subgroups are
            normal, so $G = A \times B$. In fact, we can prove that $\Z_{mn}
            \isom \Z_m \times \Z_n$ when $\gcd(m,n) = 1$ using the same
            argument where we take $A = \{\overline{0}, \overline{n},
            \overline{2n}, \ldots, \overline{(m-1)n}\}$ and similarly $B =
            \{\overline{0}, \overline{m}, \ldots, \overline{(n-1)m}\}$ and
            showing that $A \cap B = \{\overline{0}\}$ since $\gcd(m,n) = 1$.
        \item $V = \{e, (12)(34), (13)(24), (14)(23)\} \normsubgroup S_4$. This
            is clearly true since conjugating by a permutation is like applying
            the permutation to the numbers of the cycle, which will not change
            the cycle structure. Now, let $A = \{e, (12)(34)\}, B = \{e, (13)(24)\}$, then $A \cap B = \{e\}$ and $A,B \normsubgroup V$. So,
            we get that $|AB| = 2\cdot 2 = 4 = |V| \implies V = AB$ so then $V
            = A \times B \isom \Z_2 \times \Z_2$.
        \item $S_4 = VS_3$. We know that $S_3$ can be embedded into $S_4$
            easily and $V \cap S_4 = \{e\}$. So, for this to happen $|VS_3| =
            24 = |S_4|$. From above, we know $V \normsubgroup S_3$, but it is
            easy to see that $S_3 \not\normsubgroup S_4$. So, we have found an
            example of a product of groups that is not a direct product but
            still forms a group. This leads us into our discussion of
            semi-direct products.
    \end{enumerate}
\end{example}
\subsection*{Lecture 5: Finite Cyclic Groups}
We start by recounting some basic facts about finite cyclic groups. If
\(G=\langle x\rangle\) where \(|x|=n\) then \(G \isom \Z_n\) with the map
\[x^n \longleftrightarrow \bar{a}.\] Then the order of \(\bar{a}\) is given by
\[|\bar{a}|=\frac{n}{(n,a)}.\] We now show that the generators of \(\Z_n\) are
those elements relatively prime to \(n\).
\begin{align*}
  \Z_n = \langle \bar{a} \rangle &\iff |\bar{a}|=n\\
                                 &\iff \frac{n}{(n,a)}=n\\
                                 &\iff (n,a)=1.
\end{align*}
Another fact is that \(\Z_n\) has a unique subgroup of order \(d\) for each
divisor \(d\) of \(n\).
\begin{lem*}[1.2.1] Let \(R\) be a ring with \(1\).
  \begin{enumerate}
  \item[(a)] For any \(r \in R\) the map
    \begin{align*}
      \lambda_r \colon R &\to R\\
      r &\mapsto rx
    \end{align*}
    defines a homomorphism of the abelian group \((R,+)\).
  \item[(b)] For any \(r \in R^\times\), \(\lambda_r \in \Aut(R,+)\) and
    \begin{align*}
      \lambda \colon R^\times &\to \Aut(R,+)\\
      r &\mapsto \lambda_r
    \end{align*}
    is an injective group homomorphism.
  \end{enumerate}

\end{lem*}
\begin{prop*}[1.2.2]
  \[\Aut(\Z_n) \isom \Z_n^\times\]
\end{prop*}
\begin{proof}
  By Lemma 1.2.1 we have \(\lambda \colon \bar{a} \mapsto \lambda_{\bar{a}}\) is
  an injective group homomorphism. Thus we simply must show that it is
  surjective, we will use an order argument. Take any \(\alpha \in \Aut(\Z_n)\)
  since \(|\bar{1}|=n\) this means that \(|\alpha(\bar{1})|=n\). Thus
  \(|\Aut(\Z_n)| \le \varphi(n)\).
\end{proof}

\begin{prop*}[1.2.3]{Chinese Reminder Theorem}
  Let \(m,n \in \N\).
  \begin{enumerate}
  \item[(a)] The map
    \begin{align*}
      f \colon \Z_{mn} &\to \Z_n \times \Z_m\\
      [a]_{mn}&\mapsto [a]_m \times [a]_n
    \end{align*}
    is a ring homomorphism.
  \item[(b)] If \((m,n)=1\) then \(f\) is a ring isomorphism.
  \end{enumerate}
\end{prop*}
\begin{proof}[Proof of (a)]
  Just a routine calculation.
\end{proof}
Notice that this homomorphism sends \(1\) to \(1\), which needn't be the case in
general for a ring homomorphism.
\begin{proof}[Proof of (b)]
  Notice that \(f\) is surjective through order considerations. Now check that
  the kernel is trivial.
\end{proof}
Consequences of b: If \(n=\primedecomposition{p}{e}{r}\) where
\(p_1,\ldots,p_r\) are distinct primes and \(e_i \in \N\). Then
\[\Z_n \isom \prod_{i-1}^r \Z_{{p_i}^{e_i}}.\]
One can prove this by using the Chinese Reminder Theorem and induction on r.
\begin{cor*}[1.2.4]
  Let \(m,n \in \N\) with \((m,n)=1\).
  \begin{enumerate}
  \item[(a)] \(Z_{mn}^\times \isom \Z_m^\times \times \Z_n^\times\). If
    \(n=\primedecomposition{p}{e}{r}\) then
    \[\Z_n^\times \isom \prod_{i-1}^r \Z^\times_{{p_i}^{e_i}}.\]
  \item[(b)]
    \(\varphi(mn)=|\Z_{mn}^\times|=|\Z_{m}^\times\times
    \Z_{n}^\times|=|\Z_{m}^\times||\Z_{n}^\times|=\varphi(m)\varphi(n)\).
  \end{enumerate}
\end{cor*}
\begin{prop*}[1.2.5]
  For \(e \in \N\)
  \begin{enumerate}
    \item[(a)] \(\Z_{2^e}^\times \isom \Z_2 \times \Z_{2^{e-2}},\ e \ge 2\)
    \item[(b)] \(\Z_p^\times \isom \Z_{p-1}\) for every prime \(p\)
    \item[(c)] \(\Z_{p^e}^\times \isom \Z_{p^{e-1}(p-1)} \isom \Z_{p^{e-1}} \times \Z_{p-1} \) for every prime p
  \end{enumerate}
\end{prop*}

\subsection*{Lecture 6: Fundamental
  Theorem of Finitely Generated Abelian
  Groups}

\begin{thm*}[1.2.6 (Cauchy's Theorem)]
  Let \(A\) be an abelian group of
  finite order. If a prime \(p\) divides
  the order of \(A\), there exists an
  element \(a \in A\) of order \(p\).
\end{thm*}

\begin{proof}
  We will proceed by induction on
  \(|A|\). If \(|A|=p\), then
  \(A \isom \Z_p\), which means every
  non-trivial element has order
  \(p\). If \(|A| > p\) we may choose an
  \(x \in A/\{e\}\). Notice that
  \(|x| > 1\).

  \noindent Case 1:\\
  \(p \mid n \implies |a|=p\) for
  \(a=x^{\frac{n}{p}} \in \langle x
  \rangle \subset A\)\\

  \noindent Case 2:\\ \(p\) does not
  divide \(n\). Let
  \(B=\langle x \rangle \normsubgroup
  A\), and \(\bar{A}=A /B\). Then
  \[|\bar{A}|=\frac{|A|}{|B|}=\frac{|A|}{n}.\]
  Furthermore since \(p \mid |A|\)
  \[p \mid |A| \implies p \mid
    \frac{n|A|}{n}\] but since \(p\)
  does not divide \(n\) we have that
  \[p \mid \frac{|A|}{n}\implies p \mid
    |\bar{A}|.\] Now \(|\bar{A}|\) is a
  small group than \(A\), hence we may
  apply the inductive hypothesis, that
  is there exists some \(y \in \bar{A}\)
  of order \(p\). As the projection map
  \(\pi\) is surjective there must exist
  some \(z \in A\) such that
  \(\pi(z)=y\). Since \(\pi\) is a
  homomorphism \(|y| \mid |z|\), which
  means \(p \mid z\), hence \(z=pm\) for
  some \(m \in \Z\). Thus
  \(a=z^{\frac{m}{p}}\) is an element of
  order \(p\) in \(A\).
\end{proof}

\begin{prop*}[1.2.7]
  If \(A\) is a finite abelian group and
  \(m \in \N\) with \(m \mid |A|\), then
  \(A\) has a subgroup of order \(m\).
\end{prop*}
\begin{proof}
  We again proceed by induction. The
  case in which \(|A|=1\) is
  trivial. Assume \(|A|>1\) and
  \(m>1\). Choose any prime divisor
  \(p\) of \(m\). By Cauchy's Theorem
  there exists an \(a \in A\) of order
  \(p\). Set
  \(B=\langle a\rangle \unlhd A\), then
  \[|\bar{A}|=\frac{|A|}{|B|}=\frac{|A|}{p}.\]
  As \(m \mid |A|\) we have that
  \[\frac{m}{p} \mid \frac{|A|}{p}
    \implies \frac{m}{p} \mid
    \frac{|A|}{|\bar{A}|}.\] Now apply
  the inductive hypothesis on
  \(\bar{A}\), that is, there exists a
  subgroup \(\bar{H} \le \bar{A}\) of
  order \(p\). We now invoke the
  Correspondence Theorem, thus there
  exists \(H \le A\) with
  \(\bar{H}=H/B\). Then
  \[|H|=\frac{m}{p}=\frac{|H|}{|B|}=\frac{|H|}{p}
    \implies |H|=m.\]
\end{proof}

\begin{cor*}[1.2.8]
  Given a finite abelian group \(A\)
  with
  \(|A|=\primedecomposition{p}{e}{r}\)
  there exists subgroups \(B_i \le A\)
  for \(1 \le i \le r\) with
  \(|B_i|=p_i^{e_i}\) and
  \[A = B_1 \times \ldots B_r\] moreover
  \(B_i = \{a \mid |a| \text{ is a power
    of } p_i\}\).
\end{cor*}

\begin{proof}
  Subgroups \(B_i\) with
  \(|B_i|=p_i^{e_i}\) exist due to
  1.2.7.
  \[A \text{ is abelian} \implies B_i
    \normsubgroup A,\ \forall 1 \le i
    \le r\]
  \[\prod|B_i| = |A| \qquad \bigcap B_i
    = \{e\} \implies A = B_1 \times
    \ldots \times B_r\] Now we show
  uniqueness:
  \(A \isom B_1 \times \ldots \times
  B_r\) as an external direct product
  which means elemetns of \(A\) come as
  tuples \((b_1,\ldots,b_r)\), then
  \[|a|=|(b_1,\ldots,b_r)|=\text{lcm}(b_1,\ldots,b_r),\]
  which means \(|a|\) is a power of
  \(p_i\). (this needs clarification)
\end{proof}

\begin{rmk*}
  \(B_i\) are Sylow \(p_i\)-subgroups of
  \(A\)
\end{rmk*}

\begin{prop*}[1.2.9]
  Let \(A\) be a finite abelian group of
  order \(n\).
  \begin{enumerate}
  \item[(a)] If \(A\) is cyclic, and
    \(m \in \N\), then
    \[|\{a \in A \mid a^m = e\ (|a| \mid
      m)\}| =
      \begin{cases}
        m   &  m \mid n\\
        <m & m \text{ does not divide }
        n
      \end{cases}
    \]
  \item[(b)] If ( for every $m$ s.t. \(m \mid n\), we have
    \(|\{a \in A \mid a^m = e\}| \le
    m\)), then \(A\) is cyclic.
  \end{enumerate}
\end{prop*}
\begin{proof}
  Homework Three
\end{proof}

\begin{thm*}[1.2.10 (The Fundamental Theorem of Finitely Generated Abelian Groups)]
  Let \(A\) be a finite abelian group
  with
  \(|A|=\primedecomposition{p}{e}{r}\). Then
  for each \(i\) there exists a uniquely
  determined \(l_i \in \N\) such that
  \(m_{i1} < m_{i2} < \ldots <
  m_{il_i}\) that partitions \(e_i\) by
  \(e_i=m_{1i}+\ldots+m_{il_i}\) and
  \[A \isom (\Z_{{p_1}^{m_{11}}} \times
    \Z_{{p_1}^{m_{12}}} \cdots \times
    \Z_{{p_1}^{m_{1l_1}}}) \times \cdots
    \times (\Z_{{p_r}^{m_{r1}}} \times
    \Z_{{p_r}^{m_{r2}}} \cdots \times
    \Z_{{p_r}^{m_{rl_r}}}).\]
\end{thm*}

\begin{rmk*}
  Checkout Dummit and Foote 6.1 page
  197. For a proof that every abelian
  \(p\)-group is a direct product of
  cyclic \(p\)-groups.
\end{rmk*}

\begin{example*}
  Classification up to isomorphism of
  all abelian groups of order
  \(75\). First notice that
  \(72=2^3 \cdot 3^2\). The looking at
  the partitions of the exponents
  yields:
  \[\Z_8,\ \Z_4 \times \Z_2,\ \Z_2
    \times \Z_2 \times \Z_2\] and,
  \[\Z_9, \Z_3 \times \Z_3.\]
  This yields six total possibilities,
  furthermore by uniqueness these are
  exactly the abelian groups of order
  \(75\).
\end{example*}

\begin{rmk*}[1.2.11]
  Every abelian group is also a
  \(\Z\)-module. (which is a
  generalization of a vector space,
  instead of working with a field, we
  weaken the condition to working with a
  ring).
  \begin{align*}
    \Z \times A &\longrightarrow A\\
    (m,a)& \mapsto ma
  \end{align*}
  Where
  \[ma =
    \begin{cases}
      \underbrace{a+a+\ldots+a}_{m \text{ times}} & m > 0\\
      0 & m=0\\
      \underbrace{(-a)+(-a)+\ldots+(-a)}_{-m
        \text{ times}} & m < 0
    \end{cases}.
  \]
\end{rmk*}
One can check that this construction
satisfies the axioms of a
\(\Z\)-module. An abelian group is
finitely generated if and only if the
corresponding \(\Z\)-module is finitely
generated. That is to say the structure
of finitely generated abelian groups is
determined by finitely generated
\(Z\)-modules.

\begin{defn*}[1.2.12]
  An abelian group is called an
  elementary abelian \(p-group\) if
  \(|a|=p\) for all \(a \in A\{e\}\).
\end{defn*}

\begin{rmk*}[1.2.13]
  A is an elementary \(p\)-group if it
  is also a vector space over
  \(\Fp\). This allows us to import
  linear algebra. We define the scalar
  multiplication by
  \begin{align*}
    \Fp \times A &\longrightarrow A\\
    (\bar{m},a) &\longmapsto ma
  \end{align*}
  This is well defined since if
  \(\bar{m} = n \in \Fp\) then there
  exists \(k\) such that \(n=m+kp\)
  which means
  \(na=(m+kp)a=ma+k(pa)=ma+k(0)=ma\). Next
  notice that nay group homomorphism
  \(f \colon A \to A\) is automatically
  \(\Fp\)-linear:
  \(f(\bar{m}a)=\bar{m}f(a)\).
\end{rmk*}

\begin{cor}[1.2.14]
  Let \(A\) be an elementary abelian
  \(p\)-group.
  \begin{enumerate}
  \item[(a)] \(A\) is a direct sum of
    copies of \(\Z_p\)
  \item[(b)]
    \(\Aut(A) \isom \GL(A)=\{f \colon A
    \to A \mid f \text{ is a
      bijection}\}\)
  \end{enumerate}
\end{cor}

Implications for elementary abelian
\(p\)-groups: \(\Aut(A) \isom \GL(\Fp)\)
\begin{enumerate}
\item[a] \(A\) has a basis
  \(\{x_i \mid i \in I\}\)
\item[b] Any additive group is \(\Fp\)
  linear.
  \[\GL(A) \subseteq \Aut(A),\ \Aut(A)
    \subseteq \GL.\]
\end{enumerate}

\begin{example*}
  \begin{gather*}
    \Aut(\Z_2 \times \Z_2) \isom S_3 \isom \GL_2(\mathbb{F}_2)\\
    \Aut(\Z_2 \times \Z_2 \times Z_2)
    \isom \GL_3(\mathbb{F}_2)
    \rightsquigarrow \text{order 168}
  \end{gather*}
\end{example*}

\end{document}

%%% Local Variables:
%%% mode: latex
%%% TeX-master: t
%%% End:
