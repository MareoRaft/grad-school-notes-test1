\documentclass[11pt,leqno,oneside]{amsart}
%\usepackage{amsmath,amsthm}
\usepackage{amssymb,mathrsfs}
%\usepackage{ytableau}
%\ytableausetup{smalltableaux,centertableaux}
\usepackage{tikz}
\usepackage{upgreek}

%\usepackage[nohead,nofoot,centering]{geometry}

\usepackage{color}
%% Some user-defined colors
	  \definecolor{mydefi}{cmyk}{1,0,0,.5}
	  \definecolor{myred}{rgb}{.7,.1,.1}
	  \definecolor{myblue}{rgb}{.1,.1,.6}
	  \definecolor{mygreen}{rgb}{.1,.6,.1}

\usepackage[urlbordercolor={1 1 1}, pdfborder={0 0 0}, bookmarks=true,
  colorlinks=true, linkcolor=myblue, citecolor=myblue,
  urlcolor=myblue, hyperfootnotes=false]{hyperref}

\usepackage[alphabetic,abbrev]{amsrefs} % use AMS ref scheme

\addtolength{\footskip}{2\baselineskip} % to lower the page numbers


%%%%%%%%%%%%%%%%%%%%%%%%%%%%%%%%%%%%%%%%%%%%%%%%%%%%%%%%%%%%%%%%%%%
%%  MACRO DEFINITIONS:  Co-authors -- PLEASE use these!
%%%%%%%%%%%%%%%%%%%%%%%%%%%%%%%%%%%%%%%%%%%%%%%%%%%%%%%%%%%%%%%%%%%
\newcommand{\N}{{\mathbb N}} % natural numbers
\newcommand{\Z}{{\mathbb Z}} % integers
\newcommand{\Q}{{\mathbb Q}} % rational numbers
\newcommand{\R}{{\mathbb R}} % real numbers
\newcommand{\C}{{\mathbb C}} % complex numbers
\newcommand{\End}{\operatorname{End}} % endomorphisms
\newcommand{\Hom}{\operatorname{Hom}} % homomorphisms
\newcommand{\GL}{\operatorname{GL}} % general linear group
\newcommand{\B}{\mathfrak{B}} % use for the Brauer algebra
\newcommand{\Sym}{\mathfrak{S}} % symmetic group
\newcommand{\sgn}{\operatorname{sgn}} % sign
\newcommand{\T}{\mathsf{T}} % use for tableaux
\newcommand{\U}{\mathsf{U}} % use for tableaux
\newcommand{\V}{\mathsf{V}} % use for tableaux
\newcommand{\TA}{\mathsf{A}} % use for tableaux
\newcommand{\TB}{\mathsf{B}} % use for tableaux
\newcommand{\TC}{\mathsf{C}} % use for tableaux
\newcommand{\TS}{\mathsf{S}} % use for tableaux
\newcommand{\shape}{\operatorname{shape}} % shape of a tableau
\newcommand{\col}[2]{\genfrac{}{}{0pt}{1}{#1}{#2}} % column of bitableau
\newcommand{\ov}{\overline} % shorthand for a bar on a symbol
\newcommand{\dd}{\partial} % use for diagram basis; e.g. d(V_l)
\newcommand{\X}{\mathcal{X}} % use for the Gelfand-Tsetlin subalgebra
\newcommand{\JM}{\mathcal{J}} % a subalgebra of the GT-subalgebra
\newcommand{\Std}{\operatorname{Std}} % set of standard tableaux
\newcommand{\StdB}{\operatorname{StdB}} % set of standard bitableaux
\newcommand{\Orb}{\mathcal{O}} % use for orbits
\newcommand{\OS}{\ov{\Orb}} % use for orbit sums
\newcommand{\OSS}{\ov{\OS}} % use for double bar orbit sums
\newcommand{\OR}{\mathscr{R}} % orbit representatives
\newcommand{\Stab}{\operatorname{Stab}} % stabilizer
\newcommand{\rev}{\operatorname{rev}} % reverse of a cycle
\newcommand{\A}{\mathcal{A}} % the algebra
\newcommand{\fraka}{\mathfrak{a}} % Young symmetrizer
\newcommand{\frakb}{\mathfrak{b}} % Young symmetrizer
\newcommand{\frakc}{\varphi} % canonical basis
\newcommand{\yy}{\mathsf{y}} % Young symmetrizer; scaled
\newcommand{\idem}{\varepsilon} % primitive central idem in symm gp
\newcommand{\cA}{\mathcal{A}} % group algebra of symmetric group
\newcommand{\Tab}{\operatorname{Tab}} % trails in branching graph from source
\newcommand{\BG}{\mathbf{B}} % branching graph
\newcommand{\bb}{\varnothing} % the unique element of \Irr(0)
\newcommand{\res}{\operatorname{res}} % restriction
\newcommand{\Irr}{\operatorname{Irr}} % irreps
\newcommand{\Wt}{\operatorname{Wt}} % possible content vectors for an irrep
\newcommand{\trace}{\operatorname{trace}} % the trace
\newcommand{\type}{\operatorname{type}} % type = generalized shape
\newcommand{\gen}[1]{\langle #1 \rangle} % use for generating sets
\newcommand{\parm}{\updelta} % Brauer algebra parameter
\newcommand{\sep}{\,|\,} % separator for two partitions - used in tables
\newcommand{\covered}{\lessdot}
\newcommand{\qand}{\quad\hbox{and}\quad}
\swapnumbers %% put numbers in front of proclamations
\newtheorem{thm}{Theorem}[section]
\newtheorem*{thm*}{Theorem}
\newtheorem{lem}[thm]{Lemma}
\newtheorem*{lem*}{Lemma}
\newtheorem{prop}[thm]{Proposition}
\newtheorem*{prop*}{Proposition}
\newtheorem{cor}[thm]{Corollary}
\newtheorem*{cor*}{Corollary}
\newtheorem{conj}[thm]{Conjecture}
\newtheorem*{conj*}{Conjecture}

\theoremstyle{definition}
\newtheorem{defn}[thm]{Definition}
\newtheorem*{defn*}{Definition}
\newtheorem{example}[thm]{Example}
\newtheorem*{example*}{Example}
\newtheorem{examples}[thm]{Examples}
\newtheorem*{examples*}{Examples}
\newtheorem{alg}[thm]{Algorithm}
\newtheorem*{alg*}{Algorithm}
%\theoremstyle{remark}
\newtheorem{rmk}[thm]{Remark}
\newtheorem*{rmk*}{Remark}
\newtheorem{rmks}[thm]{Remarks}
\newtheorem*{rmks*}{Remarks}

%%%%%%%%%%%%%%%%%%%%%%%%%%%%%%%%%%%%%%%%%%%%%%%%%%%%%%%%%%%%%%%%%%%
\numberwithin{equation}{section}
%% The following avoids conflict between numbers of proclamations
%% and numbers of equations
\renewcommand{\theequation}{\thesection\alph{equation}}
%%%%%%%%%%%%%%%%%%%%%%%%%%%%%%%%%%%%%%%%%%%%%%%%%%%%%%%%%%%%%%%%%%%
\parskip = 2pt
\allowdisplaybreaks
\renewcommand{\labelenumi}{(\theenumi)} % use round brackets
\renewcommand{\theenumi}{\alph{enumi}} % use alphabetic enumerations


\pagestyle{plain} % suppress the running head - for working document

\title[Differential Topology]{Differential Topology}
\author{George H. Seelinger (inspired from class by Nick Kuhn)}
\date{Fall 2016}
\begin{document}
\maketitle
\section{Lecture 1}

What is differential topology? Topology is concerned with objects called
topological spaces and continuous functions between them. However, the name
differential topology suggests that calculus is somehow involved. The objects of
study in differential topology are manifolds. Manifolds are topological spaces
that ``locally'' look like $\R^n$.

\begin{example}
	Consider the set $S^2 = \{ \vec{x} \in \R^3: ||x||=1\}$. This is a sphere
	in 3 dimensions and thus can also be written as $S^2 = \{(x,y,z) :
	x^2+y^2+z^2=1\}$. The convention for $S^2$ or ``two-sphere'' comes from the
	fact that a small space of $S^2$ looks like $\R^2$.

	To make this explicit, consider two maps. The first is a map from $\R^2$ to
	a disk by $f(\vec{x}) = \frac{||x||^2}{1+||x||^2} \vec{x}$. Then consider a
	map $g(x,y) = (x,y,\sqrt{1-x^2-y^2})$. Composing these two maps will map
	$\R^2$ onto part of $S^2$. Variations on the same theme can be used to get
	other parts of the sphere.
\end{example}
\begin{defn}
	The \emph{$n$-sphere}, denoted $S^n$, is $\{x \in \R^{n+1} |\; ||x||=1 \}$. Note: The *surface* of the $n$-sphere has dimension $n$.,
\end{defn}
\begin{example}
	Continuing with the sphere, consider the tangent plane of $S^2$ at some
	point $\vec{x}$. We will denote this $T_{\vec{x}}S^2$. Note that this is a
	real vector space. Using symmetry arguments, we can see that, if $\vec{v}
	\in T_{\vec{x}}S^2$, then $\vec{v}$ is perpendicular to $\vec{x}$ and thus
	$\vec{v} \cdot \vec{x} = 0$. So, to use a concrete point,
	$T_{(1/3,2/3,2/3)} S^2 = \{(x,y,z) : x+2y+2z=0\}$.
\end{example}
\begin{example}
	Now, to think globally, consider the collection of all tangent planes of
	$S^2$. We call this the \emph{tangent bundle} of $S^2$ and denote it
	$TS^2$. We can see $TS^2 = \{(\vec{x}, \vec{v}): \vec{x} \in S^2, \vec{v}
	\in T_{\vec{x}} S^2\} = \{(\vec{x},\vec{v}) : ||\vec{x}|| = 1, \vec{x}
	\cdot \vec{v} = 0\} \subset S^2 \times \R^3 \subset \R^3 \times \R^3$.
	Finally, consider that there exists a function $p$ such that $p: TS^2 \to
	S^2$ by $(x,v) \to x$. Then, we have that $p^{-1}(x) = T_{\vec{x}}S^2$.
\end{example}
\begin{defn}
	The \emph{tangent bundle} of a manifold $M$, denoted $TM$, is $\{ (x,v) |\; x \in M and v \in T_xM \}$.
\end{defn}

Now, let $M$ be any differential manifold. Studying the function $p: TM \to M$
can tell us a lot about $M$.

\begin{example}
	A \emph{section} of $p: TS^2 \to S^2$ is a function $s: S^2 \to TS^2$ such
	that $p \circ s =$ identity. In other words, $\forall x \in S^2, s(x) \in
	T_{\vec{x}}S^2$. Note that this is a vector field. Furthermore, to be
	interesting in differential topology, $s$ should be continuous or
	differentiable.
\end{example}

A famous topological question is: could $s(x) \neq \vec{0} \in T_{\vec{x}}S^2$
for all $x \in S^2$? The famous answer is no. This is analogous to the
so-called ``hairy-ball theorem''. By contrast, the answer would be yes if $S^2$
were replaced by a circle ($S^1$), a 3-sphere ($S^3 \subset \R^4$), or a torus
($T^2$).

Given this result, clearly $TS^2$ is not $S^2 \times \R^2$. This is because,
for $S^2 \times \R^2$, we could define $s(\vec{x}) = (\vec{x}, \vec{0})$ and
$p( (\vec{x},\vec{y}) ) = \vec{x}$ which gives us $(p \circ s)(\vec{x}) =
p(s(\vec{x})) = p( (\vec{x}, \vec{0}) )= \vec{x}$.


In differential topology, one can also do ``fiberwise'' linear algebra. Let
$T^*M$ be a collection of the deual vector spaces $(T_{\vec{x}}M)^*$ where $V^* =
\{L: V \to \R\}$ with $L$ linear. A section of $T^*M \to M$ is given by $(x,L)
\to x$ where $L: T_{\vec{x}}M \to \R$ in this case. Now, we call $w: M \to
T^*M$ such that $w(x): T_{\vec{x}}M \to \R$ a collection of \emph{1-forms}.

These ideas are just a preview of what is to come in future lectures. Now, let
us move on to introducing various sorts of functions

\subsection{Functions}

\begin{defn}
	A function $f: \R^n \to \R^m$ is continuous at $\vec{a}$ if $\forall
	\epsilon > 0, \exists \delta > 0$ such that $||\vec{x}-\vec{a}|| < \delta
	\implies ||f(\vec{x}) - f(\vec{a})|| < \epsilon$.
\end{defn}
Now, recall \begin{enumerate}
	\item If $f(x) = (f_1(x), \ldots, f_m(x))$, then $f$ is continuous if and
		only if each $f_i$ is continuous. This fact follows from the basic
		construction of the product topology.
	\item Let $B_r(a) = \{x \in \R^n : ||x-a|| < r\}$. Then, $f$ is continuous
		if and only if, given open set $U \subset \R^m$, $f^{-1}(U)$ is open in
		$\R^n$.
	\item If $f: Z \to X \times Y$, $f$ is continuous if and only if $f_1: Z
		\to X$ and $f_2: Z \to Y$ are continuous, where $f_1, f_2$ are the
		obvious restrictions.
\end{enumerate}

Now, let us define what it means for a function to be differentiable.
\begin{defn}
	$f: \R^n \to \R^m$ is differentiable at $\vec{x} \in \R^n$ if there exists
	a linear function $L: \R^n \to \R^m$ so that $\lim_{\delta \vec{x} \to
	\vec{0}} \frac{||f(\vec{x}+\delta \vec{x}) - f(\vec{x}) - L(\delta
	\vec{x})||}{||\delta \vec{x}||} = 0$.
\end{defn}

\begin{thm}[a-b-dimensional tangent plane]
	Given a function $f : \\mathbb{R}^a \\to \mathbb{R}^b$ and its slope matrix $f^\prime(v)$ at a vector $v \in \mathbb{R}^a$, then the equation for the plane tangent to $f$ at $v$ is $T_v(w) = f(v) + f'(v)(w-v)$ or, in point-slope form, $\frac{T_v(w)-f(v)}{w-v} = f'(v)$.,
	% dependencies: [function, derivative, tangent],
\end{thm}
\begin{defn}
	If $p$ is a function from $TM$ to $M$, then a \emph{section} of $p$ is a right inverse of $p$.,
\end{defn}

For each $x \in M$, we can use a section $s$ to associate a vector $s(x)$.  This creates a vector field.

\begin{thm}[billiard ball theorem]
	A continuously differentiable section $s$ on $TS^2$ with $s(x) \neq 0$ for all $x \in S^2$ does not exist.,
	intuition: Can you draw a continuously differentiable, always nonzero vector field on a billiard ball?  No.,
	% // NO --> apparently TS^2 not isomorphic to S^2 \times \R^2.
	% // Ts^2 \subseteq S^2 \times \R^3 \subseteq \R^3 \times \R^3.
\end{thm}
\begin{thm}[continuously differentiable sections]
	A continuously differentiable section $s$ on $TM$ with $s(x) \neq 0$ for all $x \in M$ exists for the following possible $M$: $S^1$ (circle), $S^3$, $T^2$ (torus).
\end{thm}
\begin{defn}
	The set $U$ is \emph{open} if it can be written as a union of open balls $B$.
\end{defn}
\begin{thm}
	The set $U$ is open iff for all points in $U$, there exists an open ball around the point that's entirely contained in $U$.
\end{thm}
\begin{thm}[continuity and openness]
	$f$ is continuous iff for every open set $U \in \R^m$, $f^{-1}(U)$ is open in $\R^n$.
\end{thm}
\begin{defn}
	A \emph{topological space} (a.k.a \emph{space}) is a set $X$ and a set $O$ of subsets of $X$ that satisfy the following conditions.  Every element of $O$ is ``open'' by definition.  Further:
	\begin{itemize}
		\item Arbitrary unions of open sets are open.
		\item Finite intersections of open sets are open.
		\item The empty set is open.
		\item The entire space is open.
	\end{itemize}
\end{defn}
\begin{defn}
	$f :\R^n \to \R^m$ is \emph{differentiable} at $x \in R^n$ if there exists a linear function $df_x : \R^n \to \R^m$  such that $lim_{h \to 0} \frac{|| (f(x)-f(x+h)) - (df_x(x+h)) ||}{||h||} = 0$.  Note that $h$ is a vector.
\end{defn}
\begin{defn}
	If $f$ is differentiable at $x$ by a linear function $L$, then $L$ is the \emph{differential} of $f$ at $x$.  This is denoted $df_x$.  Using the standard basis, the differential can be expressed as a slope matrix, which is denoted $f'(x)$.
\end{defn}
\begin{thm}
	If $f$ is differentiable at $x$, then the differential $df_x$ in the $v$ direction is equal to $lim_{t \to 0} \frac{f(x+tv) - f(x)}{t}$.
\end{thm}
\begin{defn}
	For any $k \in \mathbb{N}$, a function $f:\mathbb{R}^n \\to \\mathbb{R}^m$ is \emph{$k$-continuously-differentiable}, denoted $f \in C^k$, if all iterated partial derivatives of $f$ of order $k$ exist and are continuous.
\end{defn}
\begin{thm}
	For any $k \in \N$, the function $f$ has class $C^k$ iff all of its coordinate functions $f_i$ have class $C^k$.
\end{thm}
\begin{defn}
	Given a set $V$, a \emph{distance} is a function $D : V \times V \to \R$ that satisfies, for all $x, y, z \in V$:
	\begin{enumerate}
		\item $D(x,x) = 0$
		\item If $x \neq y$, then $D(x,y) > 0$
		\item (symmetry) $D(x,y) = D(y,x)$
		\item (triangle inequality) $D(x,z) <= D(x,y) + D(y,z)$
	\end{enumerate}
\end{defn}
\begin{defn}
	A \emph{metric space} is $(X, D)$ where $X$ is a set and $D$ is a distance function on $X$.
\end{defn}
\begin{defn}[analytic functions]
	A function $f$ is \emph{analytic} if for every point $x$ in the domain, there exists a taylor series centered at $x$ which converges to $f$ in a neighborhood of $x$.  Note: The class of analytic functions is denoted $C^\omega$.
\end{defn}
\begin{thm}
	$C^\omega \subsetneq C^\infty \subsetneq \subsetneq \subsetneq C^n \subsetneq C^{n-1} \subsetneq \subsetneq \subsetneq C^0$
\end{thm}
\begin{example}
	The function $$f(x) =
	\begin{cases}
		e^{-1/x^2} &\text{if $x \neq 0$} \\
		0 &\text{if $x=0$} \\
	\end{cases}
	$$
	is in $C^\infty$, but it is NOT analytic.
\end{example}
\begin{defn}
	$K \subseteq X$ is \emph{compact} in $X$ if whenever $K \subseteq \cup_{i \in I} O$ of open sets, then there exists $J$ finite subset of $I$ such that $K \subseteq union_{j \in J} O$.
\end{defn}
\begin{thm}
	If $K$ is compact in $X$ and $f:X \to Y$ is continuous, then $f(X)$ is compact in $Y$.
\end{thm}
\begin{thm}
	If $O$ is open (closed) in $Y$ and $f:X \to Y$ is continuous, then $f^{-1}(O)$ is open (closed) in $X$.
\end{thm}
\begin{thm}
	If $K$ is compact in $X$ and $C$ is closed in $X$, then $K \cap C$ is compact in $X$.
\end{thm}
\begin{defn}
	A topological space $X$ is \emph{Hausdorff} if for every $a \neq b$ in $X$, there exists an open neighborhood $A$ of $a$ and an open neighborhood $B$ of $b$ such that $A \cap B = {}$.  Synonym: seperable.  Notation: $T_2$ is the set of all Hausdorff topological spaces.
\end{defn}
\begin{lem}
	If $X$ is Hausdorff and $K$ is compact in $X$, then $K$ is closed in $X$.
\end{lem}
\begin{defn}
	$f$ is an \emph{open} (\emph{closed}) function if the image of every open (closed) set under $f$ is open (closed).
\end{defn}
\begin{cor}
	If $f : X \to Y$ is continuous, $X$ is compact ((wait, is X a subset of Y??)), and $Y$ is Hausdorff, then $f$ is a closed function.
\end{cor}
\begin{cor}
	If $f : X \to Y$ is a continuous bijection, and $X$ is compact and $Y$ is Hausdorff, then $f$ is a homeomorphism. (That is, $f^{-1}$ is continuous).
\end{cor}
\begin{thm}
	A compact top subspace of a Hausdorff space is closed. ((ermmmm))
\end{thm}
\begin{thm}
	If $X$ and $Y$ are compact, then $X \times Y$ is compact.
\end{thm}
\begin{thm}
	$X$ is a top space and $~$ is an equivalence relation on $X$.  Then there is a topology on $X/~$ (the set of equivalence classes) as follows.  $S \subseteq X/~$ is open iff (let backwards(S) : X/~ to X) backwards(S) is open in X.  [[ f: X/~ to Y is continuous iff f of g : X to Y is contniuous.
\end{thm}
\begin{defn}
	A \emph{topological group} is a seperable space $G$ that is also a group, which satisfies:
	\begin{enumerate}
		\item The function $m : G \times G \to G$ defined by $m((g,h)) = gh$ is continuous.
		\item The function $i : G \to G$ defined by $i(g) = g^{-1}$ is continuous.
	\end{enumerate}
	See Bredon I.15 for more related definitions.
\end{defn}








\end{document}




{
	"name": "derivative of multivariable function",
	"def": "The __tangent plane__ to function $f : \mathbb{R}^n \\to \mathbb{R}^m$ at point $a$, denoted $df_a$ or $f^{\prime}(a)$, is the unique function that satisfies $\lim_{h\\to 0} \\frac{||f(a+h) - f(a) - df_a(h)||}{||h||} = 0$.",
}
{
	"def": "The __directional derivative__ of $f$ at $x$ in the $v$ direction is $df_x(v)$ which is $\lim_{h\\to 0} \\frac{f(x+hv) - f(x)}{h}$.",
}
{
	"def": "Let $U$ and $V$ be open sets in $\mathbb{R}^n$ (topological spaces, soon).  A function $f:U\\to V$ is a __homeomorphism__ iff it is a bijection and both $f$ and $f^{-1}$ are continuous.",
	"synonym": "bi-continuous function",
}
{
	"def": "Let $U$ and $V$ be open sets in $\mathbb{R}^n$.  A function $f:U\\to V$ is a __$C^r$-diffeomorphism__ iff it is a bijection and both $f$ and $f^{-1}$ are $r$-continuously-differentiable.",
}
{
	"def": "Let $U$ and $V$ be open sets in $\mathbb{R}^n$.  A function $f:U\\to V$ is a __diffeomorphism__ iff it is a bijection and both $f$ and $f^{-1}$ are differentiable.",
	"synonym": "bi-differentiable function",
}
{
	"def": "A function $f:\mathbb{R}^n \\to \\mathbb{R}^m$ is __smooth__, denoted $C^\infty$ if $f$ is $C^k$ for all $k \in \mathbb{N}$.",
	"synonym": "$\infty$-continuous",
}
{
	"thm": "If $f:\mathbb{R}^n \\to \\mathbb{R}^m$ is $C^1$ in a neighborhood of $x$, then $f$ is differentiable at $x$.",
	// no proof :(
	// need to understand this better
}
{
	"thm": "If $f:\mathbb{R}^n \\to \\mathbb{R}^m$ is $C^2$ in a neighborhood of $x$, then for all $i$ and $j$, $\\frac{\partial^2f}{\partial x_i \partial x_j}(x) = \\frac{\partial^2f}{\partial x_j \partial x_i}(x)$.",
	// no proof :(
}
{
	"name": "chain rule",
	"thm": "If $f$ is differentiable at $a$ and $g$ is differentiable at $f(a)$, then $(g \circ f)$ is differentiable at $a$, and $(g \circ f)'(a) = g'(f(a))f'(a).$",
}
{
	"name": "inverse function theorem",
	"thm": "For any $k \in \mathbb{N}^+$, if $f:\mathbb{R}^n \\to \\mathbb{R}^n$ is $C^k$ and $f^\prime(x)$ is invertible for some $x$, then $f$ is locally a diffeomorphism of class $C^k$ at $x$.",
	"counterexample": "Let

	$f(x) = \begin{cases}\item \frac x2 + x^2\sin\left(\frac 1x\right) \text{ if $x \neq 0$} \item 0 \text{ if $x=0$} \end{cases}$

	We see that $f$ is differentiable, but the derivative is not continuous (i haven't verified this).  It turns out that $f$ is not one-to-one (and therefore not a bijection).",
}
{
	"name": "local submersion theorem",
	"synonym": "implicit function theorem",
	"thm": "",
	"note": "This theorem is equivalent to the inverse function theorem.",
}
{
	"name": "local immersion theorem",
	"thm": "This is very similar to the local submersion theorem (more soon...)",
}
{
	"thm": "If f is locally diffeomorphic at x, then the slope matrix of f inverse at f(x) is given by $(f^{-1})^\prime(f(x)) = (f^\prime(a))^{-1}$.",
	"proof": "f is locally diffeomorphic at x, so f^{-1}(f(x)) = x.
	Apply the chain rule: (f^{-1})^\prime(f(x)) \cdot f^\prime(x) = 1
	Finally: (f^{-1})^\prime(f(x)) = (f^\prime(x))^{-1}",
}
{
	"thm": "A function $f : \R^a \to \R^b$ is continuous iff each of its coordinate functions $f_i$, $1<= i <=b$, are continuous.",
}

