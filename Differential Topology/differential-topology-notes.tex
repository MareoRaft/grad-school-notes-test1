\documentclass[11pt,leqno,oneside]{amsart}

\usepackage{../notes}
%%% matt experiments with theorem style:
\usepackage{amsthm}% http://ctan.org/pkg/amsthm
\newcommand{\Der}{\text{Der}}

\newtheoremstyle{mystyle}
  {20pt plus 4pt minus 4pt} % Space above
  {\topsep} % Space below
  {} % Body font
  {} % Indent amount
  {\bfseries} % Theorem head font
  {.} % Punctuation after theorem head
  {.5em} % Space after theorem head
  {} % Theorem head spec (can be left empty, meaning `normal')

\theoremstyle{mystyle} \newtheorem{thrm}[thm]{Theorem}
\theoremstyle{mystyle} \newtheorem{defi}[thm]{Definition}
%%% end experimentation

\title[Differential Topology]{Differential Topology}
\author{George Seelinger, Matthew Lancellotti\\ (class notes for Nick Kuhn's MATH 7820)}
\date{Fall 2016}
\begin{document}
\maketitle
\section{Lecture 1}

What is differential topology? Topology is concerned with objects called
topological spaces and continuous functions between them. However, the name
differential topology suggests that calculus is somehow involved. The objects of
study in differential topology are manifolds. Manifolds are topological spaces
that ``locally'' look like $\R^n$.

\begin{example}
	Consider the set $S^2 = \{ \vec{x} \in \R^3: ||x||=1\}$. This is a sphere
	in 3 dimensions and thus can also be written as $S^2 = \{(x,y,z) :
	x^2+y^2+z^2=1\}$. The convention for $S^2$ or ``two-sphere'' comes from the
	fact that a small space of $S^2$ looks like $\R^2$.

	To make this explicit, consider two maps. The first is a map from $\R^2$ to
	a disk by $f(\vec{x}) = \frac{||x||^2}{1+||x||^2} \vec{x}$. Then consider a
	map $g(x,y) = (x,y,\sqrt{1-x^2-y^2})$. Composing these two maps will map
	$\R^2$ onto part of $S^2$. Variations on the same theme can be used to get
	other parts of the sphere.
\end{example}
\begin{defi}
	The \emph{$n$-sphere}, denoted $S^n$, is $\{x \in \R^{n+1} |\; ||x||=1 \}$. Note: The *surface* of the $n$-sphere has dimension $n$.,
\end{defi}
\begin{example}
	Continuing with the sphere, consider the plane through the origin that is parallel to the tangent plane of $S^2$ at some
	point $\vec{x}$. We will denote this $T_{\vec{x}}S^2$. Note that this is a
	real vector space. Using symmetry arguments, we can see that, if $\vec{v}
	\in T_{\vec{x}}S^2$, then $\vec{v}$ is perpendicular to $\vec{x}$ and thus
	$\vec{v} \cdot \vec{x} = 0$. So, to use a concrete point,
	$T_{(1/3,2/3,2/3)} S^2 = \{(x,y,z) : x+2y+2z=0\}$.  Note also that we will abusively call this plane the \emph{tangent plane}, even though we are actually referring to the plane through $0$ instead of the plane through $\vec{x}$.
\end{example}
\begin{example}
	Now, to think globally, consider the collection of all tangent planes of
	$S^2$. We call this the \emph{tangent bundle} of $S^2$ and denote it
	$TS^2$. We can see $TS^2 = \{(\vec{x}, \vec{v}): \vec{x} \in S^2, \vec{v}
	\in T_{\vec{x}} S^2\} = \{(\vec{x},\vec{v}) : ||\vec{x}|| = 1, \vec{x}
	\cdot \vec{v} = 0\} \subset S^2 \times \R^3 \subset \R^3 \times \R^3$.
	Finally, consider that there exists a function $p$ such that $p: TS^2 \to
	S^2$ by $(x,v) \to x$. Then, we have that $p^{-1}(x) = T_{\vec{x}}S^2$.
\end{example}
\begin{defi}
	The \emph{tangent bundle} of a manifold $M$, denoted $TM$, is $\{ (x,v) |\; x \in M \;\text{and}\; v \in T_xM \}$.
\end{defi}

Now, let $M$ be any differential manifold. Studying the function $p: TM \to M$
can tell us a lot about $M$.

\begin{example}
	A \emph{section} of $p: TS^2 \to S^2$ is a function $s: S^2 \to TS^2$ such
	that $p \circ s =$ identity. In other words, $\forall x \in S^2, s(x) \in
	T_{\vec{x}}S^2$. Note that this is a vector field. Furthermore, to be
	interesting in differential topology, $s$ should be continuous or
	differentiable.
\end{example}



\subsection{Functions}

\begin{defi}
	A function $f: \R^n \to \R^m$ is continuous at $\vec{a}$ if $\forall
	\epsilon > 0, \exists \delta > 0$ such that $||\vec{x}-\vec{a}|| < \delta
	\implies ||f(\vec{x}) - f(\vec{a})|| < \epsilon$.
\end{defi}
Now, recall \begin{enumerate}
	\item If $f(x) = (f_1(x), \ldots, f_m(x))$, then $f$ is continuous if and
		only if each $f_i$ is continuous. This fact follows from the basic
		construction of the product topology.
	\item Let $B_r(a) = \{x \in \R^n : ||x-a|| < r\}$. Then, $f$ is continuous
		if and only if, given open set $U \subset \R^m$, $f^{-1}(U)$ is open in
		$\R^n$.
	\item If $f: Z \to X \times Y$, $f$ is continuous if and only if $f_1: Z
		\to X$ and $f_2: Z \to Y$ are continuous, where $f_1, f_2$ are the
		obvious restrictions.
\end{enumerate}

Now, let us define what it means for a function to be differentiable.
\begin{defi}
	$f :\R^n \to \R^m$ is \emph{differentiable} at $x \in \R^n$ if there exists a linear function $df_x : \R^n \to \R^m$  such that $\lim_{h \to 0} \frac{|| (f(x)-f(x+h)) - (df_x(h)) ||}{||h||} = 0$.  Note that $h$ is a vector, as well as $x$.
\end{defi}
\begin{defi}
	If $f$ is differentiable at $x$ by a linear function $L$, then $L$ is the \emph{differential} of $f$ at $x$.  This is denoted $df_x$.  Using the standard basis, the differential can be expressed as a slope matrix, which is denoted $f'(x)$.
\end{defi}
\begin{thrm}
	If $f$ is differentiable at $x$, then the differential $df_x$ in the $v$ direction is equal to $\lim_{t \to 0} \frac{f(x+tv) - f(x)}{t}$.
\end{thrm}

\begin{thrm}[a-b-dimensional tangent plane (warning, this is not used in our class)]
	Given a function $f : \R^a \to \R^b$ and its slope matrix $f^\prime(v)$ at a vector $v \in \R^a$, then the equation for any point $w$ on the plane tangent to $f$ at $v$ is $g_v(w) = f(v) + f'(v)(w-v)$ or, in point-slope form, $\frac{g_v(w)-f(v)}{w-v} = f'(v)$.
	% dependencies: [function, derivative, tangent],
\end{thrm}

\begin{thrm}[billiard ball theorem]
	A continuously differentiable section $s$ on $TS^2$ with $s(x) \neq 0$ for all $x \in S^2$ does not exist.,
	intuition: Can you draw a continuously differentiable, always nonzero vector field on a billiard ball?  No.,
	% // NO --> apparently TS^2 not isomorphic to S^2 \times \R^2.
	% // Ts^2 \subseteq S^2 \times \R^3 \subseteq \R^3 \times \R^3.
\end{thrm}

A famous topological question is: could $s(x) \neq \vec{0} \in T_{\vec{x}}S^2$
for all $x \in S^2$? The famous answer is no. This is analogous to the
so-called ``hairy-ball theorem''. By contrast, the answer would be yes if $S^2$
were replaced by a circle ($S^1$), a 3-sphere ($S^3 \subset \R^4$), or a torus
($T^2$).

Given this result, clearly $TS^2$ is not $S^2 \times \R^2$. This is because,
for $S^2 \times \R^2$, we could define $s(\vec{x}) = (\vec{x}, \vec{0})$ and
$p( (\vec{x},\vec{y}) ) = \vec{x}$ which gives us $(p \circ s)(\vec{x}) =
p(s(\vec{x})) = p( (\vec{x}, \vec{0}) )= \vec{x}$.


In differential topology, one can also do ``fiberwise'' linear algebra. Let
$T^*M$ be a collection of the dual vector spaces $(T_{\vec{x}}M)^*$ where $V^* =
\{L: V \to \R\}$ with $L$ linear. A section of $T^*M \to M$ is given by $(x,L)
\to x$ where $L: T_{\vec{x}}M \to \R$ in this case. Now, we call $w: M \to
T^*M$ such that $w(x): T_{\vec{x}}M \to \R$ a collection of \emph{1-forms}.


\begin{defi}
	A \emph{topological space} (a.k.a \emph{space}) is a set $X$ and a set $O$ of subsets of $X$ that satisfy the following conditions.  Every element of $O$ is ``open'' by definition.  Further:
	\begin{itemize}
		\item Arbitrary unions of open sets are open.
		\item Finite intersections of open sets are open.
		\item The empty set is open.
		\item The entire space is open.
	\end{itemize}
\end{defi}
\begin{thrm}
	The set $U$ is open iff for all points in $U$, there exists an open neighborhood of the point that's entirely contained in $U$.
\end{thrm}
\begin{defi}
	$f$ is an \emph{open} (\emph{closed}) (\emph{compact}) map if the image of every open (closed) (compact) set under $f$ is open (closed) (compact).
\end{defi}
\begin{thrm}[continuity and openness]
	$f$ is continuous iff the inverse map $f^{-1}$ is open.
\end{thrm}
\begin{defi}
	For any $k \in \N$, a function $f:\R^n \to \R^m$ is \emph{$k$-continuously-differentiable}, denoted $f \in C^k$, if all iterated partial derivatives of $f$ of order $k$ exist and are continuous.
\end{defi}
\begin{thrm}
	For any $k \in \N$, the function $f$ has class $C^k$ iff all of its coordinate functions $f_i$ have class $C^k$.
\end{thrm}
\begin{defi}
	Given a set $V$, a \emph{distance} is a function $D : V \times V \to \R$ that satisfies, for all $x, y, z \in V$:
	\begin{enumerate}
		\item $D(x,x) = 0$
		\item If $x \neq y$, then $D(x,y) > 0$
		\item (symmetry) $D(x,y) = D(y,x)$
		\item (triangle inequality) $D(x,z) \leq D(x,y) + D(y,z)$
	\end{enumerate}
\end{defi}
\begin{defi}
	A \emph{metric space} is $(X, D)$ where $X$ is a set and $D$ is a distance function on $X$.
\end{defi}
\begin{defi}
	A function $f:\R^n \to \R^m$ is \emph{smooth}, denoted $C^\infty$, if $f$ is $C^k$ for all $k \in \N$.
	Synonym: $\infty$-continuously-differentiable.
\end{defi}
\begin{defi}[analytic functions]
	A function $f$ is \emph{analytic} if for every point $x$ in the domain, there exists a taylor series centered at $x$ which converges to $f$ in a neighborhood of $x$.  Note: The class of analytic functions is denoted $C^\omega$.
\end{defi}
\begin{thrm}
	$C^\omega \subsetneq C^\infty \subsetneq \subsetneq \subsetneq C^n \subsetneq C^{n-1} \subsetneq \subsetneq \subsetneq C^0$
\end{thrm}
\begin{example}
	The function $$f(x) =
	\begin{cases}
		e^{-1/x^2} &\text{if $x \neq 0$} \\
		0 &\text{if $x=0$} \\
	\end{cases}
	$$
	is smooth, but it is NOT analytic.
\end{example}
\begin{defi}
	$K \subseteq X$ is \emph{compact} in $X$ if whenever $${K \subseteq \bigcup_{i \in I} O_i}$$ of open sets, then there exists $J$ finite subset of $I$ such that $$K \subseteq \bigcup_{j \in J} O_j$$.
\end{defi}
\begin{thrm}
	If $K$ is compact in $X$ and $C$ is closed in $X$, then $K \cap C$ is compact in $X$.
\end{thrm}
\begin{thrm}[Matt]
	If $K$ is compact in $X$ and $K \subseteq S \subseteq X$, then $K$ is compact in $S$.
\end{thrm}
\begin{conj}[Matt]
	I conjecture that the theorem above is also true for "open" instead of "compact".  Also for "closed" and ANY other topological property.  I will give 10 dollars to whoever proves or disproves this theorem.
\end{conj}
\begin{defi}
	The space $X$ has the \emph{discrete topology} if every singleton contained in $X$ is open.
\end{defi}
\begin{thrm}
	If $K$ is compact and has the discrete topology, then $K$ is finite.
\end{thrm}
\begin{defi}
	Given a space $X$, the following are all equivalent:
	\begin{itemize}
		\item $X$ is T1.
		\item For all $a \in X$, for all $b \in X$, there exists an open neighborhood of $a$ disjoint from $b$.
		\item Every singleton subset of $X$ is closed.
		\item Every finite subset of $X$ is closed.
		\item For every subset $S$ of $X$, $S$ equals the intersection of all open sets containing it.
		\item For every subset $S$ of $X$, $x \in \bar{S} \smallsetminus S$ iff every open neighborhood of $x$ contains infinitely many points in $S$.
	\end{itemize}
\end{defi}
\begin{defi}
	A topological space $X$ is \emph{Hausdorff} if for every $a \neq b$ in $X$, there exists an open neighborhood $A$ of $a$ and an open neighborhood $B$ of $b$ such that $A \cap B = \{\}$.  Synonyms: seperated.  $T_2$.

	Properties of Hausdorff spaces $H$:
	\begin{itemize}
		\item All properties of T1 spaces apply, since a T2 space is T1.
		\item Every compact subset of $H$ is closed in $H$.
	\end{itemize}
\end{defi}
\begin{thrm}
	Properties of continuous functions ($f$):
	\begin{itemize}
		\item $f^{-1}$ is open.
		\item $f^{-1}$ is closed.
		\item $f$ is compact.
	\end{itemize}
\end{thrm}
\begin{cor}
	If $f : X \to Y$ is continuous, $X$ is compact, and $Y$ is Hausdorff, then $f$ is a closed function *and* $f^{-1}$ is compact.  (basically we get all the benefits that we would get if $f^{-1}$ was a continuous function, except for the ``function'' part.)

	\textbf{Note.} Let $C$ be a closed subset of $X$.  $X$ is compact, so $C$ is compact.  $f$ is continuous, so $f(C)$ is compact.  $Y$ is Hausdorff, so $f(C)$ is closed.  $f$ is continuous, so $f^{-1}(f(C)) = C$ is closed.  This is a useful cycle, but it does not imply ``closed iff compact''.  The one case to be wary of is $K$ compact in $X$.  It will follow that $f^{-1}(f(K))$ is closed in $X$, but it is possible that $f^{-1}(f(K)) \neq K$, so we can't conclude that $K$ is closed.
\end{cor}
\begin{cor}
	If $f : X \to Y$ is a continuous bijection, and $X$ is compact and $Y$ is Hausdorff, then $f$ is a homeomorphism. (That is, $f^{-1}$ is continuous).
\end{cor}
\begin{thrm}
	If $X$ and $Y$ are compact, then $X \times Y$ is compact.
\end{thrm}
\begin{thrm}
	$X$ is a top space and $~$ is an equivalence relation on $X$.  Then there is a topology on $X/~$ (the set of equivalence classes) as follows.  $S \subseteq X/~$ is open iff (let backwards(S) : X/~ to X) backwards(S) is open in X.  [[ f: X/~ to Y is continuous iff f of g : X to Y is contniuous.
\end{thrm}
\begin{defi}
	A \emph{topological group} is a Hausdorff space $G$ that is also a group, which satisfies:
	\begin{enumerate}
		\item The function $m : G \times G \to G$ defined by $m((g,h)) = gh$ is continuous.
		\item The function $i : G \to G$ defined by $i(g) = g^{-1}$ is continuous.
	\end{enumerate}
	See Bredon I.15 for more related definitions.
\end{defi}












\begin{defi}[derivative of multivariable function]
	The \emph{tangent plane} to function $f : \R^n \to \R^m$ at point $a$, denoted $df_a$ or $f^{\prime}(a)$, is the unique function that satisfies $\lim_{h\to 0} \frac{||f(a+h) - f(a) - df_a(h)||}{||h||} = 0$.
\end{defi}
\begin{defi}
	The \emph{directional derivative} of $f$ at $x$ in the $v$ direction is $df_x(v)$ which is $\lim_{h\to 0} \frac{f(x+hv) - f(x)}{h}$.
\end{defi}
\begin{defi}
	Let $U$ and $V$ be open sets in $\R^n$ (topological spaces, soon).  A function $f:U\to V$ is a \emph{homeomorphism} iff it is a bijection and both $f$ and $f^{-1}$ are continuous.
	Synonym: bi-continuous function.
\end{defi}
\begin{defi}
	Let $U$ and $V$ be open sets in $\R^n$.  A function $f:U\to V$ is a \emph{$C^r$-diffeomorphism} iff it is a bijection and both $f$ and $f^{-1}$ are $r$-continuously-differentiable.
\end{defi}
\begin{defi}
	Let $U$ and $V$ be open sets in $\R^n$.  A function $f:U\to V$ is a \emph{diffeomorphism} iff it is a bijection and both $f$ and $f^{-1}$ are differentiable.
	Synonym: bi-differentiable function.
\end{defi}
\begin{thrm}
	If $f:\R^n \to \R^m$ is $C^1$ in a neighborhood of $x$, then $f$ is differentiable at $x$.
	% // no proof :(
	% // need to understand this better
\end{thrm}
\begin{thrm}
	If $f:\R^n \to \R^m$ is $C^2$ in a neighborhood of $x$, then for all $i$ and $j$, $\frac{\partial^2f}{\partial x_i \partial x_j}(x) = \frac{\partial^2f}{\partial x_j \partial x_i}(x)$.
	% // no proof :(
\end{thrm}



\begin{thrm}[chain rule]
	If $f$ is differentiable at $a$ and $g$ is differentiable at $f(a)$, then $(g \circ f)$ is differentiable at $a$, and $(g \circ f)'(a) = g'(f(a))f'(a)$.
\end{thrm}
\begin{thrm}[inverse function theorem]
	For any $k \in \N^+$, if $f:\R^n \to \R^n$ is $C^k$ and $f^\prime(x)$ is invertible for some $x$, then $f$ is locally a diffeomorphism of class $C^k$ at $x$.
\end{thrm}
\begin{example}
	Counterexample: Let $$f(x) =
	\begin{cases}
		\frac x2 + x^2\sin\left(\frac 1x\right) \text{ if $x \neq 0$} \\
		0 \text{ if $x=0$} \\
	\end{cases}
	$$  We see that $f$ is differentiable, but the derivative is not continuous (i haven't verified this).  It turns out that $f$ is not one-to-one (and therefore not a bijection).
\end{example}
\begin{thrm}[local submersion theorem]
	Synonym: implicit function theorem.
	Note: This theorem is equivalent to the inverse function theorem.

	function $f : \R^{k+m} \to \R^m$ is $C^1$.
	Let $x \in \R^{k+m}$ and $z \in \R^m$.
	Suppose $f'(x)$ has rank $m$.
	Then there exists $W$ nbd $x$ and $V$ nbd $z$ and $U$ open in $\R^k$ and a diffeomorphism $I : W \to U \times V$
	s.t. $f = I.\pi$ where $\pi$ is the projection from $U \times V$ onto $V$.

\end{thrm}
\begin{thrm}[local immersion theorem]
	This is very similar to the local submersion theorem (more soon...)

	function $f : \R^{m} \to \R^{m+k}$ is $C^1$.
	Let $x \in \R^{m}$ and $z \in \R^{m+k}$.
	Suppose $f'(x)$ has rank $m$.
	Then there exists $V$ nbd $x$ and $W$ nbd $z$ and $U$ open in $\R^k$ and a diffeomorphism $I : U \times V \to W$
	s.t. $f = in.I$ where $in$ is the inclusion from $V$ to $U \times V$.

\end{thrm}
\begin{thrm}
	If $f$ is locally diffeomorphic at $x$, then the slope matrix of $f^{-1}$ at $f(x)$ is given by $(f^{-1})^\prime(f(x)) = (f^\prime(a))^{-1}$.
	Proof. $f$ is locally diffeomorphic at $x$, so $f^{-1}(f(x)) = x$.
	Apply the chain rule: $(f^{-1})^\prime(f(x)) \cdot f^\prime(x) = 1$.
	Finally: $(f^{-1})^\prime(f(x)) = (f^\prime(x))^{-1}$.
\end{thrm}
\begin{thrm}
	A function $f : \R^a \to \R^b$ is continuous iff each of its coordinate functions $f_i$, $1\leq i \leq b$, are continuous.
\end{thrm}

\begin{defi}
	$X$ is \emph{connected} if whenever $X = A \cup B$ ($A$ and $B$ are both open and nonempty), then $A \cap B$ is nonempty.
\end{defi}

\begin{defi}
	A space is \emph{second-countable} if it has a countable basis.  Synonym: \emph{2nd-countable}.
\end{defi}
\begin{defi}
	A space is \emph{seperable} is it has a countable dense subset.
\end{defi}
\begin{thrm}
	Every 2nd-countable space is seperable.
\end{thrm}
\begin{defi}
	A space $M$ is an $n$-dim top \emph{manifold} if $M$ is Hausdorff and 2nd-countable, and $M$ has an open cover $\{U_i\}_{i \in I}$ such that each $U_i$ is homeomorphic (via $\phi_i$, which is itself called a \emph{chart}) to an open set in $\R^n$.  Then $\{ U_i, \phi_i \}$ is called an \emph{atlas} for $M$.
\end{defi}
\begin{thrm}
	The union of any amount of atlases is itself an atlas.
\end{thrm}

\begin{defi}
	An atlas for $M$ is \emph{smooth} if every $\phi_\alpha^{-1}.\phi_\beta$ is $C^\infty$ both ways.
\end{defi}

\begin{defi}
	Two smooth atlases $A$ and $A^\prime$ are \emph{equivalent} if for each $\alpha$, $\beta$, $\phi_\alpha^{-1}.\phi_\beta^\prime$ is $C^\infty$ both ways.
\end{defi}
\begin{rmk}
	It can be verified that the above definition really is an equivalence relation.
\end{rmk}

\begin{defi}
	A \emph{smooth} manifold ("smooth structure") is just a manifold coupled with an equivalence class of smooth atlases.
\end{defi}

\begin{thrm}
	There exist 4-dim manifolds that can't be smooth.
\end{thrm}



\begin{example}
	$M = \R$ and $A = \{ id: \R \to \R \}$.
\end{example}

\begin{example}
	$M = S^1$.  $A = {U_l, U_r, U_t, U_b}$.  For $U_t$, for example, we choose $phi_t$ to be a projection of the semicircle with open endpoints to a line segment with open endpoints.  Do the same for the others.  Looking at $\phi^{-1}$ of $\phi$'s will reveal that their composition is $C^\infty$, so we conclude that $A$ is smooth.
\end{example}

\begin{defi}
	An \emph{n-}manifold or \emph{n-dimensional} manifold is a manifold that is locally homeomorphic to $\R^n$.  (``n-'' is the adjective to ``manifold''.)
\end{defi}


\begin{defi}
	Given $M$ smooth manifold w/ atlas $A = \{ U_a, \,  \phi_a : U_a \to \R^m \}$.
	Given $N$ smooth manifold w/ atlas $B = \{ O_b, \,  \phi_b : U_b \to \R^n \}$.
	A cont function $f : M \to N$ is \emph{smooth} if for all $a$, for all $b$, $\phi_a^{-1}.f.\psi_b$ is smooth.  You might call $f$ a \emph{mersion}.
\end{defi}
\begin{defi}
	Given the same givens as above.
	$N$ and $M$ are \emph{diffeomorphic} if there exists $f : M \to N$ and $g : N \to M$ both smooth such that $f.g = id_M$ and $g.f = id_N$.
\end{defi}
\begin{defi}[Matt's version]
	$N$ and $M$ are \emph{diffeomorphic} if they are bimersive.  (That is, there exists a bimersion between them.)
\end{defi}

\begin{rmk}
	Some manifolds have NO smooth atlases.

	Some manifolds have 2 (or more) smooth atlases that are not equivalent to each other.
\end{rmk}


\begin{thrm}
	$\R^4$ has an uncountable number of smooth structures (equivalence classes of smooth atlases).  (But all other $\R^n$ have a unique smooth structure.)
\end{thrm}

\begin{rmk}
	If $(M,A)$ is a smooth $n$-manifold, and $U$ an open set in $M$, then $(U,A_U)$ is a smooth $n$-manifold, where $A_U$ is like $A$, but intersecting each set with $U$, and restricting the $\phi$ functions accordingly.
\end{rmk}

\begin{defi}
	Given $M$ smooth $m$-manifold with atlas $A = \{ U_a, \,\phi_a : U_a \to \R^m \}$.
	Given $N$ smooth $n$-manifold with atlas $B = \{ V_b, \,\psi_b : V_b \to \R^n \}$.
	$m \leq n$.
	A smooth function $f : M \to N$  is an \emph{immersion} if given any $a, b$ such that $\phi_a: U_a \to \R^m$, $\psi_b: V_b \to \R^n$, and any $x \in \phi_a(U_a \cap V_b.f^{-1})$, then $x.(\phi_a^{-1}.f.\psi_b)'$ has rank $m$.
\end{defi}

\begin{defi}
	$M$ is a \emph{submanifold} of $N$ if there exists an immersion from $M$ to $N$.
\end{defi}

\begin{thrm}
	$\R P^n \isom \R P^{n-1} \cup \R^n$.
	\textbf{Proof.}  In $\R^{n+1}$, draw a flat ``plane'' above the origin.  Each line in $\R P^n$ that has a nonzero ``z'' component will intersect this plane.  So each line maps to its intersection with this plane, and this plane is $\R^n$.  The remaining lines are all in $\R P^{n-1}$.  Wow!
\end{thrm}



If $f$ is a smooth map of manifolds, then we can differentiate, getting the linear map $d_x(\phi_a^{-1}.f.\psi_b): \R^m \to \R^n$.

\begin{thrm}
	$f$ is a smooth map of manifolds.
	$f$ is an \emph{immersion} if forall $x$, $d_x$ above is injective.  (Note: this is *local* injectivity of f) ($m \leq n$)
\end{thrm}

\begin{defi}
	$f$ is a smooth map of manifolds.
	$f$ is a \emph{submersion} if forall $x$, $d_x$ above is surjective.  (Note: this is *local* surjectivity of f) ($m \geq n$)
\end{defi}
\begin{prop}
	Every submersion is open.
\end{prop}

\begin{defi}
	An \emph{embedding} $f$ is an immersion s.t. $f : M \to f(M)$ is a homeomorphism.  Here is an equivalent, ``minimal'' definition:  An \emph{embedding} is an immersion that is injective and proper.
\end{defi}

An immersion is locally injective.  An embedding is globally injective.

\begin{thrm}
	$f$ from $M^m$ to $N^n$ is a submersion, and $f(x) = y$.  Then $f^{-1}(y)$ is a smooth $(m-n)$ dimensional manifold.
\end{thrm}

\begin{defi}
	$f$ is a smooth map of manifolds.
	$\phi_a(x) \in M$ is a \emph{regular} point if $d_x(\phi^{-1}.f.\psi)$ is onto.  Otherwise, $\phi_a(x)$ is a \emph{critical} point.
\end{defi}

\begin{defi}
	$f$ is a smooth map of manifolds.
	$y \in N$ is a \emph{regular} value if every point in $f^{-1}(y)$ is regular.  Otherwise, $y$ is a \emph{critical} value.  Note: values $y$ *not* in the image of $f$ are vacuously regular.
\end{defi}

\begin{thrm}
	This comes from the Local Submersion Theorem.  If $y \in N$ is a regular value, then $f^{-1}(y)$ is a smooth submanifold of $M$ of dimension $m-n$.
\end{thrm}

\begin{example}
	$M_n(\R)$ vectors of dim $n^2$.  $S_n(\R)$ the symmetric matrices.  Note that the matrix $A^T A$ is always symmetric.  $O(n)$ group ortho matrices.

	Let $f$ from $M_n(\R)$ to $S_n(\R)$ given by $f(A) = A^T A$.

	THEN $I_n$ in $S_n(\R)$ is a regular value.  Here's why...

	Corr. $O(n) = f^{-1}(I_n)$ is a smooth manifold of dimension $n^2 - \frac{n(n+1)}{2}$ = $\binom n2$.

	and

	$O(n) = SO(n)$ disjoiunt union $SO(n)Diagmatrix(-1 1 1 1 1 1 1 1 1)$

	proof of claim

	Given $A$ in $M_n$, find a formula $df_A$ from $M_n$ to $S_n$ ...

	$df_A(B) = \lim \frac{f(A+tB) - f(A)}{t} = \ldots = B^T A + A^T B$
	is indeed in $S_n$! (it is symmetric)

	If $A^T A = I_n$, can every symmetric matric $C$ be written in the form $C = B^T A + A^T B$.

	sneaky trick
	Guess $A^T B = \frac12 C$.
	This is possible if we let $B = \frac12 AC$.
	Then it works!
\end{example}

\begin{thrm}[Matt's]
	If $M$ and $N$ are diffeomorphic manifolds, then $M$ and $N$ are smooth.
\end{thrm}


Wed, Sept 21, 2016

Given $f:M \to N$ smooth, then there exists a $df$ (but we could call it anything) s.t. $df : TM \to TN$ and $df.P_N = P_M.f$.  (Where $P_M$ is the projection from $TM$ to $M$)  (But is this magic df the same thing as our $df$ that we already know well?)

\begin{example}
	If $M = \R^2$, then for any $x$, the tangent vector space is $\R^2$ again.  (The flat plane in $\R^2$ at the point $x$ is $\R^2$).  Therefore, $T\R^2 = \R^2 \times \R^2 = \R^2$.  The same goes for $M = \R^m$, replacing all $2$'s with $m$'s.
\end{example}

We can take the disjoint union of these distinct open sets, then wherever we WANT these open sets to overlap, we define those pairs of points to be equivalent.  Therefore, the disjoint union, when viewed as a partition via the equivalence relation, recovers the manifold that we wanted.

Let's take this a step further, using the tangent bundle too:

disjoint union ( $U_i \times \R^n$ ) has elements $(i, x \in U_i, v \in \R^n) = (i, x, v)$.
Then we define the equivalence relation as follows:
$(i, x, v) ~ (j, y, w)$ iff $x=y$ and $v.d_x(\phi_i^{-1}.\phi_j) = w$.

Note that the above $d_x$ is linear in $v$, so ~ is linear in $v$ too.

So in the partition, the $v$'s form a vector space too!  So now we have a single vector space for all of the tangent bundles combined!  Maybe.

yay!

\begin{lem}
	$T_A M$ has an atlas showing its a $2m$-dimensional manifold given
	$T_A M \supset U_i x \R^n \to \R^n x \R^n$ (where the $\to$ function is ($\phi_i \times identity$))
\end{lem}
\begin{cor}
	Suppose $A \subset B$ are atlases for $M$.  The ``canonical'' map

	$T_A M = \sqcup_{i \in I_A} U \times \R^n/~ \to \sqcup (same but B) = T_B M$
				$\to										\to$
				$M 										M$
	is a diffeo of manifolds.
\end{cor}

Some $f: M \to N$ induces $Tf : TM \to TN$ as follows:
Given $\phi_i$ a chart for $M$ about $x$.
Given $\psi_j$ a chart for $N$ about $f(x)$.
THEN
$Tf((i, x, v)) = (j, f(x), v.d_x(\phi_i^{-1}.f.\psi_j))$.

And $T(M \times N)$ ``canonically'' identifies with $TM \times TN$.


Monday, Sept 26, 2016



$df_x : T_xM \to T_yN$ is onto.

$y$ is a regular value of $f$.

Let $L = f^{-1}(y)$, a smooth ($m-n$) dimensional submanifold of $M$.

$T_xL = \ker df_x$

$x \in L \subset M$


Wednesday, Sept 28, 2016

More elegant definition of regular:
\begin{defi}
	$f$ is a smooth map of manifolds.
	$x \in M$ is a \emph{regular} point if $d_x : T_xM \to T_yN$ is onto.  Otherwise, $x$ is a \emph{critical} point.
\end{defi}

\begin{defi}
	$f$ is a smooth map of manifolds.
	$y \in N$ is a \emph{regular} value if every point in $f^{-1}(y)$ is regular.  Otherwise, $y$ is a \emph{critical} value.  Note: values $y$ *not* in the image of $f$ are vacuously regular.
\end{defi}

\begin{thrm}
	$f$ is a smooth map of manifolds.
	If $y$ is a regular value, then $f^{-1}(y)$ is a smooth submanifold of $M$.
	Also, if $x \in f^{-1}(y)$, then $T_xf^{-1}(y) = \ker df_x$.
	Also, $Tf^{-1}(y) = \{ (x,v) | f(x)=y, df_x(v) = 0\}$.
\end{thrm}

\begin{thrm}
	$f$ is a smooth map of manifolds.
	Let $L$ be a subset of $N$.
	If $M$ is a submanifold of $N$, then $f^{-1}(y) = M \cap L$.
\end{thrm}

\begin{defi}
	$f$ is \emph{transverse} to $L \subset N$, denoted $f \pitchfork L$, if $f$ is a smooth map of manifolds and whenever $f(x) = y \in L$, then $df_x(T_xM) + T_yL = T_yN$.  (yes, we are adding vector spaces).
\end{defi}

\begin{thrm}
	If $f$ is transverse to $L$, then $$\dim(M) + \dim(L) = \dim(N) + \dim f^{-1}(L)$$.
	If $M$ is transverse to $L$, then $$\dim(M) + \dim(L) = \dim(N) + \dim(M \cap L)$$.
	I think this may generalize to $$\dim(M) + \dim(L) = \dim(M \cup L) + \dim(M \cap L)$$, if the notion of taking the union of $M$ and $L$ corresponds to taking the direct sum of the tangent spaces of $M$ and $L$.

	Furthermore, if $x \in K$ and $y = f(x)$, then $T_x K = df^{-1}_x(T_y L) \subset T_x M$.
\end{thrm}

\begin{thrm}
	Given $f : U \to V$, and
	$g : V \to W$ is onto.

	Then ($f.g$ is onto iff $f(U) = \ker g$).
\end{thrm}

\begin{thrm}
	Transversailty holds iff $d(f.p)$ is onto.

	\textbf{Proof.}
	Suppose $y \in L$.
	By the local immersion thm,
	there exists $O$ open nbd of $y$, and a submersion $p : O \to \R^{n-l}$
	s.t. $p(y) = 0$ and $p^{-1}(0) = L \cap O$

	$df_x : T_xM \to T_yN$.
	$df_y : T_yN \to \R^{n-l}$.

	By prior thm, we conclude $\ker df_y = T_yL$.
\end{thrm}

Wednesday, Oct 5, 2016

\begin{defi}
	$C^\infty(M) := \{ f:M \to \R | f is smooth \}$
\end{defi}
\begin{thrm}
	$C^\infty(M)$ is a commutative $\R$-algebra. (It's a ring, a V.S., has a bilinear prod)
\end{thrm}
\begin{thrm}
	If $U$ is a submanifold of $M$, then $C^\infty(U)$ is also an $\R$-algebra.
	(is this true?  i don't know!)
\end{thrm}
\begin{thrm}
	If $U$ and $V$ both open and $U \subset V \subset M$, then
	the restriction from $C(V)$ to $C(V)$ is a morphism of $\R$-algebras.
\end{thrm}
\begin{prop}[sheaf property]
	If $U$ = union of $A$
	and $f_i$ in $C(A_i)$ s.t. $f|_{A_i \cap A_j} = f_j|_{A_i \cap A_j}$ for all $i, j$
	then there exists a UNIQUE $f \in C(U)$ s.t. $f|_{A_i} = f_i$ for all $i$.
\end{prop}

a bunch of stuff i can't read

Fix $p \in M$.
Put an equivalence relation on all smooth functions $\{ f : U \to \R | \text{smooth}\}$
with $U$ is a neighborhood of $p$.

$f:U \to \R \sim g:V \to \R$
iff
there exists $W$ s.t. $p \in W \subset U \cap V$ where $f|_W = g|_W$.

Let $C^\infty(M,p)$ = the set of equiv classes.  This is called the \emph{stalk} of the sheaf at point $p$.  Or the \emph{germ} of a function about $p$.

direct limit
$= c\lim C^\infty(U)$.
for open nieghborhoods of $p$.

If $\phi : M to N$ in smooth and $\phi(p) = q$, then we get an induced ring morhpism
$\phi^* : C^\infty(N,q) \to C^\infty(M,p)$ by
$\phi^*([f]) = [\phi.f]$

$\psi : L \to M$
$\phi : M \to N$

$(\psi.\phi)^* = \phi*.\psi* : C^\infty(n,q) \to C^\infty(L,r)$

Given $v \in T_pM$, want to define
$D_v : C^\infty(M,p) \to \R$
``directional derivative in the $v$-direction'' at $p$

Example for $M = \R_n$, $p \in \R_n$, and $v \in T_p\R_n = \R_n$.

$D_v : C^\infty(\R_n, p) \to \R$

$D_v([f]) = \lim \frac{f(p + tv) - f(p)}{t}$

properties
\begin{prop}
	\begin{enumerate}
		\item $D_{av + bw} = aD_{v} + bD_{w}$
		\item $D_v$ is linear
		\item $D_v(fg) = f*D(g) + D(f)*g$
	\end{enumerate}
\end{prop}
\begin{thrm}
	Suppose $D : C_infty(\R^n,p) \to \R$
	s.t. $D$ satisfies the properties above.  Then $D = D_v$ for some $v \in \R^n$.
\end{thrm}
proof not shown.
\begin{lem}
	Given $U \subset \R^m$ and $V \subset \R^n$ open.
	$\phi : U \to V$ is smooth.
	$p \in U$
	$q = \phi(p)$
	Let
	$w = d \phi_p(v)$ in $\R_n$

	Then
	$D_w = D_v of \phi^* : C^\infty(\R^n,q) \to \R$

	$M$ is an $n$-dimensional manifold, $p \in M$, $v \in T_pM$.
	want to define
	$D_v : C^\infty(M,p) \to \R$
	given $\phi: M \supseteq U \to \R_n$ let $D_v$ be the composit
	$C^\infty(M, p)$ to by ${\phi^{-1}}^*$ to $C^\infty(\R^n,q)$ to $D_w$ to $\R$.

	independent of coordinate chart, by lemma
\end{lem}
\begin{defi}
	A \emph{derivation} is a function that is linear and satisfies the product rule.
\end{defi}
\begin{defi}
	$\Der(M,p)$ is the set of all such derivations $D$ of the form $D: C^\infty(M,p) \to \R$.
\end{defi}
\begin{thrm}
	The map $T_pM$ to $\Der(M,p)$
	that sends $v$ to $D_v$
	is an isomorphism of vector spaces.
\end{thrm}

Monday, October 10, 2016

$M^n$ is a Manifold.
$v \in T_xM$ gives you a direction in which to take dir derivatives.
$D_v$ directional derivative in $v$-direction

\begin{defi}
	Given $p: TM \to M$, a \emph{smooth vector field} on $M$ is a smooth section to $p$.
\end{defi}
\begin{example}
	The zero function is a smooth vector field on $M$.
\end{example}
\begin{example}
	$V$ open (and connected?) in $\R^n$ is a smooth $n$-dim manifold.

	Recall that $TV \isom   V \times \R^n$.  Let $p:TV \to V$ and $\pi:V \times \R^n \to V$.

	The following map is a bijection:
	$\{\text{vector fields}\} \to \{\text{smooth functions}\}$
	$        X: V \to TV      \mapsto     X: V \to \R^n$
\end{example}
\begin{example}
	$C^\infty(M,x)$ is the set of germs about $x$.
	Each germ is an equiv class.
	The germs make up a V.S., that is, $C^\infty(M,x)$ is a V.S. over $\R$ (the operation is function multiplication).  Moreso, it is a commutative $\R$-algebra.

	Define $C^\infty(M, TM)$ as the set of vector fields on $M$.  (That is, the set of smooth $f: M \to TM$ with $f(x) \in T_xM$.)  We can add these vector fields together, and multiply them by constants in $\R$.  So we have another V.S.

	Let $F$ be one of these vector fields and $g \in C^\infty(M)$.
	Let $(gF)(x) = g(x)F(x)$.
	So $C^\infty(M)$ is a ring and $C^\infty(M, TM)$ is module over $C^\infty(M)$.  Or, more succinctly, ``$C^\infty(M, TM)$ is a $C^\infty(M)$-module''.

	Let $F_i : V to ((TV = Vx\R^n))$
	Let $Fi : V to \R^n$ be the constant vector field $Fi(x) = (((x,ei) or just ei))$ = the ith basis vector.

	then cool things happen that i missed :(
\end{example}
\begin{defi}
	A \emph{derivation} of a commutative algebra $A$ is derivation of the form $D: A \to A$.

	$\Der(A)$ is the set of all such derivations.
\end{defi}
\begin{thrm}
	The set of derivations $\Der(A)$ of a commutative algebra $A$ is an $A$-module.  That is, $\Der(A)$ is a commutative additive group, and this is a module over $A$.  (Observe that if $c \in A$, then $(cD)(x) = cD(x)$).
\end{thrm}

\begin{prop}
	vector fields on $M ---------->$ derivations on $C^\infty(M)$
	$C^\infty(M,TM) ------------> Der(C^\infty(M))$
	$x |-----------> Dx$

	This is an isomorphism of $C^\infty(M)$-modules!
\end{prop}


\begin{defi}
	The \emph{Lie bracket} of $A$ and $B$, denoted $[A,B]$, is $AB - BA$.
\end{defi}
\begin{thrm}
	Derivations admit a Lie bracket.
\end{thrm}
\begin{proof}
	For any two derivations of $A$, is
	Is $D \circ C$ in $\Der(A)$ too?
	\begin{align}
		(DC)(ab) &= D(C(a))b + C(a)D(b) + D(a)D(b) + aD(C(b)) \\
		\text{look at reverse} \\
		(CD)(ab) &= C(D(a))b + D(a)C(b) + C(a)C(b) + aC(D(b)) \\
		\text{Then} \\
		(DC - CD)(ab) &= (DC - CD)(a)b + a(DC - CD)(b)
	\end{align}
	So $(DC - CD) \in \Der(A)$.
	So $[D,C] \in \Der(A)$.
	Use some law of commutators to show that $D.C$ in $\Der(A)$ too?
\end{proof}



\begin{example}
	Vector fields admit a Lie bracket (because they are derivations)
	$X, Y \in C^\infty(M, TM)$ implies that $[X,Y] \in C^\infty(M, TM)$
\end{example}
\begin{proof}
	First write $X$ and $Y$ as derivations.  Then use the above theorem, or repeat the work of the above theorem as follows:

	Let $X(x,y) = f(x,y)d/dx$
	Let $Y(x,y) = g(x,y)d/dx$
	so $Y(h) = g(x,y)d/dx(h) = g(x,y)dh/dx$  (not $Y$ OF $h$, but $Y$ applied to $h$)

	Then $X(Y(h)) = f dg/dx dh/y + fg d2h/dxdy$
	Then $Y(X(h)) = g df/dx dh/y + gf d2h/dxdy$

	Then $(XY - YX)(h)$ = compute
	$[X,Y](h) = f dg/dx dh/y - g df/dx dh/y$
	in general:
	$[X,Y] = f dg/dx d/y - g df/dx d/y$
\end{proof}

Wednesday, October 12, 2016

Q1
Given M, does there exists a vector field X on M s.t. for all $x \in M$, X(x) \neq 0 ?

example
M = S1.  TS1 = { (z,v) | z in S1, z dot v = 0 }
yup!
X(z) = X(x,y) = (z, (-y,z)) works!

example
M = S2.
no!

example
M = S3 \subseteq \C x \C
S3 = { (z1,z2) | |z1|^2 + |z2|^2 = 1 }
yes!
X(z) = (z, iz) works!

example
M = S^{2n-1}
yes!
use the same trick!

Q2
Given M, what is the maximum number $k$ of vector fields X1,,,Xk that for every $x \in M$, {X1(x),,,Xk(x)} is linearly independent?

k = n = dim M is best possible

example
M \subseteq S3 \subseteq \H (has 4 dim)
q = x*1 + y*i + z*j + w*k
Let
X1(q) = (q, iq)
X2(q) = (q, jq)
X3(q) = (q, kq)
yes!

\begin{thrm}[J.F. Adams]
	Given n \in \N and write n = odd*2^{4d+c}
	Let p(n) = 2^c + 8d
	Then S^{n-1} admits p(n)-1 linearly independent vector fields, but not p(n).
\end{thrm}

consider we have k = n, best possible


\begin{defi}
	M is __parallelizable__ if there exists a function phi s.t.
	phi : M x Rn to TM
	by (x, t1,,,tn) mapsto (x, t1X1(x),,,tnXn(x))
	and phi is a diffeomorphism that commutes with

	phi : M x Rn to TM
	M x Rn ---pi---> M
	p: TM ----------> M
\end{defi}
\begin{thrm}
	M is \emph{parallelizable} iff the maximum number of lin. ind. vector fields = n = dim M.
\end{thrm}

weaker than parallelizable
-------------------------------

An n-dim M is __framed__ if there exists k and a vector bundle isomorphism f s.t.

TM x Rk ----f----> M x R{n+k}
TM x Rk ----p----> M
M x R{n+k} -----q----> M

s.t. f is smooth and each map TxM x Rk -----> R{n+k} is a linear isomorphism. (that is, f is a diffeomorphism)
pinverse(x) -----f-----> qinverse(x)





example
S2 is frames with k=1 by
f : TS2 x R --------> S2 x R3
((x,v), t) mapsto (x, v + tx)
<x>

observe that
TS2 \not\isom S2 x R2
but
TS2 x R \isom S2 x R3
wow!

or in general, we have
R = Coo(S2)
M = C(S2, TS2)
M is an R-module
then
M \not\isom R directsum R, as R-module
but
M directsum R \isom R directsum R directsum R, as R-module
wow!

example
RP2 not framed
CP2 not framed



dfx in (TxM)*, called the __dual vector space__ of TxM

def
cotangent bundle
----------------------
T*M = \bigcup_{x \in M} (TxM)*


observe
TM = \sqcup (Ui x Rn)/~
where (i, x, v) ~ (j, x, d(\phi_i of phi_i inverse)_x(v))
analagously,
T*M = \sqcup (Ui x (Rn)*)/~
where (j, x, w) ~ (i, x, d(\phi_j of phi_j inverse)_x(w))

\begin{defi}
	A __1-form__ on M is a section of T*M ----> M.

	in other words, for all x in M, w(x): TxM --> R and this varies "smoothly" in x
\end{defi}
notation
omega'(M) = Coo(M, T*M) = set of all 1-forms

example
U \subseteq Rn
A general vector field on U had the form
f1 d/dx1 +++ fn d/dxn
where fi : U to Rn
then
A general 1-form on U has the form
f1 dx1 +++ fn dxn
(observe that dx1 is dual to d/dx1, etc)

















\end{document}
