\documentclass[11pt,leqno,oneside]{amsart}

\usepackage{../notes}
%%% matt experiments with theorem style:
\usepackage{amsthm}% http://ctan.org/pkg/ams\begin{thrm}
\usepackage{datetime}
\usepackage{hyperref}
\usetikzlibrary{cd}
\newenvironment{dateenv}{
	\vspace{1em}
}{
	\vspace{1em}
}

\newcommand{\id}{\text{id}}
\newcommand{\minus}{\smallsetminus}
\newcommand{\Fix}{\text{Fix}}
\newcommand{\sign}{\text{sign}}
\newcommand{\Crit}{\text{Crit}}
\newcommand{\Gr}{\text{Gr}}
\renewcommand{\amalg}{\sqcup}
\newcommand{\homotopic}{\simeq}
\renewcommand{\epsilon}{\varepsilon}
\renewcommand{\d}{\partial}
\newcommand{\oo}{\infty}
\newcommand{\transverse}{\pitchfork}
\newcommand{\into}{\hookrightarrow}
\newcommand{\onto}{\twoheadrightarrow}
\newcommand{\x}{\times}
\renewcommand{\bar}{\widebar}
\newcommand{\grad}{\nabla}
\newcommand{\Der}{\text{Der}}
\newcommand{\supp}{\text{supp}}
\newcommand{\de}{\emph}
\newcommand{\mydate}[4]{
	\newdate{#1}{#2}{#3}{#4}
	\begin{dateenv}
		\hfill\displaydate{#1}
	\end{dateenv}
}

\newtheoremstyle{mystyle}
  {20pt plus 4pt minus 4pt} % Space above
  {\topsep} % Space below
  {} % Body font
  {} % Indent amount
  {\bfseries} % Theorem head font
  {.} % Punctuation after theorem head
  {.5em} % Space after theorem head
  {} % Theorem head spec (can be left empty, meaning `normal')

\theoremstyle{mystyle} \newtheorem{thrm}[thm]{Theorem}
\theoremstyle{mystyle} \newtheorem{defi}[thm]{Definition}

\everymath{\displaystyle}

%%%%%%%%%%%%%%%%%%%%%%%%%%%%%%%%% TITLE %%%%%%%%%%%%%%%%%%%%%%%%%%%%%%%%%
\title[Differential Topology]{Differential Topology}
\author{Matthew Lancellotti, George Seelinger\\ (class notes for Nick Kuhn's MATH 7820)}
\date{Fall 2016}
\begin{document}
\maketitle
\section{Lecture 1}

What is differential topology? Topology is concerned with objects called
topological spaces and continuous functions between them. However, the name
differential topology suggests that calculus is somehow involved. The objects of
study in differential topology are manifolds. Manifolds are topological spaces
that ``locally'' look like $\R^n$.
\begin{example}
	Consider the set $S^2 = \{ \vec{x} \in \R^3: ||x||=1\}$. This is a $sphere \in 3$ dimensions and thus can also be written as $S^2 = \{(x,y,z) :
	x^2+y^2+z^2=1\}$. The convention for $S^2$ or ``two-sphere'' comes from the
	fact that a small space of $S^2$ looks like $\R^2$.

	To make this explicit, consider two maps. The first is a map from $\R^2$ to
	a disk by $$f(\vec{x}) = \frac{||x||^2}{1+||x||^2} \vec{x}$$. Then consider a
	map $g(x,y) = (x,y,\sqrt{1-x^2-y^2})$. Composing these two maps will map
	$\R^2$ onto part of $S^2$. Variations on the same theme can be used to get
	other parts of the sphere.
\end{example}
\begin{defi}
	The \de{$n$-sphere}, denoted $S^n$, is $\{x \in \R^{n+1} |\; ||x||=1 \}$. Note: The *surface* of the $n$-sphere has dimension $n$.,
\end{defi}
\begin{example}
	Continuing with the sphere, consider the plane through the origin that is parallel to the tangent plane of $S^2$ at some
	point $\vec{x}$. We will denote this $T_{\vec{x}}S^2$. Note that this is a
	real vector space. Using symmetry arguments, we can see that, if $\vec{v}
	\in T_{\vec{x}}S^2$, then $\vec{v}$ is perpendicular to $\vec{x}$ and thus
	$\vec{v} \cdot \vec{x} = 0$. So, to use a concrete point,
	$T_{(1/3,2/3,2/3)} S^2 = \{(x,y,z) : x+2y+2z=0\}$.  Note also that we will abusively call this plane the \de{tangent plane}, even though we are actually referring to the plane through $0$ instead of the plane through $\vec{x}$.
\end{example}
\begin{example}
	Now, to think globally, consider the collection of all tangent planes of
	$S^2$. We call this the \de{tangent bundle} of $S^2$ and denote it
	$TS^2$. We can see $TS^2 = \{(\vec{x}, \vec{v}): \vec{x} \in S^2, \vec{v}
	\in T_{\vec{x}} S^2\} = \{(\vec{x},\vec{v}) : ||\vec{x}|| = 1, \vec{x}
	\cdot \vec{v} = 0\} \subset S^2 \times \R^3 \subset \R^3 \times \R^3$.
	Finally, consider that there exists a function $p$ such that $p: TS^2 \to
	S^2$ by $(x,v) \to x$. Then, we have that $p^{-1}(x) = T_{\vec{x}}S^2$.
\end{example}
\begin{defi}
	The \de{tangent bundle} of a manifold $M$, denoted $TM$, is $\{ (x,v) |\; x \in M \;\text{and}\; v \in T_xM \}$.
\end{defi}

Now, let $M$ be any differential manifold. Studying the function $p: TM \to M$
can tell us a lot about $M$.

\begin{example}
	A \de{section} of $p: TS^2 \to S^2$ is a function $s: S^2 \to TS^2$ such
	that $p \circ s =$ identity. In other words, $\forall x \in S^2, s(x) \in
	T_{\vec{x}}S^2$. Note that this is a vector field. Furthermore, to be
	interesting in differential topology, $s$ should be continuous or
	differentiable.
\end{example}



\subsection{Functions}

\begin{defi}
	A function $f: \R^n \to \R^m$ is continuous at $\vec{a}$ if $\forall
	\epsilon > 0, \exists \delta > 0$ such that $||\vec{x}-\vec{a}|| < \delta
	\implies ||f(\vec{x}) - f(\vec{a})|| < \epsilon$.
\end{defi}
Now, recall \begin{enumerate}
	\item If $f(x) = (f_1(x), \ldots, f_m(x))$, then $f$ is continuous if and
		only if each $f_i$ is continuous. This fact follows from the basic
		construction of the product topology.
	\item Let $B_r(a) = \{x \in \R^n : ||x-a|| < r\}$. Then, $f$ is continuous
		if and only if, given open set $U \subset \R^m$, $f^{-1}(U)$ is open in
		$\R^n$.
	\item If $f: Z \to X \times Y$, $f$ is continuous if and only if $f_1: Z
		\to X$ and $f_2: Z \to Y$ are continuous, where $f_1, f_2$ are the
		obvious restrictions.
\end{enumerate}

Now, let us define what it means for a function to be differentiable.
\begin{defi}
	$f :\R^n \to \R^m$ is \de{differentiable} at $x \in \R^n$ if there exists a linear function $df_x : \R^n \to \R^m$  such that $$\lim_{h \to 0} \frac{|| (f(x)-f(x+h)) - (df_x(h)) ||}{||h||} = 0$$.  Note that $h$ is a vector, as well as $x$.
\end{defi}
\begin{defi}
	If $f$ is differentiable at $x$ by a linear function $L$, then $L$ is the \de{differential} of $f$ at $x$.  This is denoted $df_x$.  Using the standard basis, the differential can be expressed as a slope matrix, which is denoted $f'(x)$.
\end{defi}
\begin{cor}
	If $f$ is linear, then $df = f$.  (This comes straight out of the definition of $df$!)
\end{cor}

\begin{thrm}
	If $f$ is differentiable at $x$, then the differential $df_x$ in the $v$ direction is equal to $\lim_{t \to 0} \frac{f(x+tv) - f(x)}{t}$.
\end{thrm}


\begin{thrm}[a-b-dimensional tangent plane (warning, this is not used in our class)]
	Given a function $f : \R^a \to \R^b$ and its slope matrix $f'(v)$ at a vector $v \in \R^a$, then the equation for any point $w$ on the plane tangent to $f$ at $v$ is $g_v(w) = f(v) + f'(v)(w-v)$ or, in point-slope form, $\frac{g_v(w)-f(v)}{w-v} = f'(v)$.
	% dependencies: [function, derivative, tangent],
\end{thrm}

\begin{thrm}[billiard ball theorem]
	A continuously differentiable section $s$ on $TS^2$ with $s(x) \neq 0$ for all $x \in S^2$ does not exist.,
	intuition: Can you draw a continuously differentiable, always nonzero vector field on a billiard ball?  No.,
	% // NO \to apparently TS^2 not isomorphic to S^2 \times \R^2.
	% // Ts^2 \subseteq S^2 \times \R^3 \subseteq \R^3 \times \R^3.
\end{thrm}

A famous topological question is: could $s(x) \neq \vec{0} \in T_{\vec{x}}S^2$
for all $x \in S^2$? The famous answer is no. This is analogous to the
so-called ``hairy-ball theorem''. By contrast, the answer would be yes if $S^2$
were replaced by a circle ($S^1$), a 3-sphere ($S^3 \subset \R^4$), or a torus
($T^2$).

Given this result, clearly $TS^2$ is not $S^2 \times \R^2$. This is because,
for $S^2 \times \R^2$, we could define $s(\vec{x}) = (\vec{x}, \vec{0})$ and
$p( (\vec{x},\vec{y}) ) = \vec{x}$ which gives us $(p \circ s)(\vec{x}) =
p(s(\vec{x})) = p( (\vec{x}, \vec{0}) )= \vec{x}$.


In differential topology, one can also do ``fiberwise'' linear algebra. Let
$T^*M$ be a collection of the dual vector spaces $(T_{\vec{x}}M)^*$ where $V^* =
\{L: V \to \R\}$ with $L$ linear. A section of $T^*M \to M$ is given by $(x,L)
\to x$ where $L: T_{\vec{x}}M \to \R$ in this case. Now, we call $w: M \to
T^*M$ such that $w(x): T_{\vec{x}}M \to \R$ a collection of \de{1-forms}.


\begin{defi}
	A \de{topological space} (a.k.a \de{space}) is a set $X$ and a set $O$ of subsets of $X$ that satisfy the following conditions.  Every element of $O$ is ``open'' by definition.  Further:
	\begin{itemize}
		\item Arbitrary unions of open sets are open.
		\item Finite intersections of open sets are open.
		\item The empty set is open.
		\item The entire space is open.
	\end{itemize}
\end{defi}
\begin{thrm}
	The set $U$ is open iff for all points in $U$, there exists an open neighborhood of the point that's entirely contained in $U$.
\end{thrm}
\begin{defi}
	$f$ is an \de{open} map if the image of every open (closed) (compact) set under $f$ is open (closed) (compact).
	For any topological space property, substitute ``open'' above with the name of that property, to get another definition.  Such properties include:
	\begin{itemize}
		\item open
		\item closed
		\item compact
		\item connected
		\item path-connected
		\item measure 0
	\end{itemize}
\end{defi}
\begin{thrm}[continuity and openness]
	$f$ is continuous iff the inverse map $f^{-1}$ is open.
\end{thrm}
\begin{defi}
	For any $k \in \N$, a function $f:\R^n \to \R^m$ is \\\de{$k$-continuously-differentiable}, denoted $f \in C^k$, if all iterated partial derivatives of $f$ of order $k$ exist and are continuous.
\end{defi}
\begin{thrm}
	For any $k \in \N$, the function $f$ has class $C^k$ iff all of its coordinate functions $f_i$ have class $C^k$.
\end{thrm}
\begin{defi}
	Given a set $V$, a \de{distance} is a function $D : V \times V \to \R$ that satisfies, for all $x, y, z \in V$:
	\begin{enumerate}
		\item $D(x,x) = 0$
		\item If $x \neq y$, then $D(x,y) > 0$
		\item (symmetry) $D(x,y) = D(y,x)$
		\item (triangle inequality) $D(x,z) \leq D(x,y) + D(y,z)$
	\end{enumerate}
\end{defi}
\begin{defi}
	A \de{metric space} is $(X, D)$ where $X$ is a set and $D$ is a distance function on $X$.
\end{defi}
\begin{defi}
	A function $f:\R^n \to \R^m$ is \de{smooth}, denoted $C^\infty$, if $f$ is $C^k$ for all $k \in \N$.
	Synonym: $\infty$-continuously-differentiable.
\end{defi}
\begin{defi}[analytic functions]
	A function $f$ is \de{analytic} if for every point $x$ in the domain, there exists a taylor series centered at $x$ which converges to $f$ in a neighborhood of $x$.  Note: The class of analytic functions is denoted $C^\omega$.
\end{defi}
\begin{thrm}
	$C^\omega \subsetneq C^\infty \subsetneq \subsetneq \subsetneq C^n \subsetneq C^{n-1} \subsetneq \subsetneq \subsetneq C^0$
\end{thrm}
\begin{example}
	The function $$f(x) =
	\begin{cases}
		e^{-1/x^2}	&\text{if $x \neq 0$} \\
		0			&\text{if $x=0$} \\
	\end{cases}
	$$
	is smooth, but it is NOT analytic.
\end{example}
\begin{defi}
	$K \subseteq X$ is \de{compact} in $X$ if whenever $${K \subseteq \bigcup_{i \in I} O_i}$$ of open sets, then there exists a finite $J \subseteq I$ such that $$K \subseteq \bigcup_{j \in J} O_j$$.
\end{defi}
\begin{thrm}
	If $K$ is compact in $X$ and $C$ is closed in $X$, then $K \cap C$ is compact in $X$.
\end{thrm}
\begin{thrm}[Matt]
	If $K$ is compact in $X$ and $K \subseteq S \subseteq X$, then $K$ is compact in $S$.
\end{thrm}
\begin{conj}[Matt]
	I conjecture that the theorem above is also true for ``open'' instead of ``compact''.  Also for ``closed'' and ANY other topological property.  I will give 10 dollars to whoever proves or disproves this theorem.
\end{conj}
\begin{defi}
	The space $X$ has the \de{discrete topology} if every singleton contained in $X$ is open.
\end{defi}
\begin{thrm}
	If $K$ is compact and has the discrete topology, then $K$ is finite.
\end{thrm}
\begin{defi}
	Given a space $X$, the following are all equivalent:
	\begin{itemize}
		\item $X$ is $T_1$.
		\item For all $a \in X$, for all $b \in X$, there exists an open neighborhood of $a$ disjoint from $b$.
		\item Every singleton subset of $X$ is closed.
		\item Every finite subset of $X$ is closed.
		\item For every subset $S$ of $X$, $S$ equals the intersection of all open sets containing it.
		\item For every subset $S$ of $X$, $x \in \widebar{S} \minus S$ iff every open neighborhood of $x$ contains infinitely many points in $S$.
	\end{itemize}
\end{defi}
\begin{defi}
	A topological space $X$ is \de{Hausdorff} if for every $a \neq b$ in $X$, there exists an open neighborhood $A$ of $a$ and an open neighborhood $B$ of $b$ such that $A \cap B = \{\}$.  Synonyms: seperated.  $T_2$.

	Properties of Hausdorff spaces $H$:
	\begin{itemize}
		\item All properties of $T_1$ spaces apply, since a $T_2$ space is $T_1$.
		\item Every compact subset of $H$ is closed in $H$.
	\end{itemize}
\end{defi}
\begin{thrm}
	Properties of continuous functions ($f$):
	\begin{itemize}
		\item $f^{-1}$ is open.
		\item $f^{-1}$ is closed.
		\item $f$ is compact.
		\item $f$ is connected.
		\item $f$ is path-connected.
	\end{itemize}
\end{thrm}
\begin{cor}
	If $f : X \to Y$ is continuous, $X$ is compact, and $Y$ is Hausdorff, then $f$ is a closed function *and* $f^{-1}$ is compact.  (basically we get all the benefits that we would get if $f^{-1}$ was a continuous function, except for the ``function'' part.)

	\textbf{Note.} Let $C$ be a closed subset of $X$.  $X$ is compact, so $C$ is compact.  $f$ is continuous, so $f(C)$ is compact.  $Y$ is Hausdorff, so $f(C)$ is closed.  $f$ is continuous, so $f^{-1}(f(C)) = C$ is closed.  This is a useful cycle, but it does not imply ``closed iff compact''.  The one case to be wary of is $K$ compact in $X$.  It will follow that $f^{-1}(f(K))$ is closed in $X$, but it is possible that $f^{-1}(f(K)) \neq K$, so we can't conclude that $K$ is closed.
\end{cor}
\begin{cor}
	If $f : X \to Y$ is a continuous bijection, and $X$ is compact and $Y$ is Hausdorff, then $f$ is a homeomorphism. (That is, $f^{-1}$ is continuous).
\end{cor}
\begin{thrm}
	If $X$ and $Y$ are compact, then $X \times Y$ is compact.
\end{thrm}
\begin{thrm}
	$X$ is a top space and $~$ is an equivalence relation on $X$.  Then there is a topology on $X/~$ (the set of equivalence classes) as follows.  $S \subseteq X/~$ is open iff (let backwards(S) : X/~ to X) backwards(S) is $open \in X$.  [[ f: X/~ to Y is continuous iff f of $g: X \to Y$ is contniuous.
\end{thrm}
\begin{defi}
	A \de{topological group} is a Hausdorff space $G$ that is also a group, which satisfies:
	\begin{enumerate}
		\item The function $m : G \times G \to G$ defined by $m((g,h)) = gh$ is continuous.
		\item The function $i : G \to G$ defined by $i(g) = g^{-1}$ is continuous.
	\end{enumerate}
	See Bredon I.15 for more related definitions.
\end{defi}












\begin{defi}[derivative of multivariable function]
	The \de{tangent plane} to function $f : \R^n \to \R^m$ at point $a$, denoted $df_a$ or $f^{\prime}(a)$, is the unique function that satisfies $\lim_{h\to 0} \frac{||f(a+h) - f(a) - df_a(h)||}{||h||} = 0$.
\end{defi}
\begin{defi}
	The \de{directional derivative} of $f$ at $x$ in the $v$ direction is $df_x(v)$ which is $\lim_{h\to 0} \frac{f(x+hv) - f(x)}{h}$.
\end{defi}
\begin{defi}
	Let $U$ and $V$ be open sets in $\R^n$ (topological spaces, soon).  A function $f:U\to V$ is a \de{homeomorphism} iff it is a bijection and both $f$ and $f^{-1}$ are continuous.
	Synonym: bi-continuous function.
\end{defi}
\begin{defi}
	Let $U$ and $V$ be open sets in $\R^n$.  A function $f:U\to V$ is a \de{$C^r$-diffeomorphism} iff it is a bijection and both $f$ and $f^{-1}$ are $r$-continuously-differentiable.
\end{defi}
\begin{defi}
	Let $U$ and $V$ be open sets in $\R^n$.  A function $f:U\to V$ is a \de{diffeomorphism} iff it is a bijection and both $f$ and $f^{-1}$ are differentiable.
	Synonym: bi-differentiable function.
\end{defi}
\begin{thrm}
	If $f:\R^n \to \R^m$ is $C^1$ in a neighborhood of $x$, then $f$ is differentiable at $x$.
	% // no proof :(
	% // need to understand this better
\end{thrm}
\begin{thrm}
	If $f:\R^n \to \R^m$ is $C^2$ in a neighborhood of $x$, then for all $i$ and $j$, $\frac{\partial^2f}{\partial x_i \partial x_j}(x) = \frac{\partial^2f}{\partial x_j \partial x_i}(x)$.
	% // no proof :(
\end{thrm}



\begin{thrm}[chain rule]
	If $f$ is differentiable at $a$ and $g$ is differentiable at $f(a)$, then $(g \circ f)$ is differentiable at $a$, and $(g \circ f)'(a) = g'(f(a))f'(a)$.
\end{thrm}
\begin{thrm}[inverse function theorem]
	For any $k \in \N^+$, if $f:\R^n \to \R^n$ is $C^k$ and $f'(x)$ is invertible for some $x$, then $f$ is locally a diffeomorphism of class $C^k$ at $x$.
\end{thrm}
\begin{example}
	Counter\end{example}
\begin{example}
	Let $$f(x) =
	\begin{cases}
		\frac{x}{2} + x^2\sin\left(\frac 1x\right)	&\text{if $x \neq 0$} \\
		0											&\text{if $x=0$} \\
	\end{cases}
	$$  We see that $f$ is differentiable, but the derivative is not continuous (i haven't verified this).  It turns out that $f$ is not one-to-one (and therefore not a bijection).
\end{example}
\begin{thrm}[local submersion theorem]
	Synonym: implicit function theorem.
	Note: This theorem is equivalent to the inverse function theorem.

	function $f : \R^{k+m} \to \R^m$ is $C^1$.
	Let $x \in \R^{k+m}$ and $z \in \R^m$.
	Suppose $f'(x)$ has rank $m$.
	Then there exists $W$ nbd $x$ and $V$ nbd $z$ and $U$ open in $\R^k$ and a diffeomorphism $I : W \to U \times V$
	s.t. $f = I.\pi$ where $\pi$ is the projection from $U \times V$ onto $V$.
\end{thrm}
\begin{thrm}[local immersion theorem]
	This is very similar to the local submersion theorem (more soon...)

	function $f : \R^{m} \to \R^{m+k}$ is $C^1$.
	Let $x \in \R^{m}$ and $z \in \R^{m+k}$.
	Suppose $f'(x)$ has rank $m$.
	Then there exists $V$ nbd $x$ and $W$ nbd $z$ and $U$ open in $\R^k$ and a diffeomorphism $I : U \times V \to W$
	s.t. $f = in.I$ where $in$ is the inclusion from $V$ to $U \times V$.
\end{thrm}
\begin{thrm}
	If $f$ is locally diffeomorphic at $x$, then the slope matrix of $f^{-1}$ at $f(x)$ is given by $(f^{-1})'(f(x)) = (f'(a))^{-1}$.
	Proof. $f$ is locally diffeomorphic at $x$, so $f^{-1}(f(x)) = x$.
	Apply the chain rule: $(f^{-1})'(f(x)) \cdot f'(x) = 1$.
	Finally: $(f^{-1})'(f(x)) = (f'(x))^{-1}$.
\end{thrm}
\begin{thrm}
	A function $f : \R^a \to \R^b$ is continuous iff each of its coordinate functions $f_i$, $1\leq i \leq b$, are continuous.
\end{thrm}

\begin{defi}
	$X$ is \de{connected} if whenever $X = A \cup B$ ($A$ and $B$ are both open and nonempty), then $A \cap B$ is nonempty.
\end{defi}

\begin{defi}
	A space is \de{second-countable} if it has a countable basis.  Synonym: \de{2nd-countable}.
\end{defi}
\begin{defi}
	A space is \de{seperable} is it has a countable dense subset.
\end{defi}
\begin{thrm}
	Every 2nd-countable space is seperable.
\end{thrm}
\begin{defi}
	A space $M$ is an $n$-dim top \de{manifold} if $M$ is Hausdorff and 2nd-countable, and $M$ has an open cover $\{U_i\}_{i \in I}$ such that each $U_i$ is homeomorphic (via $\phi_i$, which is itself called a \de{chart}) to an open set in $\R^n$.  Then $\{ U_i, \phi_i \}$ is called an \de{atlas} for $M$.
\end{defi}
\begin{thrm}
	The union of any amount of atlases is itself an atlas.
\end{thrm}

\begin{defi}
	An atlas for $M$ is \de{smooth} if every $\phi_\alpha^{-1}.\phi_\beta$ is $C^\infty$ both ways.
\end{defi}

\begin{defi}
	Two smooth atlases $A$ and $A'$ are \de{equivalent} if for each $\alpha$, $\beta$, $\phi_\alpha^{-1}.\phi_\beta'$ is $C^\infty$ both ways.
\end{defi}
\begin{rmk}
	It can be verified that the above definition really is an equivalence relation.
\end{rmk}

\begin{defi}
	A \de{smooth} manifold (``smooth structure'') is just a manifold coupled with an equivalence class of smooth atlases.
\end{defi}

\begin{thrm}
	There exist 4-dim manifolds that can't be smooth.
\end{thrm}



\begin{example}
	$M = \R$ and $A = \{ id: \R \to \R \}$.
\end{example}

\begin{example}
	$M = S^1$.  $A = {U_l, U_r, U_t, U_b}$.  For $U_t$, for example, we choose $\phi_t$ to be a projection of the semicircle with open endpoints to a line segment with open endpoints.  Do the same for the others.  Looking at $\phi^{-1}$ of $\phi$'s will reveal that their composition is $C^\infty$, so we conclude that $A$ is smooth.
\end{example}

\begin{defi}
	An \de{n-}manifold or \de{n-dimensional} manifold is a manifold that is locally homeomorphic to $\R^n$.  (``n-'' is the adjective to ``manifold''.)
\end{defi}


\begin{defi}
	Given $M$ smooth manifold w/ atlas $A = \{ U_a, \,  \phi_a : U_a \to \R^m \}$.
	Given $N$ smooth manifold w/ atlas $B = \{ O_b, \,  \phi_b : U_b \to \R^n \}$.
	A cont function $f : M \to N$ is \de{smooth} if for all $a$, for all $b$, $\phi_a^{-1}.f.\psi_b$ is smooth.  You might call $f$ a \de{mersion}.
\end{defi}
\begin{defi}
	Given the same givens as above.
	$N$ and $M$ are \de{diffeomorphic} if there exists $f : M \to N$ and $g : N \to M$ both smooth such that $f.g = id_M$ and $g.f = id_N$.
\end{defi}
\begin{defi}[Matt's version]
	$N$ and $M$ are \de{diffeomorphic} if they are bimersive.  (That is, there exists a bimersion between them.)
\end{defi}

\begin{rmk}
	Some manifolds have NO smooth atlases.

	Some manifolds have 2 (or more) smooth atlases that are not equivalent to each other.
\end{rmk}


\begin{thrm}
	$\R^4$ has an uncountable number of smooth structures (equivalence classes of smooth atlases).  (But all other $\R^n$ have a unique smooth structure.)
\end{thrm}

\begin{rmk}
	If $(M,A)$ is a smooth $n$-manifold, and $U$ an open set in $M$, then $(U,A_U)$ is a smooth $n$-manifold, where $A_U$ is like $A$, but intersecting each set with $U$, and restricting the $\phi$ functions accordingly.
\end{rmk}

\begin{defi}
	Given $M$ smooth $m$-manifold with atlas $A = \{ U_a, \,\phi_a : U_a \to \R^m \}$.
	Given $N$ smooth $n$-manifold with atlas $B = \{ V_b, \,\psi_b : V_b \to \R^n \}$.
	$m \leq n$.
	A smooth function $f : M \to N$  is an \de{immersion} if given any $a, b$ such that $\phi_a: U_a \to \R^m$, $\psi_b: V_b \to \R^n$, and any $x \in \phi_a(U_a \cap V_b.f^{-1})$, then $x.(\phi_a^{-1}.f.\psi_b)'$ has rank $m$.
\end{defi}

\begin{defi}
	$M$ is a \de{submanifold} of $N$ if there exists an immersion from $M$ to $N$.
\end{defi}

\begin{thrm}
	$\R P^n \isom \R P^{n-1} \cup \R^n$.
	\textbf{Proof.}  In $\R^{n+1}$, draw a flat ``plane'' above the origin.  Each line in $\R P^n$ that has a nonzero ``z'' component will intersect this plane.  So each line maps to its intersection with this plane, and this plane is $\R^n$.  The remaining lines are all in $\R P^{n-1}$.  Wow!
\end{thrm}



If $f$ is a smooth map of manifolds, then we can differentiate, getting the linear map $d_x(\phi_a^{-1}.f.\psi_b): \R^m \to \R^n$.

\begin{thrm}
	$f$ is a smooth map of manifolds.
	$f$ is an \de{immersion} if for all $x$, $d_x$ above is injective.  (Note: this is *local* injectivity of $f$) ($m \leq n$)
\end{thrm}

\begin{defi}
	$f$ is a smooth map of manifolds.
	$f$ is a \de{submersion} if for all $x$, $d_x$ above is surjective.  (Note: this is *local* surjectivity of $f$) ($m \geq n$)
\end{defi}

\begin{defi}
	An \de{embedding} is a smooth map of manifolds $f: M \to N$ s.t. $f : M \to f(M)$ is a homeomorphism.

	Note: $f$ need not be onto $N$.
\end{defi}
\begin{thrm}[minimal criteria for an embedding]
	$f$ is an \de{embedding} iff $f$ is a smooth map of manifolds that is injective and proper.
\end{thrm}
\begin{prop}
	Every embedding is an immersion.
\end{prop}

\begin{thrm}
	If $f$ from $M^m$ to $N^n$ is a submersion, and $f(x) = y$, then $f^{-1}(y)$ is an $(m-n)$ dimensional manifold.
\end{thrm}

\begin{defi}
	$f$ is a smooth map of manifolds.
	$\phi_a(x) \in M$ is a \de{regular} point if $d_x(\phi^{-1}.f.\psi)$ is onto.  Otherwise, $\phi_a(x)$ is a \de{critical} point.
\end{defi}

\begin{defi}
	$f$ is a smooth map of manifolds.
	$y \in N$ is a \de{regular} value if every point in $f^{-1}(y)$ is regular.  Otherwise, $y$ is a \de{critical} value.  Note: values $y$ *not* in the image of $f$ are vacuously regular.
\end{defi}

\begin{thrm}[Preimage theorem]
	If $y \in N$ is a regular value of a smooth map of manifolds $f$, then $f^{-1}(y)$ is a submanifold of $M$.  If $y \in f(M)$, then $f^{-1}(y)$ has dimension $m-n$.
\end{thrm}

\begin{example}
	$M_n(\R)$ vectors of dim $n^2$.  $S_n(\R)$ the symmetric matrices.  Note that the matrix $A^T A$ is always symmetric.  $O(n)$ group ortho matrices.

	Let $f$ from $M_n(\R)$ to $S_n(\R)$ given by $f(A) = A^T A$.

	THEN $I_n$ in $S_n(\R)$ is a regular value.  Here's why...

	Corr. $O(n) = f^{-1}(I_n)$ is a smooth manifold of dimension $n^2 - \frac{n(n+1)}{2}$ = $\binom n2$.

	and

	$O(n) = SO(n)$ disjoiunt union $SO(n)Diagmatrix(-1 1 1 1 1 1 1 1 1)$

	proof of claim

	Given $A$ in $M_n$, find a formula $df_A$ from $M_n$ to $S_n$ ...

	$df_A(B) = \lim \frac{f(A+tB) - f(A)}{t} = \ldots = B^T A + A^T B$
	is indeed in $S_n$! (it is symmetric)

	If $A^T A = I_n$, can every symmetric matric $C$ be written in the form $C = B^T A + A^T B$.

	sneaky trick
	Guess $A^T B = \frac12 C$.
	This is possible if we let $B = \frac12 AC$.
	Then it works!
\end{example}

\begin{thrm}[Matt's]
	If $M$ and $N$ are diffeomorphic manifolds, then $M$ and $N$ are smooth.
\end{thrm}


\mydate{d1}{21}{9}{2016}

Given $f:M \to N$ smooth, then there exists a $df$ (but we could call it anything) s.t. $df : TM \to TN$ and $df.P_N = P_M.f$.  (Where $P_M$ is the projection from $TM$ to $M$)  (But is this magic df the same thing as our $df$ that we already know well?)

\begin{example}
	If $M = \R^2$, then for any $x$, the tangent vector space is $\R^2$ again.  (The flat plane in $\R^2$ at the point $x$ is $\R^2$).  Therefore, $T\R^2 = \R^2 \times \R^2 = \R^2$.  The same goes for $M = \R^m$, replacing all $2$'s with $m$'s.
\end{example}

We can take the disjoint union of these distinct open sets, then wherever we WANT these open sets to overlap, we define those pairs of points to be equivalent.  Therefore, the disjoint union, when viewed as a partition via the equivalence relation, recovers the manifold that we wanted.

Let's take this a step further, using the tangent bundle too:

disjoint union ( $U_i \times \R^n$ ) has elements $(i, x \in U_i, v \in \R^n) = (i, x, v)$.
Then we define the equivalence relation as follows:
$(i, x, v) ~ (j, y, w)$ iff $x=y$ and $v.d_x(\phi_i^{-1}.\phi_j) = w$.

Note that the above $d_x$ is linear in $v$, so ~ is linear in $v$ too.

So in the partition, the $v$'s form a vector space too!  So now we have a single vector space for all of the tangent bundles combined!  Maybe.

yay!

\begin{lem}
	$T_A M$ has an atlas showing its a $2m$-dimensional manifold given
	$T_A M \supset U_i x \R^n \to \R^n x \R^n$ (where the $\to$ function is ($\phi_i \times identity$))
\end{lem}
\begin{cor}
	Suppose $A \subset B$ are atlases for $M$.  The ``canonical'' map

	$T_A M = \sqcup_{i \in I_A} U \times \R^n/~ \to \sqcup (same but B) = T_B M$
				$\to										\to$
				$M 										M$
	is a diffeo of manifolds.
\end{cor}

Some $f: M \to N$ induces $Tf : TM \to TN$ as follows:
Given $\phi_i$ a chart for $M$ about $x$.
Given $\psi_j$ a chart for $N$ about $f(x)$.
THEN
$Tf((i, x, v)) = (j, f(x), v.d_x(\phi_i^{-1}.f.\psi_j))$.

And $T(M \times N)$ ``canonically'' identifies with $TM \times TN$.


\mydate{d2}{26}{9}{2016}



$df_x : T_xM \to T_yN$ is onto.

$y$ is a regular value of $f$.

Let $L = f^{-1}(y)$, a smooth ($m-n$) dimensional submanifold of $M$.

$T_xL = \ker df_x$

$x \in L \subset M$

\mydate{d3}{28}{9}{2016}

More elegant definition of regular:
\begin{defi}
	$f$ is a smooth map of manifolds.
	$x \in M$ is a \de{regular} point if $d_x : T_xM \to T_yN$ is onto.  Otherwise, $x$ is a \de{critical} point.
\end{defi}

\begin{defi}
	$f$ is a smooth map of manifolds.
	$y \in N$ is a \de{regular} value if every point in $f^{-1}(y)$ is regular.  Otherwise, $y$ is a \de{critical} value.  Note: values $y$ *not* in the image of $f$ are vacuously regular.
\end{defi}

\begin{thrm}
	$f$ is a smooth map of manifolds.
	If $y$ is a regular value, then $f^{-1}(y)$ is a smooth submanifold of $M$.
	Also, if $x \in f^{-1}(y)$, then $T_xf^{-1}(y) = \ker df_x$.
	Also, $Tf^{-1}(y) = \{ (x,v) | f(x)=y, df_x(v) = 0\}$.
\end{thrm}

\begin{thrm}
	$f$ is a smooth map of manifolds.
	Let $L$ be a subset of $N$.
	If $M$ is a submanifold of $N$, then $f^{-1}(y) = M \cap L$.
\end{thrm}

\begin{defi}
	$f$ is \de{transverse} to $L \subset N$, denoted $f \transverse L$, if $f$ is a smooth map of manifolds and whenever $f(x) = y \in L$, then $df_x(T_xM) + T_yL = T_yN$.  (yes, we are adding vector spaces).
\end{defi}

\begin{thrm}
	If $f$ is transverse to $L$, then $$\dim(M) + \dim(L) = \dim(N) + \dim f^{-1}(L)$$.
	If $M$ is transverse to $L$, then $$\dim(M) + \dim(L) = \dim(N) + \dim(M \cap L)$$.
	I think this may generalize to $$\dim(M) + \dim(L) = \dim(M \cup L) + \dim(M \cap L)$$, if the notion of taking the union of $M$ and $L$ corresponds to taking the direct sum of the tangent spaces of $M$ and $L$.

	Furthermore, if $x \in K$ and $y = f(x)$, then $T_x K = df^{-1}_x(T_y L) \subset T_x M$.
\end{thrm}

\begin{thrm}
	Given $f : U \to V$, and
	$g : V \to W$ is onto.

	Then ($f.g$ is onto iff $f(U) = \ker g$).
\end{thrm}

\begin{thrm}
	Transversailty holds iff $d(f.p)$ is onto.

	\textbf{Proof.}
	Suppose $y \in L$.
	By the local immersion thm,
	there exists $O$ open nbd of $y$, and a submersion $p : O \to \R^{n-l}$
	s.t. $p(y) = 0$ and $p^{-1}(0) = L \cap O$

	$df_x : T_xM \to T_yN$.
	$df_y : T_yN \to \R^{n-l}$.

	By prior thm, we conclude $\ker df_y = T_yL$.
\end{thrm}

\mydate{d4}{5}{10}{2016}

\begin{defi}
	$C^\infty(M) := \{ f:M \to \R | f is smooth \}$
\end{defi}
\begin{thrm}
	$C^\infty(M)$ is a commutative $\R$-algebra. (It's a ring, a V.S., has a bilinear prod)
\end{thrm}
\begin{thrm}
	If $U$ is a submanifold of $M$, then $C^\infty(U)$ is also an $\R$-algebra.
	(is this true?  i don't know!)
\end{thrm}
\begin{thrm}
	If $U$ and $V$ both open and $U \subset V \subset M$, then
	the restriction from $C(V)$ to $C(V)$ is a morphism of $\R$-algebras.
\end{thrm}
\begin{prop}[sheaf property]
	If $U$ = union of $A$
	and $f_i$ in $C(A_i)$ s.t. $f|_{A_i \cap A_j} = f_j|_{A_i \cap A_j}$ for all $i, j$
	then there exists a UNIQUE $f \in C(U)$ s.t. $f|_{A_i} = f_i$ for all $i$.
\end{prop}

a bunch of stuff i can't read

Fix $p \in M$.
Put an equivalence relation on all smooth functions $\{ f : U \to \R | \text{smooth}\}$
with $U$ is a neighborhood of $p$.

$f:U \to \R \sim g:V \to \R$
iff
there exists $W$ s.t. $p \in W \subset U \cap V$ where $f|_W = g|_W$.

Let $C^\infty(M,p)$ = the set of equiv classes.  This is called the \de{stalk} of the sheaf at point $p$.  Or the \de{germ} of a function about $p$.

direct limit
$= c\lim C^\infty(U)$.
for open nieghborhoods of $p$.

If $$\phi: M \to N$$ in smooth and $\phi(p) = q$, then we get an induced ring morhpism
$\phi^* : C^\infty(N,q) \to C^\infty(M,p)$ by
$\phi^*([f]) = [\phi.f]$

$\psi : L \to M$
$\phi : M \to N$

$(\psi.\phi)^* = \phi*.\psi* : C^\infty(n,q) \to C^\infty(L,r)$

Given $v \in T_pM$, want to define
$D_v : C^\infty(M,p) \to \R$
``directional derivative in the $v$-direction'' at $p$

Example for $M = \R_n$, $p \in \R_n$, and $v \in T_p\R_n = \R_n$.

$D_v : C^\infty(\R_n, p) \to \R$

$D_v([f]) = \lim \frac{f(p + tv) - f(p)}{t}$

properties
\begin{prop}
	\begin{enumerate}
		\item $D_{av + bw} = aD_{v} + bD_{w}$
		\item $D_v$ is linear
		\item $D_v(fg) = f*D(g) + D(f)*g$
	\end{enumerate}
\end{prop}
\begin{thrm}
	Suppose $D : C_infty(\R^n,p) \to \R$
	s.t. $D$ satisfies the properties above.  Then $D = D_v$ for some $v \in \R^n$.
\end{thrm}
proof not shown.
\begin{lem}
	Given $U \subset \R^m$ and $V \subset \R^n$ open.
	$\phi : U \to V$ is smooth.
	$p \in U$
	$q = \phi(p)$
	Let
	$w = d \phi_p(v)$ in $\R_n$

	Then
	$D_w = D_v of \phi^* : C^\infty(\R^n,q) \to \R$

	$M$ is an $n$-dimensional manifold, $p \in M$, $v \in T_pM$.
	want to define
	$D_v : C^\infty(M,p) \to \R$
	given $\phi: M \supseteq U \to \R_n$ let $D_v$ be the composit
	$C^\infty(M, p)$ to by ${\phi^{-1}}^*$ to $C^\infty(\R^n,q)$ to $D_w$ to $\R$.

	independent of coordinate chart, by \end{lem}
\begin{lem}
	\end{lem}
\begin{defi}
	A \de{derivation} is a function that is linear and satisfies the product rule.
\end{defi}
\begin{defi}
	$\Der(M,p)$ is the set of all such derivations $D$ of the form $D: C^\infty(M,p) \to \R$.
\end{defi}
\begin{thrm}
	The map $T_pM$ to $\Der(M,p)$
	that sends $v$ to $D_v$
	is an isomorphism of vector spaces.
\end{thrm}

\mydate{d5}{10}{10}{2016}

\begin{defi}
	Given a differentiable manifold $M$ and the canonical $p: TM \to M$, a \de{vector field} is a section of $p$.
\end{defi}
\begin{rmk}
	For the REST of the course, all vector fields will be SMOOTH.  So whenever you see ``vector field'', interpret it as ``smooth vector field''.
\end{rmk}

$M^n$ is a Manifold.
$v \in T_xM$ gives you a direction in which to take dir derivatives.
$D_v$ directional derivative in $v$-direction

% PUT def of 0 vec field v_0 here

\begin{example}
	The zero function is a vector field on $M$.
\end{example}
\begin{example}
	$M$ open in $\R^n$ is a smooth $n$-dim manifold.

	Recall that $TM \isom   M \times \R^n$.  Let $p:TM \to M$ and $\pi:M \times \R^n \to M$.

	The following map is a bijection:
	$\{\text{vector fields}\} \to \{\text{smooth functions}\}$
	$        X: M \to TM      \mapsto     X: M \to \R^n$
\end{example}
\begin{example}
	$C^\infty(M,x)$ is the set of germs about $x$.
	Each germ is an equiv class.
	The germs make up a V.S., that is, $C^\infty(M,x)$ is a V.S. over $\R$ (the operation is function multiplication).  Moreso, it is a commutative $\R$-algebra.

	Define $C^\infty(M, TM)$ as the set of vector fields on $M$.  (That is, the set of smooth $f: M \to TM$ with $f(x) \in T_xM$.)  We can add these vector fields together, and multiply them by constants in $\R$.  So we have another V.S.

	Let $F$ be one of these vector fields and $g \in C^\infty(M)$.
	Let $(gF)(x) = g(x)F(x)$.
	So $C^\infty(M)$ is a ring and $C^\infty(M, TM)$ is module over $C^\infty(M)$.  Or, more succinctly, ``$C^\infty(M, TM)$ is a $C^\infty(M)$-module''.

	Let $F_i : V to ((TV = Vx\R^n))$
	Let $Fi : V to \R^n$ be the constant vector field $Fi(x) = (((x,ei) or just ei))$ = the ith basis vector.

	then cool things happen that i missed :(
\end{example}
\begin{defi}
	A \de{derivation} of a commutative algebra $A$ is derivation of the form $D: A \to A$.

	$\Der(A)$ is the set of all such derivations.
\end{defi}
\begin{thrm}
	The set of derivations $\Der(A)$ of a commutative algebra $A$ is an $A$-module.  That is, $\Der(A)$ is a commutative additive group, and this is a module over $A$.  (Observe that if $c \in A$, then $(cD)(x) = cD(x)$).
\end{thrm}

\begin{prop}
	Ambiguous functions which I'll just call $g$.

	$g : \{\text{smooth vector fields on $M$}\} \to $\{\text{derivations on $C^\infty(M)$}\}

	$g : C^\infty(M,TM) \to \Der(C^\infty(M))$

	$X \mapsto D_X$

	This is an isomorphism of $C^\infty(M)$-modules!
\end{prop}


\begin{defi}
	The \de{Lie bracket} of $A$ and $B$, denoted $[A,B]$, is $AB - BA$.
\end{defi}
\begin{thrm}
	Derivations are closed under the Lie bracket.
\end{thrm}
\begin{proof}
	Let $D$ and $C$ be any two derivations of $A$.
	\begin{itemize}
		\item linear:
		\begin{align}
			& (DC - CD)(cx + y) \\
			=& (DC)(cx + y) - (CD)(cx + y) \\
			=& D(C(cx + y)) - C(D(cx + y)) \\
			=& D(cC(x) + C(y)) - C(cD(x) + D(y)) \\
			=& D(cC(x)) + D(C(y)) - C(cD(x)) - C(D(y))) \\
			=& cD(C(x)) + D(C(y)) - cC(D(x)) - C(D(y))) \\
			=& cD(C(x)) - cC(D(x))+  D(C(y)) - C(D(y))) \\
			=& c(DC-CD)(x)      +    (DC-CD)(y)
		\end{align}
		\item product rule:
		\begin{align}
			(DC)(ab) &= D(C(a))b + C(a)D(b) + D(a)D(b) + aD(C(b)) \\
			\text{look at reverse} \\
			(CD)(ab) &= C(D(a))b + D(a)C(b) + C(a)C(b) + aC(D(b)) \\
			\text{Then} \\
			(DC - CD)(ab) &= (DC - CD)(a)b + a(DC - CD)(b)
		\end{align}
	\end{itemize}
	Therefore,  $(DC - CD) \in \Der(A)$, so we conclude $[D,C] \in \Der(A)$.
\end{proof}



\begin{example}
	Vector fields are closed under the Lie bracket (because they are derivations).

	$X, Y \in C^\infty(M, TM)$ implies that $[X,Y] \in C^\infty(M, TM)$
\end{example}
\begin{proof}
	\begin{enumerate}
		\item
		First write $X$ and $Y$ as derivations.

		Let $X(x,y) = f(x,y)\frac{d}{dx}$

		Let $Y(x,y) = g(x,y)\frac{d}{dx}$

		(so $Y(h) = g(x,y)\frac{d}{dx}(h) = g(x,y)\frac{dh}{dx}$  (not $Y$ OF $h$, but $Y$ applied to $h$)

Then use the above theorem, or repeat the work of the above theorem as follows:

		Then $X(Y(h)) = f \frac{dg}{dx} \frac{dh}{dy} + fg \frac{d^2h}{dxdy}$
		Then $Y(X(h)) = g \frac{df}{dx} \frac{dh}{dy} + gf \frac{d^2h}{dxdy}$

		Then $(XY - YX)(h)$ = compute
		$[X,Y](h) = f \frac{dg}{dx} \frac{dh}{dy} - g \frac{df}{dx} \frac{dh}{dy}$
		in general:
		$[X,Y] = f \frac{dg}{dx} \frac{d}{dy} - g \frac{df}{dx} \frac{d}{dy}$
	\end{enumerate}
\end{proof}

\mydate{d6}{12}{10}{2016}

Given $M$, does there exists a smooth vector field $X$ on $M$ s.t. $\forall x \in M$, $X(x) \neq 0$?

\begin{example}
	$M = S^1$.  $TS^1 = \{ (z,v) \mid z \in S^1, z \cdot v = 0 \}$
	yup!  There exists
	$X(z) = X((x,y)) = (z, (-y,x))$ works!
\end{example}
\begin{example}
	$M = S^2$.
	no! (the billiard ball)
\end{example}
\begin{example}
	$M = S^3 \subseteq \C \x \C$

	$S^3 = \{ (z_1,z_2) \mid |z_1|^2 + |z_2|^2 = 1 \}$

	yes!

	$X(z) = (z, iz)$ works!
\end{example}
\begin{example}
	$M = S^{2n-1}$
	yes!
	use the same trick!
\end{example}
Question 2:
Given $M$, what is the maximum number $k$ of vector fields $X_1,,,X_k$ that for every $x \in M$, $\{X_1(x),,,X_k(x)\}$ is linearly independent?

$k = n = \dim M$ is best possible

\begin{example}
	$M \subseteq S^3 \subseteq \H$ (has 4 dim)
	$q = x_1 + yi + zj + wk$
	Let
	$X_1(q) = (q, iq)$
	$X_2(q) = (q, jq)$
	$X_3(q) = (q, kq)$
	yes!
\end{example}
\begin{thrm}[J.F. Adams]
	Given $n \in \N$ and write $n = odd*2^{4d+c}$
	Let $p(n) = 2^c + 8d$
	Then $S^{n-1}$ has $p(n)-1$ linearly independent vector fields, but not $p(n)$.
\end{thrm}

consider we have $k = n$, best possible


\begin{defi}
	$M$ is \de{parallelizable} if there exists a diffeomorphism

	$$\phi : M \x \R^n \to TM$$
	$$(x, t_1,,,t_n) \mapsto (x, t_1X_1(x),,,t_nX_n(x))$$

	and $\phi.p$ commutes with $\pi$ in

	$$\begin{tikzcd}
		M \x \R^n \arrow[r, "\phi"] \arrow[rd, "\pi"'] & TM \arrow[d, "p"] \\
		 & M
	\end{tikzcd}$$

\end{defi}
\begin{thrm}
	$M$ is \de{parallelizable} iff the maximum number of linearly independent vector fields is $\dim M$.
\end{thrm}
\begin{defi}
	An $n$-dimensional manifold $M$ is \de{framed} if there exists $k$ and a vector bundle isomorphism $f$ s.t.

	$$\begin{tikzcd}
		TM \x \R^k \arrow[r, "f"] \arrow[rd, "p"'] & M \x \R^{n+k} \arrow[d, "q"] \\
				   & M
	\end{tikzcd}$$

	s.t. $f$ is smooth and each map $T_xM \x \R^k \to \R^{n+k}$ is a linear isomorphism. (that is, $f$ is a diffeomorphism)
	$f: p^{-1}(x) \to q^{-1}(x)$
\end{defi}
\begin{thrm}
	Every parallelizable manifold is framed. (I have not confirmed this)
\end{thrm}
\begin{example}
	$S^2$ is framed with $k=1$ by

	$$f : TS^2 \x \R \to S^2_x \R^3$$
	$$((x,v), t) \mapsto (x, v + tx)$$

	$<x>$

	observe that
	$TS^2 \not\isom S^2 \x \R^2$
	but
	$TS^2 \x \R \isom S^2 \x \R^3$
	wow!

	or in general, we have
	$\R = C^\infty(S^2)$
	$M = C(S^2, TS2)$
	$M$ is an $\R$-module
	then
	$M \not\isom \R \oplus \R$, as $\R$-module
	but
	$M \oplus \R \isom \R \oplus \R \oplus \R$, as $\R$-module
	wow!
\end{example}
\begin{example}
	$\R P^2$ not framed.
	$\C P^2$ not framed.
\end{example}

$df_x \in (T_xM)^*$, called the \de{dual vector space} of $T_xM$
$dg_x \in (T_xM)^*$ too
set of above over all smooth functions

\begin{defi}
	The \de{cotangent bundle}, denoted $T*M$, is
	$\bigcup_{x \in M} (T_xM)*$.


	observe
	$TM = \sqcup (U_i x \R^n)/~$
	where $$(i, x, v) ~ (j, x, d(\phi_i of \phi_i inverse)_x(v))$$
	analagously,
	$T*M = \sqcup (U_i x (\R^n)*)/~$
	where $$(j, x, w) ~ (i, x, d(\phi_j of \phi_j inverse)_x(w))$$
\end{defi}
\begin{defi}
	A \de{1-form} on $M$ is a section of $T*M \to M$.

	in other words, for all $x \in M$, $w(x): T_xM \to \R$ and this varies ``smoothly'' in $x$
\end{defi}
	notation
	$\omega'(M) = C^\infty(M, T*M)$ = set of all 1-forms.

\begin{example}
	$U \subseteq \R^n$
	A general vector field on $U$ had the form
	$f_1 \frac{d}{dx}1 +++ fn \frac{d}{dx}n$
	where $fi : U to \R^n$
	then
	A general 1-form on $U$ has the form
	$f1 dx1 +++ fn dxn$
	(observe that $dx1$ is dual to $\frac{d}{dx}1$, etc)
\end{example}

\mydate{d7}{17}{10}{2016}

\begin{thrm}
	$V$, $W$, $U$ are V.S. over $F$.  Every bilinear $b:V \x W \to U$ corresponds to a linear map $b` : V \to Lin(W,U)$
	such that $b`(v)(w) = b(v,w)$.
\end{thrm}
	In particular $<,> : V \x V \to \R$
	corresponds to a linear map
	$L: V \to V$*
	$L(v)(w) = <v,w>$

	If you have the POSITIVE DEFINITE property TOO, then $L$ is an isomorphism.



\begin{defi}
	$V$ is a finite dim real vector space.  Then an \de{inner product} is a function $<,> : V \x V \to \R$ that is:
	\begin{enumerate}
		\item bilinear (note this implies $<0,0> = 0$)
		\item symmetric
		\item positive definite (if $v \neq 0$, $<v,v> > 0$)
	\end{enumerate}
\end{defi}
\begin{example}
	The dot product is an inner product.
\end{example}
\begin{thrm}
	Any linear product on $\R^n$ has the form $<v,w> = v^T A w$, where $A$ is a symmetric matrix with positive eigenvalues.
\end{thrm}
\begin{defi}
	A \de{Riemannian structure} in $M$ is an inner product $<,>_x \in T_xM$ for each $x \in M$ that is ``smooth'' in the following (equivalent) ways:
	\begin{itemize}
		\item
		Given any two smooth Vector Fields $X$, $Y$ on $M$, then
		the function $M \to \R$
		$x \mapsto <X(x), Y(x)>$
		is smooth.
		\item
		The corresponding map $L$ in (below) is smooth (and an isomorphism)
		$TM ----L\to T*M$
		$TM \to M$
		$T*M \to M$
	\end{itemize}
\end{defi}
\begin{example}
	Suppose $M^n$ is embedded in $\R^N$
	Then, given $x \in M$, $T_xM$ is a sub-vector space at $\R^N$
	Restriction of the ``standard'' inner prod (dot prod) on $\R^N \to T_xM$ gives a Riemannian structure on $M$.
\end{example}
\begin{defi}
	A \de{Riemannian Manifold} is a smooth manifold paired with a Riemannian structure.
\end{defi}
\begin{example}
	Given $\alpha : [0,1] \to M$ a smooth path
	Given $\alpha' \in T_\alpha(t)M$
	Then we can define the ``length'' of a path as $\int_0^1 <\alpha'(t), \alpha'(t)>^{1/2} dt$
\end{example}
\begin{example}
	We can *define* a ``gradient'' vector field of $f$
	$\grad f(x) \in T_xM$ to be the vector such that
	$< \grad f (x), v > = df_x$ for all $v \in T_xM$
\end{example}
\begin{thrm}
		Every smooth manifold $M^n$ can be embedded in $\R^N$ for some $N \in \N$.
\end{thrm}
\begin{proof}
	We will prove the lesser theorem ``A compact smooth manifold $M^n$ embeds in $\R^N$ for $N$ large enough.'':
	$M$ compact implies there exists a finite atlas
	ALL aPHIS BELOW are actually $\phi_i$, for any $i$.  AS WELL AS aLAMBDAS AND aTHETAS AND US AND VS
	suppose $\phi$ extended to all of $M$, is

	there exists $\\overset{~}{\phi} M \to \R^n$
	s.t. $\\overset{~}{\phi}|_U = \phi$

	Then consider
	$(\overset{~}{\phi_1},,,\overset{~}{\phi_r}) : M \to \R^n \x \x \x \R^n$ (r times)

	There exists open $V \subseteq U$ with $\bar{V} = V_c \subseteq U$, and $M = \bigcup V_i$ is finite (a property of paracompact spaces).

	there exists smooth $\lambda : M \to [0,1]$
	which is 1 on $V_i$, and has support $\subseteq U$ (this is possible since $V \in U$.  So 1 on $V$ and then dips down to 0 on or before bdry of $U$)
	$\supp (\lambda)$ is the points in the domain that don't map to 0.

	Let $\psi : M \to \R^n$ be $$ \psi(x) =
	\begin{cases}
		\lambda(x)\phi(x)	&\text{if $x \in U$} \\
		0					&\text{if $x \not\in U$} \\
	\end{cases}
	$$
	(or just $\lambda(x)\overset{~}{\phi}(x)$ everywhere, but $\overset{~}{\phi}$ was only in our minds)

	$\theta: M \to (\R^{n+1})^r$ be
	$\theta() = (\psi_1(x), \lambda_1(x),,,, \psi_r(x), \lambda_r(x))$

	\begin{enumerate}
		\item[(a)]
		we see that theta is smooth:

		yes
		\item[(b)]
		we see that theta is an immersion:

		For some $x \in V = V_i$

		$M \to (\R^n x \R)^r$ ----ith projection$\to (\R^n x \R) \to \R^n$
		$x \mapsto                                                      \psi(x)$

		If $d\psi : T_xM \to \R^n$ is monic, then
		$d\theta_x$ :
		\item[(c)]
		$\theta$ is 1-1:

		$\theta(x) = \theta(y)$ implies that for all i, $\lambda(x) = \lambda(y)$
		If $x \in V$, then $\lambda(x) = 1$, so $\lambda(y) = 1$, so $y \in U$.
		Then $\phi(x) = \psi(x) = \psi(y) = \phi(y)$
		$\phi$ is injective, so $x = y$.
		wohoo!
	\end{enumerate}

	$M$ is compact, together with a,b,c, proves the theorem!
\end{proof}


\begin{thrm}[Whitney Embedding Thm]
	$N = 2n$ suffices (see above)
\end{thrm}
\begin{thrm}
	Every smooth manifold has Riemannian structures.
\end{thrm}
\begin{thrm}[Nash Embedding Theorem]
	Every Riemannian structure on $M$ arises in this way
	(we will not prove this)
\end{thrm}

\mydate{d8}{19}{10}{2016}

\section{Normal and Paracompact spaces, and partitions of unity}

\begin{defi}
	$X$ is \de{normal} if for all $A$,$B$ closed and disjoint, there exists $U$,$V$ open, disjoint with $A \subseteq U$ and $B \subseteq V$.  (The "normal" condition is also called "Axiom $T_4$".  But this is *not* the same as being a $T_4$ space.  A $T_4$ space is, by definition, $T_2$ $+$ Axiom $T_4$.)
\end{defi}
\begin{rmk}
	($T_2$ + compact) implies $T_4$
	(but we prove even better, that ($T_2$ + paracompact) implies $T_4$)
\end{rmk}
\begin{thrm}
	All paracompact manifolds are normal.
\end{thrm}

\begin{lem}[equiv def of $T_4$]
	$X$ is normal iff (Given $A \subseteq W \subseteq X$.  $A$ closed and $W$ open implies there exists $U$ open with $A \subseteq U \subseteq \bar{closure} \subseteq W$)
\end{lem}
\begin{thrm}[Urysohn's Lemma (Bredon section I.10)]
	If $X$ is $T_4$ and $A$,$B$ closed, disjoint, then there exists a continuous $f : X \to [0,1]$ with $f|_A = 0$ and $f|_B = 1$.
\end{thrm}

\begin{thrm}[Tietze Extension Theorem]
	$X$ is $T_4$. A $closed \in X$.  If $f : A to \R$ is continuous, then $f$ extends to all of $X$.
\end{thrm}


\begin{prop}
	If $X$ is $T_4$ and $\{U_1,,,U_r\}$ is a finite open cover of $X$, then for each $i$, there exists $V_i \subseteq \bar{V_i} \subseteq U_i$ such that $\{V_1,,,V_r\}$ is a cover of $X$.
\end{prop}
\begin{proof}
	$U_1 \supseteq (X \ (U_2 unionthrough U_r)) = C$, which is closed.
	By lemma, $U_1 \subseteq V_1 \subseteq \bar{V_1} \subseteq C$.
	So $\{V_1, U_2,,,,U_r \}$ is an open cover of $X$.
	Repeat for all the other $U_s$.
\end{proof}

\section{Paracompactness}[(Bredon, section I.12)]

\begin{defi}
	An open cover $O = \{U_a\}$ (this is a family of $U$'s) of $X$ is \de{locally finite} if given any $x$, the set $\{U_a | x \in U_a\}$ is finite.  Or equivalently, any infinite intersection of $U_a$'s is empty.
\end{defi}
\begin{defi}
	Given $X$, an open cover $\{V_b\}$ refines an open cover $\{U_a\}$ if every $V_b$ is contained in some $U_a$.
\end{defi}
\begin{defi}
	A $T_2$ space $X$ is \de{paracompact} if every open cover has a locally finite refinement.
\end{defi}
\begin{thrm}
	$X$ is $T_2$ and compact implies $X$ is paracompact.
\end{thrm}
\begin{prop}[Bredon, Thm I.12.5]
	Every paracompact space is normal.
\end{prop}


\begin{thrm}
	$X$ locally compact, $T_2$, and 2nd-countable implies $X$ is paracompact.
\end{thrm}
\begin{cor}
	All manifolds are paracompact.
\end{cor}

\begin{prop}[Bredon, prop I.12.9]
	If $X$ is paracompact, and $\{U\}$ is a locally finite open cover, then there exists open sets $V_a$ with $V_a \subseteq \bar{V_a} \subseteq U_a$ with $\{V_a\}$ is an open cover.
\end{prop}
\begin{defi}
	Given $\{U_a\}$ a locally finite open cover of $X$, then a \de{partition of unity} subordinate to $\{U_a\}$ is a collection of continuous functions $\phi_a : X to [0,1]$ where for each $a$, the support of $\phi_a$ is contained in $U_a$.
\end{defi}

\begin{thrm}[Bredon, thm II.10.1]
	Any smooth manifold has smooth partitions of unity for any locally finite open cover.
\end{thrm}
\begin{thrm}[Bredon, thm I.12.8]
	If $X$ is paracompact and $\{U_a\}$ is any locally finite open cover of $X$, there exists a partition of unity subordinate to $\{U_a\}$.
\end{thrm}
\begin{prop}
	If $X$ is $T_4$ and $\{U_1,,,U_r\}$ is a finite open cover of $X$, then there exists a partition of unity subordinate to $\{U_i\}$.
\end{prop}
\begin{proof}
	For each $U$, there exists open $W$, $V$ s.t. $W \subseteq \bar{W} \subseteq V \subseteq \bar{V} \subseteq U$, and $\{W_1,,,W_r\}$ still covers $X$.

	For each $U$, There exists continuous $\lambda : X \to [0,1]$ with $\lambda|_W = 1$ and $\lambda|_{X\minus V} = 0$.

	then support $\supp(\lambda) \subseteq \bar{V} \subseteq U$.

	Let $\lambda = \sum_{i=1}^{r} \lambda_i$, so $\lambda : X \to \R>=0$.

	(we want all the $\phi_i$'s to add up to 1.  not sure if this is necessary)
	Let $\phi_i = \frac{\lambda_i}{\lambda}$.  Then $\{\phi_i\}$ is a partition of unity subordinate to $\{U_i\}$.
\end{proof}


\begin{defi}
	A set $S \subseteq \R^n$ has \de{measure 0} if given any $\epsilon > 0$, there exist rectangular $n$-dimensional boxes $\{S_j\}_{j\in J}$, where $J$ is countable, s.t. $S \subseteq \bigcup_{j \in J} S_j$ and $\sum_{j \in J} \vol(S_j) < \epsilon$.
\end{defi}
\begin{defi}
	Given a set $S \subseteq \R^n$, we say that a property is true for \de{almost all} $s \in S$ if the subset $\{t \in S \mid \text{property is false for }t\}$ has measure 0.
\end{defi}
\begin{lem}
	If $f$ is a smooth function from an open subset of $\R^n$ to $\R^n$, then $f$ is measure 0.
\end{lem}
\begin{proof}
	Step 1

	$f : U \to \R^n$.

	On any compact $K \subseteq U$, the partial derivatives $\frac{df_i}{df_j}$ are bounded, as well as their magnitudes, $|\frac{df_i}{df_j}|$.

	By the Mean Value Theorem, there exists $M \in \R$ such that, for all $x,y \in K$, $|f(x) - f(y)| < M|x-y|$.

	There exists $N \in \R$ s.t. for all boxes $S_j \subseteq K$,  $f(S_j) \subseteq S_j'$ where $S_j'$ is a box with $\vol(S') < N \vol(S)$.

	Step 2

	Take an open cover $\{U_i\}$ of $U$ such that $U_i \subseteq \bar{U_i} \subseteq U$, and each $\bar{U_i}$ is compact.

	Let $S_i := S \cap U_i$

	Then $\bar{S_i} \subseteq \bar{U_i}$, so $\{S_i\}$ is compact
	$\bar{S_i}$.

	apply step 1 to $U = W_i$, $k = S_i$,

	there exists $N$

	$\mu(S) = 0$ implies $\mu(S_i) = 0$.

	Given $\epsilon > 0$, cover each $S_i$ by $S_{i_k}$ s.t. $\sum_{k} \vol(S_{i_k}) < \frac{\epsilon}{N\cdot2^i}$.

	Then $f(S_i)$ is covered by $S'_{i_k}$ s.t. $\sum_k \vol(S'_{i_k}) < \frac{\epsilon}{2^i}$.

	And so $f(S)$ is covered by $\bigcup_{i,k} S'_{i_k}$ s.t. $\sum_{i,k} \vol(S'_{i_k}) < \epsilon$.
\end{proof}
\begin{cor}
	Given any subset $S$ of an $n$-dim smooth manifold $M$, then TFAE:
	\begin{itemize}
		\item There exists an atlas $\{U\}$ for $M$ s.t. $\forall i$, $\mu(\phi_i(S \cap U_i)) = 0$.
		\item For any (smooth) chart $\phi:U \to \R^n$, $\mu(\phi(S \cap U)) = 0$.
	\end{itemize}
	In this case, we say $S$ has \de{measure 0} in $M$, and write $\mu(S) = 0$.
\end{cor}
\begin{thrm}["Mini Sards Theorem"]
	If $f : M^m \to N^n$ is a smooth map of manifolds and $m < n$, then $f(M)$ has measure $0$.
\end{thrm}
\begin{proof}
% 		Step 1
% 		Reduce to case M = U open in \R^m, and N = \R^n:

% 		{U} is an atlas for M.

% 		If, for all i, \mu(f(U_i)) = 0, then \mu(f(M)) = 0

% 		Let V_i = \phi_i(U_i) open in \R^n

% 		' inverse
% 		f(U_i) = (V_i.\phi'.f)  where \R^m supset V_i ----\phi'.f\toN^n

% 		suffices to suppose M in \R^m

% 	Step 2
% 		\R^m supset U ----f\to \R^n
% 		To show f(U) has measure 0 in \R^n

% 		U x \R^n-m ----F---- \R^n
% 		^
% 		-
% 		-
% 		Ux{0} ------f------\R^n (above)

% 		Let F(x,y) = f(x) + y

% 		F|_{U x \{0\}} = f

% 		U x {0} has measure $0 \in U$ x \R^n-m

% 		So by lemma, \mu(F(U)) = \mu(F(U x {0})) = 0



% Now let's look at when m >= n.
% maybe understand just for m = n+1...
\end{proof}
\begin{proof}
	$C = \{ x \in \R^m \mid df_x \text{ is not onto} \}$

	Consider the function $df_x$.  Using THAT as our function...

	$C_1 = \{zeros of order 1\}$

	$C_2 = \{zeros of order 2\}$

	$C \supseteq C_1 \supseteq C_2 \supseteq \supseteq \supseteq$

	lemma 1
		If $k > m/n = 1$, then $\mu(f(Ck)) = 0$.
		(The proof is similar to Mean Value Theorem lemma + related estimates)
	lemma 2
		$\mu(f(C-C1)) = 0$.
		(hint, use Fubini for lemma 2 and lemma 3)
	lemma 3
		$\mu(f(Ck-C{k+1})) = 0$.
\end{proof}
\begin{thrm}[Sard's Theorem]
	If $f$ is a smooth map of manifolds, then the set of critical values of $f$ has measure $0$.
\end{thrm}
\begin{cor}
	The regular values are dense in $N$.
\end{cor}
\begin{cor}
	If $m < n$, then $\{\text{critical values}\} = f(M)$, and $f(M)$ has measure 0.  In particular, $f$ cannot be onto.
\end{cor}
\begin{example}
	Let
	$$f : \R to S^1 \x S^1$$
	$$f(t) = (t, \sqrt(2)t) \mod Z^2$$
	and
	$$S^1 \x S^1 = \R/\Z \x \R/\Z = \R^2/\Z^2$$
	Then
	$f(\R) = \{\text{critical values}\}$ is dense in $S^1 \x S^1$.
\end{example}
\begin{example}
	There exists continuous $f : \R \to \R \x \R$ which are onto (space filling curves).
\end{example}

\mydate{d9}{26}{10}{2016}

\begin{thm}[Whitney Embedding thm]
	Every $n$-dim smooth manifold embeds in $\R^{2n}$ and immerses in $\R{2n-1}$.
\end{thm}
\begin{proof}
% 	We will only prove ``Every $n$-dim smooth manifold embeds in $\R^{2n+1}$ and immerses in $\R{2n}$
% 	instead''...

% 	Assume $M$ embedded in $\R^N$ for $N>>0$.

% 	Start with 1-1 immersion $f : M^n \to \R^N$

% 	$\pi_a$

% 	claim A:
% 	"If N > 2n + 1, then there exists nonzero a in RN s.t.
% 		1. f: Mn to RN \pi_a <a>perp is still an immersion.
% 		2. f.\pi_a is still 1-1.
% 	If N > 2N, then can still do (1)."


% 		Claim implies there exists 1-1 immersion f : M^n to \R{2n+1}

% 		If M is compact, this would be an embedding.

% 		If f is proper, this would be an embedding.

% \end{proof}
% \begin{proof}[Proof of Claim A]
% 	Let g : TM to RN be g(x,v) = df_x(v)

% 	dfx : T_xM to TRN = RN
% 	Let h : M x M x \R to RN be
% 	h(x,y,t) = t[f(x) - f(y)]

% 	by Mini-Sards Thm, the Im(g) and Im(h) have measure 0
% 	so Im(g) \cup Im(h) has measure 0
% 	so there exists nonzero a s.t. a not in Im g union Im h.

% 	verify this a works:
% 	f.\pi_a is still an immersion?
% 	d_x(f.\pi_a) = d_xf.\pi_a
% 	Fact: 0 = d_x(f.\pi_a)(v) = \pi_a(df_x(v)) iff df_x(f(v)) in <a>
% 	Fact: for all v,w, a \neq df_x(w) iff <a> = {0}

% 	is f.\pi still 1-1?
% 	consider x \neq $y \in M$
% 	0 = x.f.\pi - y.f.\pi = (x.f - y.f).\pi
% 	iff
% 	x.f - y.f in <a>
% 	(f monic implies f(x) - f(y) neq 0)
% 	iff
% 	there exists t in \R s.t. t[f(x) - f(y)] = a
% 	iff
% 	h(x,y,t) = a
% \end{proof}
% \begin{lem}
% 	There exists a proper map p "(roe)" from M to \R.
% \end{lem}
% \begin{proof}
% 					Choose countable cover {U} of M with each Ubar compact.

% 				Let {p_i : M to [0,1]} partition of unity w.r.t. {U}
% 				Let p(x) = \sum_{i=1}^oo ip_i(x).
% 				Claim: this p is proper.
% 				Fix j
% 				Suppose $x \in M$ and and p(x) < j implies there exists i < j s.t. p_i(x) > 0 iff there exists i < j with x in U_i.
% 				Thus, p'([-j,j]) \subseteq union U_i \subseteq union Uibar, which is compact

% 		Assume the given f has |f(x)| < 1 (we can always shrink f down with arctan)

% 		Let F = (f,p) : M to \R{2n+1}

% 		again find a nice nonzero a as before, and a not in <e_{2n+2}> (last basis vec) (a 0 in last coord)
% 		claim
% 		F.\pi : M to <a>perp = \R{2n+1} is proper.

% 		x.F.\pi = (x.p, x.f).\pi
% 		= (f(x), 0).\pi + (0, p(x)).\pi

% 		ALL \PIS ARE \pi_a \subseteq THIS SECTION (INCLUDING ABOVE)

% 		|a.F.\pi| <= |(x.f, 0).\pi| + |(0, x.p).\pi|
% 				<= 1 + |(0, x.p).\pi|

% 		|x_k| to oo, |p(x)| to oo implies \pi(0, p(x)) to oo implies |x_k.f.\pi| to oo

\end{proof}
\begin{thrm}
	If $M$ and $L$ are not transverse, then you can wiggle $M$ a little in the right direction, and then $M+c$ and $L$ WILL be transverse.
\end{thrm}
\begin{lem}
	Given $M \x S$ instead of $M$, and $F$ instead of $f$ in above thm.  Suppose $F \transverse L$.  Then for almost all $s \in S$, the function $F_s: M \to N$ by $F_s(x) = F(x,s)$ is transverse to $L$.
\end{lem}
\begin{lem}
	% Given \R^n instead of N in setup.
	% Let F : M x \R^n to \R^n be F(x,c) = f(x) + c
\end{lem}
\begin{proof}
	% 	Fc = fc
	% F is "obviously" a submersion
	% T_(x,z)(M x \R^n ) dF_{x,c} \R^n
	% T_xM x \R^n to \R^n
	% by (v,w) \mapsto df(v,w)
	% ``````````````````````````
\end{proof}
\begin{defi}
	A property of a function $f : X \to Y$ is \de{stable} if for every homotopy $H : X \x [0,1] \to Y$, there exists an $\epsilon > 0$ such that for all $t < \epsilon$, $H_t : X \to Y$ has the property.
\end{defi}

\mydate{d0}{31}{10}{2016}
Happy Halloween!


\begin{thrm}[Transversality Lemma, or Guilleman and Pollock Trans Thm]
	$F: M \x S \to N \supseteq L$
	Given $s \in S$, let $F_s: M \to N$.
	If $F \transverse L$, then $F_s \transverse L$ for almost all $s$.
\end{thrm}
\begin{proof}
	Since $F \transverse L$, then $K = F^{-1}(L)$ is a submanifold of $M \x S$.

	$$\begin{tikzcd}
		K \arrow[r, "F"] \arrow[rd, "\iota"] \arrow[rdd, "p", bend right] & L \arrow[rd, "\iota"] \\
		 & M \x S \arrow[r, "F"] \arrow[d, "\text{projection}"] & N \\
		 & S
	\end{tikzcd}$$

	Claim: $s \in S$ is regular for $p$ iff $F_s \transverse L$.

	The claim follows from 2 steps:
	\begin{enumerate}
		\item $s$ is regular for $p$ iff $i_s: M \to M \x S$ is $i_s(x) = (x,s)$.
		\item $i_S \transverse K$ implies $F_s \transverse L$.
	\end{enumerate}

	Let's prove each step:
	\begin{enumerate}
		\item
			$s$ is regular for $K \into M \x S \overset{p_s}{\to} S$

			$\iff$

			Given $(x,s) \in K$, the composite $T_{(x,s)}K \subseteq T_xM \x T_sS \overset{project}{\to} T_sS$ is onto.

			$\iff$

			$T_{(x,s)}K + T_xM = T_xM \x T_sS$

			$\iff$

			$T_{(x,s)}K + T_{(x,s)}(M \x \{s\}) = T_{(x,s)}(M \x S)$

			$\iff$

			$i_s \transverse K$ and $d(i_s)_x(T_xM)$
		\item
			$i_s \transverse K$

			$\iff$

			For all $(x,s) \in K$, we have

			$T_{(x,s)}K + T_xM = T_{(x,s)}(M \x S)$

			(apply $dF_{(x,s)}$ to both sides)

			$\implies$

			$dF_{(x,s)}(T_{(x,s)}K) + dF_{(x,s)}(T_xM) = T_yN$

			$\implies$

			$T_yL + dF_{(x,s)}(T_xM) = T_yN$

			$\iff$

			$F_s \transverse L$
	\end{enumerate}
\end{proof}

\begin{example}
	$M, L \subseteq \R^n$

	$$F: M \x \R^n \to \R^n$$
	$$F(x,v) = x+v$$

	$F$ is ``clearly'' a submersion, so $F \transverse L$ for all $L \in \R^n$.  So for almost all $v$, $M+v$ is transverse to $L$.
\end{example}

\begin{example}
	Suppose given $\delta: M \to S^{n-1} \subseteq \R^n$, define

	$$\overset{~}{F}: M \x \R^n \to \R^n$$
	$$\overset{~}{F}(x,s) = f(x) + s$$

	so that

	$$\overset{~}{F}|_{M \x \{0\}} = f$$

	Let $B_{1/2}$ be an open ball of radius $1/2$.
	Consider $\overset{~}{F}|_{M \x B_{1/2}} = f$.

	$f(x) \in S^{n-1}$

	$s \in B_{1/2}$

	so $\overset{~}{F}$ restricts

	...




\begin{defi}
	$f$ and $g$ are \de{homotopic}, denoted $f \homotopic g$, if there exists a homotopy $H$ with $H_0 = f$ and $H_1 = g$.
\end{defi}
other things to notice
\begin{itemize}
	\item Since $B_{1/2}$ is path connected, each $F_s$ is homotopic to $f = F_0$.
\end{itemize}
\end{example}
\begin{thm}
	$M$ compact.
	If original $f$ is an embedding, so is $F_s$ for all $s$ small enough.
\end{thm}
\begin{defi}
	Two embeddings $f,g : M \to N$ are \de{isotopic} if they are homotopic via some $H$, and for all $t \in [0,1]$, $H_t$ is an embedding.
\end{defi}
\begin{thrm}[Homotopy Transversality Theorem]
	Given smooth $f: M \to N \supseteq L$, there exists smooth $g: M \to N$ such that:
	\begin{enumerate}
		\item $g$ is smoothly homotopic to $f$
		\item $g \transverse L$
		\item If $M$ has boundary, then $dg \transverse L$
	\end{enumerate}
	Furthermore, if $f$ is an embedding, then $g$ is isotopic to $f$.
\end{thrm}
\begin{proof}
	Prove assuming $N$ compact and $N \into \R^d$:

	Let $\overset{~}{F}: M \x \R^d \to \R^d$.

	be $\overset{~}{F}(x,s) = f(x) + s$

	$$\begin{tikzcd}
		M \x \{0\} \arrow[r, "f"], \arrow[d, hook] & N \arrow[d, hook] \\
		M \x \R^d \arrow[r, "\overset{~}{F}"] & \R^d
	\end{tikzcd}$$

	Let $N(\epsilon) = \{ y+s \mid y \in N, |s| < \epsilon \} \subseteq \R^d$

	$B_{\epsilon} = \{ s \in \R^d \mid |s| < \epsilon \}$

	If $\epsilon$ is small enough, does there exist a smooth $r : N(\epsilon) \to N$ s.t. $r(y) = y$ for all $y \in N$ ?

	answer: Yes (for now just assume yes)

	then

	$\overset{~}{F}$ is a submersion, $r$ is a submersion.

	then $F = \overset{~}{F}.r : M \x B_\epsilon \to N$ is a submersion

	$\implies$

	$F \transverse L$ for all $L \subseteq N$

	$\implies$

	$F_s \transverse L$ for almost $s \in B_\epsilon$

	$F_s$ homotopic $F0 = f$

\end{proof}
\mydate{d10}{7}{11}{2016}
\begin{prop}
	(use ``smooth'' everywhere, if you wish)
	If $N$ is compact and $V$ is a vector field on $N$, then there exists $f: N \to N$ s.t.
	\begin{itemize}
		\item $f$ homotopic to $\id$
		\item $\{ \text{fixed pts of } f \} = \{ x \mid V(x) = 0 \}$
	\end{itemize}
\end{prop}
\begin{proof}
	Normalize $V$ with $v(x) = \frac{V(x)}{mag(V(x))*e}$ are vectors of magnitude $\epsilon$.
	$r$ is the retraction back onto $N$.
	Let $H(x,t) = r(x + t*v(x))$ for $t$ goes from $0$ to $e$.
	tadaa!
\end{proof}
\begin{cor}[Bredon, 9]
	If $N$ has a nowhere vanishing vector field, then there exists $f: N \to N$ with *no* fixed points, and $f$ homotopic to id.
\end{cor}

\section{manifolds with boundary}
\begin{defi}
	The definition of a \de{manifold with boundary} $M$ is the same as the definition of a manifold, but there is a nonempty set of points in the manifold that have a neighborhood diffeomorphic to $H^n$ instead of $\R^n$.  The boundary of $M$ is denoted $\d M$
\end{defi}

\begin{thrm}[Homotopy Transversality Theorem 2]
	$f: M \to N \supseteq L$, and $C \subseteq M$ is closed.
	and $f|_C \transverse L$, then there exists
	$g: M \to N$ s.t.
	\begin{itemize}
		\item $g \transverse L$
		\item $g$ homotopic to $f$ with a homotopy that is constant on $C$
		\item If $M$ has boundary, then $dg \transverse L$
	\end{itemize}
	So $g|_C = f|_C$!
\end{thrm}
\begin{lem}
	If $H^n \supset U \overset{f}{\to} V \subseteq H^n$ is a diffeomorphism between open sets, then $f( U \cap \d H^n ) = V \cap \d H^n$
	(The boundary of $U$ maps to the boundary of $V$!)
\end{lem}
\begin{prop}[Matt]
	The boundary operator $\d$ is a derivation.  That is, if $M$ and $N$ are manifolds, then $\d(M \x N) = (\d M \x N) \cup (M \x \d N)$.
\end{prop}
\begin{cor}
	If $M$ has no boundary and $N$ does, then $\d(M \x N) = M \x \d N$.
\end{cor}
\begin{example}
	Let $M$ be a circle.  Let $N$ be an interval (the endpoints are the boundary).  Then $$\d(\text{cylindar}) = \text{circle} \x \text{endpoints} = \text{one circle at each end}$$.
\end{example}
\begin{lem}
	Suppose $N$ is a manifold without boundary and $f : N \to \R$ is smooth, and $c$ is a regular value.  Then $f^{-1}([c,\oo))$ is a manifold with boundary $f^{-1}(c)$.
\end{lem}
\begin{example}
	Draw $N$.  Let $f$ be the ``height'' of each point of $N$.  Observe that the inverse image of $[c, \oo)$ has boundary!
\end{example}
\begin{lem}[Collar lemma]
	Given some $M$, then $\d M$ has an open neighborhood diffeomorphic to $\d M \x [0, \epsilon)$ for $\epsilon$ small enough.
\end{lem}


\begin{thrm}
	A connected 1-dim manifold with boundary is diffeomorphic to $\R$, $[0,\oo)$, $[0,1]$, and $S^1$.  (We will *not* prove this)
\end{thrm}
\begin{cor}
	A compact 1-dim manifold has an even number of boundary points.
\end{cor}
\begin{thrm}[Retraction theorem]
	If $M$ is a compact manifold with boundary $\d M$, then there is NO smooth retraction $r : M \to \d M$.
\end{thrm}
\begin{proof}
	Suppose there was.  By Sard's Theorem, almost all values are regular.  Since $\d M$ has positive measure, and the critical values have measure 0, there must be at least one regular value.  Let $y$ be a regular value.

	By the Preimage theorem, $r^{-1}(y)$ is a smooth manifold of dimension $\dim(M) - \dim(\d M) = 1$.  Since $\{y\}$ is closed, and $r$ is continuous, $r^{-1}(\{y\})$ is also closed.  Since $r^{-1}(\{y\})$ is closed in a compact manifold, $r^{-1}(\{y\})$ is compact.  We conclude that we have a compact one-dimensional submanifold $r^{-1}(\{y\})$.

	The boundary of our compact one-dimensional submanifold is $$\d (r ^{-1}(\{y\})) = r^{-1}(\{y\}) \cap \d M = \{x\}$$.  (Note that we concluded $x$ was the only point on the boundary by exploiting that $r$ is the identity on the boundary).   So our submanifold has exactly \textbf{1} boundary point.  But a theorem states that every compact one-dimensional manifold is diffeomorphic to a circle or a closed interval, meaning that it has an \textbf{even} number of boundary points.  This is a contradiction.
\end{proof}
\begin{thrm}[``Smooth'' Brower Fixed Point theorem]
	Any ``smooth'' $f : B^n \to B^n$ has a fixed point.  (This is in fact true for ``continuous'', but we won't prove that today)
\end{thrm}
\begin{proof}
	Suppose not.  Then for all $x \in B^n$, $f(x) \not= x$.

	Define $r : B^n \to S^{n-1}$ by sending $f(x)$ back to $x$.  Then $r$ is a smooth retraction.

	$B^n$ is compact, so this contradicts the Retraction theorem.
\end{proof}
\begin{prop}[Peter]
	If $M$ is a compact manifold, then $\d M$ is compact.
\end{prop}
\begin{proof}
	Let $M$ be a compact manifold.  Let $\d M$ be its boundary (which could even be empty).  For any open cover $\cup \theta$ of $\d M$, add a little thickness to each $\theta$, so that it becomes an open set $\theta_T$ of $M$.  (Remember that each point on the boundary has a neighborhood diffeomorphic to $H^n$, so we cover $\theta$ with these little neighborhoods and then use the properties of $H^n$ to thicken it to an open subset of $M$).  Now we have $\cup \theta_T$ is an open cover of the boundary, and together with $M \minus \d M$, we have an open cover of $M$.  Since $M$ is compact, there exists a finite subcover.  Take the finite subcover, and discard $M \minus \d M$.  The result is a finite open cover of the boundary.  Remove the thickenings, and we still have a finite open cover of the boundary.
\end{proof}

\section{intersection numbers and degree}

\begin{defi}
	Given a compact manifold $M^m$ and manifolds \\$M^m, N^n, L^l$ all with empty boundary,

	$$M^m \overset{f}{\to} N^n \supseteq L^l$$

	and $m + l = n$, then the \de{mod 2 intersection number}, denoted $I_2(f, L) \in \Z_2$, is $|g^{-1}(L)|$ mod 2 where $g \homotopic f$ and $g \transverse L$.
\end{defi}
\begin{prop}
	At least one such $g$ exists and the above is well-defined!  (you get the same cardinality for all such $g$!)
\end{prop}
\begin{cor}
	If $f,g: M \to N \supseteq L$ and $f \homotopic g$, then $I_2(f, L) = I_2(g, L)$.
\end{cor}
\begin{proof}
	\begin{itemize}
		\item Existence: Given any smooth $f$ and any $L$, the Homotopy Transversality Theorem guarantees a $g$ s.t. $g \homotopic f$ and $g \transverse L$.
		\item Well-defined: Consider both $g$ and $h$ satisfy the criterion.  Then $h \homotopic g$.  Then maybe we construct a transverse homotopy, or something.  Look up proof.
	\end{itemize}
		Let g, h each homotopic
\end{proof}
\begin{defi}
	A manifold is \de{closed} if it is compact without boundary.  (Visually, this looks like a closed container!)
\end{defi}
\begin{defi}
	$A$ and $B$ are \de{cobordant} if they are compact and there exists a manifold $W$ s.t. $\d W = A \sqcup B$.
\end{defi}
\begin{prop}
	If $A$ and $B$ are cobordant, then $A$ and $B$ are closed and $\dim(W)-1 = \dim(A) = \dim(B)$.
\end{prop}
\begin{prop}
	If $f \homotopic g$, then $f$ is cobordant to $g$.
\end{prop}
\begin{defi}
	$f$ and $g$ are \de{cobordant} maps if there exists $A$ and $B$ cobordant under $W$ and $F: W \to N$ such that
	$$F|_A = f \;\text{and}\; F|_B = g$$.
\end{defi}
\begin{prop}
	Given cobordant maps $f$ and $g$, and $L$ s.t. the mod 2 intersection numbers of $f$ and $g$ are defined, then $I_2(f, L) = I_2(g, L)$.
\end{prop}
\begin{cor}
	If $M$ is closed, $M = \d W$, $f: M \to N$ extends to $F: W \to N$, then for all $L \subseteq N$, $I_2(f, L) = 0$.  Note that this statement is actually \textbf{equivalent} to the above proposition. (i think)
\end{cor}
\begin{proof}
	$M = M \sqcup 0$, where 0 denotes the empty manifold.  Note that 0 is closed.  Then $I_2(f, L) = I_2(F|_0, L) = I_2(\{\}, L) = 0$, where $\{\}$ denotes the empty function.
\end{proof}
\begin{defi}
	If $f : M^m \to N^n$ and $N$ is connected and $M$, $N$ are closed and $m=n$, then the \de{mod 2 degree} of $f$, denoted $\deg_2(f)$, is $I_2(f, \{y\})$ for any $y \in N$.
\end{defi}

\begin{prop}
	This is well defined.  (Independent of choice of $y$).
\end{prop}
\begin{proof}
	Choose a $g \homotopic f$ s.t. $y$ is a regular value for $g$.  Look at $I_2(g, \{y\}) = |g^{-1}(y)|$.  Since $y$ is regular, we have that $|g^{-1}(y)|$ is locally constant.  Since $N$ is connected (and $M,N$ compact, see hw and midterm), then $|g^{-1}(y)|$ is globally constant.
\end{proof}
\begin{prop}
	If $f$ and $g$ are cobordant and $a=b=n$, then \\$\deg_2(f) = \deg_2(g)$.
\end{prop}
\begin{proof}
	Choose a value $y$ that is regular for both $f$ and $g$ (wiggling to $f'$ and $g'$ if necessary?).  Then $deg_2(f) = I_2(f, \{y\}) = I_2(f', \{y\}) = I_2(g', \{y\}) = I_2(g, \{y\}) = deg_2(g)$.
\end{proof}
\begin{cor}[actually equivalent to above]
	If $W$ is a manifold, $M = \d W$, and
	$f: M \to N$
	extends to $F: W \to N$,
	then $$\deg_2(f) = 0$$.
\end{cor}
\begin{cor}
	If $f_0 \homotopic f_1: M \to N$, then $\deg_2(f_0) = \deg_2(f_1)$.  (Because homotopic is more specific than cobordant)
\end{cor}

\section{Winding Numbers}

fix $z$ in $\R^n \minus M^n$.

Define $u_z : M^{n-1} \to S^{n-1}$ by
$u_z(x) = \frac{x-z}{|x-z|}$.

Picture of closed curve $M$ and a point $z$ and a ray (in the $x-z$ direction) from that point.

$$\deg_2(u_z) =
\begin{cases}
	0 	&\text{if $z$ is outside $M$} \\
	1 	&\text{if $z$ is inside $M$} \\
\end{cases}$$


\begin{thrm}[Jordan-Brouwer Separation Theorem]
	If $M$ is a compact connected hypersurface in $\R^n$, then
	\begin{itemize}
		\item $\R^n \minus M$ has exactly two connected sets $A$ and $B$.
		\item $A$ and $B$ are open.
		\item One of the sets, $A$, is bounded.
		\item $\bar{A}$ is compact.
		\item $\d\bar{A} = \d\bar{B} = M$.
	\end{itemize}
\end{thrm}
\begin{cor}
	There are exactly 2 path components at $\R^n \minus M$.
\end{cor}

\section{Orientation}

\begin{defi}
	Given $V$ a V.S. over $\R$ of dimension $n$.
	Let $\mathcal{B}(V) = \{ (v_1,,,vn) \mid \{v_1,,,vn\} \text{ is a basis of } V \}$ is the set of all ordered bases for $V$
	Introduce an equivalence relation.  Two bases are equivalent if there exists an invertible matrix with positive determinant that maps one basis to the other.   (this separates bases into "positive" and "negative")
	An \de{orientation} on $V$ is a choice of one of the two equivalence classes.  This is written $(V, [b])$ for some ordered basis $b$.
\end{defi}
\begin{example}
	The standard orientation of $\R^n$ is the equivalence class of the standard ordered basis $(e1,,,en)$.  This is written $(\R^n, [(e1,,,en)])$.
\end{example}
\begin{defi}
	If $M$ is a smooth $n$-dimensional manifold, then an orientation of $M$ will be a choice of orientation for each $T_xM$ compatible as follows.  $M$ has an atlas of coordinate charts $\{ \phi : U \to \R^n \mid d(\phi)_x : T_xM \to \R^n \text{ is orientation preserving} \}$.
\end{defi}
\begin{prop}
	A manifold admits an orientation if and only if there exists an atlas $\{ \phi : U \to \R^n \mid d(\phi)_x : T_xM \}$ s.t. for all $i,j$ we have $\phi_i^{-1}.\phi_j : \phi_i(U_i \cap U_j) \to \phi_j(U_i \cap U_j)$
	, for all $x$ in $U_i \cap U_j$, we have $d(\phi_i^{-1}.\phi_j)$ is orientation preserving.
\end{prop}
\begin{defi}
	$-M$ denotes the alternative orientation of $M$.
\end{defi}
\begin{example}
	The mobius strip is not orientable.
\end{example}

\mydate{d14}{16}{11}{2016}

Some stuff i missed.

\begin{defi}
	(still with m + l = n and M compact, L closed, all have no boundaries conditions)
	The \emph{intersection number} of $f$ and $L$, denoted $I(f, L)$, is the integer $\sum_{x \in g^{-1}(L)} o(x)$ where
	\begin{itemize}
		\item $g \homotopic f$
		\item $g \transverse L$
		\item $o(x) = \pm 1$ is the orientation of $x$
	\end{itemize}.
\end{defi}
\begin{lem}
	If $M$ is a 1-dimensional compact oriented manifold with boundary $\d M$, then $\sum_{x \in \d M} o(x) = 0$.
\end{lem}
\begin{proof}
	$g \homotopic f \homotopic h$.

	$g \transverse L$, $h \transverse L$.

	$g \transverse h : I \x M \overset{H}{\to} N$.

	There exists $K$ s.t. $J \x M \overset{\overset{~}{H}}{\to} N$.

	punch line / example.  draw the cyclindar with the level curve things on it.  Compare the level curves intersection the top and bottom boundaries.  these are $h^{-1}(L)$ and $g^{-1}(L)$:

	$\overset{~}{H}(L)^{-1} = h^{-1}(L) - g^{-1}(L) = 0$.
\end{proof}

\mydate{d17}{21}{11}{2016}

\begin{prop}
	This is well-defined (independent of choice of $g$).
\end{prop}
\begin{prop}
	If $M = \d W$, and $f: M \to N$ extends to $F: W \to N$ (i think, where $F$ is a cobordism), then for all $L$, $I(f,L) = 0$.
\end{prop}
\begin{cor}
	If $f_0 \homotopic f_1$, then $I(f_0, L) = I(f_1, L)$.

	Hypothesis for the following definition and some following things:
	$f: M \to N$. $M, N$ compact with empty boundary, oriented, $N$ connected. $\dim(M) = \dim(N)$.
\end{cor}
\begin{defi}
	$\deg(f)$ is defined to be $I(f, {y})$ for any $y \in N$.
\end{defi}
\begin{prop}
	This is well defined (independent of choice of $y$)

	(Then to calculate, choose $y$ regular)

	$\deg(f) = \sum_{\text{intersecting pts}} o(x)$ where $o(x) = +1$ if $df_x$ is orientation preserving.
\end{prop}


\begin{prop}
	If $M = \d W$ and $f$ extends to $F: W \to N$, then $\deg(f) = 0$.
\end{prop}

\begin{cor}
	$f_0 \homotopic f_1$ implies $\deg(f_1) = \deg(f_0)$.
\end{cor}
\begin{prop}
	$L \overset{f}{\to} M \overset{g}{\to} N$ and $l = m = n$.
	then
	$\deg(f.g) = \deg(f)\deg(g)$
\end{prop}
\begin{example}
	Given $d \in Z$, let $f_d : S^1 \to S^1$ (in complex plane) by $f_d(z) = z^d$.
	Then $\deg(f) = d$, as we will show!
	All pts in $S^1$ are regular
	$o(x) = +1$ for all pts if $d > 0$
	$o(x) = -1$ for all pts if $d < 0$
	whoah, so degree as we know if is related to $d$-to-$1$ functions which is related to intersection numbers!
\end{example}
\begin{example}
	$p: C \to C$ is a nonconstant polynomial
	$p$ extends to $\bar{p} : \C P^1 \to \C P^1$, where $\C P^1$ isom $\C*$ isom $S^2$.

	$\deg(\bar{p}) = \text{degree}(p)$.  Pick a regular value $c \in C$.  Regular (in the complex plane) means that $p'$ is nonzero, so we get that $\bar{p}^{-1}(c) = \{ z \mid p(z) = c \}$, which has $d$ distinct solutions.
\end{example}
\begin{thrm}
	If $M^n$ is compact manifold without boundary, then there exists $f_d : M \to S^n$ of degree $d$.
\end{thrm}

% "new" stuff starts here.  go over :)

\section{Lefschetz numbers of self maps}

\begin{defi}
	Given the set $M$, the \de{diagonal} of $M$ is $$\Delta M := \{ (x,x) \mid x \in M \}$$
\end{defi}
\begin{defi}
	Given the function $f: A \to B$, the \de{graph} of $f$ is $$\Gr(f) := \{ (x, f(x)) \mid x \in A \}$$
\end{defi}
\begin{rmk}
	If $f: M \to M$, then
	\begin{itemize}
		\item $\Delta M \subseteq M \x M$
		\item $\Gr(f) \subseteq M \x M$
		\item $\dim(\Delta M) = \dim(M)$
		\item $\dim(\Gr(f)) = \dim(M)$
		\item $\dim(M \x M) = 2\dim(M)$
	\end{itemize}
\end{rmk}
\begin{defi}
	$f$ is \de{Lefschetz} if it is a smooth map of manifolds $f: M \to M$ and (TFAE)
	\begin{itemize}
		\item $Gr(f) \transverse \Delta M$ in $M \x M$
		\item for every fixed point $x$ of $f$, the intersection is transverse.
		\item for every fixed point $x$ of $f$, $1$ is not an eigenvalue of $df_x$
	\end{itemize}
\end{defi}

\begin{thrm}
	If $f: M \to M$ is Lefschetz and $M$ is compact, then $f$ has a finite number of fixed points.
\end{thrm}
\begin{defi}
	If $f : M \to M$ is a smooth map of manifolds, $M$ is compact, and $\d M = 0$, then the \de{Lefschetz number} of the map $f$ is $$L(f) := I(\Delta M, \Gr(f))$$
\end{defi}
\begin{prop}
	If the Lefshetz number of $f$ is defined, then $$L(f) = (-1)^n I(Gr(f), \Delta M)$$ where $n = \dim(M)$.
\end{prop}
\begin{thrm}
	If $f$ is Lefschetz, then $$L(f) = \sum_{x \in \Fix(f)} o(x,x)$$

	Note that $o(x,x) = o((x,x)) = o((x,f(x)))$, where $(x, f(x))$ is an element of $\Delta M \cap \Gr(f)$.
\end{thrm}

\begin{lem}
	If $f$ is Lefschetz, then the orientation number
	$$o(x,x) =
	\begin{cases}
		+1 &\text{if $df_x - I$ preserves orientation} \\
		-1 &\text{otherwise} \\
	\end{cases}$$
\end{lem}
\begin{thrm}
	If $f$ is Lefschetz, then $$L(f) = \sum_{x \in \Fix(f)} \text{sign}(df_x - I)$$ where $df_x - I : T_xM \overset{~}{\to} T_xM$.
\end{thrm}
\begin{example}
	$f : S^2 \to S^2$ is the rotation by $\pi/g$ radians about the $z$-axis.

	The fixed points are $e_3$ and $-e_3$.

	The tangent space at each fixed point is $\R^2$.

	A positive basis for $T_{e_3}S^2$ is $(e_1, e_2)$.

	A positive basis for $T_{-e_3}S^2$ is $(-e_1, e_2)$.

	Define $df_{e_3} : \R^2 \to \R^2$ by

	$df_{e_3}(e_1) = e_2$ and $df_{e_3}(e_2) = -e_1$
	% $e2 \mapsto -e1$

	Interpret this over the positive basis:

	$$\begin{pmatrix}
	0 & -1 \\
	1 & 0
	\end{pmatrix}$$

	So $df_{e_3} - I$ has this matrix w.r.t. the positive basis:

	$$\begin{pmatrix}
	-1 & -1 \\
	1 & -1
	\end{pmatrix}$$

	We get $\det(\text{above}) = 2 > 0$.

	Now define $df_{-e_3} : \R^2 \to \R^2$ by
	$-e_1 \mapsto e_2$ and
	$e_2 \mapsto -e_1$.

	Interpret this over the positive basis:

	$$\begin{pmatrix}
	0 & 1 \\
	-1 & 0
	\end{pmatrix}$$

	So $df_{-e_3} - I$ has this matrix w.r.t. the positive basis:

	$$\begin{pmatrix}
	-1 & 1 \\
	-1 & -1
	\end{pmatrix}$$

	As before, $\det(\text{above}) = 2 > 0$. We conclude that $L(f) = +2$.

	So we get f homotopic to id, and L(f) = 2.
\end{example}

\begin{prop}
	\begin{enumerate}
		\item If $f \homotopic g$, then $L(f) = L(g)$.

		\item If $f: M \to M$ has no fixed points, then L(f) = 0.

		\item If $L(f) \neq 0$, then every $g \homotopic f$ must have at least one fixed point.
	\end{enumerate}
\end{prop}

\begin{thm}
	Let $M$ be a manifold.
	\begin{itemize}
		\item Every vector field is homotopic to $\id_M$.
		\item Any two vector fields on $M$ are homotopic.
	\end{itemize}
\end{thm}
\begin{proof}
	For any (smooth) vector field, if we let it flow just a little bit, we get a function homotopic to $\id$. The continuous transition from $\id$ to the vector field is a homotopy!
\end{proof}

\textbf{Note.}
Since $L(\id) \neq 0$, then the vector field function also has fixed points. This proves the smooth hairy ball theorem!

\begin{rmk}
	If $L(f) = 3$, then $f$ has 3 more positive fixed points than negative fixed points.
\end{rmk}

\mydate{d18}{28}{11}{2016}

\section{The Euler characteristic}

\begin{defi}
	The \de{Euler characteristic} $\chi(M)$ of a compact manifold $M$ is (TFAE):
	\begin{itemize}
		\item $L(\id_M)$
		\item $I(\Delta M, \Delta M)$
		\item $\sum_{x \in \Fix(f)} \sign(\det(df_x - I))$
		\item $I(v_0(M), v_0(M))$
		\item $I(v_0(M), v(M))$ for any generate vector field $v$
		\item $\sum_{x \in Z(v)} o(x)$ (where Z(v) is the zeros of v)
		\item $\sum_{x \in \Crit(f)} (-1)^{index(x)}$
	\end{itemize}
\end{defi}

\begin{defi}
	The \de{zero} vector field is the vector field defined by $$v_0(x) := (x, 0)$$ for all $x$.
\end{defi}

\begin{prop}
	If $v$ is a vector field, then $v$ is an embedding (using the formal definition $v: M \to M \x TM$).
\end{prop}

\begin{defi}
	A vector field $v$ is \de{nondegenerate} (or \de{generate}) if $$v(M) \transverse v_0(M)$$.
\end{defi}

\begin{prop}
	If $M$ is compact and $v$ generate, then $$v(M) \cap v_0(M) = \{ x \in M \mid v(x) = 0 \}$$ is finite.
\end{prop}

\begin{defi}
	$\bar{v}$ is $v$, but only looking at 2nd coordinate.
\end{defi}

\begin{lem}
	If $U$ in $\R^n$ is open, $v$ is a smooth vector field on $U$, and $\bar{v}(x) = 0$, then $v$
	is generate in a neighborhood of $x$ iff $d\bar{v}_x : \R^n \to \R^n$ is invertible.
\end{lem}

\begin{defi}
	If $v$ is a vector field, then $x$ is a \de{nondegenerate} 0 for $v$ if $d\bar{v}_x$ is invertible.
\end{defi}

\begin{lem}
	If $v$ is a generate vector field and (given, or always?) there exists $H_v : M \x [0, \epsilon) \to M$, then $f_t: M \to M$ defined by $f_t(x) = H(x,t)$ is Lefschetz for small nonzero $t$.
\end{lem}

\section{Morse functions}

\begin{rmk}
	If $f : M \to \R$, then $x$ is a critical point iff $df_x = 0$.
\end{rmk}

\begin{example}
	Look at some smooth surface in $\R^3$.
	Note that we can figure out local max, local mins with certainty by taking the gradient / Jacobian, then taking the determinant of that, which is the curvature.  Positive curvature is local max/min.  Negative curvature is saddle.  Zero is flat in any direction (not just tangent).

		case 1:
			$U$ open in $\R^n$
			$f: U \to \R$

			$x \in U$ is a critical point iff ($\del(f)(x) = 0$)
			% x is a generate critical point of \del^2(x) = H_p(x) is invertible (Hessian).

		case 2:
			general $M$
			$f: M \to \R$

			$x \in M$ critical point.  $x$ is a generate critical point of $\phi(x)$ is generate for $\phi^{-1}.f$

			% M superset U --\phi\to \phi(?) \subseteq \R^n

		checking this is independent of choice of coc chart ancink (amounts?) to proving
\end{example}

\mydate{d19}{30}{11}{2016}

\begin{defi}
	Let $f : \R^n \to \R$ be smooth.  The point $p$ is \emph{critical} if for all $i$ from 1 to $n$, $$\frac{df}{dx_i}(p) = 0$$
\end{defi}

\begin{defi}
	The \de{Hessian} of $f: \R^n \to \R$ is $$\left( \frac{d^2f}{dx_i dx_j} \right)_{i,j}$$

	Note that the Hessian is related to the Jacobian, and the term Hessian can abusively refer to the determinant of the Hessian, just like the Jacobian.
\end{defi}
\begin{thrm}
	All Hessians are symmetric.
\end{thrm}

\begin{defi}
	A critical point $p$ is \emph{nondegenerate} if the Hessian $H(p)$ is invertible.
\end{defi}


\begin{defi}
	A smooth map of manifolds $f: M \to \R$ is \de{Morse} if all critical points are generate.
\end{defi}

\begin{example}
	The following are generate critical points.
	\begin{itemize}
		\item a "sink" has all vectors pointing to the critical point
		\item a "half-sink" "half-source" e.g., the vector field $< x, -y >$
	\end{itemize}
\end{example}

\begin{example}[degenerate point of a vector field]
	Define $f: \R \to \R$ by
	$$f(x) = x^2.$$ Then
	$$\del(f)(x) = 3x^2.$$
	There is a critical point for $f$ at $x = 0$.
	All the vectors are going in the same direction, and they shrink to 0 at 0, then grow again.
\end{example}

% Matt, I moved this before the formal statement of the lemma.
Roughly speaking, the Morse Lemma states that a smooth enough function near a critical point can be expressed as a quadratic form after a suitable change of coordinates.

\begin{lem}[Morse]
	If $f: M \to \R$ is a smooth map of manifolds and $p$ is a generate critical point, then near $p$, $$f(x_1,,,x_n) = x_1^2 +++ x_n^2$$ (maybe minus?) with some change of basis.
\end{lem}

We will \emph{not} define genus rigorously in this course.

\begin{defi}
	The \de{genus} $g$ of a closed manifold $M$ is the number of holes.
\end{defi}
\begin{thrm}
	If $M$ is a closed manifold with genus $g$, then $$\chi(M) = 2 - 2g$$
\end{thrm}
\begin{cor}
	If $g \neq 1$, then every vector field on $M_g$ has a zero.
\end{cor}

\begin{thrm}
	If $M$ has odd dimension, then $\chi(M) = 0$.
\end{thrm}
\begin{proof}
	$$\chi(M) = I(\Delta M, \Delta M) = (-1)^{\dim(M)^2} \cdot I(\Delta M, \Delta M)$$
	If the dimension of $M$ is odd, then $\chi(M) = -\chi(M)$.
	So $\chi(M) = 0$.
\end{proof}

\begin{defi}
	The \de{index} of a point is defined in several ways (TFAE):
	\begin{itemize}
		\item When you visualize the surface, the index is the number of axis directions in which the surface goes downwards.
		\item When you look at the (del f, or the laplacian, or maybe the hessian?), the index is the number of negative eigenvalues (of the hessian?).
		\item When you zoom in and approximate $f(x)$ as $x_1^2 + x_2^2 - x_3^2 + x_4^2$ (up to a change in basis), the index is the number of negative signs.
	\end{itemize}
\end{defi}









\end{document}
