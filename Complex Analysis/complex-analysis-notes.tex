\documentclass[11pt,leqno,oneside]{amsart}
%\usepackage{amsmath,amsthm}
\usepackage{amssymb,mathrsfs}
%\usepackage{ytableau}
%\ytableausetup{smalltableaux,centertableaux}
\usepackage{tikz}
\usepackage{upgreek}

%\usepackage[nohead,nofoot,centering]{geometry}

\usepackage{color} 
%% Some user-defined colors
      \definecolor{mydefi}{cmyk}{1,0,0,.5}
      \definecolor{myred}{rgb}{.7,.1,.1}
      \definecolor{myblue}{rgb}{.1,.1,.6}
      \definecolor{mygreen}{rgb}{.1,.6,.1}

\usepackage[urlbordercolor={1 1 1}, pdfborder={0 0 0}, bookmarks=true,
  colorlinks=true, linkcolor=myblue, citecolor=myblue,
  urlcolor=myblue, hyperfootnotes=false]{hyperref}

\usepackage[alphabetic,abbrev]{amsrefs} % use AMS ref scheme

\addtolength{\footskip}{2\baselineskip} % to lower the page numbers


%%%%%%%%%%%%%%%%%%%%%%%%%%%%%%%%%%%%%%%%%%%%%%%%%%%%%%%%%%%%%%%%%%%
%%  MACRO DEFINITIONS:  Co-authors -- PLEASE use these! 
%%%%%%%%%%%%%%%%%%%%%%%%%%%%%%%%%%%%%%%%%%%%%%%%%%%%%%%%%%%%%%%%%%%
\newcommand{\N}{{\mathbb N}} % natural numbers
\newcommand{\Z}{{\mathbb Z}} % integers
\newcommand{\Q}{{\mathbb Q}} % rational numbers
\newcommand{\R}{{\mathbb R}} % real numbers
\newcommand{\C}{{\mathbb C}} % complex numbers
\newcommand{\End}{\operatorname{End}} % endomorphisms
\newcommand{\Hom}{\operatorname{Hom}} % homomorphisms
\newcommand{\GL}{\operatorname{GL}} % general linear group
\newcommand{\B}{\mathfrak{B}} % use for the Brauer algebra
\newcommand{\Sym}{\mathfrak{S}} % symmetic group
\newcommand{\sgn}{\operatorname{sgn}} % sign
\newcommand{\T}{\mathsf{T}} % use for tableaux
\newcommand{\U}{\mathsf{U}} % use for tableaux
\newcommand{\V}{\mathsf{V}} % use for tableaux
\newcommand{\TA}{\mathsf{A}} % use for tableaux
\newcommand{\TB}{\mathsf{B}} % use for tableaux
\newcommand{\TC}{\mathsf{C}} % use for tableaux
\newcommand{\TS}{\mathsf{S}} % use for tableaux
\newcommand{\shape}{\operatorname{shape}} % shape of a tableau
\newcommand{\col}[2]{\genfrac{}{}{0pt}{1}{#1}{#2}} % column of bitableau
\newcommand{\ov}{\overline} % shorthand for a bar on a symbol
\newcommand{\dd}{\partial} % use for diagram basis; e.g. d(V_l)
\newcommand{\X}{\mathcal{X}} % use for the Gelfand-Tsetlin subalgebra
\newcommand{\JM}{\mathcal{J}} % a subalgebra of the GT-subalgebra
\newcommand{\Std}{\operatorname{Std}} % set of standard tableaux
\newcommand{\StdB}{\operatorname{StdB}} % set of standard bitableaux
\newcommand{\Orb}{\mathcal{O}} % use for orbits
\newcommand{\OS}{\ov{\Orb}} % use for orbit sums
\newcommand{\OSS}{\ov{\OS}} % use for double bar orbit sums
\newcommand{\OR}{\mathscr{R}} % orbit representatives
\newcommand{\Stab}{\operatorname{Stab}} % stabilizer
\newcommand{\rev}{\operatorname{rev}} % reverse of a cycle
\newcommand{\A}{\mathcal{A}} % the algebra
\newcommand{\fraka}{\mathfrak{a}} % Young symmetrizer
\newcommand{\frakb}{\mathfrak{b}} % Young symmetrizer
\newcommand{\frakc}{\varphi} % canonical basis
\newcommand{\yy}{\mathsf{y}} % Young symmetrizer; scaled
\newcommand{\idem}{\varepsilon} % primitive central idem in symm gp
\newcommand{\cA}{\mathcal{A}} % group algebra of symmetric group
\newcommand{\Tab}{\operatorname{Tab}} % trails in branching graph from source
\newcommand{\BG}{\mathbf{B}} % branching graph
\newcommand{\bb}{\varnothing} % the unique element of \Irr(0) 
\newcommand{\res}{\operatorname{res}} % restriction
\newcommand{\Irr}{\operatorname{Irr}} % irreps
\newcommand{\Wt}{\operatorname{Wt}} % possible content vectors for an irrep
\newcommand{\trace}{\operatorname{trace}} % the trace
\newcommand{\type}{\operatorname{type}} % type = generalized shape
\newcommand{\gen}[1]{\langle #1 \rangle} % use for generating sets
\newcommand{\parm}{\updelta} % Brauer algebra parameter
\newcommand{\sep}{\,|\,} % separator for two partitions - used in tables 
\newcommand{\covered}{\lessdot}
\newcommand{\qand}{\quad\hbox{and}\quad}
\newcommand{\Arg}{\operatorname{Arg}}
\swapnumbers %% put numbers in front of proclamations
\newtheorem{thm}{Theorem}[section]
\newtheorem*{thm*}{Theorem}
\newtheorem{lem}[thm]{Lemma}
\newtheorem*{lem*}{Lemma}
\newtheorem{prop}[thm]{Proposition}
\newtheorem*{prop*}{Proposition}
\newtheorem{cor}[thm]{Corollary}
\newtheorem*{cor*}{Corollary}
\newtheorem{conj}[thm]{Conjecture}
\newtheorem*{conj*}{Conjecture}

\theoremstyle{definition}
\newtheorem{defn}[thm]{Definition}
\newtheorem*{defn*}{Definition}
\newtheorem{example}[thm]{Example}
\newtheorem*{example*}{Example}
\newtheorem{examples}[thm]{Examples}
\newtheorem*{examples*}{Examples}
\newtheorem{alg}[thm]{Algorithm}
\newtheorem*{alg*}{Algorithm}
%\theoremstyle{remark}
\newtheorem{rmk}[thm]{Remark}
\newtheorem*{rmk*}{Remark}
\newtheorem{rmks}[thm]{Remarks}
\newtheorem*{rmks*}{Remarks}

%%%%%%%%%%%%%%%%%%%%%%%%%%%%%%%%%%%%%%%%%%%%%%%%%%%%%%%%%%%%%%%%%%%
\numberwithin{equation}{section} 
%% The following avoids conflict between numbers of proclamations 
%% and numbers of equations
\renewcommand{\theequation}{\thesection\alph{equation}} 
%%%%%%%%%%%%%%%%%%%%%%%%%%%%%%%%%%%%%%%%%%%%%%%%%%%%%%%%%%%%%%%%%%%
\parskip = 2pt
\allowdisplaybreaks
\renewcommand{\labelenumi}{(\theenumi)} % use round brackets
\renewcommand{\theenumi}{\alph{enumi}} % use alphabetic enumerations


\pagestyle{plain} % suppress the running head - for working document

\title[Complex Analysis]{Complex Analysis}
\author{George H. Seelinger (inspired from class by David Sherman)}
\date{Fall 2016}
\begin{document}
\maketitle
\section{Lecture 1}
What makes complex analysis different than calculus and real analysis?

\begin{enumerate}
    \item Over the complex field, all polynomials factor completely.
    \item All the functions are infinitely differentiable and can thus be written as a power series.
    \item It is easier to describe important physics problems using complex analysis to solve partial differential equations.
    \item Some integrals can be solved using visual tricks and some integrals that are not solvable using calculus/real analysis omit solutions using complex analysis.

\end{enumerate}
\begin{example}
    The polynomial $x^2+10x+100$ cannot be factored in terms of real factors since its discriminant is negative. However, this can be factored over the complexes.
\end{example}
\begin{example}
    Describing physical phenemenon, such as the flow of a river over a log, is much easier using complex analysis.
\end{example}
\begin{example}
    Solving $\int_{-\infty}^{\infty} \frac{\cos 2x}{x^2+1} dx$ is almost impossible for a standard calculus student and in real analysis, one can show that the integral is finite, but solving it is difficult. In complex analysis, one can easily show that it is $\pi^2/e$. 
\end{example}
\begin{example}
    $\int_{\gamma} \frac{\cos z}{z} dz = 2 \pi i $ times the number of counter clockwise rotations around $(0,0)$ in the complex plane.
\end{example}
\section{Lecture 2}
When doing complex analysis, it is important to develop a geometric intuition
for what is happening when you see functions of complex variables. The
following are some examples:
\begin{table}
    \centering
    \begin{tabular}{|c|c|}
        \hline
        $|z-w|$ & The distance from $z$ to $w$ \\
        $|z-1|=|z-i|$ & The locus of points equidistance from $1$ and $i$, ie a line. \\
        $|z-1|=3-|z-i|$ & The distance from $z$ to 1 and $z$ to $i$ add to 3, ie a parabola. \\
        $|z-1|=3+|z-i|$ & The upper part of a hyperbols (because the complex ordering). \\
        $|z-1|=2|z-i|$ & A circle (points that are twice as far from $1$ as from $i$). \\
        $|z-1|=\frac{2}{|z-i|}$ & An oval shape called a Cassini oval (product of distances is constant). \\
        \hline
    \end{tabular}
    \caption{Some real functions in complex form}
    \label{tab:func-descs}
\end{table}
Also note that $\ov{z}$ is $z$ reflected over the x-axis. 

In $\R$, there are many ways to define $e^x$. One way is $e^x =
\sum_{n=0}^\infty \frac{x^n}{n!}$. Also, recall the taylor series expansions of
$\sin x$ and $\cos x$. We can define all these functions in complex $x$ using
the taylor series expansions. 
\[
    e^{ix} = 1 + ix + \frac{(ix)^2}{2!} + \frac{(ix)^3}{3!} + \cdots = \cos x + i \sin x
\]
When $x$ is real, $|e^{ix}| = |\cos x + i \sin x| = \sqrt{\cos^2 x + \sin^2 x}
= 1$, so our rules in $\R$ still hold. 

This formulation allows us to come up with the polar form of looking at complex
numbers. This boils down to $z = r e^{i \theta}$ where $r = |z|$ and $\theta
\in \arg z$. Note that $\arg z$ is the \emph{set} of (real) $\theta$ that solve
the equation. However, since $\arg$ is multi-valued, we say that $\Arg \in
\arg$ is the value that is between $(-\pi, \pi]$. 

From this description, we get that $e^z e^w = e^{z+w}$ (follows from the power series). In particular, $(e^z)^n = e^{nz}, n \in \N$, This gives us \[
    \cos n \theta + i \sin n \theta = e^{in\theta} = (e^{i\theta})^n = (\cos \theta + i \sin \theta)^n
\]
This formula is often called \emph{DeMoivre's formula} and from it, we can
recover trig identities by expanding and equating the real and imaginary parts. 

Overall, polar form is good for understanding multiplication and powers. For example, $\frac{1}{z} = \frac{1}{r} \exp{-i \theta}$. 

We can also take roots effectively using polar form (look it up).

Finally, we can look at the stereographic projection (see textbook for picture) using the one-point compactification of $\C$, namely $\C^* = \C \cup \{\infty\} \cong S^2$. 

We also get some nice correspondances from actions on the sphere as in the table.
    \begin{tabular}{|c|c|}
        \hline
        Equator & $|z| = 1$ \\
        Upper hemisphere & $\{|z| > 1\} \cup \{\infty\}$ \\
        Lines of lattitude & Circle centered at origin \\
        Lines of longitude & Rays from origin \\
        Circles on $S^2$ that do not contain the north pole & Circles on $C^*$ \\
        Circles on $S^2$ that contain the north pole & Lines on $C^*$ \\
        \hline
    \end{tabular}

\end{document}
