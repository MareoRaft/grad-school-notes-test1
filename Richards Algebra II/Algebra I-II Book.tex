\documentclass[11pt]{book}

\usepackage[margin=1.0in]{geometry}

\usepackage{amsmath,amssymb,amsthm}

\usepackage{mathrsfs}
%\usepackage{dsfont}
\usepackage{times}
\usepackage{epsfig}
\usepackage{enumitem}
\usepackage{mathabx}

\usepackage{csquotes}
\MakeOuterQuote{"}
\usepackage{graphicx}
\usepackage{caption}
\usepackage{subcaption}
\usepackage{setspace}

\usepackage{tikz}
\usepackage{tikz-cd}
\usepackage{pgfplots}

\usepackage{etoolbox,xparse}
\usepackage{float}
\floatstyle{plain}%boxed
\restylefloat{figure}
\usepackage{chngcntr}
\counterwithin{chapter}{part}

\newcounter{counter}


\newtheorem{theorem}[counter]{Theorem}   \newtheorem*{theorem*}{Theorem}   \newtheorem{lemma}[counter]{Lemma}   \newtheorem{corollary}[counter]{Corollary}
\newtheorem{proposition}[counter]{Proposition}   \newtheorem{problem}[counter]{Problem}   \newtheorem*{proposition*}{Proposition}   \newtheorem*{lemma*}{Lemma}

\theoremstyle{definition}   \newtheorem{defn}[counter]{Definition} %These theorem environments are not numbered separately
\newtheorem*{defn*}{Definition}
\newtheorem{blank}[counter]{}   \newtheorem{remark}[counter]{Remark(s)}   \newtheorem*{remark*}{Remark(s)}   \newtheorem{generalization}[counter]{Generalization}
\newtheorem{consequence}[counter]{Consequence(s)}   \newtheorem*{consequence*}{Consequence(s)}   \newtheorem*{problem*}{Problem}   \newtheorem{notation}[counter]{Notation}
\newtheorem*{notation*}{Notation}   \newtheorem{example}[counter]{Example(s)}   \newtheorem*{example*}{Example(s)}   \newtheorem*{warning}{Warning} \newtheorem*{corollary*}{Corollary}
\newtheorem*{question}{Question}   \newtheorem*{answer}{Answer}   \newtheorem{modification}[counter]{Modification}   \newtheorem{numitem}[counter]{}

\newcommand{\ov}{\overline}   \newcommand{\wt}{\widetilde}
\newcommand{\blt}{$\bullet$}   \newcommand{\tn}{\textnormal}   \newcommand{\tb}{\textbf}   \newcommand{\mbb}{\mathbb}
\newcommand{\bs}{\backslash}   \newcommand{\A}{\mathcal{A}}   \newcommand{\sy}{\textnormal{Syl}}   \newcommand{\size}[1]{\left| #1 \right|}
\newcommand{\zx}[1]{(\z/#1\z)^{\times}}   \newcommand{\zn}[1]{\z/#1\z}   \newcommand{\pr}[1]{\textbf{Problem #1.}}   \newcommand{\abc}{(\alph*)}
\newcommand{\nsg}{\mathrel{\unlhd}}   \newcommand{\ind}{\parindent24pt}   \newcommand{\vn}{\varnothing}
\newcommand{\ve}{\varepsilon}   \newcommand{\im}{\textnormal{im }}   \newcommand{\re}{\textnormal{Re }}   \newcommand{\mb}[1]{\mathbf{#1}}
\newcommand{\lra}{\leftrightarrow}   \newcommand{\0}{\mathbf{0}}   \newcommand{\mc}[1]{\mathcal{#1}}   \newcommand{\hra}{\hookrightarrow}   \newcommand{\hla}{\hookleftarrow}
\newcommand\myeq{\mathrel{\overset{\makebox[0pt]{\mbox{\normalfont\tiny\sffamily \textrm{def}}}}{=}}}
\newcommand\lheq{\mathrel{\overset{\makebox[0pt]{\mbox{\normalfont\tiny\sffamily \textrm{L'H}}}}{=}}}
\newcommand{\mymatrix}[2]{\left( \begin{array}{#1} #2 \end{array} \right)}   \newcommand{\myvec}[1]{\left( \begin{array}{c} #1 \end{array} \right)}

\newcommand{\hm}{homomorphism}   \newcommand{\hms}{homomorphisms}   \newcommand{\iso}{isomorphism}
\newcommand{\isos}{isomorphisms}   \newcommand{\auto}{automorphism}   \newcommand{\autos}{automorphisms}   \newcommand{\ds}[2]{#1^{(#2)}}   \newcommand{\lcs}[2]{#1^{[#2]}}
\newcommand{\pf}[1]{$ #1 = p_1^{e_1} p_2^{e_2} \cdots p_r^{e_r}$, with $p_1,p_2,\dots,p_r$ distinct primes and $e_1,e_2,\dots,e_r\in \N$}
\newcommand{\inn}{\textnormal{Inn}}   \newcommand{\out}{\textnormal{Out}}   \newcommand{\car}{\textnormal{ char }}   \newcommand{\id}{\textnormal{id}}   \newcommand{\triv}{\{e\}}
\newcommand{\tl}{\triangleleft}   \newcommand{\sd}[1]{\rtimes_{#1}}   \newcommand{\x}{^{\times}}   \newcommand{\cyc}[1]{\begin{pmatrix} #1 \end{pmatrix}}
\newcommand{\gen}[1]{\langle #1 \rangle}   \newcommand{\stab}[2]{\tn{Stab}_{#1}(#2)}   \newcommand{\fix}[2]{\tn{Fix}_{#1}(#2)}   \newcommand{\op}{^{\tn{op}}}
\newcommand{\hooklongrightarrow}{\lhook\joinrel\longrightarrow}   \newcommand{\twoheadlongrightarrow}{\relbar\joinrel\twoheadrightarrow}
\newcommand{\ses}[5]{1 \longrightarrow #1 \overset{#2}{\hooklongrightarrow} #3 \overset{#4}{\twoheadlongrightarrow} #5 \longrightarrow 1}
\renewcommand{\arrowvert}{\arrow}   \renewcommand{\i}{\mathbf{i}}   \renewcommand{\j}{\mathbf{j}}   \renewcommand{\k}{\mathbf{k}}    \renewcommand{\H}{\mathbb{H}}  \renewcommand{\hom}{\tn{Hom}}
\newcommand{\ann}[2]{\textnormal{Ann}_{#1}(#2)}


\DeclareMathOperator{\pow}{\mathcal{P}}   \DeclareMathOperator{\C}{\mathbb{C}}
\DeclareMathOperator{\col}{\mathcal{C}}   \DeclareMathOperator{\F}{\mathbf{F}}   \DeclareMathOperator{\h}{\mathbb{H}}   \DeclareMathOperator{\bF}{\mathbf{F}}
\DeclareMathOperator{\R}{\mathbb{R}}   \DeclareMathOperator{\N}{\mathbb{N}}   \DeclareMathOperator{\z}{\mathbb{Z}}   \DeclareMathOperator{\Q}{\mathbb{Q}}
\DeclareMathOperator{\ra}{\rightarrow}   \DeclareMathOperator{\Poly}{\mathbf{P}}   \DeclareMathOperator{\spn}{\textnormal{span}}   \DeclareMathOperator{\aut}{\textnormal{Aut}}
\DeclareMathOperator{\znx}{(\z/n/Z)^{\times}}   \DeclareMathOperator{\lcm}{\textrm{lcm}}   \DeclareMathOperator{\inv}{^{-1}}   \DeclareMathOperator{\sub}{\subseteq}

\newenvironment{prf}{\paragraph{\textit{Proof}}}{\hfill$\square$}

\newcommand{\vs}{\vspace{8pt}}
\newcommand{\hs}{\hspace{8pt}}

\numberwithin{counter}{chapter}

\title{Algebra}
\author{ Transcribed from the lectures of \\ \ \\ Professor Peter Abramenko \\ \ \\ University of Virginia}
\date{Fall 2016}


\begin{document}

\frontmatter
\maketitle
\tableofcontents
\mainmatter

\ \vspace{100pt}
\begin{quote}
\centering
\textit{A study of groups, rings, fields, modules, tensor products, and multilinear functions.}
\end{quote}

\part{Group Theory}


\part{Commutative Rings}


\part{Modules}

\chapter{Basic Concepts}

In the following, $R$ always denotes a (not necessarily commutative) ring with 1.

\vs

\begin{defn}
An abelian group $(M,+)$ together with a map $R \times M \ra M$, $(r,m) \mapsto rm$, is called a \tb{left $\mb{R}$-module} if
\begin{enumerate}
\item[(1)] $(r+s)m = rm + sm$
\item[(2)] $r(m+n) = rm + rn$
\item[(3)] $(rs)m = r(sm)$
\item[(4)] $1 m = m$
\end{enumerate}
for all $r,s \in R$ and $m,n \in M$. It follows from these conditions that
\begin{enumerate}
\item[$\bullet$] $0_R m = 0_M$
\item[$\bullet$] $r0_m = 0_m$
\item[$\bullet$] $(-1)m = -m$
\end{enumerate}
for all $m \in M$, $r \in R$.
\end{defn}

\vs

\begin{example}\
\begin{enumerate}
\item[(a)] If $R$ is a field, then the $R$-modules are precisely the $R$-vector spaces.
\item[(b)] Any abelian group $(M,+)$ is a (left) $\z$-module with the scalar multiplication defined by $k m = m + m + \dots + m $ ($k$ times) and $-k m = -m -m - \dots - m $ ($k times$).
\item[(c)] $\{0\}$ is an $R$-module for any ring $R$.
\item[(d)] $R$ is (left) $R$-module.
\item[(e)] For $n \in \N$, and $R^n := \left\{ \myvec{r_1 \\ \vdots \\ r_n} \bigg| r_i \in R \right\}$ with componentwise addition and scalar multiplication is a left $R$-module.

For (f) and (g) below, $F$ denotes a field and $V = F^n$.

\item[(f)] $V$ is a $M_n(F)$-module with matrix multiplication.
\item[(g)] Fix $A \in M_n(F)$. Then $V$ is an $F[x]$-module with $f(x) v := f(A) v$. We denote this module by $_A V$.
\end{enumerate}
\end{example}

\vs

From now on, $M$ denotes an $R$-module.

\vs

\begin{defn}
A subgroup $N$ of $M$ is called an ($R$-)submodule of $M$, denoted $N \leq M$, if $rn \in N$ for all $r \in R$ and $n \in N$. A subset $N \sub M$ is therefore a submodule if
\begin{enumerate}
\item[(1)] $0_m \in N$
\item[(2)] $n + n' \in N$ for all $n,n' \in N$
\item[(3)] $rn \in N$ for all $r \in R$, $n \in N$
\end{enumerate}
A submodule of the left $R$-module $R$ is called a \tb{left-ideal} of $R$. If $R$ is commutative, then any left ideal is an ideal.
\end{defn}

\vs

\begin{defn}
Let $S \sub M$. Define the \tb{annihilator of $\mb{S}$ in $\mb{R}$} by
	\[\ann{R}{S} := \{ r \in R \mid r s = 0 \ \forall \ s \in S \} \leq R \]
If $S \leq M$, then $\ann{R}{S}$ is a two-sided ideal of $R$. Indeed, for $a \in \ann{R}{S}, x \in S, r \in R$, $(ar)x = a(rx) = 0$ and $(ra)x = r(ax) = r0 = 0$, so $ar,ra \in \ann{R}{S}$.
\end{defn}

\vs

\begin{defn}
Let $M,N$ be $R$-modules. A map $\phi : M \ra N$ is called an \tb{$\mb{R}$-module homomorphism} or \tb{$\mb{R}$-linear} if $\phi(m+m') = \phi(m) + \phi(m')$ and $\phi(rm) = r\phi(m)$ for all $m,m' \in M$ and $r \in R$. Clearly, $\ker \phi := \phi\inv(0_N) \leq M$, $\im \phi \leq N$, and $\phi$ is injective if and only if $\ker \phi = \{0_M\}$. $\phi$ is called an \tb{$\mb{R}$-module isomorphism} if $\phi$ is bijective, in which case $\phi\inv : N \ra M$ is also an $R$-module isomorphism.
\end{defn}

\vs

\begin{lemma}
Let $\phi : M \ra N$ be $R$-linear.
\begin{enumerate}
\item[(a)] If $M' \leq M$, then $\phi(M') \leq N$, and if $N' \leq N$, then $\phi\inv(N') \leq M$.
\item[(b)] (Correspondence Theorem) Assume $\phi$ is surjective. Then $\phi$ induces a bijection $\{M' \leq M \mid \ker \phi \leq M\} \ra \{N' \leq N\}, M' \mapsto \phi(M')$ with inverse $N' \mapsto \phi\inv(N')$.
\end{enumerate}
\end{lemma}

\begin{proof}
(a) is straightforward. With (a) established, the induced maps in (b) are well-defined. If $N' \leq N$, then $\phi \phi\inv(N') \sub N'$ (always). To show reverse inclusion, let $x \in N'$. There exists $y \i m$ with $x = \phi(y) \implies y \in \phi\inv(N') \implies x \in \phi \phi\inv(N')$.

Now, let $\ker \phi \leq M' \leq M$. $M' \sub \phi\inv \phi(M')$ always. To show reverse inclusion, let $x \in \phi\inv \phi(M')$. Then $\phi(x) \in \phi(M')$, so there exists $m \in M'$ with $\phi(x) - \phi(m) \implies x-m \in \ker \phi \leq M' \implies x \in M'$.
\end{proof}

\vs

\begin{defn}
If $N \leq M$, the \tb{quotient module} $M/N$ is the abelian quotient group $M/N$ with scalar multiplication defined by $r(m+N) := rm + N$ (well-defined). The map $\pi : M \ra M/N$, $m \mapsto m+N$, is an $R$-linear surjection.
\end{defn}

\vs

\begin{corollary}
Each submodule $L$ of $M/N$ can be uniquely written in the form $L = M'/N$ for some $N \leq M' \leq M$, namely $M' = \pi\inv(L)$.
\end{corollary}

\vs

\begin{proposition}[Isomorphism Theorem for Modules]\
\begin{enumerate}
\item[(a)] If $\phi : M \ra M'$ is $R$-linear, then $M/\ker \phi \cong \im \phi$ via $m + \ker \phi \mapsto \phi(m)$.
\item[(b)] If $L,N \leq M$, then $L+N := \{\ell + n \mid \ell \in L, n \in N\} \leq M$, and $(L+N)/N \cong L / L \cap N$ (consider $L \ra (L+N)/N, \ell \mapsto \ell + N$).
\item[(c)] If $N \leq L \leq M$, then $(M/N)/(L/N) \cong M/L$ ($M/N \ra L/N$, $m+N \mapsto m+L$).
\end{enumerate}
\end{proposition}

\vs

\begin{defn}\
\begin{enumerate}
\item[(a)] For $R$-modules $M,N$, we set $\hom_R(M,N) := \{f : M \ra N \mid f \tn{ is $R$-linear}\}$. Check: $\hom_R(M,N)$ is an abelian group with $(f+g)(m) := f(m) + g(m)$ for $f,g \in \hom_R(M,N)$. If $R$ is commutative, then $\hom_R(M,N)$ is an $R$-module with $(rf)(m) := r f(m)$.

\item[(b)] $\tn{End}_R(M) := \hom_R(M,M)$ is a ring with $(f \cdot g)(m) := f(g(m))$ and is called the \tb{endomorphism ring of $\mb{M}$}.
\end{enumerate}
\end{defn}

\vs

\begin{example}
Consider $R$ as a left $r$-module. We ask what is the relationship between the rings $\tn{End}_R(R)$ and $R$? Let us define two maps $\rho : R \ra \tn{End}_R(R), a \mapsto (\rho_a : r \ra ra)$ and $\sigma : \tn{End}_R(R) \ra R$, $f \mapsto f(1)$. One should verify that $\sigma \circ \rho = \id_R$ and $\rho \circ \sigma = \id_{\tn{End}_R(R)}$. \\

$\rho$ is clearly additive, but $\rho_{ab} = \rho_b \circ \rho_a$ instead of $\rho_{ab} = \rho_a \circ \rho_b$. Thus $\rho$ is not a ring homomorphism, but it is very close to one. We now introduce the "opposite ring" $R\op$, which is the same abelian group as $R$ with multiplication reversed, i.e., $a \times b = ba$. One should verify that $(R,+,\times)$ is a ring with 1. We conclude that the map $\rho : \tn{End}_R(R) \ra R\op$, $a \mapsto (\rho_a : r \mapsto ra)$, is a ring isomorphism.
\end{example}

\vs

\begin{defn}
For $m_1,\dots,m_n \in M$, we set
	\[\gen{m_1,\dots,m_n}_R = \left\{\sum_{i=1}^n r_i m_i \mid r_i \in R \right\} \leq M \]
which is called the \tb{submodule of $\mb{M}$ generated by $\mb{m_1,\dots,m_n}$}. This is the smallest submodule of $M$ containing $m_1,\dots,m_n$. We say that $m_1,\dots,m_n$ \tb{generate} or \tb{span} $M$ if $M = \gen{m_1,\dots,m_n}_R$, and in this case we say $M$ is \tb{finitely generated}. $M$ is called \tb{cyclic} if $M = \gen{m}_R$ for some $m \in M$, and we often write $M = Rm$. \\

If $S = \{m_i \mid i \in I\} \sub M$, we define
	\[\gen{S}_R := \bigcup_{S' \sub S, S' \tn{ finite}} \tn{S'}_R = \left\{\sum_{i \in I} r_i m_i \mid \tn{all } r_i \in R, \tn{ almost all $r_i = 0$} \right\} \]
Check: $\gen{S}_R$ is the smallest submodule of $M$ containing $S$.
\end{defn}

\vs


\begin{remark}\
\begin{enumerate}
\item[(a)] If $M$ is cyclic and $m \in M$ with $M = Rm$, we consider the surjective $R$-linear map $\phi : R \ra M, r \mapsto rm$. By 3.1.9,
	\[M \cong R/ \ker \phi = M/ \ann{R}{m} \]
If $R$ is commutative, then $\ann{R}{m} = \ann{R}{M}$, but this is not true in general.

\begin{example*}
Let $R = M_n(F)$ for $F$ a field, and consider $V = F^n$ as an $R$-module. Let $m = e_1 = \mymatrix{c}{1 \\ 0 \\ \vdots \\ 0}$. Then $V = Rm$ is a cyclic $R$-module, but
	\[\ann{R}{e_1} = \left\{\mymatrix{cccc}{0 & \ast & \cdots & \ast \\ 0 & \ast & \cdots & \ast \\ \vdots & \vdots & \ddots & \vdots \\ 0 & \ast & \cdots & \ast} \in M_n(F) \right\} \ne \{O\} = \ann{R}{V} \]
\end{example*}
\item[(b)] If $M = \gen{m_1,\dots,m_n}_R$ is finitely generated, then $M$ is the quotient of $R^n$ for some $m \in \N$. Indeed,
	\[\phi : R^n \ra M, \quad \mymatrix{c}{r_1 \\ r_2 \\ \vdots \\ r_n} \mapsto \sum_{i=1}^n r_i m_i \]
is $R$-linear, so $M \cong R^n / \ker \phi$.

\item[(c)] If $M$ is finitely generated, then any $R$-linear image of $M$ is also finitely generated, since $\phi(\gen{m_1,\dotsm_n}_R) = \gen{\phi(m_1),\dots,\phi(m_n)}_R$ for an $R$-linear map $\phi$ defined on $M$.
\end{enumerate}
\end{remark}

\vs

\begin{example}\
\begin{enumerate}
\item[(a)] $R = \gen{1}_R$ always. If $R$ is a non-Noetherian commutative ring, it has ideals which are not finitely generated. This shows that a submodule of a finitely generated ring needn't be finitely generated.

\item[(b)] $R^n = \gen{e_1,\dots,e_n}_R$ with $e_i = \mymatrix{c}{a_1 \\ a_2 \\ \vdots \\ a_n}$ where $a_j = \delta_{ij}$.

\item[(c)] Put $M = F^n$ for $F$ a field, and let $R = M_n(F)$ or $R = F[x]$ as in 3.1.2 (g). Regardless, $V = \gen{e_1,\dots,e_n}_R$.
\end{enumerate}
\end{example}

\vs

\begin{defn}
If $J$ is a left $R$-submodule (i.e., left ideal) of $R$, then we define
	\[JM := \gen{\{ jm \mid j \in J, m \in m\}}_R = \left\{ \sum_{\tn{finite}} j_i m_i \mid j_i \in J, m_i \in M \right\} \]
\end{defn}

\vs

\begin{remark}
If $J$ is a two-sided ideal ($J \nsg R$), then $R/J$ is a ring with well-defined multiplication $(r+j)(s+j) = rs +J$, and $M/JM$ becomes an $(R/J)$-module with well-defined scalar multiplication
	\[(r+J)(m+JM) = rm + JM \]
\end{remark}

\subsection*{Direct Sums and Products}

\vs

\begin{defn}
The (external) \tb{direct sum} $M \oplus N$ of the left $R$-modules $M,N$, which is also the \tb{direct product} $M \times N$, is the group $M \oplus N$ with scalar multiplication $r(m,n) := (rm,rn)$.
\end{defn}

\vs

\begin{proposition}
For $M$ with submodules $L,N$, the following are equivalent:
\begin{enumerate}
\item[(i)] $L+N = M$ and $L \cap N = \{0\}$.
\item[(ii)] Each $m \in M$ can be written uniquely in the form $m = \ell + n$ for $\ell \in L$ and $n \in N$.
\item[(iii)] $M \cong L \oplus N$ (external direct sum) via $(\ell,n) \mapsto \ell + n$.
\end{enumerate}
\end{proposition}

\begin{proof}
Straightforward.
\end{proof}

\vs

\begin{example*}\
\begin{enumerate}
\item[(a)] $R^2 = R \oplus R$ (external)  $= \{(r,0)\} \oplus \{(0,r')\}$ (internal)
\item[(b)] $R^ = R \oplus R^{n-1} = R \oplus R \oplus \dots \oplus R$
\item[(c)] $\z$ is not the direct sum of two nonzero $\z$-submodules, since if $\{0\} \lneq L,N \leq \z$, then $L \cap N \ne \{0\}$.
\end{enumerate}
\end{example*}

\vs

\begin{defn}
An $R$-module $M$ is called \tb{indecomposable} if it cannot be written as the direct sum of two nonzero submodules (e.g., $\z$). $M \ne \{0\}$ is called \tb{irreducible} or \tb{simple} if $\{0\}$ and $M$ are the only submodules of $M$ (irreducible $\implies$ indecomposable).
\end{defn}

\vs

We can generalize in a straightforward way the construction of direct sums with finitely many summands: $\bigoplus_{i=1}^n M_i :- \{(m_1,\dots,m_n) \mid m_i \in M_i\}$ (external direct sum). For $M_1,\dots,M_n \leq M$, $M = \bigoplus_{i=1}^n M_i$ (internal direct sum) iff $M = \sum_{i=1}^n M_i$ and $M_i \cap \sum_{1 \leq j \leq n, j \ne i} M_j = \{0\}$.

\vs

\begin{defn}
Let $I$ be any (possibly infinite) index set and $(M_i \in I)$ a family of $R$-modules. The \tb{direct product} of $(M_i \mid i \in I)$ is the Cartesian product $\prod_{i \in I} M_i$ with componentwise addition and scalar multiplication. The \tb{direct sum} $\bigoplus_{i=1}^n M_i$ is the submodule of $\prod{i \in I} M_i$ consisting of all $(m_i)_{i \in I} \in \prod_{i \in I} M_i$ such that $m_i = 0$ for almost all $i \in I$.
\end{defn}

\vs

We define the following $R$-module homomorphisms
\begin{align*}
\pi_j & : \prod_{i \in I} M_i \ra M_j, \quad (m_i)_{i \in I} \mapsto m_j \\
\iota_j &: M_j \ra \bigoplus_{i \in I} M_i,  \quad m \mapsto (\delta_{ij}m)_{i \in I}
\end{align*}

\vs

\begin{remark}
The direct product and direct sum of $R$-modules are characterized by the following universal properties.
\begin{enumerate}
\item[(a)] (Universal Property of the Direct Product) For any $R$-module $N$ and family $(\phi_j)_{j \in I}$ of $R$-module homomorphisms $\phi_j : N \ra M_j$, there exists a unique homomorphism $\phi : N \ra \prod_{i \in I} M_i$ such that $\pi_j \circ \phi = \phi_j$ for all $j \in I$. Namely, $\phi(n) = (\phi_j(n))_{j \in I}$. \quad
\begin{tikzcd}
N \arrow[dotted]{r}{\exists \ ! \ \phi}
\arrow[swap]{d}{\phi_j} & \prod_{i \in I} M_i
\arrow{dl}{\pi_j} \\
M_j
\end{tikzcd}

\item[(b)] (Universal Property of the Direct Sum) For any $R$-module $N$ and family $(\phi_j)_{j \in I}$ of $R$-module homomorphisms $\phi_j : M_j \ra N$, there exists a unique homomorphism $\phi : \bigoplus_{i \in I} M_i \ra N$ such that $\phi \circ \iota_j = \phi_j$ for each $j \in I$. Namely, $\phi((m_i)_{i \in I}) = \sum_{i \in I} \phi_i(m_i)$. \quad
\begin{tikzcd}
N & \bigoplus_{i \in I} M_i \arrow[dotted,swap]{l}{\exists \ ! \ \phi} \\
M_j \arrow{u}{\phi_j} \arrow[hookrightarrow,swap]{ur}{\iota_j}
\end{tikzcd}
\end{enumerate}
\end{remark}

\vs

\begin{example*}
Let $M$ be an $R$-module and $(m_i)_{i \in I}$ a family of elements of $M$. For each $i \in I$, define $\phi_i : R \ra M$ by $r \mapsto rm_i$. By the universal property (b) above, there exists a unique homomorphism $\phi : \bigoplus_{i \in I} R \ra M$ such that $\phi((r_i)_{i \in I}) = \sum_{i \in I} r_i m_i$.
\end{example*}

\vs

\begin{lemma}
Let $(M_i)_{i \in I}$ be a family of $R$-modules and $N$ another $R$-module. Then
\begin{enumerate}
\item[(a)] $\hom_R(\bigoplus_{i \in I} M_i,N) \cong \prod_{i \in I} \hom_R(M_i,N)$
\item[(b)] $\hom_R(N,\prod_{i \in I} M_i) \cong \prod_{i \in I} \hom_R(N,M_i)$
\end{enumerate}

These isomorphisms are as abelian groups for arbitrary rings and as $R$-modules for commutative rings $R$.
\end{lemma}

\begin{proof}\
\begin{enumerate}
\item[(a)] The bijective correspondence is given as follows. To each $(\phi_i)_{i \in I} \in \prod_{i \in I} \hom_R(M_i,N)$, we associate the unique $\phi \in \hom_R(\bigoplus_{i \in I} M_i,N)$ granted by the universal property of the direct sum. Conversely, each $\phi \in \hom_R(\bigoplus_{i \in I} M_i,N)$ is uniquely determined by $(\phi \circ \iota_i)_{i \in I} \in \prod_{i \in I} \hom_R(M_i,N)$. It is routine to verify that this correspondence is additive and that it is $R$-linear when $R$ is commutative.

\item[(b)] To each $(\phi_i)_{i \in I} \in \prod_{i \in I} \hom_R(N,M_i)$, we associate the unique $\phi \in \hom_R(N,\prod_{j \in I} M_j)$ granted by the universal property of the direct product. Conversely, each $\phi \in  \hom_R(N,\prod_{j \in I} M_j)$ is uniquely determined by $(\pi_i \circ \phi)_{i \in I} \in \prod_{i \in I} \hom_R(N,M_i)$. It is again straightforward to verify that this correspondence is additive and that it is $R$-linear when $R$ is commutative.
\end{enumerate}
\end{proof}

\vs

\subsection*{Free Modules}

\vs

\begin{defn*}
Let $M$ be an $R$-module. Elements $m_1,\dots,m_n \in M$ are called \tb{linearly independent} if for all $r_1,\dots,r_n \in R$, we have that
	\[\sum_{i = 1}^n r_i m_i = 0 \iff r_1 = r_2 = \dots = r_n = 0 \]
More generally, a subset $S \sub M$ is called \tb{lienarly independent} if every finite subset of $S$ is linearly independent.
\end{defn*}

\vs

\begin{remark}
If $S = \{m_i \mid i \in I\} \sub M$ is linearly independent, then the map $\bigoplus_{i \in I} R \ra \gen{S}_R \sub M$, $(r_i)_{i \in I} \mapsto \sum_{i \in I} r_i m_i$ is an isomorphism.
\end{remark}

\vs

\begin{defn}
A subset $S \sub M$ is called an $\mb{R}$\tb{-basis} of the $R$-module $M$ if $\gen{S}_R = M$ and $S$ is linearly independent. Such a module $M$ admitting a basis is called \tb{free}. By convention, $\{0\}$ is a free $R$-module with basis $\vn$.
\end{defn}

\vs

\begin{example}\
\begin{enumerate}
\item[(a)] If $R$ is a \emph{skew field}, then any $R$-module (i.e., $R$-vector space) is free (the same argument as for vector spaces over a field).
\item[(b)] For any ring $R \ne \{0\}$, the left $R$-module $R$ is free with basis $\{1\}$.
\item[(c)] For $R \ne \{0\}$, $\bigoplus_{i \in I} R$ is free with basis $\{e_i \mid i \in I\}$, where $e_j = (\delta_{ij})_{i \in I}$. If $I$ is finite with $|I| = n \in \N$, then $\bigoplus_{i \in I} R \cong R^n$.

\tb{Warning:} If $I$ is infinite, the $R$-module $\prod_{i \in I} R$ is NOT generally free. As an example, see Exercise 24 on page 358 of [DF]. Here, $R = \z$, and one shows that $\prod_{i \in \z^+} M_i = \prod_{i \in \N} \z$ is not a free $\z$-module. The proof of this involves some set theory, e.g., countable unions of countable sets are countable sets and $\{(a_i)_{i \in \N} \mid a_i \in \{0,1\}\}$ is uncountable. In (c) at the end, it should read "depending on $\ov{x}$".

\item[(d)] For $2 \leq n \in \N$, $\z / n\z$ is not a free $\z$-module since $n x = 0$ for all $x\ in \z / n\z$.

\item[(e)] $(\Q,+)$ is not a free $\z$-module since
	\begin{enumerate}
	\item[(i)] $\Q$ is not finitely generated (in particular, is not cyclic)
	\item[(ii)] Any $S \sub \Q$ with $|S| \geq 2$ is $\z$-linearly dependent.
	\end{enumerate}
\end{enumerate}
\end{example}

\vs

\begin{proposition}
For $S = \{m_i \mid i \in I\} \sub M$, the following are equivalent:
\begin{enumerate}
\item[(i)] $S$ is an $R$-basis of $M$.
\item[(ii)] The $R$-module homomorphism $\phi : \bigoplus_{i \in I} R \ra M, (r_i)_{i \in I} \mapsto \sum_{i \in I} r_i m_i$, is an isomoprhism.
\item[(iii)] Every element in $M$ has a unique representation of the form $m = \sum_{i \in I} r_i m_i$ where $r_i \in R$ and $r_i = 0$ for almost all $i \in I$.
\item[(iv)] (Universal Property of Free Modules) For any $R$-module $N$ and any map $f : S \ra N$, there exists a unique homomorphism $\phi : M \ra N$ such that $\phi|_S = f$. \quad
\begin{tikzcd}
S \arrow[hookrightarrow]{r}
\arrow[swap]{d}{f} & M \arrow[dotted]{dl}{\exists \ ! \ \phi} \\
N
\end{tikzcd}
\end{enumerate}
\end{proposition}

\vs

\begin{corollary}\
\begin{enumerate}
\item[(a)] Every free $R$-module $M$ is isomorphic to $\bigoplus_{i \in I} R$ for some set $I$.
\item[(b)] If $M,M'$ are two free $R$-modules with bases $S,S'$ such that $|S| = |S'|$, then $M \cong M'$.
\end{enumerate}
\end{corollary}

\vs

\begin{remark}[Rank of a Free $R$-module]
If $M$ is a free $R$-module and $B,B'$ are two bases of $M$, it is not generally true that $|B| = |B'|$. As an example, see Exercise 27 on page 358 of [DF], which provides a ring $R$ such that $R \cong R^n$ for all $n \in \N$. However, for a free $R$-module $M$ with bases $B,B'$, one can conclude that $|B| = |B'|$ in the following cases:
\begin{enumerate}
\item[(i)] $R$ is a skew field.
\item[(ii)] $R$ is commutative (Exercise)
\item[(iv)] If $B$ or $B'$ is infinite (Exercise: $B$ infinite $\implies$ $M$ is not finitely generated).
\end{enumerate}

In each of these cases, we can well-define the \tb{rank} of $M$ to be $\tn{rk}_R(M) = |B|$ for any basis $B$ of $M$. E.g., $\tn{rk}_n(R^n) = n$ if $R$ is commutative. If $M$ is not finitely generated, we simply write $\tn{rk}_R(M) = \infty$.
\end{remark}

\end{document}
