\documentclass[11pt]{book}

\usepackage[margin=1.0in]{geometry}

\usepackage{amsmath,amssymb,amsthm}

\usepackage{mathrsfs}
%\usepackage{dsfont}
\usepackage{times}
\usepackage{epsfig}
\usepackage{enumitem}
\usepackage{mathabx}

\usepackage{csquotes}
\MakeOuterQuote{"}
\usepackage{graphicx}
\usepackage{caption}
\usepackage{subcaption}
\usepackage{setspace}

\usepackage{tikz}
\usepackage{tikz-cd}
\usepackage{pgfplots}

\usepackage{etoolbox,xparse}
\usepackage{float}
\floatstyle{plain}%boxed
\restylefloat{figure}
\usepackage{chngcntr}
\counterwithin{chapter}{part}

\newcounter{counter}


\newtheorem{theorem}[counter]{Theorem}   \newtheorem*{theorem*}{Theorem}   \newtheorem{lemma}[counter]{Lemma}   \newtheorem{corollary}[counter]{Corollary}
\newtheorem{proposition}[counter]{Proposition}   \newtheorem{problem}[counter]{Problem}   \newtheorem*{proposition*}{Proposition}   \newtheorem*{lemma*}{Lemma}

\theoremstyle{definition}   \newtheorem{defn}[counter]{Definition} %These theorem environments are not numbered separately 
\newtheorem*{defn*}{Definition}
\newtheorem{blank}[counter]{}   \newtheorem{remark}[counter]{Remark(s)}   \newtheorem*{remark*}{Remark(s)}   \newtheorem{generalization}[counter]{Generalization}
\newtheorem{consequence}[counter]{Consequence(s)}   \newtheorem*{consequence*}{Consequence(s)}   \newtheorem*{problem*}{Problem}   \newtheorem{notation}[counter]{Notation}
\newtheorem*{notation*}{Notation}   \newtheorem{example}[counter]{Example(s)}   \newtheorem*{example*}{Example(s)}   \newtheorem*{warning}{Warning} \newtheorem*{corollary*}{Corollary}
\newtheorem*{question}{Question}   \newtheorem*{answer}{Answer}   \newtheorem{modification}[counter]{Modification}   \newtheorem{numitem}[counter]{}

\newcommand{\ov}{\overline}   \newcommand{\wt}{\widetilde}
\newcommand{\blt}{$\bullet$}   \newcommand{\tn}{\textnormal}   \newcommand{\tb}{\textbf}   \newcommand{\mbb}{\mathbb}
\newcommand{\bs}{\backslash}   \newcommand{\A}{\mathcal{A}}   \newcommand{\sy}{\textnormal{Syl}}   \newcommand{\size}[1]{\left| #1 \right|}
\newcommand{\zx}[1]{(\z/#1\z)^{\times}}   \newcommand{\zn}[1]{\z/#1\z}   \newcommand{\pr}[1]{\textbf{Problem #1.}}   \newcommand{\abc}{(\alph*)}
\newcommand{\nsg}{\mathrel{\unlhd}}   \newcommand{\ind}{\parindent24pt}   \newcommand{\vn}{\varnothing}
\newcommand{\ve}{\varepsilon}   \newcommand{\im}{\textnormal{im }}   \newcommand{\re}{\textnormal{Re }}   \newcommand{\mb}[1]{\mathbf{#1}}
\newcommand{\lra}{\leftrightarrow}   \newcommand{\0}{\mathbf{0}}   \newcommand{\mc}[1]{\mathcal{#1}}   \newcommand{\hra}{\hookrightarrow}   \newcommand{\hla}{\hookleftarrow}
\newcommand\myeq{\mathrel{\overset{\makebox[0pt]{\mbox{\normalfont\tiny\sffamily \textrm{def}}}}{=}}}
\newcommand\lheq{\mathrel{\overset{\makebox[0pt]{\mbox{\normalfont\tiny\sffamily \textrm{L'H}}}}{=}}}
\newcommand{\mymatrix}[2]{\left( \begin{array}{#1} #2 \end{array} \right)}   \newcommand{\myvec}[1]{\left( \begin{array}{c} #1 \end{array} \right)}

\newcommand{\hm}{homomorphism}   \newcommand{\hms}{homomorphisms}   \newcommand{\iso}{isomorphism}
\newcommand{\isos}{isomorphisms}   \newcommand{\auto}{automorphism}   \newcommand{\autos}{automorphisms}   \newcommand{\ds}[2]{#1^{(#2)}}   \newcommand{\lcs}[2]{#1^{[#2]}}
\newcommand{\pf}[1]{$ #1 = p_1^{e_1} p_2^{e_2} \cdots p_r^{e_r}$, with $p_1,p_2,\dots,p_r$ distinct primes and $e_1,e_2,\dots,e_r\in \N$}
\newcommand{\inn}{\textnormal{Inn}}   \newcommand{\out}{\textnormal{Out}}   \newcommand{\car}{\textnormal{ char }}   \newcommand{\id}{\textnormal{id}}   \newcommand{\triv}{\{e\}}
\newcommand{\tl}{\triangleleft}   \newcommand{\sd}[1]{\rtimes_{#1}}   \newcommand{\x}{^{\times}}   \newcommand{\cyc}[1]{\begin{pmatrix} #1 \end{pmatrix}}
\newcommand{\gen}[1]{\langle #1 \rangle}   \newcommand{\stab}[2]{\tn{Stab}_{#1}(#2)}   \newcommand{\fix}[2]{\tn{Fix}_{#1}(#2)}
\newcommand{\hooklongrightarrow}{\lhook\joinrel\longrightarrow}   \newcommand{\twoheadlongrightarrow}{\relbar\joinrel\twoheadrightarrow}
\newcommand{\ses}[5]{1 \longrightarrow #1 \overset{#2}{\hooklongrightarrow} #3 \overset{#4}{\twoheadlongrightarrow} #5 \longrightarrow 1}
\renewcommand{\arrowvert}{\arrow}   \renewcommand{\i}{\mathbf{i}}   \renewcommand{\j}{\mathbf{j}}   \renewcommand{\k}{\mathbf{k}}    \renewcommand{\H}{\mathbb{H}}  \renewcommand{\hom}{\tn{Hom}}   
\newcommand{\ann}[2]{\textnormal{Ann}_{#1}(#2)}


\DeclareMathOperator{\pow}{\mathcal{P}}   \DeclareMathOperator{\C}{\mathbb{C}}
\DeclareMathOperator{\col}{\mathcal{C}}   \DeclareMathOperator{\F}{\mathbf{F}}   \DeclareMathOperator{\h}{\mathbb{H}}   \DeclareMathOperator{\bF}{\mathbf{F}}
\DeclareMathOperator{\R}{\mathbb{R}}   \DeclareMathOperator{\N}{\mathbb{N}}   \DeclareMathOperator{\z}{\mathbb{Z}}   \DeclareMathOperator{\Q}{\mathbb{Q}}
\DeclareMathOperator{\ra}{\rightarrow}   \DeclareMathOperator{\Poly}{\mathbf{P}}   \DeclareMathOperator{\spn}{\textnormal{span}}   \DeclareMathOperator{\aut}{\textnormal{Aut}}
\DeclareMathOperator{\znx}{(\z/n/Z)^{\times}}   \DeclareMathOperator{\lcm}{\textrm{lcm}}   \DeclareMathOperator{\inv}{^{-1}}   \DeclareMathOperator{\sub}{\subseteq}

\newenvironment{prf}{\paragraph{\textit{Proof}}}{\hfill$\square$}

\newcommand{\vs}{\vspace{8pt}}
\newcommand{\hs}{\hspace{8pt}}

\numberwithin{counter}{chapter}

\title{Algebra}
\author{ Transcribed from the lectures of \\ \ \\ Professor Peter Abramenko \\ \ \\ University of Virginia}
\date{Fall 2016}


\begin{document}

\frontmatter
\maketitle
\tableofcontents
\mainmatter

\ \vspace{100pt}
\begin{quote}
\centering
\textit{A study of groups, rings, fields, modules, tensor products, and multilinear functions.}
\end{quote}

\part{Group Theory}


\part{Commutative Rings}


\part{Modules}

\chapter{Basic Concepts}

In the following, $R$ always denotes a (not necessarily commutative) ring with 1. 

\vs

\begin{defn}
An abelian group $(M,+)$ together with a map $R \times M \ra M$, $(r,m) \mapsto rm$, is called a \tb{left $\mb{R}$-module} if
\begin{enumerate}
\item[(1)] $(r+s)m = rm + sm$
\item[(2)] $r(m+n) = rm + rn$
\item[(3)] $(rs)m = r(sm)$
\item[(4)] $1 m = m$
\end{enumerate}
for all $r,s \in R$ and $m,n \in M$. It follows from these conditions that
\begin{enumerate}
\item[$\bullet$] $0_R m = 0_M$
\item[$\bullet$] $r0_m = 0_m$
\item[$\bullet$] $(-1)m = -m$
\end{enumerate}
for all $m \in M$, $r \in R$.
\end{defn}

\vs

\begin{example}\ 
\begin{enumerate}
\item[(a)] If $R$ is a field, then the $R$-modules are precisely the $R$-vector spaces.
\item[(b)] Any abelian group $(M,+)$ is a (left) $\z$-module with the scalar multiplication defined by $k m = m + m + \dots + m $ ($k$ times) and $-k m = -m -m - \dots - m $ ($k times$). 
\item[(c)] $\{0\}$ is an $R$-module for any ring $R$.
\item[(d)] $R$ is (left) $R$-module. 
\item[(e)] For $n \in \N$, and $R^n := \left\{ \myvec{r_1 \\ \vdots \\ r_n} \bigg| r_i \in R \right\}$ with componentwise addition and scalar multiplication is a left $R$-module. 

For (f) and (g) below, $F$ denotes a field and $V = F^n$.

\item[(f)] $V$ is a $M_n(F)$-module with matrix multiplication.
\item[(g)] Fix $A \in M_n(F)$. Then $V$ is an $F[x]$-module with $f(x) v := f(A) v$. We denote this module by $_A V$.
\end{enumerate}
\end{example}

\vs

From now on, $M$ denotes an $R$-module.

\vs

\begin{defn}
A subgroup $N$ of $M$ is called an ($R$-)submodule of $M$, denoted $N \leq M$, if $rn \in N$ for all $r \in R$ and $n \in N$. A subset $N \sub M$ is therefore a submodule if
\begin{enumerate}
\item[(1)] $0_m \in N$
\item[(2)] $n + n' \in N$ for all $n,n' \in N$
\item[(3)] $rn \in N$ for all $r \in R$, $n \in N$
\end{enumerate}
A submodule of the left $R$-module $R$ is called a \tb{left-ideal} of $R$. If $R$ is commutative, then any left ideal is an ideal. 
\end{defn}

\vs

\begin{defn}
Let $S \sub M$. Define the \tb{annihilator of $\mb{S}$ in $\mb{R}$} by
	\[\ann{R}{S} := \{ r \in R \mid r s = 0 \ \forall \ s \in S \} \leq R \]
If $S \leq M$, then $\ann{R}{S}$ is a two-sided ideal of $R$. Indeed, for $a \in \ann{R}{S}, x \in S, r \in R$, $(ar)x = a(rx) = 0$ and $(ra)x = r(ax) = r0 = 0$, so $ar,ra \in \ann{R}{S}$. 
\end{defn}

\vs

\begin{defn}
Let $M,N$ be $R$-modules. A map $\phi : M \ra N$ is called an \tb{$\mb{R}$-module homomorphism} or \tb{$\mb{R}$-linear} if $\phi(m+m') = \phi(m) + \phi(m')$ and $\phi(rm) = r\phi(m)$ for all $m,m' \in M$ and $r \in R$. Clearly, $\ker \phi := \phi\inv(0_N) \leq M$, $\im \phi \leq N$, and $\phi$ is injective if and only if $\ker \phi = \{0_M\}$. $\phi$ is called an \tb{$\mb{R}$-module isomorphism} if $\phi$ is bijective, in which case $\phi\inv : N \ra M$ is also an $R$-module isomorphism. 
\end{defn}

\vs

\begin{lemma}
Let $\phi : M \ra N$ be $R$-linear.
\begin{enumerate}
\item[(a)] If $M' \leq M$, then $\phi(M') \leq N$, and if $N' \leq N$, then $\phi\inv(N') \leq M$. 
\item[(b)] (Correspondence Theorem) Assume $\phi$ is surjective. Then $\phi$ induces a bijection $\{M' \leq M \mid \ker \phi \leq M\} \ra \{N' \leq N\}, M' \mapsto \phi(M')$ with inverse $N' \mapsto \phi\inv(N')$.
\end{enumerate}
\end{lemma}

\begin{proof}
(a) is straightforward. With (a) established, the induced maps in (b) are well-defined. If $N' \leq N$, then $\phi \phi\inv(N') \sub N'$ (always). To show reverse inclusion, let $x \in N'$. There exists $y \i m$ with $x = \phi(y) \implies y \in \phi\inv(N') \implies x \in \phi \phi\inv(N')$. 

Now, let $\ker \phi \leq M' \leq M$. $M' \sub \phi\inv \phi(M')$ always. To show reverse inclusion, let $x \in \phi\inv \phi(M')$. Then $\phi(x) \in \phi(M')$, so there exists $m \in M'$ with $\phi(x) - \phi(m) \implies x-m \in \ker \phi \leq M' \implies x \in M'$.
\end{proof}

\vs

\begin{defn}
If $N \leq M$, the \tb{quotient module} $M/N$ is the abelian quotient group $M/N$ with scalar multiplication defined by $r(m+N) := rm + N$ (well-defined). The map $\pi : M \ra M/N$, $m \mapsto m+N$, is an $R$-linear surjection. 
\end{defn}

\vs

\begin{corollary}
Each submodule $L$ of $M/N$ can be uniquely written in the form $L = M'/N$ for some $N \leq M' \leq M$, namely $M' = \pi\inv(L)$.
\end{corollary}

\vs

\begin{proposition}[Isomorphism Theorem for Modules]\ 
\begin{enumerate}
\item[(a)] If $\phi : M \ra M'$ is $R$-linear, then $M/\ker \phi \cong \im \phi$ via $m + \ker \phi \mapsto \phi(m)$.
\item[(b)] If $L,N \leq M$, then $L+N := \{\ell + n \mid \ell \in L, n \in N\} \leq M$, and $(L+N)/N \cong L / L \cap N$ (consider $L \ra (L+N)/N, \ell \mapsto \ell + N$). 
\item[(c)] If $N \leq L \leq M$, then $(M/N)/(L/N) \cong M/L$ ($M/N \ra L/N$, $m+N \mapsto m+L$).
\end{enumerate}
\end{proposition}

\vs

\begin{defn}\ 
\begin{enumerate}
\item[(a)] For $R$-modules $M,N$, we set $\hom_R(M,N) := \{f : M \ra N \mid f \tn{ is $R$-linear}\}$. Check: $\hom_R(M,N)$ is an abelian group with $(f+g)(m) := f(m) + g(m)$ for $f,g \in \hom_R(M,N)$. If $R$ is commutative, then $\hom_R(M,N)$ is an $R$-module with $(rf)(m) := r f(m)$. 

\item[(b)] $\tn{End}_R(M) := \hom_R(M,M)$ is a ring with $(f \cdot g)(m) := f(g(m))$ and is called the \tb{endomorphism ring of $\mb{M}$}. 
\end{enumerate}
\end{defn}


\end{document}