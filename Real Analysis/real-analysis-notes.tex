\documentclass[11pt,leqno,oneside]{amsbook}

\usepackage{../notes}
\numberwithin{thm}{section}
\renewcommand{\P}{\mathcal{P}}
\renewcommand{\A}{\mathcal{A}}
\newcommand{\M}{\mathcal{M}}
\newcommand{\F}{\mathcal{F}}
\newcommand{\Ep}{\mathcal{E}}
\newcommand{\Top}{\mathcal{T}}
%%%%%%%%%%%%%% BEGIN CONTENT: %%%%%%%%%%%%%%

\title[Real Analysis]{Real Analysis}
\author{George H. Seelinger (inspired from class by Abdelmalek Abdesselam)}
\date{Spring 2017}
\begin{document}
\maketitle
\section{Measures}
\subsection*{(1/19/2017) Lecture 1}
Riemann integration is great, but it is insufficient for many classes
of integrands. Lebegue integration seeks to solve this problem and is
the main topic of the first part of this course. The main technique of
Riemann integration is to slice the domain into intervals, resulting in vertical rectangles, and add up
their area whereas Lebegue integration slices the target and then
looks at the parts of the domain that hit the target, say some set $A$
and calculates the height of the target slice times the size of the set
$A$. However, such a calculation requires a notion of measure, which
leads us to the topic of measure theory. There are two main questions
we ask ourselves.
\begin{thm}[Question]
  For what kind of sets do we have a reasonable notion of size? This
  leads us to the idea of measurable sets.
\end{thm}
\begin{thm}[Question]
  What is that notion of size? This leads us to the idea of measures.
\end{thm}
\subsection{$\sigma$-algebras}
\begin{defn}
Let $X \neq \varnothing$ be some set and $\P(x)$ be the set of all
subsets of $X$ (called the power set). Furthermore, let $\A \subset
\P(X)$ with $\A \neq \varnothing$. Then, $\A$ is called an \emph{algebra} of
sets on $X$ if $\A$ is closed under finite unions and taking
complements.
\end{defn}
\begin{lem}
  $\varnothing, X \in \A$. This follows by taking a set in $\A$ and
  unioning it with its complement to get $X$ and then taking the
  complement of $X$.
\end{lem}
\begin{defn}
  $\A \subset \P(X)$ is a \emph{$\sigma$-algebra} on $X$ if and only if
  $\A$ is an algebra of sets which is closed under countable
  unions. That is, for all sequences $(A_n)_{n \geq 0}$ of elements of $\A$,
  $\cup_{n=0} A_n \in \A$.
\end{defn}
\begin{thm}
  $\A$ is a $\sigma$-algebra iff the following are satisfied.
  \begin{enumerate}
    \item $\varnothing \in \A$,
    \item for all nonempty countable sequences $(A_n)_{n \geq 0}$ such that $A_n \in \A$, we have
      $\cup_{n=0}^\infty A_n \in \A$.
    \item for all $A \in \A$, $A^c \in \A$,
  \end{enumerate}
\end{thm}
\begin{example}
  Let $X$ be a set. Then
  \begin{enumerate}
  \item $\A = \{\varnothing,X\}$ is a $\sigma$-algebra.
  \item $\A = \P(X)$ is a $\sigma$-algebra.
  \item if $X = \{1,2,3\}$, we have that $\A =
    \{\varnothing,X,\{1\},\{2,3\}\}$ is a $\sigma$-algebra.
  \end{enumerate}
\end{example}
\begin{prop}
  If $\A$ is a $\sigma$-algebra of $X$, then $\A$ is closed under
  countable intersections because \[
    \bigcap_{n=0}^\infty A_n = \left( \bigcup_{n=0}^\infty A_n^c \right)^c
  \]
\end{prop}
\begin{defn}
  Let $X$ be a set. Suppose $\Ep \subset \P(A)$. $\M(\Ep)$ is the
  smallest $\sigma$-algebra on $X$ which contains $\Ep$.  It is called the
  \emph{$\sigma$-algebra generated by $\Ep$}.
\end{defn}
Of course, this leaves us to ask, does this even exist and if so, is
it unique? We must prove it.
\begin{proof}
  Take $\mathcal{Z} = \{\A \subset \P(x) \mid \A \text{ a }
  \sigma\text{-algebra and } \Ep \subset \A\}$. Then $\mathcal{Z}
  \subset \P(\P(X))$ and take a partial order on $\mathcal{Z}$ as
  containment. Now, we need to show that there exists a smallest
  element of $\mathcal{Z}$. Let us define \[
    \M(\Ep) = \bigcap_{\A \in \mathcal{Z}} \A
  \]
  Then we will show $\M(\Ep)$ is a $\sigma$-algebra.
  \begin{enumerate}
  \item We note that $\mathcal{Z}$ is nonempty because $\P(X) \in \mathcal{Z}$.
  \item For all $\A \in \mathcal{Z}$, $\A$ is a $\sigma$-algebra, so
    $\varnothing \in \A$. This gives us that $\varnothing \in \cap_{\A
      \in \mathcal{Z}} \A = \M(\Ep)$.
  \item Let $E \in \M(\Ep)$. Then, for all $\A \in \mathcal{Z}$, we have that
    $E \in \A$. Thus, $E^c \in \A$ so $E^c \in \cap_{\A \in \mathcal{Z}} \A$.
  \item Suppose $(A_n)_{n \geq 0}$ and $\forall n, A_n \in
    \M(\Ep)$. Then, for all $\A \in \mathcal{Z}$, we have that $\forall n \in
    A_n \in \A \implies \cup_{n \geq 0} A_n \in \A$ for all $\A
    \in \mathcal{Z}$. Thus, $(\cup_{n \geq 0} A_n) \in \cap_{\A \in \mathcal{Z}} \A$.
  \end{enumerate}
  Thus, $\M(\Ep)$ is a $\sigma$-algebra. Moreover, for all $\A \in
  \mathcal{Z}$, we have that $\Ep \subset \A$ and so $\Ep \subset \cap_{\A \in
    \mathcal{Z}} \A = \M(\Ep)$. Thus, $\M(\Ep) \in \mathcal{Z}$. \\

  Now, we show it is the smallest element. If $\mathscr{B} \in
  \mathcal{Z}$, then clearly $\M(\Ep) = \cap_{\A \in \mathcal{Z}} A
  \subset \mathscr{B}$. Thus, $\M(\Ep)$ is the smallest element of
  $\mathcal{Z}$.
\end{proof}
\begin{example}
  Let $X = \{1,2,3\}$ and $\Ep = \{\{1\}\}$. Then, $\M(\Ep) = \{X,
  \varnothing, \{1\},\{2,3\}\}$.
\end{example}
\begin{defn}
  Given $X$ a topological space and $\Ep$ the set of all open subsets of $X$,
  then $\B_X := \M(\Ep)$ is the \emph{Borel $\sigma$-algebra of $X$}.
\end{defn}
\begin{defn}
  Given,
  \begin{itemize}
  \item a set $X$,
  \item a family of sets $(X_\alpha)_{\alpha \in A}$,
  \item a family of functions $(f_\alpha)_{\alpha \in A}, f_\alpha: X
    \to X_\alpha$,
  \item a family of $\sigma$-algebras $(\M_\alpha)_{\alpha \in A}$ such
    that for all $\alpha$, $\M_\alpha$ is a $\sigma$-algebra on
    $X_\alpha$,
  \end{itemize}
  the \emph{$\sigma$-algebra on $X$ generated by all of this data} is
  $\M := \M(\Ep)$ where $\Ep = \{f_\alpha^{-1}(z) : \alpha \in A, z \in
  \M_\alpha\}$.
\end{defn}
\begin{example}
  Let $X = \{0,1\}^3$, $A = \{1,3\}$, and $X_1 = X_3 =
  \{0,1\}$. Furthermore, let $f_1\colon  X \to X_1$ be given by
  $(x_1,x_2,x_3) \mapsto x_1$ and similarly let $f_3\colon  X \to X_3$ be given by $(x_1,x_2,x_3) \mapsto x_3$. Finally, let
  $\M_1 = \M_3 = \P(\{0,1\})$. Then, the $\sigma$-algebra defined
  above will yield the set of sets of tuples indexed by the first and
  third coordinates. In other words, this set is in bijection with the
  set $\P(\{0,1\}^2)$.
\end{example}
\subsection*{(1/23/2017) Lecture 2}
\begin{example}[Product $\sigma$-algebra]
  Let $(x_\alpha)_{\alpha \in A}$ be an indexed family of sets as
  described above. Then, for each $\alpha \in A$, let $M_\alpha$ be a
  $\sigma$-algebra on $X_\alpha$. Given \[
    X = \prod_{\alpha \in A} X_\alpha
  \]
  we take $f_\alpha = \pi_\alpha$, that is, the projection function onto the $\alpha$ component. \[
    \pi_\alpha\colon  X \to X_\alpha
  \] \[
    (x_\beta)_{\beta \in A} \mapsto x_\alpha.
  \]
  This gives a $\sigma$-algebra on $X$ called the \emph{product
    $\sigma$-algebra}, denoted \[
    \bigotimes_{\alpha \in A} \M_\alpha.
  \]
  In more concise terms, we say that \[
    \bigotimes_{\alpha \in A} \M_\alpha = \M(\Ep), \ \Ep =
    \{\pi_{\alpha}^{-1}(E_\alpha) \mid \alpha \in A, E_\alpha \in
    \M_\alpha \}.
  \]
  Also, to clear up any ambiguity, we note that \[
    \pi_\alpha^{-1}(E_\alpha) = E_\alpha \times \prod_{\beta \in A
      \setminus \{\alpha\}} X_\alpha.
  \]
\end{example}
\begin{example}
  If $A = \{1, \ldots, n\}$ and $X = X_1 \times \cdots \times X_n$,
  then \[
    \bigotimes_{\alpha \in A} \M_\alpha = \M_1 \otimes \cdots \otimes \M_n
  \]
\end{example}
\begin{prop}
  If $A$ is at most countable, then $\bigotimes_{\alpha \in A}
  \M_\alpha$ is generated by $\prod_{\alpha \in A} E_\alpha$ where
  $E_\alpha \subset \M_\alpha$.
\end{prop}
\begin{proof}
  Let \[
    \Ep_1 = \{\pi_\alpha^{-1}(E_\alpha) \mid \alpha \in A, E_\alpha
    \in \M_\alpha\}
  \]
  and \[
    \Ep_2 = \{\prod_{\alpha \in A} E_\alpha \mid (E_\alpha)_{\alpha
      \in A} \text{ family of set such that } \forall \alpha \in A,
    E_\alpha \in \M_\alpha \}.
  \]
  By definition, \[
    \bigotimes_{\alpha \in A} \M_\alpha = \M(\Ep_1)
  \]
  and we seek to show $\M(\Ep_1) = \M(\Ep_2)$.
  \begin{itemize}
  \item ($\subset$) It is enough to show $\Ep_1 \subset \Ep_2 \subset
    \M(\Ep_2)$. Let $\alpha \in A$, $E_\alpha \in \M_\alpha$. Then \[
      \pi_\alpha^{-1}(E_\alpha) = \prod_{\beta \in A} F_\beta
    \]
    where $F_\beta = E_\alpha$ if $\beta = \alpha$ and $F_\beta =
    X_\alpha$ if $\beta \neq \alpha$. Thus, $\Ep_1 \subset \Ep_2$

  \item ($\supset$) It is enough to show that $\Ep_2 \subset
    \M(\Ep_1)$. Let $Y = \prod_{\alpha \in A} E_\alpha \in \Ep_2$
    where, for all $\alpha \in A$, $E_\alpha \in \M_\alpha$. Because
    $A$ is at most countable, we have that \[
      Y = \bigcap_{\alpha \in A} \pi_\alpha^{-1}(E_\alpha).
    \]
    So, if $x = (x_\alpha)_{\alpha \in A}$, then $x \in
    \pi_\alpha^{-1}(E_\alpha) (\in \Ep_1) \ \iff \ x_\alpha \in
    E_\alpha$. Thus, $Y \in \M(\Ep_1)$ because $Y$ is (at most) countable
    intersection of elements of $\M(\Ep_1)$, which is a $\sigma$-algebra.
  \end{itemize}
\end{proof}
\begin{prop}
  Suppose we have $(x_\alpha)_{\alpha \in A}$ and $\Ep_\alpha \subset
  \P(X_\alpha)$. Let $\M_\alpha = \M(\Ep_\alpha)$ in $X_\alpha$. Then
  \begin{itemize}
  \item $\bigoplus_{\alpha \in A} \M_\alpha = \M(\F_1)$ where $\F_1 =
    \{\pi_\alpha^{-1}(E_\alpha) \mid \alpha \in X, E_\alpha \in \Ep_\alpha\}$
  \item If, in addition, $A$ is at most countable, then \\
  $\bigoplus_{\alpha \in A} \M_\alpha = \M(\F_2)$ where $\F_2 = \{\prod_{\alpha \in A} E_\alpha \mid \forall \alpha
    \in A, E_\alpha \in \Ep_\alpha\}$
  \end{itemize}
\end{prop}
\begin{proof}
  The proof of this proposition is omitted. One can refer to
  proposition 1.4 on page 23 of Folland to get an idea of how to prove this.
\end{proof}
\begin{prop}
  Let $X_1, \ldots, X_n$ be metric spaces and \[
    X = \prod_{j=1}^n X_j.
  \]
  Then,
  \begin{enumerate}
  \item $\B_{X_j} \subset \B_X$.
  \item If, in addition, the $X_j$'s are seperable, then \[
      \bigotimes_{j=1}^n \B_{X_j} = \B_X.
    \]
    Recall that a space $Y$ is separable if and only if $Y$ has a
    countable dense subset.
  \end{enumerate}
\end{prop}
\begin{cor}
  \[
    \bigotimes_{j=1}^n \B_\R = \B_{\R^n}
  \]
\end{cor}
\begin{proof}[Proof of Corollary]
  $\R$ is seperable.
\end{proof}
\begin{proof}[Proof of Proposition]
  The proof of this proposition is also ommitted but can be found
  under the proof of Proposition 1.5 on page 23 of Folland.
\end{proof}
\subsection*{(1/24/2017) Lecture 3}
\subsection{Point-Set Topology}
\begin{defn}
  Let $X$ be a set. A \emph{topology} on $X$ is a set of ``open sets''
  $\Top \in \P(X)$ such that
  \begin{itemize}
  \item $\varnothing, X \in \Top$.
  \item $\Top$ is stable under finite intersections.
  \item $\Top$ is stable by arbitrary unions.
  \end{itemize}
  We call the pair $(X,\Top)$ a \emph{topological space}.
\end{defn}
\begin{rmk}
  If $(X,d)$ is a metric space, there exists a canonically defined
  topology, $\Top(d)$, given by the characterization that \[
    U \in \Top(d) \ \iff \ \forall x \in U, \exists r > 0 \text{ such
      that } B(x,r) \subset U.
  \]
\end{rmk}
If $d_1,d_2$ are two metrics on $X$, we say that $d_1$ and $d_2$ are
equivalent (topologically) if and only if $\Top(d_1) = \Top(d_2)$.
\begin{example}
  On $\R^n$, let \[
    d_1(x,y) = \max_{1 \leq i \leq n} \left| x_i - y_i \right|
  \]
  and \[
    d_2(x,y) = \sqrt{\sum_{i=1}^n (x_i-y_i)^2}.
  \]
  Then, $d_1(x,y) \leq d_2(x,y) \leq \sqrt{n} d_1(x,y)$ for all $x,y
  \in \R^n$ implies that $\Top(d_1) = \Top(d_2)$. We can see this
  because if $U \in \Top(d_1)$ with $x \in U$, then there is an $r$
  such that $B_2(x,r) \subset B_1(x,r) \subset U$ and thus $U \in
  \Top(d_2)$. Conversely, if $U \in \Top(d_2)$ with $x \in U$, then
  there is an $r$ such that $B_1(x,\frac{r}{\sqrt{n}}) \subset
  B_2(x,r) \subset U$.
\end{example}
Now, let us consider the product of topological spaces.
\begin{defn}
  Let $(X_\alpha, \Top_\alpha)_{\alpha \in A}$ be a family of
  topological spaces. Then, if $X = \prod_{\alpha \in A} X_\alpha$,
  there is a standard topolgy $\Top$ on $X$ called the \emph{product
    topology}. Let $\pi_\alpha\colon  X \to X_\alpha$ be given by
  $(x_\beta)_{\beta \in A} \mapsto x_\alpha$. Then, $\Top$ is the
  weakest (coarsest) topology such that for all $\alpha \in A$,
  $\pi_\alpha$ is continuous.
\end{defn}
In keeping with constructions similar to $\sigma$-algebras, we note
the following
\begin{prop}
  If $Y$ is a set and $\Ep \subset \P(Y)$, then there is a smallest (coarsest)
topology $\Top(\Ep)$ on $Y$ which contains $\Ep$.
\end{prop}
\begin{proof}
  The proof of this fact is relatively straight forward and can be
  found in almost any point-set topology reference.
\end{proof}
\begin{defn}
  Let $(X,\Top)$ and $(X',\Top')$ be two topological spaces. A map $f\colon
  X \to X'$ is called \emph{continuous} if and only if for all $U' \in
  \Top'$, $f^{-1}(U') \in \Top$.
\end{defn}
Given this definition, we can come up with a more concise description
of the product topology as follows.
\begin{defn}[Concise definition of product topology]
  Given the same setup as the previous definition of product topology,
  $\Top$ must be such that, for all $\alpha$ and $U_\alpha \in
  \Top_\alpha$, \[
    \pi_\alpha^{-1}(U_\alpha) \in \Top
  \]
  Thus, we say that the product topology is $\Top = \Top(\Ep)$
  where \[
    \Ep = \{\pi_{\alpha}^{-1}(U_\alpha) \mid \alpha \in A, U_\alpha
    \in \Top_\alpha\}
  \]
\end{defn}
\begin{rmk}
  Let $(X_1,d_1), \ldots, (X_n,d_n)$ be a finite collection of metric
  spaces. The product distance on $V = X_1 \times \cdots \times X_n$
  is given by \[
    d(x,y) = \min_{1 \leq i \leq n} d(x_i,y_i)
  \]
  Then, $\Top(d)$ is the product topology coming from
  $(X_1,\Top(d_1)), \ldots, (X_n,\Top(d_n))$. The proof of this is an
  exercise.
\end{rmk}
Now, let us restrict ourselves to the countable product case. Let
$(X_n,d_n)_{n \geq 1}$ be a sequence of metric spaces and
$(X_n,\Top(d_n))$ be a topological space. Then, we can consider \[
  X = \prod_{n=1}^\infty X_n
\]
with $\Top$ the product topology. Then, we want to find $d$ the
distance on $X$ such that $\Top(d) = \Top$. The metric has to be such
that, given any $x = (x_n)_{n \geq 1}$ and $y = (y_n)_{n \geq 1}$,
$d(x,y)$ will actually converge. Thus, we let \[
  d(x,y) = \sum_{n=1}^\infty 2^{-n} \min(1,d_n(x_n,y_n))
\]
\begin{lem}
  If $(X,d)$ is a metric space, then \[
    \tilde{d}(x,y) = \min(1,d(x,y))
  \]
  is also a metric and, moreover, they define the same topology.
\end{lem}
\begin{proof}
  The proof of this lemma is relatively straight-forward, but
  conceptually follows from the fact that topologies encode local
  information, and locally these metrics agree. One also has to show
  that this is actually a metric, but this also is not much work.
\end{proof}
\begin{prop}
  Let $(X_n,d_n)_{n \geq 1}$ be a sequence of metric spaces. Then, $U
  \in \Top$ is an open set in $X = \prod_{n=1}^\infty X_n$ if and only
  if for all $x \in U$, there exist $k \geq 0$ and $i_1, < \cdots <
  i_k \in \N$ and $r > 0$ such that \[
    B = \{y \in X \mid d(x_{i_1},y_{i_1}) < \cdots <
    d(x_{i_k},y_{i_k}) < r\} \subset U
  \]
\end{prop}
\begin{proof}
  For the proof, one notes that a $B$ of this form can also be written \[
    B = \pi_{i_1}^{-1}(B(x_i,r)) \cap \cdots \cap \pi_{i_k}^{-1}(x_{i_k},r)
  \]
  which is a finite intersection. Thus, $B \in \Top$. Going the other
  direction, let $U \in \Top(d)$ and $x \in U$. Then, there exists an
  $r > 0$ such that \[
    \{y \in X \mid d(x,y) < r\} \subset U.
  \]
  Now, let $k$ be such that $\sum_{n=k+1}^\infty 2^{-n} <
  \frac{r}{2}$ and let $i_1 = 1, \ldots, i_k = k$. Then, pick an $s$
  such that $0 < s < 1$ and $s < r$. We can then set \[
    B = \{y \mid d_1(x_1,y_1) < \frac{s}{2}, \ldots, d_k(x_k,y_k) < \frac{s}{2} \}.
  \]
  Of $y \in B$, then \[
    d(x,y) = \sum_{n=1}^k 2^{-n} \min(1,d_n(x_n,y_n)) +
    \sum_{n=k+1}^\infty 2^{-n} \min(1,d_n(x_n,y_n)).
  \]
  By our construction, each $\min(1,d_n(x_n,y_n))$ in the first sum will be less than
  $\frac{r}{2}$ and each $\min(1,d_n(x_n,y_n)) \leq 1$. Thus, we get
  that \[
    d(x,y) < \sum_{n=1}^k 2^{-n} \frac{r}{2} + \sum_{n=k+1}^\infty
    2^{-n} < \frac{r}{2} + \frac{r}{2} = r
  \]
\end{proof}
\subsection*{(1/26/2017) Lecture 4}
\subsection{Measurable spaces}
\begin{defn}
  Given a set $X$ and a $\sigma$-algebra $\A$ on $X$, we call the pair
  $(X,\A)$ a \emph{measurable set}.
\end{defn}
In the language of categories, we have objects and the morphisms
between the objects. To put our discussion in context, here is a table
of some categories with their objects and morphisms.
\begin{center}
  \begin{tabular}{|c|c|c|}
    \hline && \\
    Category name& Objects & Morphisms \\
    \hline &&\\
    Linear algebra & Vector spaces & Linear maps \\
    Group theory & Groups & Group homomorphisms \\
    Topology & Topological spaces $(X,\Top)$ & Continuous maps \\
    Measurable sets & $(X,\A)$ & Measurable maps (\S 2.1 Folland) \\
    \hline
  \end{tabular}
\end{center}
This leads us to the question, what is a measurable map?
\begin{defn}
  Given two measurable spaces $(X,\A)$ and $(Y,\mathcal{B})$, we say
  that $f\colon  X \to Y$ is an $(\A,\mathcal{B})$ measurable map if and
  only if for all $z \in \mathcal{B}$, $f^{-1}(z) \in \A$.
\end{defn}
With this definition, we can now provide another characterization of
product $\sigma$-algebras like we did for product topologies.
\begin{prop}
  Let $X = \prod_{\alpha \in A} X_\alpha$ with $(X_\alpha,\M_\alpha)$
  a family of measurable spaces. Then $\bigotimes_{\alpha \in A} \M_\alpha$ is the
  smallest $\sigma$-algebra which makes all the natural maps
  $\pi_\alpha\colon  X \to X_\alpha$ measurable.
\end{prop}
At this point, we provide note that we are headed in the direction of
an idea called a measure space $(X,\A,\mu)$ where $(X,\A)$ is a
measurable space and $\mu$ is a ``measure'' defined on sets in
$\A$. Given a $f\colon  X \to \R$ that is measurable, we will be able to
define $\int f \in \R$. The most important examples of this are $X =
\R, \R_n, \Q_p, \Q_p^n$. However, in probability theory and quantum
field theory, $X = \R^\R$ (the set of functions from $\R$ to $\R$) is
also incredibly important.
\begin{example}
  Let $A = X$ and for all $\alpha \in A$, let $Y_\alpha = Y$. Then
  $Y^X = \prod_{\alpha \in A} Y_\alpha$. This means that we can think
  of a $y \in Y^X$ as $y = (y_\alpha)_{\alpha \in A}$ where $y\colon  A \to
  \bigcup_{\alpha \in A} Y_\alpha$ such that for all $\alpha \in A$, $y(\alpha)
  = y_\alpha \in Y_\alpha$.
\end{example}
We now ask, what $\sigma$-algebra can one put on $X = \R^\R$? One
example would be \[
  \bigotimes_{\alpha \in \R} \B_\R
\]
which is the product $\sigma$-algebra coming from the Borel
$\sigma$-algebra on $\R$.
\begin{thm}
  Given $(X_n)_{n \geq 1}$ a sequence of seperable metric spaces, and
  $X = \prod_{n \geq 1} X_n$ with the product topology, then \[
    \bigotimes_{n=1}^\infty \B_{X_n} = \B_X
  \]
\end{thm}
\begin{proof}[Sketch of proof]
  The key thing is to build a countable dense subset of $X$. Our
  hypothesis is for all $n$, $Z_n$ is a countable dense subset in
  $X_n$. So, let us assume for all $n$ that $X_n \neq \varnothing$, so
  there exists $*_n \in X_n$. Then, let \[
    Z = \bigcup_{N \geq 1} \left( Z_1 \times Z_2 \times \cdots \times
      Z_N \times \{*_{N+1}\} \times \{*_{N+2}\} \times \cdots \right).
  \]
  This is a dense countable subset in $X$.
\end{proof}
\begin{thm}
  Let $(X_\alpha,\M_\alpha)_{\alpha \in A}$ be an indexed family of
  measurable sets. Then, for all $y \in \bigotimes_{\alpha \in A}
  \M_\alpha = \M(\Ep)$, there exist at most countable $B \subset A$
  such that there exist $Y_B \in \bigotimes_{\alpha \in B} \M_\alpha$
  where $Y = Y_B \times \prod_{\alpha \in A \setminus B}
  X_\alpha$. \\

  Thus, if $X = \prod_{\alpha \in A} X_\alpha$, we have a measurable
  map into $\prod_{\alpha \in B} X_\alpha$ given by the restriction of
  the $x \in X$ of the form $x\colon  A \to \bigcup X_\alpha$ such that
  $x(\alpha) = x_\alpha \in X_\alpha$ to $x|_B\colon  B \to \bigcup X_\alpha$
  for $B \subset A$.
\end{thm}
\begin{proof}[Sketch of proof]
  Let \[
    \bigotimes_{\alpha \in A} Y_\alpha = \M(\Ep) \text{ where } \Ep =
    \{\pi_\alpha^{-1}(E_\alpha) \mid \alpha \in A, E_\alpha \in M_\alpha\}
  \] and let \[
    \mathcal{N} = \left\{ Y \in \P(X) \mid \exists \text{ at most countable
    } B \subset A, \exists Y_B \in \bigotimes_{\alpha \in B} X_\alpha,
    Y = Y_B \times \prod_{\alpha \in A \setminus B} X_\alpha \right\}
  \]
  \begin{enumerate}
  \item ($\Ep \subset \mathcal{N}$.) If $y =
    \prod_\beta^{-1}(E_\beta)$ for some $\beta \in A, E_\beta \in
    \M_beta$, then $y = E_\beta \times \prod_{\alpha \in A \setminus
      \{\beta\}} X_\alpha$, where, since $B = \{\beta\}$, $E_\beta =
    Y_beta$. Thus, $\bigotimes_{\alpha \in \{\beta\}} \M_\alpha =
    \M_\beta$.
  \item ($\mathcal{N}$ is a $\sigma$-algebra.)
    \begin{itemize}
    \item ($X \in \mathcal{N}$.) Let $Y = X, B = \varnothing$. Then,
      $\bigotimes_{\alpha \in B} X_\alpha = \{\pmb{\varnothing}\}$
      where $\pmb{\varnothing}: \varnothing \to \varnothing$ is the
      empty function. Thus, $Y = X = Y_\varnothing \times
      \prod_{\alpha \in A} X_\alpha$ and so $X \in \mathcal{N}$.
    \item (Closed under complement.) We have that \[
        \left( Y_B \times \prod_{\alpha \in A \setminus B} X_\alpha
        \right)^c = Y_B^c \times \prod_{\alpha \in A \setminus B} X_\alpha
      \]
      and $Y_B^c \in \bigoplus_{\alpha} \M_\alpha$.
    \item (Closed under countable union.) Let $Y_n \in \mathcal{N}$
      and $B_n, Y_{B_n}$ such that $Y_n = Y_b \times \prod_{\alpha \in
      A \setminus B} X_\alpha$. Then, let $Y = \bigcup_{n=1}^\infty
    Y_n$. Then, we need to find $B$ countable and $Y_B$ of the right
    form. Let $B = \bigcup_{n=1}^\infty B_n$. So, $B_n \subset B
    \subset A$. Then, for all $n$, \[
      Y_n = \left( Y_{B_n} \times \prod_{\alpha \in B \setminus B_n}
        X_\alpha \right) \times \prod_{\alpha \in A \setminus B} X_\alpha.
    \]
    We also note that $x|_A = (x|_B)|_{B_n}$ (What?!) Thus,
    \[
      Y = \bigcup_{n \geq 1} Y_n = \bigcup_{n \geq 1} \left[ \left(
          Y_{B_n} \times \prod_{\alpha \in B \setminus B_n} X_\alpha
        \right) \times \prod_{\alpha \in A \setminus B} X_\alpha
      \right] = \left[ \bigcup_{n \geq 1} \left( Y \times
          \prod_{\alpha \in B \setminus B_n} X_\alpha \right) \right]
      \times \prod_{\alpha \in A \setminus B} X_\alpha.
    \]
    Note that \[
       Y_{B_n} \times \prod_{\alpha \in B \setminus B_n}
        X_\alpha  \subset \prod_{\alpha \in B} X_\alpha
      \]
      and that \[
        \bigcup_{n \geq 1} \left( Y \times
          \prod_{\alpha \in B \setminus B_n} X_\alpha \right) = Y_B
        \in \bigotimes_{\alpha \in B} \M_\alpha.
      \]
    Thus, $B = \bigcup_{n \geq 1} B_n$ is countable.
    \end{itemize}
  \end{enumerate}
\end{proof}
\begin{example}
  $\R^\R = \{f\colon  \R \to \R\}$ has \[
    \bigotimes_{x \in \R} \B_\R \propsubset \B_{\R^\R} \text{ with
      product topology.}
  \]
  The reason is because $\{f(x) = 0\} \in \B_{\R^\R}$ a closed set,
  but not in $\bigotimes_{x \in \R} \B_\R$ by the theorem above.
\end{example}
The moral of the story is that for measurable topological spaces,
countable products are good but uncountable products are bad. \\

Switching gears, we note that if $X = \R$, $\B_\R = \M(\Ep)$ can have
many different $\Ep$. For instance $\Ep$ could be all open sets, all
open intervals, all sets of the form $(a, \infty)$ with $a \in \R$ or
even $a \in \Q$. So, we have a reduction that $\B_\R = \M(\{(a,b] \mid
-\infty \leq a \leq b < \infty\})$.
\begin{defn}
  Let $X$ be a set and $\Ep \subset \P(X)$. Then $\Ep$ is called an
  elementary family if
  \begin{enumerate}
  \item $\varnothing \in \Ep$,
  \item $E,F \in \Ep \ \implies \ E \cap F \in \Ep$,
  \item and $E \in \Ep \ \implies \ E^c$ is a finite disjoint union of
    elements in $\Ep$.
  \end{enumerate}
\end{defn}
\end{document}
