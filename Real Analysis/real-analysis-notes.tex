\documentclass[11pt,leqno,oneside]{amsbook}

\usepackage{../notes}
\numberwithin{thm}{section}
\renewcommand{\P}{\mathcal{P}}
\renewcommand{\A}{\mathscr{A}}
\newcommand{\M}{\mathcal{M}}
\newcommand{\Ep}{\mathscr{E}}
%%%%%%%%%%%%%% BEGIN CONTENT: %%%%%%%%%%%%%%

\title[Real Analysis]{Real Analysis}
\author{George H. Seelinger (inspired from class by Abdelmalek Abdesselam)}
\date{Spring 2017}
\begin{document}
\maketitle
\section{Measures}
\subsection*{(1/19/2017) Lecture 1}
Riemann integration is great, but it is insufficient for many classes
of integrands. Lebegue integration seeks to solve this problem and is
the main topic of the first part of this course. The main technique of
Riemann integration is to slice the domain into intervals, resulting in vertical rectangles, and add up
their area whereas Lebegue integration slices the target and then
looks at the parts of the domain that hit the target, say some set $A$
and calculates the height of the target slice times the size of the set
$A$. However, such a calculation requires a notion of measure, which
leads us to the topic of measure theory. There are two main questions
we ask ourselves.
\begin{thm}[Question]
  For what kind of sets do we have a reasonable notion of size? This
  leads us to the idea of measurable sets.
\end{thm}
\begin{thm}[Question]
  What is that notion of size? This leads us to the idea of measures.
\end{thm}
\begin{defn}
Let $X \neq \varnothing$ be some set and $\P(x)$ be the set of all
subsets of $X$ (called the power set). Furthermore, let $\A \subset
\P(X)$ with $\A \neq \varnothing$. Then, $\A$ is called an \emph{algebra} of
sets on $X$ if $\A$ is closed under finite unions and taking
complements.
\end{defn}
\begin{lem}
  $\varnothing, X \in \A$. This follows by taking a set in $\A$ and
  unioning it with its complement to get $X$ and then taking the
  complement of $X$.
\end{lem}
\begin{defn}
  $\A \subset \P(X)$ is a \emph{$\sigma$-algebra} on $X$ if and only if
  $\A$ is an algebra of sets which is closed under countable
  unions. That is, for all sequences $(A_n)_{n \geq 0}$ of elements of $\A$,
  $\cup_{n=0} A_n \in \A$.
\end{defn}
\begin{thm}
  $\A$ is a $\sigma$-algebra iff the following are satisfied.
  \begin{enumerate}
    \item $\varnothing \in \A$,
    \item for all nonempty countable sequences $(A_n)_{n \geq 0}$ such that $A_n \in \A$, we have
      $\cup_{n=0}^\infty A_n \in \A$.
    \item for all $A \in \A$, $A^c \in \A$,
  \end{enumerate}
\end{thm}
\begin{example}
  Let $X$ be a set. Then
  \begin{enumerate}
  \item $\A = \{\varnothing,X\}$ is a $\sigma$-algebra.
  \item $\A = \P(X)$ is a $\sigma$-algebra.
  \item if $X = \{1,2,3\}$, we have that $\A =
    \{\varnothing,X,\{1\},\{2,3\}\}$ is a $\sigma$-algebra.
  \end{enumerate}
\end{example}
\begin{prop}
  If $\A$ is a $\sigma$-algebra of $X$, then $\A$ is closed under
  countable intersections because \[
    \cap_{n=0}^\infty A_n = \left( \cup_{n=0}^\infty A_n^c \right)^c
  \]
\end{prop}
\begin{defn}
  Let $X$ be a set. Suppose $\Ep \subset \P(A)$. $\M(\Ep)$ is the
  smallest $\sigma$-algebra on $X$ which contains $\Ep$.  It is called the
  \emph{$\sigma$-algebra generated by $\Ep$}.
\end{defn}
Of course, this leaves us to ask, does this even exist and if so, is
it unique? We must prove it.
\begin{proof}
  Take $\mathcal{Z} = \{\A \subset \P(x) \mid \A \text{ a }
  \sigma\text{-algebra and } \Ep \subset \A\}$. Then $\mathcal{Z}
  \subset \P(\P(X))$ and take a partial order on $\mathcal{Z}$ as
  containment. Now, we need to show that there exists a smallest
  element of $\mathcal{Z}$. Let us define \[
    \M(\Ep) = \cap_{\A \in \mathcal{Z}} \A
  \]
  Then we will show $\M(\Ep)$ is a $\sigma$-algebra.
  \begin{enumerate}
  \item For all $\A \in \mathcal{Z}$, $\A$ is a $\sigma$-algebra, so
    $\varnothing \in \A$. This gives us that $\varnothing \in \cap_{\A
      \in \mathcal{Z}} \A = \M(\Ep)$.
  \item Let $E \in \M(\Ep)$. Then, for all $\A \in \mathcal{Z}$, we have that
    $E \in \A$. Thus, $E^c \in \A$ so $E^c \in \cap_{\A \in \mathcal{Z}} \A$.
  \item Suppose $(A_n)_{n \geq 0}$ and $\forall n, A_n \in
    \M(\Ep)$. Then, for all $\A \in \mathcal{Z}$, we have that $\forall n \in
    A_n \in \A \implies \cup_{n \geq 0} A_n \in \A$ for all $\A
    \in \mathcal{Z}$. Thus, $(\cup_{n \geq 0} A_n) \in \cap_{\A \in \mathcal{Z}} \A$.
  \end{enumerate}
  Thus, $\M(\Ep)$ is a $\sigma$-algebra. Moreover, for all $\A \in
  \mathcal{Z}$, we have that $\Ep \subset \A$ and so $\Ep \subset \cap_{\A \in
    \mathcal{Z}} \A = \M(\Ep)$. Thus, $\M(\Ep) \in \mathcal{Z}$. \\

  Now, we show it is the smallest element. If $\mathscr{B} \in
  \mathcal{Z}$, then clearly $\M(\Ep) = \cap_{\A \in \mathcal{Z}} A
  \subset \mathscr{B}$. Thus, $\M(\Ep)$ is the smallest element of
  $\mathcal{Z}$.
\end{proof}
\begin{example}
  Let $X = \{1,2,3\}$ and $\Ep = \{\{1\}\}$. Then, $\M(\Ep) = \{X,
  \varnothing, \{1\},\{2,3\}\}$.
\end{example}
\begin{defn}
  Given $X$ a topological space and $\Ep$ the set of all open subsets of $X$,
  then $\B_X := \M(\Ep)$ is the \emph{Borel $\sigma$-algebra of $X$}.
\end{defn}
\begin{defn}
  Given,
  \begin{itemize}
  \item a set $X$,
  \item a family of sets $(X_\alpha)_{\alpha \in A}$,
  \item a family of functions $(f_\alpha)_{\alpha \in A}, f_\alpha: X
    \to X_\alpha$,
  \item a family of $\sigma$-algebras $(\M_\alpha)_{\alpha \in A}$ such
    that for all $\alpha$, $\M_\alpha$ is a $\sigma$-algebra on
    $X_\alpha$,
  \end{itemize}
  the \emph{$\sigma$-algebra on $X$ generated by all of this data} is
  $\M := \M(\Ep)$ where $\Ep = \{f_\alpha^{-1}(z) : \alpha \in A, z \in
  \M_\alpha\}$.
\end{defn}
\begin{example}
  Let $X = \{0,1\}^3$, $A = \{1,3\}$, and $X_1 = X_3 =
  \{0,1\}$. Furthermore, let $f_1: X \to X_1$ be given by
  $(x_1,x_2,x_3) \mapsto x_1$ and similarly let $f_3: X \to X_3$ be given by $(x_1,x_2,x_3) \mapsto x_3$. Finally, let
  $\M_1 = \M_3 = \P(\{0,1\})$. Then, the $\sigma$-algebra defined
  above will yield the set of sets of tuples indexed by the first and
  third coordinates. In other words, this set is in bijection with the
  set $\P(\{0,1\}^2)$.
\end{example}
\end{document}
