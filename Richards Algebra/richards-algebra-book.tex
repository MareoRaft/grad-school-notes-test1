\documentclass[11pt]{book}

\usepackage[margin=1.0in]{geometry}

\usepackage{amsmath,amssymb,amsthm}

\usepackage{mathrsfs}
%\usepackage{dsfont}
\usepackage{times}
\usepackage{epsfig}
\usepackage{enumitem}
\usepackage{mathabx}

\usepackage{csquotes}
\MakeOuterQuote{"}
\usepackage{graphicx}
\usepackage{caption}
\usepackage{subcaption}
\usepackage{setspace}

\usepackage{tikz}
\usepackage{tikz-cd}
\usepackage{pgfplots}

\usepackage{etoolbox,xparse}
\usepackage{float}
\floatstyle{plain}%boxed
\restylefloat{figure}
\usepackage{chngcntr}
\counterwithin{chapter}{part}

\newcounter{counter}


\newtheorem{theorem}[counter]{Theorem}   \newtheorem*{theorem*}{Theorem}   \newtheorem{lemma}[counter]{Lemma}   \newtheorem{corollary}[counter]{Corollary}
\newtheorem{proposition}[counter]{Proposition}   \newtheorem{problem}[counter]{Problem}   \newtheorem*{proposition*}{Proposition}   \newtheorem*{lemma*}{Lemma}

\theoremstyle{definition}   \newtheorem{defn}[counter]{Definition} %These theorem environments are not numbered separately
\newtheorem*{defn*}{Definition}
\newtheorem{blank}[counter]{}   \newtheorem{remark}[counter]{Remark(s)}   \newtheorem*{remark*}{Remark(s)}   \newtheorem{generalization}[counter]{Generalization}
\newtheorem{consequence}[counter]{Consequence(s)}   \newtheorem*{consequence*}{Consequence(s)}   \newtheorem*{problem*}{Problem}   \newtheorem{notation}[counter]{Notation}
\newtheorem*{notation*}{Notation}   \newtheorem{example}[counter]{Example(s)}   \newtheorem*{example*}{Example(s)}   \newtheorem*{warning}{Warning} \newtheorem*{corollary*}{Corollary}
\newtheorem*{question}{Question}   \newtheorem*{answer}{Answer}   \newtheorem{modification}[counter]{Modification}   \newtheorem{numitem}[counter]{}

\newcommand{\ov}{\overline}   \newcommand{\wt}{\widetilde}
\newcommand{\blt}{$\bullet$}   \newcommand{\tn}{\textnormal}   \newcommand{\tb}{\textbf}   \newcommand{\mbb}{\mathbb}
\newcommand{\bs}{\backslash}   \newcommand{\A}{\mathcal{A}}   \newcommand{\sy}{\textnormal{Syl}}   \newcommand{\size}[1]{\left| #1 \right|}
\newcommand{\zx}[1]{(\z/#1\z)^{\times}}   \newcommand{\zn}[1]{\z/#1\z}   \newcommand{\pr}[1]{\textbf{Problem #1.}}   \newcommand{\abc}{(\alph*)}
\newcommand{\nsg}{\mathrel{\unlhd}}   \newcommand{\ind}{\parindent24pt}   \newcommand{\vn}{\varnothing}
\newcommand{\ve}{\varepsilon}   \newcommand{\im}{\textnormal{im }}   \newcommand{\re}{\textnormal{Re }}   \newcommand{\mb}[1]{\mathbf{#1}}
\newcommand{\lra}{\leftrightarrow}   \newcommand{\0}{\mathbf{0}}   \newcommand{\mc}[1]{\mathcal{#1}}   \newcommand{\hra}{\hookrightarrow}   \newcommand{\hla}{\hookleftarrow}
\newcommand\myeq{\mathrel{\overset{\makebox[0pt]{\mbox{\normalfont\tiny\sffamily \textrm{def}}}}{=}}}
\newcommand\lheq{\mathrel{\overset{\makebox[0pt]{\mbox{\normalfont\tiny\sffamily \textrm{L'H}}}}{=}}}
\newcommand{\mymatrix}[2]{\left( \begin{array}{#1} #2 \end{array} \right)}   \newcommand{\myvec}[1]{\left( \begin{array}{c} #1 \end{array} \right)}

\newcommand{\hm}{homomorphism}   \newcommand{\hms}{homomorphisms}   \newcommand{\iso}{isomorphism}
\newcommand{\isos}{isomorphisms}   \newcommand{\auto}{automorphism}   \newcommand{\autos}{automorphisms}   \newcommand{\ds}[2]{#1^{(#2)}}   \newcommand{\lcs}[2]{#1^{[#2]}}
\newcommand{\pf}[1]{$ #1 = p_1^{e_1} p_2^{e_2} \cdots p_r^{e_r}$, with $p_1,p_2,\dots,p_r$ distinct primes and $e_1,e_2,\dots,e_r\in \N$}
\newcommand{\inn}{\textnormal{Inn}}   \newcommand{\out}{\textnormal{Out}}   \newcommand{\car}{\textnormal{ char }}   \newcommand{\id}{\textnormal{id}}   \newcommand{\triv}{\{e\}}
\newcommand{\tl}{\triangleleft}   \newcommand{\sd}[1]{\rtimes_{#1}}   \newcommand{\x}{^{\times}}   \newcommand{\cyc}[1]{\begin{pmatrix} #1 \end{pmatrix}}
\newcommand{\gen}[1]{\langle #1 \rangle}   \newcommand{\stab}[2]{\tn{Stab}_{#1}(#2)}   \newcommand{\fix}[2]{\tn{Fix}_{#1}(#2)}
\newcommand{\hooklongrightarrow}{\lhook\joinrel\longrightarrow}   \newcommand{\twoheadlongrightarrow}{\relbar\joinrel\twoheadrightarrow}
\newcommand{\ses}[5]{1 \longrightarrow #1 \overset{#2}{\hooklongrightarrow} #3 \overset{#4}{\twoheadlongrightarrow} #5 \longrightarrow 1}
\renewcommand{\arrowvert}{\arrow}   \renewcommand{\i}{\mathbf{i}}   \renewcommand{\j}{\mathbf{j}}   \renewcommand{\k}{\mathbf{k}}    \renewcommand{\H}{\mathbb{H}}


\DeclareMathOperator{\pow}{\mathcal{P}}   \DeclareMathOperator{\C}{\mathbb{C}}
\DeclareMathOperator{\col}{\mathcal{C}}   \DeclareMathOperator{\F}{\mathbf{F}}   \DeclareMathOperator{\h}{\mathbb{H}}   \DeclareMathOperator{\bF}{\mathbf{F}}
\DeclareMathOperator{\R}{\mathbb{R}}   \DeclareMathOperator{\N}{\mathbb{N}}   \DeclareMathOperator{\z}{\mathbb{Z}}   \DeclareMathOperator{\Q}{\mathbb{Q}}
\DeclareMathOperator{\ra}{\rightarrow}   \DeclareMathOperator{\Poly}{\mathbf{P}}   \DeclareMathOperator{\spn}{\textnormal{span}}   \DeclareMathOperator{\aut}{\textnormal{Aut}}
\DeclareMathOperator{\znx}{(\z/n/Z)^{\times}}   \DeclareMathOperator{\lcm}{\textrm{lcm}}   \DeclareMathOperator{\inv}{^{-1}}   \DeclareMathOperator{\sub}{\subseteq}

\newenvironment{prf}{\paragraph{\textit{Proof}}}{\hfill$\square$}

\newcommand{\vs}{\vspace{8pt}}
\newcommand{\hs}{\hspace{8pt}}

\numberwithin{counter}{chapter}

\title{Algebra I}
\author{ Transcribed from the lectures of \\ \ \\ Professor Peter Abramenko \\ \ \\ University of Virginia}
\date{Fall 2016}


\begin{document}

\frontmatter
\maketitle
\tableofcontents
\mainmatter

\ \vspace{100pt}
\begin{quote}
\centering
\textit{A study of groups, rings, fields, modules, tensor products, and multilinear functions.}
\end{quote}

\part{Group Theory}

\ \\

\chapter{Some Basic Facts}

\section*{Normalizers and Centralizers} \vs

\begin{defn}
For $A$ a subset of $G$ and $x \in G$, we set
	\begin{align*}
	N_G(A) &= \{g \in G \mid gAg\inv = A\}, \\
	C_G(A) &= \{g \in G \mid gag\inv = a \ \forall \ a \in A\}, \\
	C_G(x) &= C_G(\{x\}), \\
	Z(G) &= \{g \in G \mid gag\inv = x \ \forall \ a \in G\}
	\end{align*}
\end{defn}

\vs

\begin{remark}[about normalizers] \ \\
\begin{enumerate}
\item[(a)] If $A \leq G$, then $A \nsg N_G(A)$. In fact, $N_G(A)$ is the biggest subgroup of $G$ in which $A$ is normal.
\item[(b)] If $A \leq G$, then $A \nsg G$ iff $N_G(A) = G$.
\end{enumerate}
\end{remark}

\vs

\begin{remark}[about centralizers]
\item[(a)] If $A \leq G$, then $A$ is abelian if and only if $A \leq C_G(A)$.
\item[(b)] If $A \leq Z(G)$, then $A \nsg G$.
\end{remark}

\vs

\begin{example}
Consider the group $S_n$. $Z(S_n) = \{id\}$ for $n \geq 3$. To show this, let $\sigma \in S_n$, $\sigma \ne e$. We show that $\sigma \notin Z(S_n)$. Then there exists $i,j \in \{a,2,\dots,n\}$, $i \ne j$, such that $\sigma(i) = j$. Choose $k \in \{1,2,\dots,n\} \bs  \{i,j\}$. Then
	\[\sigma \circ (ik) \circ \sigma\inv (j) = (\sigma(i)\, \sigma(k))(j) = (j\, \sigma(k))(j) = \sigma(k) \ne j = (ik)(j), \]
since $k \ne i$.
\end{example}

\vs

\begin{lemma}
If $G$ is a group and $H \leq Z(G)$, then $G/H$ is cyclic implies that $G$ is abelian.
\end{lemma}

\begin{prf}
Since $G/H$ is cyclic, $G/H = \langle xH \rangle$ for some $x \in G$. Let $a,b \in G$. Then there exists $m,n \in \z$ such that $aH = x^mH$ and $bH = x^nH$. That is, there exist $h,h' \in H$ such that $a = x^mh$ and $b = x^nh'$. Therefore, recalling that $h,h' \in H \leq Z(G)$,
	\begin{align*}
	ab = (x^mh)(x^nh') = x^{m+n}(hh') = x^{n+m}(h'h) = (x^nh')(x^mh) = ba
	\end{align*}
\end{prf}

\vs

\begin{remark*}
If $H \leq Z(G)$ and $G/H$ is \textit{abelian}, it does not necessarily follow that $G$ is abelian. As an example, consider $Q = \{\pm 1, \pm i, \pm j, \pm k\}$. Note that $Z(Q) = \{\pm 1\}$ and $Q / Z(Q) \cong \z_2 \times \z_2$, which is abelian, but $Q$ is not abelian.
\end{remark*}

\vs

\begin{lemma}
Let $G$ be a group and $A,B \leq G$.
	\begin{enumerate}
	\item[(a)] $AB \leq G$ if and only if $AB = BA$.
	\item[(b)] If $A \leq N_G(B)$ or $B \leq N_G(A)$, then $AB \leq G$.
	\end{enumerate}
\end{lemma}

\begin{prf}
\begin{enumerate}
\item[(a)] $(\implies)$ If $AB \leq G$, then $AB = (AB)\inv = B\inv A\inv = BA$. \\
$(\impliedby)$ Suppose $AB = BA$. Certainly $e \in AB$, since $e \in A\cap B$ and $e = ee \in AB$. $AB$ is closed under inversion, since $(AB)\inv = B\inv A\inv = BA = AB$. Finally, $(AB)(AB) = A(BA)B = A(AB)B = A^2B^2 = AB$, so $AB \leq G$.

\item[(b)] WLOG, assume $A \leq N_G(B)$. We claim that $AB = BA$. Let $a \in A, b \in B$. Since $a \in A \leq N_G(B)$, $aba\inv = b'$ for some $b' \in B$, hence $ab = b'a \in BA$, so $AB \sub BA$. Similarly, $a\inv ba = b''$ for some $b'' \in B$, so $ba = ab'' \in AB$, so $BA \sub AB$, hence $AB = BA$. It follows from part (a) that $AB \leq G$.
\end{enumerate}
\end{prf}

\vs

\begin{lemma}
If $A$ and $B$ are finite subgroups of $G$, then
	\[|AB| = \frac{|A||B|}{|A\cap B} \]
\end{lemma}

\begin{prf}
Since $A \cap B \sub A$, we can talk about $A / (A \cap B)$. Take a complete set $A' = \{a_1,a_2,\dots,a_n\}$ of representatives of $A/(A\cap B)$, so that $A = \bigsqcup_{I=1}^n a_i(A\cap B)$. By Lagrange's theorem,
	\[n = \left| A/(A\cap B) \right| = \frac{|A|}{|A\cap B|} \]
Consider $f : A' \times B \ra AB$, $(a_i,b) \mapsto a_ib$. \\

$f$ is injective. Suppose $a_ib = a_jb'$ for some $a_i,a_j \in A, b,b' \in B$. Then $b'b\inv = a_ia_j\inv \in A \cap B$, so $a_i (A\cap B) = a_j (A \cap B)$, hence $i = j$. But $a_i b = a_i b' \implies b = b' \implies (a_i,b) = (a_j,b')$. \\

$f$ is surjective. Let $a \in A$, $b \in B$. There exists $1 \leq i \leq n$ and $x \in A\cap B$ such that $a = a_i x$, from which it follows that $ab = a_i (xb) \in AB$. \\

Finally, \[|AB| = |A' \times B| = \left( \frac{|A|}{|A\cap B|} \right) |B| = \frac{|A||B|}{|A \cap B|} \]
\end{prf}

\vs
\begin{example}\ \\
\item[(a)] $G = S_3$, $A = \langle (12) \rangle$, $B = \langle (13) \rangle$, $A \cap B = \{id\}$. Then $|AB| = 4 \not|\ 6 = |G| \implies AB \not\leq G$.
\item[(b)] $G = S_4$, $A = S_3$, $B = \langle \cyc{1 \ 2 \ 3 \ 4} \rangle \leq G$, $A \cap B = \{id\}$. Then
	\[|AB| = |A||B| = 6 \cdot 4 = 24 = |G| \implies AB = G \]
Notice, however, that $A \not\leq N_G(B)$ and $B \not\leq N_G(A)$.
\end{example}

\vs

\begin{lemma}
Let $G$ be a group and $A,B \leq G$.
\begin{enumerate}
\item[(a)] $A \leq B \implies [G : A] = [G : B] [B : A]$
\item[(b)] $[A : A \cap B] \leq [G : B]$
\item[(c)] $[G : A \cap B] \leq [G : A] [G : B]$
\end{enumerate}
\end{lemma}

\begin{prf}
We show part (b) and leave the rest as exercises. Define $f : A / (A\cap B) \ra G/B$ by $a(A\cap B) = aB$. \\

$f$ is well-defined. Suppose $a(A\cap B) = a'(A \cap B)$. Then $a' = ab$ for some $b \in A\cap B \leq B$, so $aB = a'B$. \\

$f$ is injective. Let $a,a' \in A$, and suppose $aB = a'B$. Then $a' = ab$ for some $b \in B$, so $b = a\inv a' \in A\cap B$, thus $a(A\cap B) = a'(A \cap B)$. \\

It follows that
	\[\left| \frac{A}{A\cap B} \right| \leq |G/B| \iff [A : A \cap B] \leq [G : B] \]
\end{prf}

\ \\

\section*{The Isomorphism Theorems}

\ \\

\begin{theorem}[First Isomorphism Theorem]
Let $\phi : G \ra H$ be a group homomorphism. Then
	\[G/\ker \phi \cong \im \phi \]
\end{theorem}

\begin{prf}
Put $N = \ker \phi$. Since $N \nsg G$, define $\overset{\sim}{\phi} : G/ N$ by $\overset{\sim}{\phi} (gN) = \phi(g)$. \\

$\overset{\sim}{\phi}$ is well-defined. Suppose $gN = g'N$. Then $g' = gn$ for some $n \in N = \ker \phi$, so
	\[\overset{\sim}{\phi}(g'N) = \phi(gn) = \phi(g)\phi(n) = \phi(g) = \overset{\sim}{\phi}(gN) \]

$\overset{\sim}{\phi}$ is a group homomorphism. If $g,g' \in G$, then
	\[\overset{\sim}{\phi}((gN)(g'N)) = \overset{\sim}{\phi}((gg')N) = \phi(gg') = \phi(g)\phi(g') = \overset{\sim}{\phi}(gN) \overset{\sim}{\phi}(g'N) \]

That $\overset{\sim}{\phi}$ is surjective is clear by its definition. \\

$\overset{\sim}{\phi}$ is injective. Suppose $\overset{\sim}{\phi}(gN) = e_h$ for some $g \in G$. Then $\phi(g) = e_H$, so $g \in \ker \phi = N$. That is, the kernel of $\overset{\sim}{\phi}$ is trivial.
\end{prf}

\vs

\begin{corollary}[Second Isomorphism Theorem]
Let $A,B \leq G$, $A \leq N_G(B)$. Then
	\[\frac{AB}{B} \cong \frac{A}{A\cap B} \]
\end{corollary}

\begin{prf}
$AB \leq G$ by 1.1.6. $A,B \leq N_G(B) \implies AB \leq N_G(B) \leq G \implies B \nsg AB$. Similarly, $A \cap B \nsg A$. Apply the First Isomorphism Theorem to
	\[\phi : A \ra \frac{AB}{B}, \qquad a \mapsto aB \]
You should verify that $\phi$ is a surjective group homomorphism with $\ker \phi = A \cap B$.
\end{prf}

\vs

\begin{corollary}[Third Isomorphism Theorem]
Let $A,B \nsg G$, $A \leq B$. Then
	\[G/B \cong \frac{G/A}{B/A} \]
\end{corollary}

\begin{prf}
Apply 1.1.10 to
	\[\phi : G/A \ra G/B, \qquad gA \mapsto gB, \]
verifying that $\phi$ is a surjective group homomorphism with $\ker \phi = B/A $.
\end{prf}

\vs

\begin{theorem}[Correspondence Theorem]
Let $G$ be a group, $N \nsg G$, $\ov{G} = G/N$. Define $\pi : G \ra G/N, g \mapsto gN$. Let $\mc{S}_N = \{A \leq G \mid N \leq A\}$. For each $A \in \mc{S}_N$, set $\ov{A} = A/N$. Finally, let $\ov{\mc{S}}$ denote the set of all subgroups of $\ov{G}$. Then for $A,B \in \mc{S}_N$, we have
	\begin{enumerate}
	\item[(a)] $A \leq B \iff \ov{A} \leq \ov{B}$
	\item[(b)] $A \nsg G \iff \ov{A} \nsg \ov{G}$
	\item[(c)] $A \leq B \implies [B:A] = [\ov{B} : \ov{A}]$
	\item[(d)] $\ov{(A \cap B)} = \ov{A} \cap \ov{B}$
	\item[(e*)] $\phi : \mc{S}_N \ra \ov{\mc{S}}, A \mapsto \ov{A}$ is a bijective map.
	\item[(f)] $f : \{A \in \mc{S}_N \mid A \nsg G\} \ra \{\textnormal{normal subgroups of } \ov{G}\}, A \mapsto \ov{A}$ is a bijection.
	\end{enumerate}
\end{theorem}

\begin{proof} \
\begin{enumerate}
\item[(a)] $[\implies]$ Suppose $A \leq B$. Since $\pi$ is a group homomorphism, $\pi(A) \leq \pi (B) \leq G/N$. But $\pi(A) = \{aN \mid a \in A\} = \ov{A}$ and $\pi(B) = \{bN \mid b \in B\} = \ov{B}$, so $\ov{A} \leq \ov{B}$. \\
$[\impliedby]$ Suppose $\ov{A} \leq \ov{B}$. Then $\pi\inv (\ov{A}) \leq \pi \inv(\ov{B}) \leq G$. Clearly, $A \leq \pi\inv(\ov{A})$. Since $N \leq A$, we have
	\begin{align*}
	g \in \pi\inv(\ov{A}) & \iff gN = aN \textnormal{ for some } a \in A \\
	&\iff \textnormal{ for some } n \in N, g = an \in A \\
	&\implies \pi\inv(\ov{A}) \sub A \\
	&\implies \pi\inv(\ov{A}) = A, \textnormal{ since } A \sub \pi\inv({A})
	\end{align*}
That is, $gN \in \ov{A} \iff g \in A$. A similar argument shows that $\pi\inv({B}) = B$, hence $A \leq B$. \\

\item[(b)] $[\implies]$ Suppose $A \nsg G$. For all $g \in G, a \in A$, we have $(gN)(aN)(gN)\inv = (gag\inv) N \in \ov{A}$, whence $\ov{A} \nsg \ov{G}$. \\
$[\impliedby]$ Suppose $\ov{A} \nsg \ov{G}$. For all $g \in G, a \in A$, we have $(gag\inv)N = (gN)(aN)(gN)\inv \in \ov{A} \implies gag\inv \in A$, as shown above. Therefore, $A \nsg G$. \\

\item[(c)] Suppose $A \leq B$. By the third isomorphism theorem, $\ov{B}/\ov{A} = (B/N)/(A/N) \cong B/A \implies [B : A] = [\ov{B} : \ov{A}]$. \\

\item[(d)] Since $A\cap B \leq G$ and $\ov{A} \cap \ov{B} \leq \ov{G}$, it follows from the proof of part (a) that
	\begin{align*}
	xN \in \ov{A\cap B} &\iff x \in A \cap B \\
	&\iff x \in A \textnormal{ and } x \in B \\
	&\iff xN \in \ov{A} \tn{ and } xN \in \ov{B} \\
	&\iff xN \in \ov{A} \cap \ov{B}
	\end{align*}
Therefore, $\ov{(A \cap B)} = \ov{A} \cap \ov{B}$. \\

\item[(e)] Consider $\psi : \ov{\mc{S}} \ra \mc{S}_N, H \mapsto \pi\inv(H) \leq G$. We claim that $\phi \circ \psi = Id_{\ov{\mc{S}}}$ and $\psi \circ \phi = Id_{\mc{S}_N}$. \\
Let $H \in \ov{\mc{S}}$, so that $H \leq \ov{G}$. Then $\psi(H) = \pi\inv (H) \leq G$, and
	\[ (\phi\circ \psi)(H) = \phi(\pi \inv(H)) = \pi(\pi\inv(H)) = H, \]
since $\pi$ is surjective. It remains to be shown that $\psi \circ \phi = Id_{\mc{S}_N}$. Let $A \in \mc{S}_N$, so that $N \leq A \leq G$. We claim that $\pi\inv(\pi(A)) = A$. Clearly, $A \sub \pi\inv(\pi(A))$. Suppose that $x \in \pi\inv(\pi(A))$, so that $xN \in \ov{A} = A/N$. Then $xN = aN$ for some $a \in A$, hence there exists $n \in N$ such that $x = an \in A$, since $N \leq A$. Therefore, $\pi\inv(\pi(A)) \sub A$, from which it follows that $\pi\inv(\pi(A) = A$. So we have
	\[(\psi\circ \phi)(A) = \pi\inv(\pi(A)) = A, \]
which shows that $\psi \circ \phi = Id_{\mc{S}_N}$. This proves (e). \\

\item[(f)] This follows directly from (e) and (b).
\end{enumerate}
\end{proof}


\section*{Commutators}

\vs

\begin{defn}
The \textbf{commutator} of two elements $x,y \in G$ is defined as $[x,y] = xyx\inv y\inv$.
\end{defn}

\vs

\begin{warning}
The commutator of $x$ and $y$ is sometimes defined to be $[x,y] = x\inv y\inv xy$.
\end{warning}

\vs

\begin{lemma}
For $x,y,z \in G$, we obtain
	\begin{enumerate}
	\item[(a)] $[x,y] = e \iff xy = yx \iff [x\inv,y\inv] = e$
	\item[(b)] $[x,y]\inv = [y,x]$
	\item[(c)] $z[x,y]z\inv = [zxz\inv,zyz\inv]$
	\item[(d)] If $\phi : G \ra H$ is a group homomorphism, then $[\phi(x),\phi(y)] = \phi([x,y])]$.
	\item[(e)] $xy = [x,y]yx = yx[x\inv,y\inv]$
	\item[(f)] $[xy,z] = x[y,z]x\inv [x,z]$
	\item[(g)] $[x,yz] = [x,y]y[y,z]y\inv$
	\end{enumerate}
\end{lemma}

\vs

\begin{defn}
For $A,B \leq G$, the \textbf{commutator} of $A$ and $B$ is the subgroup $[A,B]$ of $G$ defined by
	\[[A,B] = \langle \{[a,b] \mid a \in A, b \in B\}\rangle \]
$[G,G]$ is called the \textbf{commutator subgroup of $G$.} (sometimes denoted by $G'$).
\end{defn}

\vs

\begin{remark}
Let $A,B \leq G$. \\
\begin{enumerate}
\item[(a)] $[A,B] = \{e\} \iff ab = ba \ \forall\ a\in A, b \in B \iff A \leq C_G(B) \iff B \leq C_G(A)$. In this case, we say that $A$ and $B$ commute.
\item[(b)] $[G,G] = \{e\} \iff G$ is abelian $\iff Z(G) = G$.
\item[(c)] $[A,B] = [B,A]$.
\item[(d)] $[A_n,A_n] = A_n$ for $n \geq 5$. Indeed, for all distinct $i,j,k,l,m \in \{1,2,\dots,n\}$,
	\[[\cyc{i\ k\ l},\cyc{k\ j\ m}] = \cyc{i\ j\ k} \]
Since $A_3$ is abelian, $[A_3,A_3] = \{e\}$. What is $[A_4,A_4]$? (Exercise)
\end{enumerate}
\end{remark}

\vs

\begin{example}
\ \\
\begin{enumerate}
\item[(a)] $[Q,Q] = \{1,-1\} = Z(Q) = [\langle i \rangle, \langle j \rangle] $. $[i,j] = iji\inv j\inv = ij(-i)(-j) = i^2j^2 = -1$.
\item[(b)] $[A_n,\langle \cyc{1 \ 2} \rangle] = A_n$ for $n \geq 2$. For $i \geq 3$, $[\cyc{1 \ i \ 2},\cyc{1 \ 2}] = \cyc{1 \ 2 \ i}$. So $[A_n,\langle \cyc{1 \ 2}\rangle] = \langle \{\cyc{1 \ 2 \ i} \mid 3 \leq i \leq n \} \rangle = A_n$ (see HW 1).
\item[(c)] $[S_n,S_n] = A_n$ for $n \geq 2$.
\end{enumerate}
\end{example}

\vs

\begin{lemma}
Let $A,B \leq G$.
	\begin{enumerate}
	\item[(a)] $A \leq N_G(B) \iff [A,B] \leq B$
	\item[(b)] $B \leq N_G(A) \iff [A,B] \leq A$
	\item[(c)] $A,B \nsg G \implies [A,B] \nsg G$ and $[A,B] \nsg A\cap B$
	\end{enumerate}
\end{lemma}

\begin{prf}
	\begin{enumerate}
	\item[(a)] $[\implies]$ Suppose $A \leq N_G(B)$. Let $a\in A$, $b \in B$. Then $[a,b] = (aba\inv)b\inv \in B$, so $[A,B] \leq B$. \\
	$[\impliedby]$ Suppose $[A,B] \leq B$. Let $a\in A, b \in B$. Then $[a,b] = (aba\inv)b\inv \in B \implies aba\inv \in B$, from which it follows that $A \leq N_G(B)$. \\

	\item[(b)] Similar to the above. \\

	\item[(c)] Suppose $A,B \nsg G$. Let $g \in G, a\in A, b\in B$. Then $g[a,b]g\inv = [gag\inv,gbg\inv] \in [A,B] \implies g[A,B]g\inv \sub [A,B]$, hence $[A,B] \nsg G$. $[A,B] \leq A\cap B$ follows from (a) and (b), hence $[A,B] \nsg A\cap B$.
	\end{enumerate}
\end{prf}

\vs

\begin{proposition}
If $H \leq G$, then $[G,G] \leq H$ if and only if $H \nsg G$ and $G/H$ is abelian.
\end{proposition}

\begin{prf}
$[\implies]$ Suppose $H \leq G$ and $[G,G] \leq H$. Then for any $g \in G$, $h \in H$, we have $[g,h] = (ghg\inv)h\inv \in H \iff ghg\inv \in H$, so $H \nsg G$. Let $x,y \in G$. Then $[xH,yH] = (xH)(yH)(xH)\inv(yH)\inv = [x,y]H = H$, hence $G/H$ is abelian. \\

$[\impliedby]$ Suppose $H \nsg G$ and $G/H$ is abelian. Let $x,y \in G$. Then $[xH,yH] = [x,y]H = H \iff [x,y] \in H$. Thus $[G,G] \leq H$.
\end{prf}

\ \\

\section*{Direct Products}

\vs

\begin{defn}
The (external) \textbf{direct product} of two groups $A$ and $B$ is defined as the se $A \times B$ (Cartesian product) with multiplication given by $(a,b)(a',b') = (aa',bb')$.
\end{defn}

With this multiplication, it is routine to check that $A \times B$ is a group.

\vs

\begin{notation*}
If $A$ and $B$ are abelian groups, we write $A \oplus B$ instead of $A \times B$.
\end{notation*}

\vs

\begin{lemma}
Let $G_1,G_2$ be groups, $H_1 \leq G_2$, $H_2 \leq G_2$. Then
	\begin{enumerate}
	\item[(a)] $H_1 \times H_2 \leq G_1 \times G_2$
	\item[(b)] $H_1 \times H_2 \nsg G_1 \times G_2 \iff H_1 \nsg G_1$ and $H_2 \nsg G_2$. In this case,
		\[\frac{G_1 \times G_2}{H_1 \times H_2} \cong \frac{G_1}{H_1} \times \frac{G_2}{H_2} \]
	\end{enumerate}
\end{lemma}

\textit{Proof of (b).}

The equivalence statement follows from $(g_1,g_2)(h_1,h_2)(g_1,g_2)\inv = (g_1h_1g_1\inv,g_2h_2g_2\inv)$ for all $g_1 \in G_1,g_2 \in G_2, h_1 \in H_1, h_2 \in H_2$. Consider $\phi : G_1 \times G_2 \ra (G_1/H_1) \times (G_2/H_2)$, $(g_1,g_2) \mapsto (g_1H_1,g_2H_2)$. It is a routine exercise to show that $\phi$ is a surjective group homomorphism with $\ker \phi = H_1 \times H_2$. Once this is verified, the first isomorphism theorem gives $(G_1\times G_2)/(H_1\times H_2) \cong (G_1/H_1)\times (G_2/H_2)$. $\hfill \square$

\vs

\begin{warning}
Not every subgroup of $G_1\times G_2$ is of the form $H_1 \times H_2$ for $H_1 \leq G_1, H_2 \leq G_2$. For example, consider $\z_2 \times \z_2$ and $\gen{(1,1)} \leq \z_2 \times \z_2$.
\end{warning}

\vs

\begin{remark}[Universal Property of Direct Product]
Let $A,B$ be groups. Define the projections $\pi_A : A \times B \ra A$, $(a,b) \mapsto a$ and $\pi_B : A \times B \ra B$, $(a,b) \mapsto b$, which are easily seen to be surjective group \hms. $\pi_A$ and $\pi_B$ satisfying the following universal property: for any group $G$ and pair of group \hms $\phi_A : G \ra A, \phi_B : G \ra B$, there exists a unique group \hm $\phi : G \ra A \times B$ such that $\pi_A\circ \phi = \phi_A$ and $\pi_B \circ \phi = \phi_B$. That is, the following diagram commutes: \\
%	\begin{tikzpicture}
%	\node at (-3,0) (G) {$G$};
%	\node at (0,2) (A) {$A$};
%	\node at (0,-2) (B) {$B$};
%	\node at (3,0) (AB) {$A \times B$};
%	\draw[->] (G)--(A) node[anchor=south,midway]{$\phi_A\ $};
%	\draw[->] (G)--(B) node[anchor=north,midway]{$\phi_B$};
%	\draw[->>] (AB)--(A) node[anchor=south,midway]{$\ \pi_A$};
%	\draw[->>] (AB)--(B) node[anchor=north,midway]{$\quad \pi_B$};
%	\draw[->,dotted] (G)--(AB) node[anchor=south,midway]{$\exists !\ \phi : G \ra A\times B$};
%	\end{tikzpicture}

	\begin{center}
	\begin{tikzcd}
	&A \\
	G \arrow{ur}{\phi_A}
	\arrow[dotted]{rr}{\exists ! \phi : G \ra A \times B}
	\arrow{dr}[swap]{\phi_B}
	&&A\times B \arrow[two heads]{ul}{\pi_A}
	\arrow[two heads]{dl}{\pi_B} \\
	&B
	\end{tikzcd}
	\end{center}
More generally, if $\{A_i\}_{i\in I}$ is a family of groups, then $\prod_{i\in I} A_i$ satisfies a universal property (Exercise).
\newpage
\end{remark}

Note also that if $A$ and $B$ are groups, then
	\begin{enumerate}
	\item[\blt] $A \cong A' \myeq \{(a,e_B) \mid a \in A\}$
	\item[\blt] $B \cong B' \myeq \{(e_A,b) \mid b \in B\}$
	\item[\blt] $A'B' = A\times B$
	\item[\blt] $A'\cap B' = \{e\}$
	\item[\blt] $A'B' \nsg A \times B$
	\end{enumerate}
\ \\

\section*{Internal Direct Products}

\vs

\begin{defn}
A group $G$ is the (internal) \textbf{direct product} of the two subgroups $A,B \leq G$, denoted by $G = A \times B$, if the following conditions are satisfied:
	\begin{enumerate}
	\item[(i)] $AB = G$
	\item[(ii)] $A\cap B = \{e\}$
	\item[(iii)] $A,B \nsg G$
	\end{enumerate}

\vs

\begin{remark*}
If $A\times B$ is an external direct product, then $A \times B$ is an internal direct product of $A'$ and $B'$, as above.
\end{remark*}

\vs

\begin{example}\
\begin{enumerate}
\item[(a)] $\z_6 = A \times B$, where $A = \{\ov{0},\ov{3}\}, B = \{\ov{0},\ov{2},\ov{4}\}$. A good exercise is to show that $\z_{mn} \cong \z_m\times \z_n$ iff $(m,n) = 1$.
\item[(b)] Let $V = \{e,\cyc{1\ 2}\cyc{3\ 4},\cyc{1\ 3}\cyc{2\ 4},\cyc{1\ 4}\cyc{2\ 3}\} \nsg S_4$, $A = \{e,\cyc{1\ 2}\cyc{3\ 4}\}$, $B = \{e, \cyc{1\ 3}\cyc{2\ 4}\} \leq S_4$. Then $V = A \times B$.
\item[(c)] $S_4 = VS_3$, where $S_3 = \{\sigma \in S_4 \mid \sigma(4)=4\}$. Here, $VS_3$ is a group, but it is not a direct product.
\end{enumerate}
\end{example}
\end{defn}

\vs

\begin{proposition}
If $G$ is the internal direct product of two subgroups $A,B \leq G$, then $G$ is isomorphic to the direct product $A \times B$.
\end{proposition}

\begin{prf}
Define $\mu : A \times B \ra G$ by $\mu((a,b)) = ab$. Recall from Lemma 1.1.19 that $A$ and $B$ commute, since $[A,B] \sub A\cap B = \{e\}$.

$\mu$ is a group \hm. Let $(a,b),(a',b') \in A \times B$. Then
	\[\mu((a,b),(a',b')) = \mu((aa',bb')) = a(a'b)b' = (ab)(a'b') = \mu((a,b))\mu((a',b')) \]

$\mu$ is surjective. This is clear from our assumption that $G = AB$.

$\mu$ is injective. Let $(a,b) \in \ker \mu$. Then $ab = e \implies a = b\inv \in A\cap B = \{e\}$, hence $(a,b) = (e,e) = e_{A\times B}$. Thus $\ker \mu$ is trivial and $\mu$ is injective.
\end{prf}

\vs

\begin{corollary}
If $G$ is the internal direct product $G = A \times B$, then
	\begin{enumerate}
	\item[(a)] $A$ and $B$ commute.
	\item[(b)] Any $g \in G$ can be uniquely written in the form $g = ab$, $a \in A$, $b \in B$.
	\end{enumerate}
\end{corollary}

\ \\

\section*{Automorphisms}

\vs

\begin{defn}
An \textbf{automorphism} of a group $G$ is a group \iso \ $\alpha : G \ra G$. The set
	\[\aut (G) \myeq \{\alpha : G \ra G \mid \alpha \tn{ is an automorphism} \} \]
is a group under composition.
\end{defn}

\vs
\begin{example}
\begin{enumerate}\
\item[(a)] $\aut(\z) \cong \{\pm 1\}$
\item[(b)] $\aut(\z_n) \cong (\z_n)\x$
\item[(c)] If $A,B$ are groups, then $\aut(A) \times \aut(B) \hookrightarrow \aut(A \times B)$ via $(\alpha, \beta) \mapsto ((a,b) \mapsto (\alpha(a),\beta(b))$.
\item[(d)] $\aut(\z_2\times \z_2) \cong S_3$
\item[(e)] For any group $G$ and any $g \in G$, conjugation with $g$ defines an automorphism $\kappa_g : G \ra G$, $x \mapsto gxg\inv$ with $(\kappa_g)\inv = \kappa_{g\inv}$.
\end{enumerate}
\end{example}

\vs

\begin{defn}
An element $\alpha \in \aut(G)$ is called an \textbf{inner automorphism} if there exists $g \in G$ such that $\alpha(x) = gxg\inv$ for all $x \in G$. We then define the set
	\[\inn(G) \myeq \{\alpha \in \aut(G) \mid \alpha \tn{ is an inner \auto} \} \]
\end{defn}

\vs

\begin{lemma}Let $G$ be a group.
\begin{enumerate}
\item[(a)] $\inn(G) \cong G/Z(G)$
\item[(b)] $\inn(G) \nsg \aut(G)$
\end{enumerate}
\end{lemma}

\begin{prf}
Exercise.
\end{prf}

\vs

\begin{remark*}
$\out(G) \myeq \aut(G) / \inn(G)$ is called the group of \textbf{outer automorphisms} of $G$.
\end{remark*}

\vs

\begin{example}\
\item[(a)] $\inn(G) = \{\id\} \iff G = Z(G) \iff G$ is abelian.
\item[(b)] $\inn(S_n) \cong S_n$ for $n \geq 3$, since $Z(S_n) = \{\id\}$ by 1.1.4.
\item[(c)] $\inn(S_3) = \aut(S_3)$. To see this, note that any $\alpha \in \aut(S_3)$ permutes the elements $(1\ 2),(1\ 3),(2\ 3)$. Since $S_3$ is generated by these transpositions, every automorphism $\alpha $ is uniquely determined by $\alpha((1\ 2)),\alpha((1\ 3)),\alpha((2\ 3))$. Thus
	\[|\aut(S_3)| \leq 6 = |\inn(S_3)| \leq |\aut(S_3)| \implies \inn(S_3) = \aut(S_3) \]
\item[(d)] With some more work, one can show that
	\[\inn(S_3) = \aut(S_3) \]
for $n \ne 6$, and $|\out(S_6)| = [\aut(S_6) : \inn(S_6)] = 2$.
\end{example}

\vs

\begin{remark}\
\item[(a)] If $\alpha \in \aut(G)$ and $g \in G$, then $|\alpha(g)| = |g|$, since $g^m = e$ iff $(\alpha(g))^m = e$.
\item[(b)] If $\alpha \in \aut(G)$, then $\alpha(Z(G)) = Z(G)$. Let $g \in Z(G)$. Then
	\begin{align*}
	& gx = xg \quad \forall\ x \in G \\
	\implies & \alpha(x)\alpha(g) = \alpha(g)\alpha(x) \quad \forall\ x \in G \\
	\implies & y\alpha(g) = \alpha(g)y \quad \forall\ y \in G \\
	\implies & \alpha(g) \in Z(G) \\
	\implies & \alpha(Z(G)) \sub Z(G)
	\end{align*}
Apply the same argument to $\alpha\inv \in \aut(G)$ to obtain $\alpha\inv(Z(G)) \sub Z(G) \iff Z(G) \sub \alpha(Z(G))$, hence $\alpha(Z(G)) = Z(G)$.

\item[(c)] If $\alpha \in \aut(G)$, then $\alpha([G,G]) = [G,G]$, since $\alpha([x,y]) = [\alpha(x),\alpha(y)]$ for all $x,y \in G$.
\end{remark}

\vs

\begin{defn}
A subgroup $H \leq G$ is called \textbf{characteristic}, denoted $H \car G$, if $\alpha(H) = H$ for all $\alpha \in \aut(G)$.
\end{defn}

\vs

\begin{example*}
$\{e\},G,Z(G),[G,G]$
\end{example*}

\vs

\begin{lemma}
Let $G$ be a group, $A,B,H \leq G$.
\begin{enumerate}
\item[(a)] $H \car G \implies H \nsg G$
\item[(b)] $A \car H$ and $H \car G \implies A \car G$
\item[(c)] If $A \leq H$, then $A \car H \nsg G \implies A \nsg G$
\item[(d)] $A \car G$ and $B \car G \implies [A,B] \car G$
\end{enumerate}
\end{lemma}

\vs

\begin{example}\
\item[(a)] If $G = \z_2 \times \z_2$, then any order 2 subgroup of $G$ is normal but not characteristic.
\item[(b)] Let $G = S_4$, $H = \{\id,(1\ 2)(3\ 4),(1\ 3)(2\ 4),(1\ 4)(2\ 3)\} \nsg S_4$. Verify that
	\[H = [A_4,A_4] \car A_4 = [S_4,S_4] \car S_4\]
We have $A \myeq \{\id,(1\ 2)(3\ 4)\} \nsg H \car S_4$, but $A \not\nsg S_4$. In particular, $A \nsg H \nsg S_4$, but $A \not\nsg S_4$, so the relation $\nsg$ is not transitive.
\end{example}

\chapter{Finite Abelian Groups}

\vs

Recall Euler's $\phi$ function $\phi : \z \ra \N$ defined by $\phi(n) \myeq |\{1 \leq m \leq n \mid (n,m) = 1\}| = |\z_n\x|$.

\vs

\begin{lemma}
Let $R$ be a ring with $1$.
	\begin{enumerate}
	\item[(a)] For all $f \in R$, the map $\lambda_r : R \ra R$ given by $\lambda_r(x) = rx$ defines a group \hm\ from $(R,+)$ into $(R,+)$.
	\item[(b)] If $r \in R\x$, then $\lambda_r \in \aut((R,+))$ and the map $\lambda : R\x \ra \aut(R,+)$ given by $\lambda(r) = \lambda_r$ is an injective group \hm.
	\end{enumerate}
\end{lemma}

\begin{proof}\
\begin{enumerate}
\item[(a)] Follows from the distributive law of $R$. Note also that $\lambda_r \circ \lambda_s = \lambda{rs}$ for all $r,s \in R$.
\item[(b)] If $r \in R\x$, then $r\inv \in R$, so $\lambda_r \circ \lambda_{r\inv} = \lambda_1 = Id_R$. Thus $\lambda_r$ is bijective and is in $\aut(R,+)$. $\lambda : R\x \ra \aut(R)$ is a group \hm\ because $\lambda(rs) = \lambda_{rs} = \lambda_r \circ \lambda_s = \lambda(r) \circ \lambda(s)$ for all $r,s \in \R\x$. To see that $\lambda$ is injective, let $r \in \ker \lambda$. Then $1 = \lambda_r(1) = r 1 = r$, so $r = 1$.
\end{enumerate}
\end{proof}

\vs

\begin{proposition}
$\aut(\z_n) = \z_n\x$.
\end{proposition}

\begin{proof}
By 1.2.1, $\z_n\x$ injects into $\aut(\z_n)$ via $\ov{a} \mapsto \lambda_{\ov{a}}$. We need to show that $\lambda$ is surjective. Let $\alpha \in \aut(\z_n)$. $|\ov{1}| = n$, so $|\alpha(\ov{1})| = n$, hence $\alpha(\ov{1}) = \ov{a}$, where $(a,n) = 1$. From here, you should verify that $\alpha = \lambda_{\ov{a}}$.
\end{proof}

\vs

\begin{proposition}[Chinese Remainder Theorem]
Let $m,n \in \N$.
	\begin{enumerate}
	\item[(a)] The map $f : \z_{mn} \ra \z_m \times \z_n$ given by $f([a]_{mn}) = ([a]_m,[a]_n)$ is a ring \hm.
	\item[(b)] If $(m,n) = 1$, then $f$ is an \iso.
	\end{enumerate}
\end{proposition}

\begin{proof}\
\begin{enumerate}
\item[(a)] $f$ is well-defined. If $[a]_{mn} = [b]_{mn}$, then $mn \mid (a-b)$, so in particular $m \mid (a-b)$ and $n \mid (a-b)$, hence $[a]_m = [b]_m$ and $[a]_n = [b]_n$. $f$ immediately inherits additivity and multiplicity from the existing ring structures.

\item[(b)] Since $|\z_{mn}| = mn = |\z_m \times \z_n|$, $f$ is bijective if and only if $f$ is injective. Let $[a]_{mn} \in \ker f$. Then $n \mid a$ and $m \mid a$, so $nm \mid a$, since $(n,m) = 1$, so $[a]_{mn} = [0]_{mn}$.
\end{enumerate}
\end{proof}

\vs

\begin{consequence*}
If $n = p_1^{e_1} p_2^{e_2} \cdots p_r^{e_r}$, where $p_1,p_2,\dots,p_r$ are distinct primes and $e_1,e_2,\dots,e_n \in \N$, then $\z_n \cong \z_{p_1^{e_1}} \times \z_{p_2^{e_2}} \times \dots \times \z_{p_r^{e_r}}$ as rings by simple induction and 1.2.3.
\end{consequence*}

\vs

\begin{corollary}
Let $m,n \in \N$ with $(m,n) = 1$. Then
	\begin{enumerate}
	\item[(a)] $\z_{mn}\x \cong \z_m\x \times \z_n\x$.
	\item[(b)] $\phi(mn) = |\z_{mn}\x| = |\z_m\x \times \z_n\x| = \phi(m) \phi(n)$.
	\end{enumerate}
\end{corollary}

\vs

\begin{proposition}
Let $e \in \N$ and $p$ a prime.
	\begin{enumerate}
	\item[(a)] $\z_{1^e}\x \cong \z_2 \times \z_{2^{e-2}}$, if $e \geq 2$.
	\item[(b)] $\z_p\x \cong \z_{p-1}$.
	\item[(c)] If $p$ if odd, then $\z_{p^e}\x \cong \z_{p^{e-1}(p-1)}$.
	\end{enumerate}
\end{proposition}

\begin{proof}
Exercise.
\end{proof}

\vs

\begin{proposition}[Cauchy's Theorem for Finite Abelian Groups]
Let $A$ be an abelian group of order $n$, and let $p$ be a prime divisor of $n$. Then $A$ contains an element of order $p$.
\end{proposition}

\begin{proof}
By induction on $n = |A|$. If $n = 1,2$, the result is certainly true, as $A$ is cyclic. Suppose $n \geq 3$ and that the result is true for all abelian groups of order at most $n-1$. Let $x \in A \bs\{e\}$, and put $|a| = m$. There are two cases.
	\begin{enumerate}
	\item[1. ] $p \mid m$. Then $a^{m/p}$ has order $p$.
	\item[2. ] $p \nmid m$. Then $(p,m) = 1$. Since $|A/\gen{a}| = n/m$ and $p \mid n$, we have that $p \mid n/m$, hence by our inductive hypothesis  there exists $\ov{x} \in A/ \gen{a}$ with $|\ov{x}| = p$. But then $p \mid |x|$, so we are back to case one. That is, $x^{|x|/p}$ has order $p$.
	\end{enumerate}
\end{proof}

\begin{proposition}
If $A$ is a finite abelian group and $m \in \N$ with $m \mid |A|$, then $A$ has a subgroup of order $m$.
\end{proposition}

\begin{proof}
By induction on $|A|$. The case $|A| = 1$ is trivial. We may now assume $|A| < m > 1$. Choose any prime divisor $p$ of $m$. By 1.2.6, there exists $a \in A$ with $|a| = p$. Set $\ov{A} = A / \gen{a}$, so that $|\ov{A}| = |A|/p$. Since $m \mid |A|$, we have $\frac{m}{p} \mid \frac{|A|}{p}$. We apply the induction hypothesis to $\ov{A}$ to show that there exists $\ov{H} \leq \ov{A}$ with $|\ov{H}| = \frac{m}{p}$. By the correspondence theorem, $\ov{H} = H/\gen{a}$ for a unique subgroup $H$ of $A$, and $\frac{m}{p} = |\ov{H}| = \frac{|H|}{p}$, so $|H| = m$.
\end{proof}

\vs

\begin{corollary}
Let $A$ be a finite abelian group with \pf{|A|}. Then there exist subgroups $B_i \leq A$, $1 \leq i \leq r$, with $|B_i| = p_i^{e_i}$ and $A = B_1 \times B_2 \times \dots \times B_r$. Moreover, $B_i = \{a \in A \mid |A| \tn{ is a power of } p_i\}$ for each $1 \leq i \leq r$.
\end{corollary}

\begin{proof}
The existence of $B_i \leq A$ with $|B_i| = p_i^{e_i}$ follows from 1.2.7. Since $A$ is abelian, $B_i \nsg A$ for each $1 \leq i \leq r$. Let $1 \leq i \leq r$. Now certainly $B_1 \cdots B_{i-1} B_{i+1} \cdots B_r \nsg A$, and $|B_1 \cdots B_{i-1} B_{i+1} \cdots B_r| \mid \prod_{j\ne i} p_j^{e_j}$, so $(|B_i|,|B_1 \cdots B_{i-1} B_{i+1} \cdots B_r|) = 1$. Therefore,
	\begin{align*}
	&B_i \cap B_1 \cdots B_{i-1} B_{i+1} \cdots B_r = \{e\} \qquad \forall \ 1 \leq i \leq r \\
	\implies & |B_1 B_2 \dots B_r| = |B_1| |B_2| \dots |B_r| = |A| \\
	\implies &A = B_1 \times B_2 \times \dots \times B_r
	\end{align*}
Finally, let $a \in A$. Then $a = b_1 b_2 \dots b_r$ for some unique $b_i \in B_i$, $1 \leq i \leq r$, so $|a|$ is a power of $p_i$ if and only if $b_1 = b_2 = \dots = b_{i-1} = b_{i+1} = \dots = b_r = e \iff a \in B_i$.
\end{proof}

\vs

\begin{remark*}
The $B_i$ above are the Sylow $p_i$-subgroups of $A$.
\end{remark*}

\vs

	\begin{lemma*}
	Let $A$ be a finite abelian group of order $n$. Suppose that for every divisor $d$ of $n$, $A$ has a unique subgroup of order $d$. Then $A$ is cyclic.
	\end{lemma*}

	\begin{proof}
	By induction on $n = |A|$. If $n=1$, then $A$ is trivially cyclic. Suppose $|A| = n > 1$, and the result is true for all abelian groups of order at most $n-1$. In particular, all proper subgroups of $A$ must be cyclic. Fix a prime divisor $p$ of $n$, and define $\phi_p : A \ra A$ by $\phi_p(a) = p a$. Since $A$ is abelian, $\phi_p$ is a group homomorphism. Since every nonidentity element of $\ker \phi_p$ has order $p$, and by assumption $A$ has at exactly one subgroup of order $p$, we must have that $|\ker \phi_p| = p$. By the first isomorphism theorem for groups,
		\[ G/ \ker \phi_p \cong \im \phi_p \leq G, \]
	so $G$ contains a subgroup $H \myeq \im \phi_p$ of order $n/p$. Since $H$ is a proper subgroup of $G$, $H$ is cyclic, so $H = \gen{x}$ for some $x \in G$. Moreover, $x \in \im \phi_p$, so $x = y^p$ for some $y \in G$. There are now two cases.
		\begin{enumerate}
		\item[(i)] $p \mid \frac{n}{p}$. We claim that $|y| = n$. Put $|y| = l$. Clearly, $p \mid l$. If $l < n$, then $\frac{l}{p} < \frac{n}{p}$, so $e = y^l = (y^p)^{l/p} = x^{l/p}$, contradicting the fact that $|x|=n/p$. Since $y^n = (y^p)^{n/p} = x^{n/p} = e$, we have $|y| = n$, hence $A = \gen{y}$ is cyclic.

		\item[(ii)] $p \nmid \frac{n}{p}$. Then $(p,\frac{n}{p}) = 1$. By Cauchy's theorem, there exists $a \in A$ with $|a| = p$. Therefore, $|ay| = |a| |y| = p \left( \frac{n}{p} \right) = n$, so $A = \gen{ay}$ is cyclic.
		\end{enumerate}
	\end{proof}
	\begin{proposition}
	Let $A$ be a finite abelian group of order $n$.
	\begin{enumerate}
	\item[(a)] If $A$ is cyclic and $m \in \N$, then
		\[|\{a \in A \mid a^m = e\}| = \begin{cases}
												m, \quad \tn{if } m \mid n \\
												< m, \quad \tn{if } m \nmid n
									   \end{cases} \]
	\item[(b)] If for all $m \mid n$, $|\{a \in A \mid a^m = e\}| \leq m$, then $A$ is cyclic.
	\end{enumerate}
	\end{proposition}

\begin{prf}
For $m \in \N$, set $A_m \myeq \{a \in A \mid a^m = e\}$.
	\begin{enumerate}
	\item[(a)] Suppose $A = \gen{a}$ is cyclic. We claim that $|A_m| = (n,m)$. Put $d \myeq (n,m)$. For $1 \leq k \leq n$, we have
		\begin{align*}
		a^k \in A_m &\iff n \mid km \\
		&\iff \frac{n}{d} \mid k \frac{m}{d} \\
		&\iff \frac{n}{d} \mid k \\
		&\iff k = \frac{n}{d}, \frac{2n}{d}, \dots, \frac{dn}{d} = n
		\end{align*}
	From this, we see that $|A_m| = d$. If $m \mid n$, then $|A_m| = d = m$. If $m \nmid n$, then $d$ is a proper divisor of $m$, so $d < m$, and therefore $|A_m| = d < m$.

	\item[(b)] By $1.2.7$, $A$ has a subgroup of order $m$ for each divisor $m$ of $n$. Since any such subgroup is clearly contained in $A_m$, we have that $|A_m| = m$ for each divisor $m$ of $n$. That is, $A$ has a unique subgroup of order $m$ for each divisor $m$ of $n$. By the lemma, $A$ is cyclic.
	\end{enumerate}
\end{prf}

\vs

\begin{theorem}[Fundamental Theorem of Finite Abelian Groups]
Let $A$ be a finite abelian group with \pf{|A|}. Then, for each $1 \leq i \leq r$, there exist uniquely determined $l_i \in \N$ and $m_{i_1} \leq m_{i_2} \leq \dots \leq m_{i_{l_i}} \in \N$ such that $e_i = \sum_{j=1}^{l_i} m_{i_j}$ and $A \cong B_1 \times B_2 \times \dots \times B_r$, where each $B_i$ is of the form $B_i \cong \z_{p_i}^{m_{i_1}} \times \z_{p_i}^{m_{i_2}} \times \dots \times \z_{p_i}^{m_{i_{l_i}}}$.
\end{theorem}

\vs

\begin{remark*}
A proof of the Fundamental Theorem of Finite Abelian groups can be found in Dummit and Foote section 6.1 (p. 197). The $m_{i_j}$ are unique because $m_{i_j} = |\{b \in B_i \mid |b| = p_i^j\}|$. Later, we will be able to prove a more general theorem using "elementary divisors".
\end{remark*}

\vs

\begin{example*}
Classify, up to isomorphism, all abelian groups of order 72. $72 = 3^2 \cdot 2^3$, so there are six isomorphism classes:
	\begin{align*}
	&\z_8 \times \z_9, \z_4 \times \z_2 \times \z_9, \z_2 \times \z_2 \times \z_2 \times \z_9, \\
	&\z_8 \times \z_3 \times \z_3, \z_4 \times \z_2 \times \z_3 \times \z_3, \z_2 \times \z_2 \times \z_2 \times \z_3 \times \z_3
	\end{align*}
\end{example*}

\vs

\begin{remark}
Any abelian group, finite or infinite, is a $\z$-module.
\end{remark}

\vs

\begin{defn}
Let $p$ be a prime. An abelian group $A$ is called an \textbf{elementary abelian $p$-group} if $|a| = p$ for all $a \in A \bs \{e\}$.
\end{defn}

\vs

\begin{remark}
If $A$ is an elementary abelian $p$-group, then $A$ is a vector space over $\mathbb{F}_p$. Moreover, any group \hm\ $f : A \ra A$ is automatically $\mathbb{F}_p$-linear.
\end{remark}

\vs

\begin{corollary}
Let $A$ be an elementary abelian $p$-group.
	\begin{enumerate}
	\item[(a)] $A$ is a direct sum of copies of $(\z_p,+)$.
	\item[(b)] $\aut(A) = GL(A) = \{f : A \ra A \mid f \tn{ is bijective and $\mathbb{F}_p$-linear}\}$.
	\end{enumerate}
In particular, if $|A| = p^n$, then $\aut(A) \cong GL_n(\mathbb{F}_p)$.
\end{corollary}

\begin{proof}\
\begin{enumerate}
\item[(a)] $A$ has a basis $(e_i)_{i \in I}$, so $A = \bigoplus_{i\in I} \mathbb{F} e_i \cong \bigoplus_{i \in I} \z_p$.
\item[(b)] Clearly, $GL_A \sub \aut(A)$, and $\aut(A) \sub GL(A)$ by 1.2.13.
\end{enumerate}
\end{proof}

\vs

\begin{lemma}
If $F$ is a finite field of order $q$, then $|GL_n(F)| = \prod_{j=0}^{n-1} (q^n - q^j) = |GL_n(F)|$.
\end{lemma}

\begin{proof}[Proof sketch]
Let $M \in M_n(F)$. Put $M = \cyc{v_1 \ v_2 \dots v_n}$. Then $M \in GL_n(F)$ if and only if $(v_1, v_2, \dots v_n)$ is a basis for $F^n$. Thus there are $q^n-1$ choices for $v_1 \in F^n \bs \{0\}$, $q^n-q$ choices for $v_2 \in F^n \bs \spn(v_1)$,..., $q^n-q^{k-1}$ choices for $v_k \in F^n \bs \spn(v_1,v_2,\dots,v_{k-1})$, ..., and $q^{n}-q^{n-1}$ choices for $v_n \in F^n\bs\spn(v_1,v_2,\dots,v_{n-1})$.
\end{proof}



\chapter{Nilpotent and Solvable Groups}

\begin{defn}
Let $G$ be a group.
\begin{enumerate}
\item[(a)] The \textbf{lower (or descending) central series} of $G$ is defined inductively by
	\begin{align*}
	G^{[0]} &= G, \\
	G^{[1]} &= [G,G], \\
	&\vdots \\
	G^{[i+1]} &= [G, G^{[i]}]
	\end{align*}
$G$ is called \textbf{nilpotent} if there exists $n \in \N_0$ such that $G^{[n]} = \{e\}$. We say that $G$ is nilpotent of class $n \in \N_0$ if $G^{[n]} = \{e\}$ and $G^{[k]} \ne \{e\}$ for all $0 \leq k \leq n-1$.

\item[(b)] The \textbf{derived (or commutator) series} of $G$ is defined inductively by
	\begin{align*}
	G^{(0)} &= G, \\
	G^{(1)} &= [G,G], \\
	&\vdots \\
	G^{(i+1)} &= [G^{(i)},G^{(i)}]
	\end{align*}
$G$ is called \textbf{solvable} if there exists $n \in \N_0$ such that $G^{(n)} = \{e\}$.
\end{enumerate}
\end{defn}

\vs

\begin{remark}\
\begin{enumerate}
\item[(a)] Obviously, $\ds{G}{i} \sub \lcs{G}{i}$ for all $i \in \N_0$, so if $G$ is nilpotent, then $G$ is solvable.
\item[(b)] If $G$ is abelian, then $[G,G] = \{e\}$, so $G$ is nilpotent of class $0$ if $G=\{e\}$, and $G$ is nilpotent of class $1$ if $G \ne \{e\}$.
\item[(c)] $G$ is nilpotent of class 2 if and only if $\lcs{G}{i} = [G,G] \ne \{e\}$ and $[G, \lcs{G}{1}] = \{e\}$, which is true if and only if $\lcs{G}{i} \leq Z(G)$.
\item[(d)] Recalling 1.1.35, $A,B \car G \implies [A,B] \car G$, so $\lcs{G}{i}, \ds{G}{i} \car G$ for all $i \in \N_0$.
\end{enumerate}
\end{remark}

\vs

\begin{lemma}
Suppose $G \ne \{e\}$.
\begin{enumerate}
\item[(a)] If $G$ is nilpotent, then $Z(G) \ne \triv$.
\item[(b)] If $G$ is solvable, then there exists an abelian subgroup $A \ne \triv$ such that $A \car G$.
\end{enumerate}
\end{lemma}

\begin{prf}\
\begin{enumerate}
\item[(a)] There exists $n \in \N$ such that $\lcs{G}{n-1} \ne \triv$ and $\lcs{G}{n} = \triv$. That is, $\triv = \lcs{G}{n} = [G,\lcs{G}{n-1}] \implies \triv \subsetneq \lcs{G}{n-1} \leq Z(G)$.
\item[(b)] There exists $n \in \N$ such that $\ds{G}{n-1} \ne \triv$ but $\ds{G}{n} = \triv \implies \ds{G}{n-1} \ne \{e\}$ is abelian. As observed, $A \myeq \ds{G}{n-1} \car G$.
\end{enumerate}
\end{prf}

\vs

\begin{example}
\
\begin{enumerate}
\item[(a)] The quaternion group $Q$ is such that $Z(Q) = \{1,-1\} = [Q,Q]$, so $Q$ is nilpotent of class 2.
\item[(b)] $S_n$ is not nilpotent for $n \geq 3$ because $Z(S_n) = \triv$ for $n \geq 3$. However, $S_3$ and $S_4$ are solvable: $\ds{S_3}{1} = A_3$, $\ds{S_3}{2} = \triv$, $\ds{S_4}{1} = A_4$, $\ds{S_4}{2} = [A_4,A_4] = V$, $\ds{S_4}{3} = \triv$.
\item[(c)] $A_4$ is solvable but not nilpotent (Exercise).
\item[(d)] $S_n, A_n$ are not solvable for $n \geq 5$, since $[S_n,S_n] = A_n, [A_n,A_n] = A_n$ (simple).
\item[(e)] By application of the Sylow Theorem, if $|G| < 60$, then $G$ is solvable. If $|G| = 60$ and $G$ is not solvable, then $G \cong A_5$.
\item[(f)] Let $F$ be a field. We define
	\begin{align*}
	U_n(F) &\myeq \left\{ \mymatrix{ccccc}{1 & * & * & \cdots & * \\
										 0 & 1 & * & \cdots & * \\
										 0 & 0 & 1 & \cdots & * \\
										 \vdots & \vdots & \vdots & \ddots & \vdots \\
										 0 & 0 & 0 & \cdots & 1}  \Bigg| * \in F \right\} \leq GL_n(F) \\
	B_n(F) &\myeq \left\{ \mymatrix{ccccc}{\bullet & * & * & \cdots & * \\
										   0 & \bullet & * & \cdots & * \\
										   0 & 0 & \bullet & \cdots & * \\
									   	   \vdots & \vdots & \vdots & \ddots & \vdots \\
										   0 & 0 & 0 & \cdots & \bullet}  \Bigg| \bullet \in F\x, * \in F \right\} \leq GL_n(F) \\
	D_n(F) &\myeq \left\{ \mymatrix{ccccc}{\bullet & 0 & 0 & \cdots & 0 \\
											   0 & \bullet & 0 & \cdots & 0 \\
											   0 & 0 & \bullet & \cdots & 0 \\
										   	   \vdots & \vdots & \vdots & \ddots & \vdots \\
											   0 & 0 & 0 & \cdots & \bullet}  \Bigg| \bullet \in F\x \right\} \leq GL_n(F)
	\end{align*}
\end{enumerate}
\end{example}

\vs

\section*{Some More Facts}

\vs

\begin{proposition*}
If $G$ is a finite $p$-group, then $G$ is nilpotent (proof to come later).
\end{proposition*}

\vs

\begin{theorem*}[Burnside]
If $|G| = p^nq^m$, where $p,q$ are primes, then $G$ is solvable.
\end{theorem*}

\vs

\begin{theorem*}[Feit-Thompson (1963)]
Every finite group of odd order is solvable.
\end{theorem*}

\vs

\begin{proposition}
If $G$ is nilpotent and $H < G$ is a proper subgroup of $G$, then $N_G(H) \ne H$.
\end{proposition}

\begin{proof}
We have $G = \lcs{G}{0} \ne H$, but $\lcs{G}{n} = \{e\} \leq H$ for some $n \in N$, so there exists $m \in \N$ such that $\lcs{G}{m-1} \not\leq H$ but $\lcs{G}{m} \leq H$. Thus $[H,\lcs{G}{m-1}] \leq [G,\lcs{G}{m-1}] = \lcs{G}{m} \leq H$, which implies that $\lcs{G}{m-1} \leq N_G(H)$ by 1.1.19 (b), but $\lcs{G}{m-1} \not\leq H$, so $H < N_G(H)$.
\end{proof}

\vs

\begin{defn}
A (finite or infinite) sequence of \hms\ $\dots G_{k-1} \overset{f_{i-1}}{\longrightarrow} G_i \overset{f_i}{\longrightarrow} G_{i+1} \ra \dots$ is called \tb{exact} if $\im f_{i-1} = \ker f_i$ for each $i$. A \tb{short exact sequence} (SES) is an exact sequence of the form $\ses{N}{f_1}{G}{f_2}{Q}$. \\

\noindent Exactness means that
	\begin{enumerate}
	\item[(a)] $f_1$ is injective, so that $N \cong f_1(N) \leq G$.
	\item[(b)] $f_2$ is surjective.
	\item[(c)] $f_1(N) = \ker f_2$.
	\item[(d)] $Q \cong G/f_1(N)$.
	\end{enumerate}
\end{defn}

\vs

\begin{remark*}\
\begin{enumerate}
\item[(a)] If $\phi : G \ra H$ is a group \hm, then $\ses{\ker \phi}{}{G}{}{\phi(G)}$ is a SES.
\item[(b)] If $N \nsg G$, then $\ses{N}{}{G}{}{G/N}$ is a SES.
\end{enumerate}
\end{remark*}

\vs

\begin{proposition}
Let $G$ be a group, $H \leq G$, and $\ses{N}{f_1}{G}{f_2}{Q}$ a SES. Then
	\begin{enumerate}
	\item[(a)] If $G$ is nilpotent/solvable, then so is $H$.
	\item[(b)] If $G$ is nilpotent/solvable, then so is $Q$ (any homomorphic image of $G$).
	\item[(c)] If $N$ and $Q$ are solvable, then $G$ is solvable.
	\item[(d)] If $Q$ is nilpotent and $f_1(n) \leq Z(G)$, then $G$ is nilpotent.
	\end{enumerate}
\end{proposition}

\begin{proof}\
\begin{enumerate}
\item[(a)] Verify inductively: $\ds{H}{i} \leq \ds{G}{i}$, $\lcs{H}{i} \leq \lcs{G}{i}$ for all $i \in \N_0$.
\item[(b)] Verify inductively: Because $f_2$ is surjective, $f_2(\ds{G}{i}) = \ds{Q}{i}$ for all $i \in \N_0$.
\item[(c)] Because $Q$ is solvable, there exists $n \in \N_0$ such that $\ds{Q}{n} = \{e\}$. Recall that $f_2(\ds{G}{i}) = \ds{Q}{i}$ for all $i \in \N_0$, so
	\begin{align*}
	&\ds{G}{n} \leq \ker f_2 = \im f_2 \cong N \ \tn{(solvable)} \\
	\implies &\ds{G}{n} \ \tn{is solvable} \\
	\implies &G \ \tn{is solvable}
	\end{align*}
\item[(d)] Since $Q$ is nilpotent, there exists $n \in \N_0$ such that $\lcs{Q}{n} = \{e\}$. Now $f_2(\lcs{G}{i}) = \lcs{Q}{i}$ for all $i \in \N_0$, so
	\begin{align*}
	f_2(\lcs{G}{n}) &= \lcs{Q}{n} = \triv \\
	\implies \lcs{G}{n} &\leq \ker f_2 = \im f_1 \leq Z(G) \\
	\implies \lcs{G}{n+1} &= [G, \lcs{G}{n}] \leq [G,Z(G)] = \triv
	\end{align*}
\end{enumerate}
\end{proof}

\vs

\begin{remark}
1.3.7 (c) is not true with "solvable" replaced by "nilpotent". For example, consider the SES $\ses{A_3 \cong \z_3}{}{S_3}{}{S_3/A_3 \cong \z_2}$. $z_2,z_3$ are even abelian, but $S_3$ is not nilpotent.
\end{remark}

\vs

\begin{corollary}
Let $A,B \leq G$, $A \leq N_G(B)$, with $A,B$ solvable. Then $AB$ is a solvable of $G$.
\end{corollary}

\begin{proof}
Exercise.
\end{proof}

\vs

\begin{remark*}
1.3.9 is not true with "solvable" replaced by "nilpotent". For example, consider $G = S_3$, $B = A_3$, $A = \gen{(1\ 2)}$. Then $G = AB$, but $G$ is not nilpotent. However, $A,B$ are abelian, so they are nilpotent.
\end{remark*}

\vs

\begin{lemma}
Let $G_1, G_2$ be groups, and let $G_1 \times G_2$ be the external direct product.
	\begin{enumerate}
	\item[(a)] If $A_1, B_1 \leq G_1$ and $A_2, B_2 \leq G_2$, then
		\[[A_1\times A_2, B_1 \times B_2] = [A_1,B_1] \times [A_2,B_2] \]
	\item[(b)] $G_1 \times G_2$ is solvable/nilpotent if and only if $G_1$ and $G_2$ are both solvable/nilpotent.
	\end{enumerate}
\end{lemma}

\begin{prf}\
\begin{enumerate}
\item[(a)] Exercise.
\item[(b)] $\ses{G_1}{}{G_1\times G_2}{}{G_2}$ is a SES, so the result about solvability follows from 1.3.7. \\

We now treat nilpotency. Note that $G_1, G_2 \leq G_1 \times G_2$ are nilpotent if $G_1 \times G_2$ is nilpotent. Now, assume $G_1$ and $G_2$ are nilpotent. Let's say that $G_1$ is nilpotent of class $c_1$ and $G_2$ is nilpotent of class $c_2$. Put $c = \max(c_1,c_2)$. Then by part (a) above, $G_1 \times G_2$ is nilpotent of class $c$, since $\lcs{(G_1 \times G_2)}{i} = \lcs{G_1}{i} \times \lcs{G_2}{i}$ (use (a) and induction on $c$).
\end{enumerate}
\end{prf}

\vs

\begin{proposition}
If $A$ and $B$ are nilpotent normal subgroups of $G$, then $AB$ is nilpotent (and normal) in $G$.
\end{proposition}

\begin{proof}
WLOG, assume $G = AB$. Say that $A$ is nilpotent of class $c$ and $B$ is nilpotent of class $d$. WLOG, assume $A,B \ne \triv$.

We proceed by induction on $c+d$. Recall that $Z(A) \car A \nsg G \implies Z(A) \nsg G$, $Z(B) \car B \nsg G \implies Z(B) \nsg G$. Consider $\ov{A} \myeq A/Z(A)$, which is nilpotent of class at most $c-1$. Indeed, $\triv = \lcs{A}{c} = [A,\lcs{A}{c-1}] \implies \lcs{A}{c-1} \leq Z(A)$. Hence $\lcs{\ov{A}}{c-1} = \lcs{A}{c-1}Z(A)/Z(A) = \{e\}$ in $A/Z(A)$. $\ov{G} \myeq G/Z(A) = AB/Z(A) = \ov{A}(BZ(A)/Z(A))$, where $BZ(A)Z(A) \cong B/B\cap Z(A)$ is nilpotent of class at most $d$. Similarly, $G/Z(B) = (AZ(B)/Z(B))(B/Z(B))$ is a product of two normal nilpotent subgroups whose sum of nilpotency classes is at most $c+d-1$. We therefore apply the induction hypothesis to get that $G/Z(A)$ and $G/Z(B)$ are both nilpotent.

Now consider the homomorphisms $G/Z(A)\cap Z(B) \hookrightarrow G/Z(A) \times G/Z(B)$, $g(Z(A)\cap Z(B)) \mapsto (gZ(A),gZ(B))$ and the natural projection $G/Z(A)\cap Z(B) \twoheadrightarrow G/Z(G)$, $gZ(A)\cap Z(B) \mapsto gZ(G)$. $G/Z(A)\cap Z(B)$ is nilpotent since it is isomorphic to a subgroup of a nilpotent group, and $G/Z(G)$ is nilpotent since it is the homomorphic image of the nilpotent group $G/Z(A)\cap Z(B)$. But then $G/Z(G)$ is nilpotent implies that $G$ is nilpotent by 1.3.7 (d).
\end{proof}

\vs

\begin{remark*}
This proof shows that $AB$ is nilpotent of class at most $c+d$.
\end{remark*}

\vs

\section*{Lower and Upper Central Series}

\vs

\begin{question}
If $N \nsg G$, what is the smallest normal subgroup $N'$ of $G$ which is contained in $N$ such that $N/N' \leq Z(G/N)$?
\end{question}

\begin{answer}
$N' = [G,N] \nsg G$, and also $[G,N] \nsg N$ by 1.1.19.
\end{answer}

\vs

Now, if we start with $N=G$, $N' = [G,G]$, $N'' = \lcs{G}{2}, \dots$, then the result is what we call the \textit{lower central series.}

\vs

\begin{question}
If $N \nsg G$, what is the largest subgroup $N'$ of $G$ such that $N \leq N'$ and $N'/N \leq Z(G/N)$?
\end{question}

\begin{answer}
The subgroup of $G$ containing $N$ which corresponds to $Z(G/N)$ (think correspondence theorem), so $N'/N = Z(G/N)$.
\end{answer}

\vs

\begin{defn}
For a group $G$, the \tb{upper central series} is defined inductively as follows: $Z_0(G) \myeq \triv, Z_1(G) \myeq Z(G)$, and $Z_{i+1}(G)$ is the unique subgroup of $G$ such that  $Z_{i+1}(G)/Z_i(G) = Z(G/Z_i(G))$ for $i \in \N$.
\end{defn}

\vs

\begin{remark}
\begin{enumerate}\
\item[(a)] One can show that $Z_i(G) \car G$ for all $i \in \N_0$.
\item[(b)] $G$ is nilpotent if and only if there exists $n \in \N_0$ such that $Z_n(G) = G$.
\item[(c)] Define for any group $G$ $c(G) \myeq \min \{n \in \N_0 \mid \lcs{G}{n} = \triv \}$, where $c(G) \myeq \infty$ if no such $n$ exists, and $c'(G) \myeq \min\{n \in \N_0 \mid Z_n(G) = G\}$ ($\myeq \infty$ if no such $n$ exists). Then $c(G) = c'(G)$.
\end{enumerate}
\end{remark}



\chapter{Simple Groups and Composition Series}

\begin{defn}
A group $G \ne \triv$ is called \tb{simple} if $G$ and $\triv$ are the only normal subgroups of $G$.
\end{defn}

\vs

\begin{example}
For any prime $p$, $\z_p$ is simple (proof: Lagrange).
\end{example}

\vs

\begin{remark}\
\begin{enumerate}
\item[(a)] A group $G$ is both solvable and simple if and only if $G$ is cyclic of prime order (proof: exercise).
\item[(b)] If $G$ is simple and not cyclic of prime order, then
	\begin{enumerate}
	\item[(i)] $G$ is not solvable (by (a)).
	\item[(ii)] $G$ is not abelian.
	\item[(iii)] $Z(G), [G,G] \ne \triv$.
	\item[(iv)] $[G,G] = G, Z(G) = \triv$ (because $G$ is simple).
	\end{enumerate}
\item[(c)] If $G$ is simple and $\phi : G \ra H$ is a group \hm, then either $\ker \phi = \triv$ or $\ker \phi = G$.

The first "interesting" examples of simple groups are
	\begin{enumerate}
	\item[(i)] $A_n$ for $n \geq 5$
	\item[(ii)] $PSL_n(F) \myeq SL_n(F)/Z(SL_n(F))$ is simple for $n \geq 2$ provided $(n,|F|) \ne (2,2),(2,3)$.
	\end{enumerate}
\item[(d)] For $1 \leq i \ne j \leq n$ and $\lambda \in F$, define $E_{ij}(\lambda)$ to be the matrix such with $ij$-th entry $\lambda$, diagonal entries $1$, and $0$ elsewhere. Let $E_{ij}(F) = \{E_{ij}(\lambda) \mid \lambda \in F\}$.
\end{enumerate}
\end{remark}

\vs

\begin{theorem}
$A_n$ is simple for $n \geq 5$.
\end{theorem}

\begin{proof}
Assume $\{id\} \ne N \nsg A_n$. We want to show that $N = A_n$. First, note that it suffices to show that $N$ contains a 3-cycle. The argument goes as follows. Assume $(i \ j \ k) \in N$, and let $(i'\ j'\ k') \in A_n$ be any 3-cycle. There exists $\pi \in S_n$ such that $\pi(i\ j\ k) \pi\inv = (i'\ j'\ k')$. If $\pi \in A_n$, then $(i'\ j'\ k') \in N$. Assume $\pi \notin A_n$. Then $\pi (i\ j) \in A_n$, and $(j\ i\ k) = (i\ j\ k)\inv \in N$, so
	\[\pi(i\ j) (j\ i\ k) (i\ j) \pi\inv = \pi (i\ j\ k) \pi\inv = (i'\ j'\ k') \in N \]

We now distinguish cases depending on $id \ne \sigma \in N$. Note that for any $\alpha \in A_n$, $[\sigma,\alpha] = \sigma (\alpha \sigma\inv \alpha\inv) \in N$.

\begin{enumerate}
\item[\tb{Case 1.}] The factorization of $\sigma$ into disjoint cycles contains a cycle of length at least 4. Then there exist distinct $i,j,k,l \in I_n \myeq \{1,2,\dots,n\}$ with $\sigma(i)=j$, $\sigma(j)=k$, $\sigma(k)=l$, and $\sigma(l)=i$. Set $\alpha \myeq (i\ j\ k) \in A_n$. Then
	\[ [\sigma,\alpha] = \sigma (i\ j\ k) \sigma\inv (j\ i\ k) = (j\ k\ l)(j\ i\ k) = (i\ l\ j) \]

\item[\tb{Case 2.}] $\sigma$ contains a 3-cycle $i\ j\ k)$ for some distinct $i,j,k \in I_n$, so $\sigma(i) = j, \sigma(j)=k, \sigma(k)=i$. If $\sigma$ fixes all elements of $I_n\bs \{i,j,k\}$, then $\sigma = (i\ j\ k) \in N$. Otherwise, there exists $l \in I_n\bs \{i,j,k\}$ such that $m \myeq \sigma(l) \ne l$. Set $\alpha \myeq (i\ j\ l)$. Then
	\[ [\sigma,\alpha] = \sigma (i\ j\ l) \sigma\inv (j\ i\ l) = (j\ k\ m)(j\ i\ l) = (i\ l\ k\ m\ j) \in N \]
We now apply case 1 to $[\sigma,\alpha]$ to show that $N$ contains a 3-cycle.

\item[\tb{Case 3.}] $\sigma$ is a product of disjoint transpositions and fixes at least one element of $I_n$. Then there exist distinct $i,j,k \in I_n$ such that $\sigma(i) = j, \sigma(j=i)$, $\sigma(k)=k$. Set $\alpha \myeq (i\ j\ k)$ so that
	\[[\sigma,\alpha] = \sigma(i\ j\ k)\sigma\inv (j\ i\ k) = (j\ i\ k)(j\ i\ k) = (i\ j\ k) \in N \]

\item[\tb{Case 4.}] $\sigma$ is a product of disjoint transpositions and does not fix any element of $I_n$. Then there exist $i,j,k,l \in I_n$ with $\sigma(i)=j$, $\sigma(j) = i$, $\sigma(k)=l$, and $\sigma(l)=k$. Set $\alpha \myeq (i\ j\ k)$. Then
	\[[\sigma,\alpha] = \sigma(i\ j\ k)\sigma\inv(j\ i\ k) = (j\ i\ l)(j\ i\ k) = (i\ k)(j\ l) \in N \]
Since $n \geq 5$, we apply case 3 to $[\sigma,\alpha]$ to show that $N$ contains a 3-cycle.
\end{enumerate}
\end{proof}

\vs

\begin{corollary}
$S_n, A_n$ are not solvable for $n \geq 5$.
\end{corollary}

\begin{proof}
Consider the derived series for $S_n$: $\ds{S_n}{1} = [S_n,S_n] = A_n$, and $A_n$ is both simple and non-abelian, therefore $\ds{S_n}{i} = A_n$ for all $i \in \N$.
\end{proof}

\vs

\begin{example}[Some Simple Linear Groups]
Let $F$ be a field. Can $GL_n(F)$ be simple? If $|F| > 2$, the answer is no, because $SL_n(F) \tl GL_n(F)$. If $|F|= 2$, then $SL_n(F) = GL_n(F)$. From now on, assum $n,|F|) \ne (2,2), (2,3)$. Next question: Can $SL_n(F)$ be simple? Only when $Z(SL_n(F)) = \{ \lambda I \mid \lambda^n=1\}$.

\begin{theorem*}
$PSL_n(F) \myeq SL_n(F)/Z(SL_n(F))$ is simple if $n,|F|) \ne (2,2), (2,3)$.
\end{theorem*}

\vs

We know that $|GL_n(F_q)| = \prod_{i=0}^{n-1} (q^n-q^i)$, and $\ses{SL_n(F_q)}{}{GL_n(F_q)}{\det}{F_q\x}$ is a SES, which implies that $|SL_n(F_q)| = \frac{1}{q-1} |GL_n(F_q)|$. Let $r_n(F_q) \myeq |\{\lambda \in F\x \mid \lambda^n=1\}|$. Then $|PSL_n(F_q)| = \frac{1}{(q-1)r_n(F_q)} |GL_n(F_q)|$. A few more facts include:
	\begin{list}{$\bullet$}{}
	\item $GL_3(F_2) = SL_3(F_2) = PSL_3(F_2)$
	\item $GL_2(F_2) \cong S_2$ (solvable)
	\item $GL_2(F_3)$ and $PSL_2(F_3) \cong A_4$ are both solvable.
	\item $PSL_2(F_4) \cong A_5 \cong PSL_2(F_5)$
	\item There is one isomorphism class of groups of order 168, namely $PSL_2(F_7) \cong GL_3(F_2)$.
	\item $PSL_2(F_9) \cong A_6$
	\item The only simple non-abelian groups of order at most 360 are, up to isomorphism, $A_5, GL_3(F_2), A_6$.
	\item There are other simple matrix groups, which are finite if $F$ is finite, and they are called simple groups of "Lie type".
	\item The finite simple groups are: $\z_p$, $A_n$ $(n \geq 5)$, simple groups of Lie type (e.g. $PSL_n(F_q)$), and 26 sporadic finite simple groups.
	\end{list}

The program for studying finite simple groups is:
	\begin{enumerate}
	\item[1. ] Classify all finite simple groups (done).
	\item[2. ] "Extension Problem": Given groups $A,B$, how many non-isomorphic $G$ are there such that a SES $\ses{A}{}{G}{}{B}$ exists?
	\end{enumerate}
\end{example}

\vs

\begin{defn}
Let $G$ be a group. A finite sequence $\{e\} = G_0 \leq G_1 \leq \dots \leq G_n = G$ is called
	\begin{enumerate}
	\item[(a)] a \tb{subnormal series} if $G_{i-1} \nsg G_i$ for all $1 \leq i \leq n$. In this case, the $Q_i \myeq G_i/G_{i-1}$ are called the \tb{quotients} of the subnormal series.
	\item[(b)] a \tb{normal} series if $G_i \nsg G$ for all $1 \leq i \leq n$.
	\item[(c)] a \tb{composition series} if it is a subnormal series and all of the quotients $Q_i$ are simple.
	\end{enumerate}
\end{defn}

\vs

\begin{remark}\
\begin{enumerate}
\item[(a)] Every finite group $G$ has a composition series (seen by induction on $|G|$).
\item[(b)] If $G$ is infinite, it needn't have a composition series. For example, $\z$ has no composition series.
\item[(c)] A finite group is solvable if and only if it has a composition series whose quotients are cyclic of prime order.
\item[(d)] If $G$ admits a composition series, it does NOT necessarily follow that a subgroup $H \leq G$ admits a composition series. However, if $H \nsg G$, the statement is true (exercise).
\item[(e)] A group $G$ may have different composition series. For example, $G = A_5 \times A_6$ has $\{e\} \tl A_5 \tl G$ and $\{e\} \tl A_6 \tl G$.
\end{enumerate}
\end{remark}

\vs

\begin{theorem}[Jordan-H\"older]
Assume that $\{e\} = G_0 \tl G_1 \tl \dots \tl G_n = G$ and $\{e\} = G_0' \tl G_1' \tl \dots \tl G_m' = G$ are both composition series for the group $G$. Then $n=m$, and there exists a permutation $\pi \in S_n$ such that $G_i/G_{i-1} \cong G_{\pi(i)}/G_{\pi(i)-1}$ for all $1 \leq i \leq n$.
\end{theorem}

\begin{proof}
See Chapter 4 of Rotman.
\end{proof}



\chapter{Group Actions}

In this chapter, $G$ denotes a group and $X$ denotes a set acted upon by $G$.
\vs

\begin{defn}
We say that $G$ \tb{acts on} $X$ (on the left) if there is a map $G \times X \ra X$, $(g,x) \mapsto G\cdot x$ such that
	\begin{enumerate}
	\item[(i)] $g\cdot(h\cdot x)) = gh \cdot x$ for all $g,h \in G, x \in X$.
	\item[(ii)] $e \cdot x = x$ for all $x \in X$.
	\end{enumerate}
\end{defn}

\vs

\begin{consequence}
For each $g\in G$, we have a map $\pi_g : X \ra X$, $x \mapsto g\cdot x$. $(i)$ above gives $\pi_g \circ \pi_h (x) = \pi_{gh}(x) \implies \pi_g \circ \pi_h = \pi_{gh}$, and $(ii)$ gives $\pi_e = \id_X \implies \pi_g \circ \pi_{g\inv} = \id_x = \pi_{g\inv} \circ \pi_{g} \implies \pi_g \in S(X)$. Hence $\phi : G \ra S(X)$, $g \mapsto \pi_g$ is a group homomorphism.

Conversely, if $\phi : G \ra S(X)$ is a group \hm, then $G\times X \ra X, (g,x) \mapsto \phi(g)(x)$ defines a $G$-action on $X$. The concepts of $G$ acting on $X$ and \hms $G\ra S(X)$ are equivalent.

We call the $G$-action \tb{faithful} if $\ker \phi = \{e\}$.
\end{consequence}

\vs

\begin{generalization}
If $X$ carries additional algebraic structure (e.g., group, ring, vector space), we say that $G$ acts on $X$ if $G$ preserves this additional structure. That is, $\pi_g \in \aut (X)$, and so $\phi : G \ra \aut(X)$ is a restricted group \hm. A special case of this is when $X$ is a vector space. In this case, $\aut(X) = GL(X)$, and $\phi : G \ra GL(X)$ is called a \tb{linear representation.}
\end{generalization}


\vs

\begin{example}
\
\begin{enumerate}
\item[(a)] $S_n$ acts on $I_n$ by definition ($\sigma\cdot j = \sigma(j), \phi = \id_{S_n}$).
\item[(b)] $GL_n(F)$ acts on $F^n$ by matrix multiplication.
\item[(c)] If $G$ acts on $X$ and $H \leq G$, then $H$ acts on $X$ by $\phi_H : H \ra S(X)$.
\item[(d)] $G$ acts on the set $X=G$ by left multiplication: $g\cdot x = gx$. Note that here $\pi_g$ is just a permutation of $X$ and NOT a group \hm.
\item[(e)] If $H \leq G$, we consider the action of $G$ on $G/H$ by left multiplication: $G\times G/H \ra G/H$, $(g,g'H) \mapsto gg'H$.
\item[(f)] $G$ acts on the group $X=G$ by conjugation: $g\cdot x = gxg\inv \implies \pi_g = \kappa_g : G \ra G, x \mapsto gxg\inv$ (group automorphism).
\end{enumerate}
\end{example}

\vs

\begin{defn}
If $G$ acts on $X$, the \tb{stabilizer} of $x \in X$ is defined as
	\[G_x \myeq \{g \in G \mid g\cdot x = x\} \leq G \]
So $G_x = G$ if and only if $x$ is a fixed point of $G$. If $Y \sub X$, the \tb{stabilizer} of $Y$ is
	\[\stab{G}{Y} \myeq \{g \in G \mid g\cdot Y = Y\} \]
The \tb{fixer} of $Y \sub X$ is the set
	\[\fix{G}{Y} \myeq \{g \in G \mid g\cdot y = y \ \forall \ y \in Y\} = \cap_{y\in Y} G_y \]
\end{defn}

\vs

\begin{example}
\
\begin{enumerate}
\item[(a)] Let $G = S_n$. $G_n = \{\sigma \in S_n \mid \sigma(n) = n\} \cong S_{n-1}$ for $n \in I_n$.
\item[(b)] $GL_n(F)_0 = GL_n(F)$, so $0$ is a fixed point of this action.
\item[(c)] Let $H \leq G$, where $G$ acts on $X$ and $x\in X$. Then $H_x = H\cap G_x \sub H$.
\item[(d)] Let $G$ act on $G$ my left multiplication. Then $G_x = \{e\}$ for all $x \in X$. This is an example of a \tn{free} action.
\item[(e)] For $H \leq G$, let $G$ act on $G/H$ by left multiplication. Then for $x = gH \in G/H$, $G_x = gHg\inv$.
\item[(f)] Let $G$ act on $G$ by conjugation, $x \in G$. Then $G_x = \{g \in G \mid gxg\inv = x\} C_G(x)$. If $H sub G$, then $\stab{G}{H} = \{g \in G \mid gHg\inv = H\} = N_G(H)$, and $\fix{G}{H} = \{g\in G\mid ghg\inv = h \ \forall \ h \in H\} = C_G(H)$.
\end{enumerate}
\end{example}

\vs

\begin{remark}
Let $G$ act on the set $X$.
\begin{enumerate}
\item[(a)] If $x \in X$, then $G_{g\cdot x} = g G_x g\inv$.

$[\supseteq]$ Let $h \in G_x$. Then $h \cdot x = x$, so $ghg\inv \cdot (g\cdot x) = (ghg\inv g) \cdot x = (gh) \cdot x = g\cdot (h\cdot x) = g\cdot x$, hence $ghg\inv \in G_{g\cdot x}$.

$[\sub]$ Applying the above argument to $g\inv$ and $g \cdot x$ gives us $g\inv \cdot G_{g\cdot x} g \sub G_{g\inv \cdot (g\cdot x)} \implies G_{g\cdot x} \sub g G_x g\inv$.

\item[(b)] Let $\phi : G \ra S(X)$ be the associated group \hm. Since $\ker \phi = \bigcap_{x\in X} G_x$, $G$ acts faithfully on $X$ if and only if $\bigcap_{x\in X} G_x = \{e\}$.
\end{enumerate}
\end{remark}

\vs

\begin{corollary}[Cayley's Theorem]
If $G$ is a finite group of order $n$, then $G$ is isomorphic to a subgroup of $S_n$.
\end{corollary}

\begin{proof}
The action of $G$ on itself by left multiplication is faithful since $G_x = \{e\}$ for all $x \in G$, so the associated group \hm $\phi : G \ra S(G) \cong S_n$ is injective.
\end{proof}

\vs

\begin{proposition}
If $H \leq G$ with $[G : H] = n$, then there exists a group \hm\ $\phi : G \ra S_n$ with $G_1 = H$ and $\ker \phi = \bigcap_{g\in G} gHg\inv$.
\end{proposition}

\begin{proof}
$G$ acts by left multiplication on $X = G/H$. Let $\phi' : G \ra S(X)$ be the associated group \hm. Choose any bijective map $f : X \ra I_n$ satisfying $f(H) = 1$. Define $\psi : S(X) \ra S_n$ by $\pi \mapsto f \circ \pi \circ f\inv$ (injective). Define $\phi = \psi \circ \phi' : G \ra S(X) \ra S_n$, a group homomorphism with $\ker \phi = \ker \phi'$. Since $F(H) = 1$, we get $G_1 = G_H = H$.
\end{proof}

\vs

\begin{proposition}
Let $G$ be a finite group with $|G| = p_1 p_2 \cdots p_n$, all $p_i$ primes, and $p_1 \leq p_2 \leq \dots \leq p_n$. If $H \leq G$ is such that $[G : H] = p_1$, then $H$ is a normal subgroup of $G$.
\end{proposition}

\begin{proof}
Consider the group homomorphism $\phi : G \ra S(G/H) \cong S_{p_1}$. By 1.5.9, $\ker \phi \leq H$. We want to show that $\ker \phi = H$. Assume $K \myeq \ker \phi \ne H$. Then $[G : K] = [G : H][H : K] = p_1 [H : K]$ and $1 \ne [H : K] \mid |H| \mid |G|$. There exists $1 \leq i \leq n$ such that $p_i \mid [H : K] \implies p_1p_i \mid [G : K] = |G/K|$, but $G/K \hookrightarrow S_{p_1}$ by the first isomorphism theorem, so $|G/K| \mid |S_{p_1}| = p_1!$, hence $p_1p_i \mid p_1! \implies p_i \mid (p_1-1)!$, which is a contradiction because $p_i \geq p$ is a prime. Hence $K = H \implies H \nsg G$.
\end{proof}

\vs

\begin{defn}
The \tb{$\mb{G}$-orbit} of $x \in X$ is $G\cdot x \myeq \{g\cdot x \in X \mid g \in G\}$. The action of $G$ on $X$ is called \tb{transitive} if $G\cdot x = X$ for all $x \in X$.
\end{defn}

\vs

\begin{example*}[of transitive actions]
\
\begin{list}{$\bullet$}{}
\item $G$ acting on itself by multiplication
\item $G$ acting on $G/H$ by left multiplication
\item $S_n$ acting on $I_n$. In fact, the action of $\gen{(1\ 2\ \dots n)}$ on $I_n$ is transitive.
\end{list}
\end{example*}

\vs

\begin{lemma}[Orbit-Stabilizer Lemma]
For any $x \in X$, $|G\cdot x| = [G : G_x]$.
\end{lemma}

\begin{proof}
Consider the map $f : G/G_x \ra G\cdot x, gG_x \mapsto g\cdot x$.

$f$ is well-defined. Suppose $h=gs$ for some $S \in G_x$. Then $f(hG_x) = h \cdot x = (gs) \cdot x = g\cdot(s\cdot x) = g\cdot x = f(g G_x)$.

$f$ is injective. Suppose $f(gG_x) = f(hG_x)$. Then $g\cdot x = h\cdot x \implies h\inv g \in G_x \implies g G_x = h G_x$.

$f$ is surjective. Given $g\cdot x \in G\cdot x$, $g G_x \overset{f}{\longmapsto} g\cdot x$.

Since $f$ is bijective, $[G : G_x] = |G/G_x| = |G \cdot x|$.
\end{proof}

\vs

\begin{proposition}
Let $G$ act on the set $X$.
	\begin{enumerate}
	\item[(a)] $X$ is the disjoint union of $G$-orbits.
	\item[(b)] If $X$ is finite and $G_{x_1},G_{x_2},\dots,G_{x_r}$ are the distinct $G$-orbits in $X$, then
		\[|X| = \sum_{i=1}^r |G\cdot x_i| = \sum_{i=1}^r [G : G_{x_i}] \]
	\end{enumerate}
\end{proposition}

\begin{proof}
\
\begin{enumerate}
\item[(a)] Define the relation $\sim$ on $X$ by $x \sim y$ if $x \in G \cdot y$. It is easily verified that $\sim$ is an equivalence relation on $X$ with $[x]_{\sim} = G\cdot x$ for all $x \in X$.
\item[(b)] This follows easily from (a) and the Orbit-Stabilizer Lemma.
\end{enumerate}
\end{proof}

\vs

Let us now specialize to the situation in which $G$ acts on itself by conjugation. Stabilizers have the form $G_x = \{g \in G \mid gxg\inv = x\} = C_G(x)$, and an orbit $G\cdot x = \{gxg\inv \mid g\in G\}$ is the conjugacy class of $x\in G$. By 1.5.12, $|G\cdot x| = [G : C_G(x)]$ and $G\cdot x = \{x\} \iff C_G(x) = G \iff X \in Z(G)$.

\vs

\begin{theorem}[Class Equation]
Let $G$ be a finite group with $X_1,x_2,\dots,x_r$ distinct conjugacy class representatives of $G\bs Z(G)$. Then
	\[ |G| = |Z(G)| + \sum_{i=1}^r [G : G_{x_i}], \]
where $1 < [G : G_{x_i}] \mid |G|$ for each $1 \leq i \leq r$.
\end{theorem}

\begin{proof}
Let $x_{r+1},x_{r+2},d\dots,x_s$ be the elements of $Z(G)$. Then $s-r = |Z(G)|$ and $G\cdot x_i = \{x_i\}$ for all $r+1 \leq i \leq s$. Now $G = \bigsqcup_{i=1}^s G\cdot x_i$, so by 1.5.13 we have
	\[ |G| = |Z(G)| + \sum_{i=1}^r [G : C_G(x_i)] \]
If $1 \leq i \leq r$, then $x_i \notin Z(G) \implies [G : G_{x_i}] > 1$.
\end{proof}

\vs

\begin{modification}
Let $G$ be a finite group, $N\nsg G$, and $x \in G$. Then either $G\cdot x \cap N = \vn$ or $G\cdot x \sub N$. By letting $G$ act on $N$ by conjugation, we get
	\[|N| = |N \cap Z(G)| + \sum_{i=1}^t [G : C_G(y_i)], \]
where the $y_i$ are distinct representatives of the $G$-conjugacy classes contained in $N\bs Z(G)$.
\end{modification}

\vs

\begin{example}
Let $G = S_4$, so that $Z(G) = \{e\}$ by 1.1.4. The conjugacy classes in $S_4$ (or really $S_n$) correspond to "cycle types". For more on this, see [DF] p. 125.

Let $x_1 = \gen{(1\ 2)}$, $x_2 = \gen{(1\ 2)(3\ 4)}$, $x_3 = \gen{(1\ 2\ 3)}$, $x_4 = \gen{(1\ 2\ 3\ 4)}$, distinct conjugacy class representatives for $S_4$. Recall that
	\[|C_G(x_i)| = \frac{|G|}{[G : C_G(x_i)]} = \frac{|G|}{|G \cdot x_i|} \]
	\begin{enumerate}
	\item[$\mb{i=1.}$ ] $|C_G(x_1)| = 24/6 = 4 \implies C_G(x_i) = C_G(x_1) = \gen{(1\ 2)(3\ 4)} \cong \z_2 \times \z_2$.
	\item[$\mb{i=2.}$ ] $|C_G(x_2)| = 24/3 = 8$. Noting that $(1\ 2)(3\ 4), (1\ 2\ 3\ 4) \in C_G(x_2)$, it is not very difficult to see that $C_G(x_2) = \gen{(1\ 2)}\gen{(1\ 2\ 3\ 4)} \cong D_8$.
	\item[$\mb{i=3.}$ ] $|C_G(x_3)| = 24/3 = 8 \implies C_G(x_3) = \gen{(1\ 2\ 3)}$.
	\item[$\mb{i=4.}$ ] $|C_G(x_4)| = 24/6 = 4 \implies C_G(x_4) = \gen{(1\ 2\ 3\ 4)}$.
	\end{enumerate}
\end{example}

\vs

\begin{theorem}
Let $G$ be a finite group, $p$ a prime, and $n\in \N$ such that $p^n \mid |G|$. Then there exists a subgroup $H$ of $G$ with $|H| = p^n$.
\end{theorem}

\begin{proof}
By induction on $|G|$. We choose $|G|=2$ as the base case, so $H \myeq G$ shows that the base case holds. For the inductive step, consider the class equation
	\[ |G| = |Z(G)| + \sum_{i+1}^r [G : C_G(x_i)], \]
with $x_1,\dots,x_r$ as in 1.5.14. We have two cases.
	\begin{enumerate}
	\item[First Case. ] There exists $1 \leq i \leq r$ such that $p \nmid [G : C_G(x_i)]$. Put $p^n \mid |G| = |C_G(x_i)| [G : C_G(x_i)]$, so $p^n \mid |C_G(x_i)| < |G|$, so we apply the induction hypothesis to show that there exists $H \leq C_G(x_i) < G$ with $|H| = p^n$.
	\item[Second Case. ] $p \mid [G : C_G(x_i)]$ for all $1 \leq i \leq r$. Then we must have $p \mid |Z(G)|$. By 1.2.6, there exists $g \in Z(G)$ with $|g| = p$. Set $N \myeq \gen{g}$, where $|N| = p$. Consider $\ov{G} \myeq G/N$. If $n=1$, then we are done. Assume $n \geq 2$. Then $p^{n-1} \mid |\ov{G}| = |G|/p$, so we apply the induction hypothesis to $\ov{G}$ to show that there exists $\ov{H} \leq \ov{G}$ with $|\ov{H}| = p^{n-1}$. By the correspondence theorem, $\ov{H} = H/N$ for some $N \leq H \leq G$, so $|H| = p^{n-1} |N| = p^n$.
	\end{enumerate}
\end{proof}

\vs

\begin{corollary}[Cauchy's Theorem]
Let $G$ be a finite group and $p$ a prime such that $p \mid |G|$. Then there exists $g \in G$ with $|g| = p$.
\end{corollary}

\vs

\begin{defn}
A finite group $G$ is called a $p$-group ($p$ a prime) if $|G|=p^n$ for some $n \in \N_0$.
\end{defn}

\vs

\section*{$p$-Groups ($p$ Prime)}

\vs

\begin{theorem}
Let $G$ be a $p$-group.
\begin{enumerate}
\item[(a)] Any subgroup $H \leq G$ is a $p$-group.
\item[(b)] $Z(G) \ne \{e\}$ if $G \ne \{e\}$.
\item[(c)] $G$ is nilpotent.
\item[(d)] If $G \ne \{e\}$, then the following are equivalent for a subgroup $H$ of $G$:
	\begin{enumerate}
	\item[(i)] $H$ is a maximal subgroup of $G$.
	\item[(ii)] $[G : H] = p$.
	\end{enumerate}
In this case, $H\nsg G$.
\end{enumerate}
\end{theorem}

\begin{proof}
\
\begin{enumerate}
\item[(a)] Lagrange.
\item[(b)] Just look at the class equation.
\item[(c)] By induction on $|G|$. The case $|G| = 1$ is trivial. If $G \ne \{e\}$, then $Z(G) \ne \{e\}$ by part (b). Then $Z(G)$ is nilpotent and $Z(G)$ is nilpotent by the induction hypothesis, so $G$ is nilpotent 1.3.7 (d).
\item[(d)] $[(ii) \implies (i)]$ by a simple index argument. $[(i) \implies (ii)]$ $G$ is nilpotent by (c) and $H < G$ is maximal, so $H < N_G(H)$ by 1.3.5, hence $N_G(H) = G \implies H \nsg G$ as $H$ is maximal. Set $\ov{G} = G/H$, a nontrivial $p$-group, so that $p \mid |\ov{G}|$. By 1.5.17, there exists $\ov{U} \leq \ov{G}$ with $|\ov{U}| = p$. By the correspondence theorem, there exists $H \tl U \nsg G$ such that $U/H = \ov{U}$. Since $H$ is maximal, $U=G$, and so $p = [U : H] = [G : H]$.
\end{enumerate}
\end{proof}

\vs

\begin{remark}
The same argument as in $(ii) \implies (i)$ above shows that for any group $G$ and any $H \leq G$, if $[G : H] = p$, then $H$ is maximal in $G$. On the other hand, neither of the following statements are true in general:
	\begin{enumerate}
	\item[(i)] If $H$ is a maximal subgroup of $G$, then $[G : H] = p$. As a counterexample, $S_3 \leq S_4$ is maximal with $[S_4 : S_3] = 4$.
	\item[(ii)] If $[G : H] = p$, then $H\nsg G$. As a counterexample, consider 1.5.16, where $[S_4 : C_{S_4}(x_2)] = 3$, but $C_{S_4}(x_2) \not\nsg S_4$.
	\end{enumerate}
\end{remark}

\vs

\begin{proposition}
Let $G$ be a $p$-group.
	\begin{enumerate}
	\item[(a)] If $\triv \ne N \nsg G$, then $N \cap Z(G) \ne \{e\}$.
	\item[(b)] If $N \nsg G$ with $|N| = p^m$ for some $m \in \N$, then for all $1 \leq l \leq m$, there exists a subgroup $H \leq n$ with $|H| = p^l$ and $H \nsg G$.
	\item[(c)] If $H < G$, then there exists $H < U \leq G$ with $[U : H] = p$.
	\end{enumerate}
\end{proposition}

\begin{proof}
Exercise.
\end{proof}

\vs

\begin{proposition}
If $|G| = p^2$, then $G$ is abelian and either $G \cong \z_{p^2}$ or $G \cong \z_p \times \z_p$.
\end{proposition}

\begin{proof}
$Z(G) \ne \triv$ by 1.5.20 (b), so $|Z(G)| = p$ or $|Z(G)| = p^2$. In either case, $G/Z(G)$ is cyclic, whence by 1.1.5 $G$ is abelian. There are now two cases for the abelian group $G$:
	\begin{enumerate}
	\item[(i)] $G$ is cylic. Then $G \cong \z_{p^2}$.
	\item[(ii)] Every $e \ne g \in G$ is such that $|g| = p$. Choose $e \ne x \in G$ and $y \in G\bs \gen{x}$, so $|x| = |y| = p$. Since $\gen{y} \ne \gen{x}$, and both groups have prime order, we have $\gen{x} \cap \gen{y} = \triv$. Both subgroups are normal since $G$ is abelian, so we have $G = \gen{x} \times \gen{y} \cong \z_p \times \z_p$.
	\end{enumerate}
\end{proof}

\vs

\begin{remark*}
Consider $p$-groups $G$ where $|G| = p^n, n=1,2,3$. Which $G$ are abelian?
\begin{enumerate}
\item[$\mb{n=1.}$ ] Then $G \cong \z_p$.
\item[$\mb{n=2.}$ ] All such $G$ are abelian by 1.5.23.
\item[$\mb{n=3.}$ ] We invoke the not-yet-proven Fundamental Theorem of Finite Abelian groups to say that the abelian groups of order $p^3$ are, up to isomorphism: $\z_{p^3}, \z_{p^2} \times \z_p, \z_p\times \z_p \times \z_p$.
\end{enumerate}
\end{remark*}

\vs

\begin{question}
Are there non-abelian groups of order $p^3$?
\end{question}

\begin{answer}
Yes, there are precisely 2 (up to isomorphism) for each prime $p$. When $p=2$, these are $Q, D_8$. When $p\geq 3$, it turns out that these are $\z_{p^2} \rtimes \z_p$ and $(\z_p \times \z_p) \rtimes \z_p$.
\end{answer}



\chapter{Semi-Direct Products}

\begin{defn}
Let $B$ be a group which acts on another group $A$, and let $\phi : B \ra \aut(A)$ be the corresponding \hm, so that $b \cdot a = \phi(b)(a)$ and $b\cdot(aa') = (b\cdot a)(b\cdot a')$ for all $b\in B$ and $a,a' \in A$. In particular, $b\cdot e_A = e_A$ for all $b \in B$.

The \tb{(external) semi-direct product} $A \sd{\phi} B$ of $A$ and $B$ with respect to $\phi$ is defined as the set $A \times B$ (Certesian product) with the following multiplication: $(a_1,b_1)(a_2,b_2) \myeq (a_1 (b_1\cdot a_2), b_1 b_2)$.

With this multiplication, it is easily verified that $A \sd{\phi} B$ is a group with identity element $(e_A,e_B)$ and $(a,b)\inv = (b\inv \cdot a\inv,b\inv)$ for all $a\in A, b \in B$.
\end{defn}

\vs

\begin{remark}
Let $A,B,\phi$ be given as above.
\begin{enumerate}
\item[(a)] The subscript $\phi$ in $A\sd{\phi} B$ is often omitted if $\phi$ is understood. However, $A\sd{\phi} B \cong A\sd{\psi}B$ is NOT true in general.
\item[(b)] Sometimes the semi-direct product is written $B \overset{{\huge \ltimes}}{\null_{\tiny (\phi)}} A$. In either case, the "triangle" always points to the normal subgroup.
\item[(c)] $A \sd{\phi}B$ is NEVER abelian if $\phi \ne \id_A$. If $\phi \ne \id_A$, choose $b \in B$ and $a \in A$ such that $b\cdot a \ne a$. Then $(e_A,b)(a,e_B) = (b\cdot a,b) \ne (a,b) = (a,e_B)(e_A,b)$.
\item[(d)] $|A\sd{\phi}B| = |A\times B| = |A| |B|$.
\item[(e)] $A \lra A\sd{\phi}B$, $a \mapsto (a,e_B)$ and $B \lra A\sd{\phi}B$, $b \mapsto (e_A,b)$ are both injective group \hms. We often identify $A$ and $B$ with their images in $G \myeq A\sd{\phi}B$. These subgroups have the following properties:
	\begin{enumerate}
	\item[(i)] $G = AB$
	\item[(ii)] $A\cap B = \triv$
	\item[(iii)] $A \nsg G$ (We do NOT have $B \nsg G$ unless $\phi = \id_A$). This is seen from the following:
		\begin{align*}
		(e_A,b)(a,e_B)(e,A,b\inv) &= (b\cdot e_A,b)(e_A,b\inv) \\
		&= (b^2 \cdot e_A,e_B) \\
		&\in A
		\end{align*}
	However,
		\begin{align*}
		(a,e_B)(e_A,b)(a\inv,e_B) &= (a,b)(a\inv,e_B) \\
		&= (a(b\cdot a\inv),b) \in B \\
		\iff b\cdot a\inv &= a\inv
		\end{align*}
	\end{enumerate}
\end{enumerate}
\end{remark}

\vs

\begin{defn}[+ Proposition]
Let $G$ be a group and $A,B \leq G$ satisfying
	\begin{enumerate}
	\item[(i)] $G = AB$
	\item[(ii)] $A\cap B = \triv$
	\item[(ii)] $A \nsg G$
	\end{enumerate}
Then $G$ is called the \tb{(internal) semi-direct product} of $A$ and $B$, written $G = A \rtimes B$ (or $B \ltimes A$). Here, $B$ acts on $A$ by conjugation: $b\cdot a = bab\inv$, so the corresponding homomorphism is $\phi : B \ra \aut(A), b \mapsto \kappa_b|_A$. With this action of $B$ on $A$, we consider the external semidirect product $A\sd{\phi}B$, and we get a group isomorphism $\mu : A \sd{\phi}B \ra G, (a,b) \mapsto ab$.
\end{defn}

\begin{proof}
$\mu$ is a \hm.
	\begin{align*}
	\mu((a_1,b_1)(a_1,b_2)) &= \mu(a_1(b_1\cdot a_2), b_1 b_2) \\
	&= \mu((a_1(b_1a_2b_1\inv),b_1,b_2)) \\
	&= a_1(b_1 a_2 b_1\inv) b_1 b_2 \\
	&= (a_1b_1) (a_2 b_2) \\
	&= \mu((a_1,b_1)) \mu((a_2,b_2))
	\end{align*}

$\mu$ is surjective by (i).

$\mu$ is injective. Let $(a,b) \in \ker \mu$. Then $e = \mu(a,b) = ab \implies a = b\inv \in A \cap B = \{e\} \implies a = b = e \implies (a,b) = (e,e)$.
\end{proof}

\vs

\begin{corollary}
Assume $G = A \sd{} B$ (internal semi-direct product).
	\begin{enumerate}
	\item[(a)] Every $g \in G$ has a unique representation of the form $g = ab$ for $a \in A, b \in B$.
	\item[(b)] If $A,B$ are solvable, then $G$ is solvable.
	\end{enumerate}
\end{corollary}

\begin{proof}\
\begin{enumerate}
\item[(a)] $\mu$ is bijective.
\item[(b)] Follows from 1.3.9.
\end{enumerate}
\end{proof}

\vs

\begin{example}[Internal Semi-Direct Products]
\
\begin{enumerate}
\item[(a)] $S_3 = \gen{(1\ 2\ 3)} \sd{} \gen{(1\ 2)} \cong \z_3 \sd{} \z_2$ (not nilpotent)
\item[(b)] $S_4 = V \sd{} S_3 \cong (\z_2 \times \z_2) \sd{} (\z_3 \sd{} \z_2)$, where $V = \{\id,(1\ 2)(3\ 4),(1\ 3)(2\ 4),(1\ 4)(2\ 3)\}$ is the unique normal subgroup of $S_4$ other than $A_4$.
\item[(c)] $A_4 = V \sd{} \gen{(1\ 2\ 3)} \cong (\z_2 \times \z_2) \sd{} \z_3$
\item[(d)] Recall ???: $C_{S_4}(x_2) = \gen{(1\ 2\ 3\ 4)} \sd{} \gen{(1\ 2)} \cong \z_4 \sd{} \z_2 \cong D_8$
\item[(e)] $S_n = A_n \sd{} \gen{(1\ 2)} \cong A_n \sd{} \z_2$
\item[(f)] $A_n$ does not admit a proper semi-direct product decomposition for $n \geq 5$.
\item[(g)] $Q$ has no proper decomposition as a semi-direct since $A \cap B \ne \{1\}$ for all $\{1\} \ne A,B \leq Q$.
\item[(h)] $B_n(F) = U_n(F) \sd{} D_n(F)$
\end{enumerate}
\end{example}

\vs

\begin{example}[External Semi-Direct Products]
\
\begin{enumerate}
\item[(a)] Consider the ring $R = \z[\frac{1}{2}] = \{\frac{z}{2^n} \mid z \in \z, n \in \N_0\}$ and let $A$ be the abelian group $A = (R,+)$. Left-multiplication by 2 defines an automorphism of $A$, since $2 \in \z[\frac{1}{2}]\x$. Let $B = \gen{b}$, $|b| = \infty$, be an infinite cyclic group, and define the homomorphism $\phi : B \ra \aut(A), b^m \mapsto (a \mapsto 2^m a)$. Now, consider $G = A \sd{\phi} B$. $G = \gen{1,b}$, but $A$ is NOT finitely generated. Indeed, let $a_1,a_2,\dots,a_r \in A$. Then $a_i = \frac{z_i}{2^{n_i}}$, $z_i \in \z, n_i \in \N_0, 1 \leq i \leq r$. If $n = \max_{i=1}^r n_i$, then
	\[\gen{a_1,a_2,\dots,a_r} \sub \frac{1}{2^n} \z \subsetneq \z\left[\frac{1}{2}\right] \]
This also demonstrates that $N_G(H) \ne \{g \in G \mid gAg\inv \sub H\}$ is not true, e.g., $g = b, H = \z \sub G \implies b\z b\inv = 2\z \sub \z$, but $b \notin N_G(H)$. Indeed,
	\begin{align*}
	(0,b)(z,1)(0,b\inv) &= (2z,b)(0,b\inv) \\
	&= (2z,1)
	\end{align*}

\item[(b)] (Non-abelian groups of order $p^3$). Let $A = \z_{p^2}$ or $A = \z_p \times \z_p$, $B = \gen{b}, |b| = p$. Then
	\[ |\aut(A)| = \begin{cases}
	p(p-1), \quad & A = \z_{p^2} \\
	(p^2-1)(p^2-p), \quad & A = \z_p \times \z_p
	\end{cases} \]
In both cases, $p \mid |\aut(A)|$, so by Cauchy's theorem there exists $\beta \in \aut(A)$ with $|\beta| = p$. Define $\phi : B \ra \aut(A)$ by $b^m \mapsto \beta^m$. In both cases (for $A$), we get a group $A \sd{\phi} B$ of order $|A| |B| = p^3$, non-abelian since $\beta \ne \id_A$.
\end{enumerate}
\end{example}

\vs

\begin{remark*}
The previous example covers all possible non-abelian groups of order $p^3$ with the exception of $Q$ ($p=2$).
\end{remark*}

\vs

\section*{Split Exact Sequences}

\vs

\begin{defn}
A short exact sequence $\ses{N}{f}{G}{p}{Q}$ is called \tb{split} if there exists a homomorphism ("section") $s : Q \ra G$ such that $p \circ s = \id_Q$, so that $s$ is injective and $Q \cong S(Q) \leq G$.
\end{defn}

\vs

\begin{example*}
\
\begin{enumerate}
\item[(a)] The SES $\ses{\gen{i}}{}{Q_8}{}{\z_2}$ is not split.
\item[(b)] The SES $\ses{SL_n(F)}{}{GL_n(F)}{}{F\x}$ is split.
\item[(c)] If $G = A \sd{\phi} B$, then $\ses{A}{}{G}{}{B}$ is a split SES. Conversely, if $\ses{N}{f}{G}{p}{Q}$ is a split SES with section $s$, then $G = f(N) \sd{} s(Q)$ (Exercise).
\end{enumerate}
\end{example*}

\vs

\begin{defn}[Dihedral Groups]
For any abelian group $A$, the inversion map $i : A \ra A, a \mapsto a\inv$ is an automorphism of $A$ with $i^2 = \id_A$. Let $B = \gen{b}, |b|=2$, and consider the homomorphism $\phi : B \ra \aut(A), b \mapsto i, e_B \mapsto \id_A$, so that we get the group $A \sd{\phi} B$. If $A = \gen{a}$ is cyclic, we call $D = A \sd{\phi}B$ a \tb{dihedral group.}
\begin{enumerate}
\item[\tb{Notation.}] If $A \cong \z$, then we write $A \sd{\phi}B = D_\infty$. If $A \cong \z_n$, then we write $A \sd{\phi}B = D_{2n}$.
\item[\tb{Elements.}] If $|A| = n$, then $|D_{2n}| = 2n$.
\item[\tb{Multiplication.}] We now write elements of $D$ as $a^ib^j$ instead of $(a^i,b^j)$. We have for all $i,j \in \z$,
	\begin{align*}
	a^ia^j &= a{i+j}, \\
	a^i(a^jb) &= a^{i+j} b, \\
	(a^jb)a^i &= a^j(ba^ib\inv)b = a^j(a^{-i})b = a^{j-i}b, \\
	(a^ib)(a^jb) &= a^{i-j}b
	\end{align*}
Note that $(a^ib)^2 = e$ for all $i \in \z$, so $|a^ib| = 2$ for all $i \in \z$.
\end{enumerate}
\end{defn}

\vs

\begin{remark}[On Dihedral Groups]
\
\begin{enumerate}
\item[(a)] $\aut(\z) = \{\id,i\}$, so $\id \ne \phi : B \ra \aut(\z)$ is uniquely determined. However, if $8 \leq n < \infty$, there are in general different possibilities for a nontrivial homomorphism $\psi : B \ra \aut(\z_n)$ such that $\z_n \sd{\psi} B \ncong D_{2n}$.
\item[(b)] $D_2 \cong \z_2$, $D_4 \cong \z_2 \times \z_2$, $D_6 \cong \z_3 \sd{}\z_2$, $D_8 \cong \z_4 \sd{} \z_2$. In the cases of $D_6$ and $D_8$, the nontrivial homomorphism $\phi : B \ra \aut(\z_{3,4})$ is unique.
\item[(c)] Any dihedral group is solvable by 1.6.4 (b), since $\z,\z_n,\z_2$ are solvable.
\item[(d)] $D_\infty$ is not nilpotent, and $D_{2n}$ is nilpotent if and only if $n$ is a power of 2 (Exercise).
\item[(e)] Geometric interpretation: If $n \geq 3$ is finite, then $D_{2n}$ can be considered as the group of symmetries of a regular n-gon $P$ by associating $a$ with a counterclockwise rotation by $\frac{2\pi}{n}$ and associating $a^jb$, $0 \leq j \leq n-1$, with the reflections.
\item[\tb{Consequence}] $\{\tn{Symmetries of }P\} \ra S_n$, where $S_n$ is thought of as the set of permutations of the $n$ vertices of $P$. Thus, $S_n$ contains a subgroup isomorphic to $D_{2n}$ for $n \geq 3$, e.g., $D_8 \hra S_4$.
\item[\tb{Algebraically:}] Put $a = (1\ 2\dots n) \in S_n$, and
	\[b = \begin{cases}
	\left(\frac{n}{2},\frac{n}{2}+1\right), \qquad & n \tn{ even} \\
	\left( \frac{n-1}{2},\frac{n+3}{2}\right), & n \tn{ odd}
	\end{cases}\]
If $n = \infty$, then $D_{\infty}$ can be interpreted as the group of symmetries of the real line which stabilize $\z$ by associating $a^j$ with the translation $x \mapsto x+j$ and $a^jb$ with the point reflection about $\frac{j}{2}$ for each $j \in \z$.
\end{enumerate}
\end{remark}

\vs

\begin{proposition}
Let $G$ be a group such that $|G| = pq$, where $p>q$ are primes. Then
\begin{enumerate}
\item[(a)] $G = \z_p \sd{} \z_q$.
\item[(b)] If $q \nmid p-1$, then $G \cong \z_p \times \z_q \cong \z_{pq}$.
\end{enumerate}
\end{proposition}

\begin{proof}
\
\begin{enumerate}
\item[(a)] Cauchy's theorem, there exist $A,B \leq G$ with $|A| = p$, $|B| = q$. Since $[G : A = q]$, $A \nsg G$ by 1.5.10.
	\begin{enumerate}
	\item[(i)] \[\size{AB} = \frac{|A||B|}{|A\cap B|} = pq = |G| \implies G = AB\]
	\item[(ii)] $A \cap B = \{e\}$ since $(p,q) = 1$.
	\item[(iii)] As already noted, $A \nsg G$.
	\end{enumerate}

\item[(b)] $|\aut(A)| = |\z_p\x| = p-1$. If $q \nmid p-1$, then $\aut(A)$ does not have an element of order $q$, so the only homomorphism $B \ra \aut(A)$ is trivial, so $G = A \sd{}B = A \times B \cong \z_p \times \z_q \cong \z_{pq}$.
\end{enumerate}
\end{proof}

\vs

\chapter{Sylow Theorems and Applications}



The idea of this chapter is to analyze the structure of a finite group $G$ by investigation its maximal $p$-subgroups. Throughout this chapter, $G$ is always a finite group and $p$ is a prime.

\vs

\begin{defn}
A \tb{Sylow $\mb{p}$-subgroup} of $G$ is a maximal $p$-subgroup of $G$. Denote by $\sy_p(G)$ the set of Sylow $p$-subgroups of $G$. It is clear that:
	\begin{enumerate}
	\item[(i)] If $p \nmid |G|$, then $\{e\}$ is the only Sylow $p$-subgroup of $G$.
	\item[(ii)] If $U \leq G$ is any $p$-subgroup of $G$, then there exists $H \in \sy_p(G)$ with $U \leq H$.
	\item[(iii)] If $|G| = p^em$, $p \nmid m$, $i \in \N_0$, $m \in \N$, and $H$ is a subgroup of $G$ with $|H| = p^e$, then $H \in \sy_p(G)$.
	\end{enumerate}
\end{defn}

\vs

\begin{question}
\
\begin{enumerate}
\item[(a)] If $H \in \sy_p(G)$, is in necessarily true that $|H| = p^em$ as in $(iii)$ above?
\item[(b)] If $H_1,H_2 \in \sy_p(G)$, are $H_1$ and $H_2$ necessarily isomorphic?
\end{enumerate}
\end{question}

A partial answer to (a) is that we know, by 1.5.17, that there exists $P \in \sy_p(G)$ with $|P| = p^e$.

\vs

\begin{lemma}
If $P \in \sy_p(G)$ and $H \leq G$ is a $p$-subgroup with $H \leq N_G(P)$, then $H \leq P$.
\end{lemma}

\begin{proof}
Because $H \leq N_G(P)$, $HP \leq G$. Now $HP$ is certainly a $p$-subgroup of $G$, $P \leq HP$, and $P$ is a maximal $p$-subgroup of $G$, so we necessarily have $HP = P \implies H \leq P$.
\end{proof}

\vs

\begin{theorem}[Sylow]
Let $G$ be a finite group with $|G| = p^em$, $e \in \N_0$, $m\in \N$, and $p \nmid m$. Then
\begin{enumerate}
\item[(a)] If $P,P' \in \sy_p(G)$, then there exists $g \in G$ such that $P' = gPg\inv$.
\item[(b)] $|\sy_p(G)| = [G : N_G(P)] \mid m$ for any $P \in \sy_p(G)$.
\item[(c)] $|\sy_p(G)| \equiv 1 \pmod{p}$.
\end{enumerate}
\end{theorem}

\begin{proof}
Fix $P \in \sy_p(G)$ with $|P| = p^e$. Such $P$ exists by 1.5.17. Set $X = \{gPg\inv \mid g \in G\} \sub \sy_p(G)$, and let $G$ act on $X$ by conjugation.
\begin{enumerate}
\item[(1)] $G$ acts transitively on $X$ (by definition), and $G_P = \{g \in G \mid gPg\inv = P\} = N_G(P)$.

\item[(2)] $|X| = |G \cdot P| = [G : G_p] = [G : N_G(P)]$ by 1.5.12 (Orbit-Stabilizer).

\item[(3)] $[G : N_G(P)][N_G(p) : P] = [G : P] = m$, so $|X| = [G : N_G(P)] \mid m$.

\item[(4)] It follows from (3) that $p \nmid |X|$, since $p \nmid m$.
\end{enumerate}

Now take any $H \in \sy_p(G)$, and let $H$ also act on $X$ by conjugation. The action of $H$ needn't be transitive. Let $H\cdot P_1, H \cdot P_2, \dots, H\cdot P_r$ be the distinct $H$-orbits in $X$. In particular, $P_1,P_2,\dots,P_r \in X$, and each have order $p^e$.

\begin{enumerate}
\item[(5)] $|X| = \sum_{1=1}^r |H \cdot P_i|$ by 1.5.13.

\item[(6)] $|H \cdot P_i| = 1 \iff H = P_i$. The $\impliedby$ direction is clear. If $|H \cdot P_i| = 1$, i.e., $H \cdot P_i = \{p_i\}$, then $hP_ih\inv = P_i$ for all $h \in H$, so $H \leq N_G(P_i)$. By 1.7.2, $H \leq P_i$. By both $H,P_i$ are maximal $p$-subgroups of $G$, so we must have $H = P_i$.

\item[(7)] If $|H \cdot P_i| \ne 1$, then $p \mid |H \cdot P_i|$, because $1 \ne |H \cdot P_i| = [H : H \cap N_G(P_i)] \mid |H|$, and $H$ has $p$-power order, so $p \mid |H \cdot P_i|$.

\item[(8)] $H = P_i$ for precisely one $i \in \{1,2,\dots,r\}$, and $|X| \cong 1 \pmod{p}$. If $H \ne P_i$ for all $1 \leq i \leq r$, then it would follow that $|X| \equiv 0 \pmod{p}$, contradiction (4).
\end{enumerate}

In particular, we have proved that $H \in \sy_p(G) \implies H \in X$. Hence $\sy_p(G) = X$, which proves (a), and (c) follows from (8) and (b) follows from (8) and (3).
\end{proof}

\vs

\begin{remark*}
If $P' \in \sy_p(G)$, then $P' = gPg\inv$ for some $g \in G$ by (a), so $n_G(P'_ = gN_G(P')g\inv \implies |N_G(P')| = |N_G(P)| \implies [G : N_G(P')] = [G : N_G(P)]$.
\end{remark*}

\vs

\begin{corollary}
$\sy_p(G) = \{H \leq G \mid |H| = p^e\}$, where $|G| = p^em, e \in \N_0, m\in \N,p \nmid m$.
\end{corollary}

\vs

\begin{corollary}
For $P \in \sy_p(G)$, the following are equivalent:
\begin{enumerate}
\item[(i)] $P \car G$.
\item[(ii)] $P \nsg G$.
\item[(iii)] $|\sy_p(G)| = 1$.
\end{enumerate}
\end{corollary}

\vs

\begin{corollary}
If $P \in \sy_p(G)$, then $N_G(N_g(P)) = N_G(P)$.
\end{corollary}

\begin{proof}
$P \nsg N_G(P)$ by definition, so by 1.7.5 $P \car N_G(P) \nsg N_G(N_G(P))$, thus $P \nsg N_G(N_G(P))$ by 1.1.35 (c). Therefore, $N_G(N_G(P)) \leq N_G(P) \implies N_G(P) = N_G(N_G(P))$.
\end{proof}

\vs

\begin{theorem}
Let $G \ne \{e\}$ be a finite group with $|G| = p_1^{e_1}\cdots p_r^{e_r}$, $p_1,\dots,p_r$ distinct prime divisors of $|G|$, $P_i \in \sy_{p_i}(G)$ for $1 \leq i \leq r$. The following are equivalent:
\begin{enumerate}
\item[(i)] $G$ is nilpotent.
\item[(ii)] $N_G(H) \ne H$ for all $H < G$.
\item[(iii)] $P_i \nsg G$ for all $1 \leq i \leq r$.
\item[(iv)] $G = P_1 \times \dots \times P_r$ (internal semi-direct product).
\end{enumerate}
\end{theorem}

\begin{proof}\
$(i) \implies (ii)$ follows from 1.3.5.

$(ii) \implies (iii)$ follows from 1.7.6.

$(iii) \implies (v)$. $P_i \nsg G$ for all $i$ implies that every product of the $P_i's$ is a normal subgroup of $G$. By induction on $r$ it is easy to show that $|P_1 \cdots P_r| = |P_1| \cdots |P_r|$ and $P_1 \cap \dots \cap P_r = \{e\} \implies P_r \cdots P_r \cong P_1 \times \dots P_r$. In particular, $|P_1 \cdots P_r| = p_1^{e_1} \cdots p_r^{e_r}$, so $G = P_1 \times \dots P_r$.

$(iv) \implies (i)$. Each $P_i$ is a $p_i$-subgroup of $G$, so each $P_i$ is nilpotent by 1.5.20 (c), and direct products of nilpotent groups are nilpotent by 1.3.10 (together with easy induction), so $G$ is nilpotent.
\end{proof}

\vs

\begin{notation*}
Let $G$ be a finite group and $p$ a prime divisor of $|G|$. Let $n_p(G) = n_p = |\sy_p(G)| = [G : N_G(P)] \mid m$, where $|G| = p^em$, $p\nmid m$, and $P \in \sy_p(G)$. If we write $P_p$, we mean that $P_p \in \sy_p(G)$.
\end{notation*}

\vs

\begin{lemma}[Generalization of 1.6.10]
If $|G| = p^em$, where $p > m$, then $P_p \nsg G$ (hence $n_p = 1$). If, additionally, there exists $H \leq G$ with $|H| = m$, then $G = P_p \sd{} H$.
\end{lemma}

\begin{proof}
We know $n_p \mid m$ and $n_p \equiv 1 \pmod{p}$. If $n_p \ne 1$, then $n_p > p > m$, which is a contradiction. Therefore, $P_p \nsg G$. If $H \leq G$ and $|H| = m$, then $H \cap P_p = \triv$, and $|P_pH| = p^em = |G| \implies G = P_p H$, hence $G = P_p \sd{} H$.
\end{proof}

\vs

\begin{example*}
Suppose $|G| = 100 = 2^25^2 \implies n_5 = 1 \implies G \cong P_5 \sd{} P_2$. Moreover, $G$ is solvable because groups of order $25 = 5^2$ are solvable and $G/P_5$ is solvable (order $2^2$).
\end{example*}

\vs

\begin{proposition}
Let $|G| = pqr$, with primes $p > q > r$. If $P = P_p$, $Q = P_q$, and $R = P_r$, then $P,PQ \nsg G$, and hence $G = (P \sd{} Q) \sd{} R \cong (\z_p \sd{} \z_q) \sd{} \z_r$. In particular, $G$ solvable.
\end{proposition}

\begin{proof}
Claim 1: $P \nsg G$ or $Q \nsg G$. We have $n_p \mid qr$ and $n_p \equiv 1 \pmod{p} \implies n_p = 1,qr$. $N_q \mid pr$ and $n_q \equiv 1 \pmod{q} \implies q = 1,p,pr$. Assume that $n_p,n_q \ne 1$. Then $n_p = qr \implies$ there are $qr$ subgroups of order $p$ and $qr(p-1)$ elements of order $p$ in $G$, and $n_q \geq p \implies$ there are at least $p(q-1)$ elements of order $q$ in $G$. But together this implies that $G$ has at least
	\begin{align*}
	(p-q)qr - p(q-1) &= pqr - qr + p(q-1) \\
	&> pqr - qr + qr \tag{$p > q, p-1 \geq r$} \\
	&= pqr
	\end{align*}
elements, which is a contradiction. This proves the first claim. \\

Claim 2: $P \nsg G$. Since $P \nsg G$ or $Q \nsg G$, we have $PQ \leq G$. Now Now $[G : PQ] = r$, the smallest prime divisor of $|G|$, so $PQ \nsg G$. By the same argument, $P \nsg PQ \iff P \car PQ \nsg G \implies P \nsg G$. It follows that $PQ = P \sd{} Q$ and $G = PQ \sd{}R = (P \sd{} Q) \sd{}R \cong (\z_p \sd{} \z_q) \sd{} \z_r$.
\end{proof}

\vs

\begin{example}
\
\begin{enumerate}
\item[(a)] Suppose $|G| = 30 = 2 \cdot 3 \cdot 5$. Then $G \cong (\z_5 \sd{} \z_3) \sd{} \z_2$. Moreover, $3 \nmid |\aut(\z_5)| = 4 \implies G \cong \z_{15} \sd{} \z_2$, and $\z_3,\z_5 \nsg \z_15 \implies \z_3,\z_5 \car \z_{15} \nsg G \implies \z_3,\z_5 \nsg G \implies n_3=n_5=1$.
\item[(b)] Suppose $|G| = 12 = 2^2 \cdot 3$. If $n_3 \ne 1$, then $n_3 = 4 \implies $ there are $8$ elements of order 3 in $G$, so the 4 remaining elements must comprise some $P_2$. That is, $n_2 = 1$.
\item[(c)] Suppose $|G| = 56 = 2^3 \cdot 7$. Similarly, if $n_7 \ne 1$, then $n_7 = 8 \implies n_2 = 1$.
\item[(d)] Exercise: If $|G| = p^2 q$, where $p \ne q$ are primes, then $n_p = 1$ or $n_q = 1$.
\item[(e)] If $|G| = p^3 q$, it does not necessarily follow that $n_p = 1$ or $n_q = 1$. For example, if $G = S_4$, then $|G| = 2^3 \cdot 3$, but $n_2 = 3$ (3 copies of $D_8$) and $n_3 = 4$ (8 elements of order 3).
\end{enumerate}
\end{example}

\vs

\begin{lemma}\
\begin{enumerate}
\item[(a)] If $|G| = p^e m > p$, with $e,m \in \N$ and $p \nmid m$, and if $p^e \nmid (m-1)!$, then $G$ is not simple.
\item[(a)] If, additionally, any group $H$ with $|H| \mid |G|$ and $|H| < |G|$ is solvable, then $G$ is solvable.
\end{enumerate}
\end{lemma}

\vs

\begin{proof}
\
\begin{enumerate}
\item[(a)] Let $P \in \sy_p(G)$, so that $[G : P] = m$. By 1.5.9, there exists a group homomorphism $\phi : G \ra S_n$ with $K \myeq \ker \phi \leq P$. If $K = G$, then $G = P$ is a $p$-group with $|G| = p^e > p$, so $G$ is nilpotent and not simple.

Assume $K \ne \triv$. Then $G$ injects into $S_m$ so that $|G| \mid |S_m| - m! \implies p^e \mid m! \implies p^e \mid (m-1)!$, since $p \nmid m$. This is a contradiction.

Thus, if $K \ne G$, then $\triv \leq K \tl G$, so $G$ is not simple

\item[(b)] By (a), there exists $\triv \ne N \tl G$, so we have a SES $\ses{N}{}{G}{}{G/N}$. By assumption, both $N$ and $G/N$ are solvable, so $G$ is solvable by 1.3.7.
\end{enumerate}

\vs

\begin{example}
\
\begin{enumerate}
\item[(a)] Suppose $|G| = 24,48 = 2^3 \cdot 3, 2^4 \cdot 3$, so $p = 2, m = 3$, and $p^2, p^3 \nmid m$. By part (a) above, $G$ is not simple. Using fact that groups of order at most 12 are solvable, we get that $G$ is solvable if $|G| = 24 \implies$ $G$ is solvable if $|G| = 48$ (groups of order 16 are certainly solvable).

\item[(b)] Let $|G| = 72 = 2^3 \cdot 3^2$, so that $n_3 = 1, 4$. If $n_3 = 1$, then $P_3 \nsg G$. $n_3 = 4 \iff [G : N_G(P_3)] = 4 \implies$ there exists a group homomorphism $\phi : G \ra S_4$ with $\ker \phi \leq N_G(P_3) \ne G$. Now $\ker \phi \ne \triv$, other $G$ would inject into $S_4$, which is impossible by an order argument. Therefore, $G$ is not simple. It is also true that $G$ is solvable (Exercise).

\item[(c)] $|G| = 112 = 2^4 \cdot 7$. Then $[G : P_2] = 7$, so there exists a group homomorphism $\phi : G \ra S_7$ with $\ker \phi \leq P_2 \ne G$. If $K \myeq \ker \phi \ne \triv$, then $G$ is not simple.

Suppose $K = \triv$. Then we have an embedding of $G$ into $S_7$. Regarding $G$ as a subgroup of $S_7$, we observe that $G \nleq A_7$ because $112 \nmid |A_7|$, so $\triv \neq G \cap A_7 \tl G$, hence $G$ is not simple.
\end{enumerate}
\end{example}
\end{proof}

\vs

\begin{remark}(About groups of small order)
\
\begin{enumerate}
\item[(a)] If $|G| < 60$, then $G$ is solvable.
\item[(b)] For $|G| = 60$, the following are equivalent:
	\begin{enumerate}
	\item[(i)] $G$ is simple.
	\item[(ii)] $n_5 = 6$.
	\item[(iii)] $G \cong A_5$.
	\end{enumerate}
$(ii) \implies (i)$ by [DF] Proposition 21, page 145, and $(i) \implies (iii)$ by Proposition 23. Examples of such groups include $PSL_2(F_5), SL_2(F_4) = PSL_2(F_4)$.
\end{enumerate}
\end{remark}

\vs

\begin{remark}(About more general theorems)
\
\begin{enumerate}
\item[(a)] If all Sylow subgroups of a finite group are cyclic, then $G$ is solvable (even a semi-direct product of two cyclic subgroups).
\item[(b)] If $|G| = p_1 p_2 \cdots p_r$, with $p_1, \dots, p_r$ distinct primes, then $G$ is solvable (follows from (a)).
\item[(c)] If $|G| = p^2qr$, with $p,q,r$ distinct primes and $|G| \ne 60$, then $G$ is solvable.
\end{enumerate}
\end{remark}



\chapter{Free Groups and Presentations}

\begin{defn}[Construction of Free Groups]
Let $X \ne \vn$ by any set. We introduce the set $X \inv = \{x\inv \mid x \in X\}$ with $X \cap X\inv = \vn$ and the map $X \ra X\inv, x \mapsto x\inv$ bijective. Set $Y = X \sqcup X\inv$ (More formally, associate $X$ with $X \times \{1\}$ and $X\inv$ with $X \times \{-1\}$, and put $Y = X \times \{1,-1\}$). We introduce the ("inversion") map $\iota : Y \ra Y$, $x \mapsto x\inv, x\inv \mapsto x$. Write $\iota (y) = y\inv$ for all $y \in Y$.

Think of $Y$ as an "alphabet". Put $W(X) \myeq \{\tn{\tb{words} in } Y\} = \{(y_1,y_2,\dots,y_n) \mid n \in \N_0, y_i \in Y\}$. For $n=0$ we have the empty word, which we will denote by $1 = (\ )$. We will also often denote $(y_1,\dots,y_n)$ by $y_1 \cdots y_n$.

Define \tb{multiplication} on $W(X)$ by concatenation. That is,
	\[(y_1,\dots,y_n)(y_{n+1},\dots,n_m) = (y_1,\dots,y_m) \]
It is immediately clear that this multiplication is associative and 1 is a two-sided identity element (At this point, $W(X)$ is a "monoid").

Call $w,v \in W(X)$ \tb{elementary equivalent}, denoted $w \lra v$, if $w =w_1 w_2$ and $v = w_1 y y\inv w_2$, or $v = w_1 w_2$ and $w = w_1 y y\inv w_2$ for some $w_1,w_2 \in W(X)$, $y \in Y$. This relation is by definition symmetric.

Now, we say that $w,v \in W(X)$ are \tb{equivalent}, denoted $w \sim v$, if there exists a finite chain of elementary equivalences $w = w_0 \lra w_1 \lra \dots \lra w_n = v$. It is straightforward to check that $\sim$ is an equivalence relation on $W(X)$. For $w \in W(X)$, denote the $\sim$-equivalence class of $w$ by $[w]$.

Put $F(X) \myeq W(X)/\sim = \{[w] \mid w \in W(X)\}$. Observe that if $w \sim w'$ and $v \sim v'$, then $wv \sim w'v \sim w'v'$. Thus, we have a well-defined multiplication on $F(X)$ given by $[w][v] = [wv]$.

We now check that $F(X)$ is a group under this multiplication, called the \tb{free group on $\mb{X}$}. If $|X| = n$, we write $F(X) = F_n$, the \tb{free group of rank $\mb{n}$.}
\begin{enumerate}
\item[$\bullet$] Multiplication on $F(X)$ is associative. For all $w,v,u \in W(X)$, we have
	\[([w][v])[u] = [wv][u] = [(wv)u] = [w(vu)] = [w][vu] = [w]([v][u]) \]
\item[$\bullet$] $[1]$ is certainly a two-sided inverse for $F(X)$.
\item[$\bullet$] Inverse elements. Given $w = y_1 \cdots y_n \in W(X)$, put $w\inv = y_n \inv \cdots y_1 \inv$, so that
	\begin{align*}
	[w][w\inv] &= [(y_1\cdots y_n)(y_n\inv \cdots y_1)] \\
	&= [(y_1 \cdots y_{n-1})(y_ny_n\inv)(y_{n-1}\inv \cdots y_1\inv)] \\
	&= [(y_1 \cdots y_{n-1})(y_{n-1}\inv \cdots y_1\inv)] \\
	&\vdots \\
	&= [1]
	\end{align*}
Similarly, $[w\inv][w] = [1]$.
\end{enumerate}
\end{defn}

\vs

\begin{proposition}[Universal Property of Free Groups]
For any group $G$ and any map $f : X \ra G$, there exists a unique group homomorphism $\phi : F(X) \ra G$ such that $\phi \circ \iota = f$. That is, the following diagram commutes:

\begin{center}
\begin{tikzcd}
X \arrow[hookrightarrow]{r}{\iota}
\arrow{d}{f}
& F(X) \arrow[dotted]{dl}{\phi} \\
G
\end{tikzcd}
\end{center}
\end{proposition}

\begin{proof}
If such $\phi$ exists, then for any $[w] \in F(X)$, with $w = x_1^{\ve_1} \dots x_n^{\ve_n}$, $x_i \in X$, $\ve_i \in \{1,-1\}$, we get
	\begin{align*}
	\phi([w]) &= \phi([x_1^{\ve_1} \dots x_n^{\ve_n}]) \\
	&= \phi([x_1]^{\ve_1} \cdots [x_n]^{\ve_n}) \\
	&= \phi([x_1])^{\ve_1} \cdots \phi([x_1])^{\ve_n} \\
	&= f(x_1)^{\ve_1} \cdots f(x_n)^{\ve_n}
	\end{align*}
Since $\phi$ is completely determined by $f$, $\phi$ is uniquely determined.

We now treat existence. Define $\wt{\phi} : W(X) \ra G$ by $\wt{\phi}(x_1^{\ve_1}) \dots x_n^{\ve_n}) = f(x_1)^{\ve_1} \cdots f(x_n)^{\ve_n}$. At this point, there are no issues with $\wt{\phi}$ being well-defined.

Suppose $w, v \in W(X)$ are elementary equivalent ($w \lra v$). Without loss of generality, say $w = w_1 w_2$ and $v = w_1 y y\inv w_2$ for some $w_1,w_1 \in W(x)$, $y \in Y$. Then
	\begin{align*}
	\wt{\phi}(v) &= \wt{\phi}(w_1 y y\inv w_2) \\
	&= f(w_1) f(y) f(y)\inv f(w_2) \\
	&= f(w_1) f(w_2) \\
	&= \wt{\phi}(w)
	\end{align*}
By induction on the length of elementary equivalences, it is easily shown that $w \sim v \implies \wt{\phi}(w) = \wt{\phi}(v)$. Thus, $\wt{\phi}$ induces a well-defined map $\phi : F(X) \ra G$, $\phi([w]) = \wt{\phi}(w)$.

$\phi$ is a homomorphism.
	\begin{align*}
	\phi([w][v]) &= \phi([wv]) \\
	&= \wt{\phi}(wv) \\
	&= \wt{\phi}(w) \wt{\phi}(v) \\
	&= \phi([w]) \phi([v])
	\end{align*}

Finally $(\phi \circ \iota)(x) = \phi([x]) = \wt{\phi}(x) = f(x)$ for all $x \in X$.
\end{proof}

\vs

\begin{remark}
\
\begin{enumerate}
\item[(a)] $\iota$ is injective: Suppose $x_1 \ne x_1 \in X$, and consider a map $f : X \ra \z$ such that $f(x_i) = i$ for $i = 1,2$. By the proposition, $f$ extends to a group homomorphism $\phi : F(X) \ra \z$ such that $\phi([x_1]) = 1 \ne 2 = \phi([x_2]) \implies [x_1] \ne [x_2]$.

In the following, we identify $X$ with $\iota(X) \sub F(X)$.

\item[(b)] $\gen{x} = F(X)$ by construction $F(X)$, i.e., every element of $F(X)$ is of the form $[x_1^{\ve_1} \dots x_n^{\ve_n}] = [x_1]^{\ve_1} \cdots [x_n]^{\ve_n} \in \gen{X}$.

\item[(c)] $F(X)$ is characterized, up to isomorphism, by its universal property. Assume the pair $(F',\iota ')$, with $F'$ a group and $\iota' : X \ra F'$, has the property that for all groups $G$ and any map $f' : X \ra G$ there exists a unique group homomorphism $\phi : F' \ra G$ such that $\phi \circ \iota' = f'$, then $F' \cong F(X)$. Consider the following diagrams, which by assumption commute:
\begin{center}
\[
\begin{array}{cc}
\begin{tikzcd}
X \arrow[hookrightarrow]{r}{\iota}
\arrow[swap]{d}{\iota'}
& F(X) \arrow[dotted]{dl}{\exists ! \ \phi} \\
F'
\end{tikzcd} &
\begin{tikzcd}
X \arrow[hookrightarrow]{r}{\iota'}
\arrow[swap]{d}{\iota}
& F' \arrow[dotted]{dl}{\exists ! \ \phi'} \\
F(X)
\end{tikzcd}
\end{array}\]
\end{center}

That is, there exist group homomorphisms $\phi : F(X) \ra F'$ and $\phi' : F' \ra F(X)$ with $\phi \circ \iota = \iota'$ and $\phi' \circ \iota' = \iota$. Then $(\phi' \circ \phi) \circ \iota = \phi' \circ (\phi \circ \iota) = \phi' \circ \iota = \iota = \id_{F(X)} \circ \iota$:
\begin{center}
\[
\begin{array}{cc}
\begin{tikzcd}
X \arrow[hookrightarrow]{r}{\iota}
\arrow[swap]{d}{\iota'}
& F(X) \arrow[dotted]{dl}{\phi' \circ \phi} \\
F(X)
\end{tikzcd} &
\begin{tikzcd}
X \arrow[hookrightarrow]{r}{\iota'}
\arrow[swap]{d}{\iota}
& F(X) \arrow[dotted]{dl}{\id_{F(X)}} \\
F(X)
\end{tikzcd}
\end{array}\]
\end{center}

By the uniqueness part of the universal property of $F(X)$, $\phi' \circ \phi = \id_{F(X)}$. Similarly, we have that $\phi \circ \phi' = \id_{F'}$, hence $\phi' = \phi\inv$, so $\phi, \phi'$ are isomorphisms.

\item[(d)] If $X,X'$ are sets with $|X| = |X'|$, then $F(X) \cong F(X')$ (Exercise).
\end{enumerate}
\end{remark}

\vs

\begin{corollary}
Every group is (up to isomorphism) a quotient of a free group.
\end{corollary}

\begin{proof}
Let $G$ be a group, and let $X \sub G$ be such that $\gen{X} = G$ (e.g. $X = G$). Let $\iota : X \hra F(X)$ and $i : X \hra G$ be inclusions. By the universal property, there exists a homomorphism $\phi : F(X) \ra G$ such that $\phi(x) = x$ for all $x \in X$. Clearly $\phi$ is surjective, since $\phi(F(X)) \supseteq \phi(\gen{X}) = \gen{X} = G$. By the first isomorphism theorem, $G \cong F(X) / \ker \phi$.
\begin{tikzcd}
X \arrow[hookrightarrow]{r}{\iota}
\arrow[hookrightarrow,swap]{d}{i}
& F(X) \arrow[dotted]{dl}{\exists ! \ \phi} \\
G
\end{tikzcd}
\end{proof}

\vs

\begin{remark}[Reduced Words]
An element $w \in W(X)$ is called \tb{reduced} if $w$ has no subwords of the form $y y\inv$, $y \in Y$. Put $W_0(X) = \{ w \in W(X) \mid w \tn{ is reduced}\}$. Let $\psi : W_0(X) \ra F(X)$ be $\psi(w) = [w]$. Clearly, $\psi$ is surjective. What does require a bit of work to show is that $\psi$ is also injective, hence we may identify the elements of $F(X)$ with those of $W_0(X)$.
\end{remark}

\vs

\begin{defn}
Let $G$ be a group and $S \sub G$. Let $\mc{N} = \{N \nsg G \mid S \sub N\}$. We define the \tb{normal closure of $\mb{S}$ in $\mb{G}$} to be the set $\gen{S}_G = \bigcap_{N \in \mc{N}} N$, i.e., the smallest normal subgroup of $G$ which contains $S$. It is an exercise to show that
	\[\gen{S}_G = \{g_1s_1^{\ve_1}g_1\inv \cdots g_n s_n^{\ve_n} g_n\inv \mid n \in \N_0, g_i \in G, s_i \in S, \ve_i \in \{1,-1\}\} \]
\end{defn}

\vs

\begin{defn}
A \tb{presentation} $\gen{X \mid R}$ of a group $G$ consists of a map $f : X \ra G$ (often an embedding) and a subset $R \sub F(X)$ such that
	\begin{enumerate}
	\item[(a)] $\gen{f(X)} = G$
	\item[(b)] $\gen{R}_{F(X)} = \ker \phi$, where $\phi : F(X) \ra G$ is the homomorphism induced by the universal property of free groups:
	\begin{tikzcd}
	X \arrow[hookrightarrow]{r}{\iota}
	\arrow[swap]{d}{f}
	& F(X) \arrow[dotted]{dl}{\phi} \\
	G
	\end{tikzcd} $R$ is often called a set of "defining" relations.
	\end{enumerate}
\end{defn}

\vs

\begin{remark}
\
\begin{enumerate}
\item[(a)] Usually one thinks of $X$ as embedded in $G$ and does not distinguish (if $f$ is injective) between $X$ and $F(X)$. But one has to be careful in general: The relations in $R$ can force identities between the elements of $X$.

\item[(b)] Technically, a "relation" is just an element of $R$, and hence of $F(X)$. That is, a relation is of the form $[w]$ for some $w \in W(X)$. If $[w] = [x_1^{\ve_1} \cdots x_n^{\ve_n}] \in R$, this means that $f(x_1)^{\ve_1} \cdots f(x_n)^{\ve_n} = 1$ in $G$.

Usually, if $w = x_1^{\ve_1} \cdots x_n^{\ve_n}$ is a reduced word and $[w] \in R$, we also write $x_1^{\ve_1} \cdots x_n^{\ve_n} \in R$. In the presentation $\gen{X \mid R}$, in order to emphasize that $\phi(x_1^{\ve_1} \cdots x_n^{\ve_n}) = 1$, one often writes $x_1^{\ve_1} \cdots x_n^{\ve_n} = 1$ (in $G$). One also often writes $w_1 = w_2$ instead of $w_1 w_2\inv \in R$.

\begin{example*}
$S_3 = \gen{(1\ 2), (2\ 3)}$ has the presentation(s)
	\begin{align*}
	S_3 &= \gen{a,b \mid a^2 = b^2 = 1, (ab)^2 = 1} \\
	&= \gen{a,b \mid ^2 = b^2 = 1, aba = bab} \\
	&= \gen{a,b \mid a^2, b^2, ababab} \\
	&= \gen{a,b \mid [a^2], [b^2], [ababab]}
	\end{align*}
The homomorphism $\phi : \{a,b\} \ra S_3$ of this presentation is given by $\phi(a) = (1\ 2), \phi(b) = (2\ 3)$. $\ker \phi = \gen{R}_{F(X)}$.
\end{example*}

\item[(c)] Given $X$ and $R \sub F(X)$, there is always a group having the presentation $\gen{X \mid R}$, namely $\wt{G} \myeq F(X) / \gen{R}_G$. It immediately clear that $\wt{G}$ has the presentation $\gen{X \mid R}$:
\begin{tikzcd}
X \arrow{r}{f}
\arrow[hookrightarrow,swap]{d}{\iota}
& F(X)/\gen{R}_{F(X)} \\
F(X) \arrow[dotted,swap]{ur}{\tn{projection}}
\end{tikzcd}

If $G$ is given with a generating set $X$, in order to decide whether $G$ has the presentation $\gen{X \mid R}$ for some $R \sub F(X)$, one has to do the following:
\begin{enumerate}
\item[(i)] Check whether for any reduced word $w$ with $[w] \in R$, the equation $w = e$ holds in $G$.
\item[(ii)] If (i) is true, there exists a surjective homomorphism $\wt{\phi} : \wt{G} \ra G$, where $\wt{G} = F(X)/\gen{R}_{F(X)}$. Define $\wt{\phi} : \wt{G} \ra G$ by $\wt{\phi}(\wt{g}) = \phi(g)$ for all $g \in F(X)$ with $p(g) = \wt{g}$ (well-defined, check). Clearly, $\ker p = \gen{R}_{F(X)} \leq \ker \phi \nsg F(X)$.
\begin{tikzcd}
F(X) \arrowvert[two heads]{r}{\phi}
\arrowvert[two heads,swap]{d}{p}
& G \\
\wt{G} \arrowvert[dotted,swap]{ur}{\wt{\phi}}
\end{tikzcd}

\item[(iii)] Verify (often difficult) that $\wt{\phi}$ is injective.
\end{enumerate}

For finite groups $G$, the injectivity of $\wt{\phi}$ can sometimes by established by verifying $|\wt{G}| \leq |G|$.

For infinite groups, one can sometimes use "normal forms" for the elements in $G$ to verify that $\wt{\phi}$ is injective.
\end{enumerate}
\end{remark}

\vs

\begin{example}
\
\begin{enumerate}
\item[(a)] Consider $\gen{a,b \mid a^2 = b^2, a^3 = b^3}$. Note that $b^3 = a^3 = b^2 = a \implies b = a$, so $\gen{a,b} = \gen{a \mid \ } \cong \z$.

\item[(b)] Let us establish a presentation for $S_3 = \gen{(1\ 2), (2\ 3)}$. Put
	\[\wt{G} = \gen{a,b \mid a^2 = b^2 = 1, (ab)^3 = 1 \iff aba = bab} \]
Claim: $G' \myeq \{e,a,b,ab,ba,aba=bab\} = \wt{G}$. Observe that $aG' \sub G'$ and $b G' \sub G' \implies \wt{G} = \gen{a,b} \sub G' \implies \wt{G} = G'$. The consequence is that $|\wt{G}| \leq 6$.

We have a surjective homomorphism $\wt{\phi} : \wt{G} \ra S_3$, $a \mapsto (1\ 2), b \mapsto (2\ 3)$  (by verifying the relations hold in $S_3$). Since $|\wt{G}| \leq 6 = |S_3|$, $\wt{\phi}$ is injective $\implies \wt{\phi}$ is bijective $\implies \wt{G} \cong S_3$, hence $S_3$ has the presentation $\gen{a,b \mid a^2 = b^2 = 1, aba = bab}$.

\item[(c)] The quaternion group $Q$ has the presentation $\gen{a,b \mid a^4 = b^4 = 1,a^2 = b^2, aba\inv = b\inv (\implies b^4 = 1)}$ (Exercise). The strategy: show that $|\wt{G}| \leq 8$.

\item[(d)] $\z_m \times \z_n \cong \gen{a,b \mid a^m = b^n = 1, ab = ba}$.

\item[(e)] $D_{2n} \cong \gen{a,b \mid a^n = b^2 = 1, bab = a\inv}$.

\item[(f)] $D_\infty \cong \gen{a,b \mid b^2 = 1, bab = a\inv} \cong \gen{b,c \mid b^2 = c^2 = 1} \cong \z_2 * \z_2$ ("free product"). The $c$ in the second presentation corresponds to $ba$ in the first presentation.

\item[(g)] $\z^2 \cong \gen{a,b \mid ab = ba}$.
\end{enumerate}
\end{example}

\vs

\begin{proposition}
Let $G$ be a group with presentation $\gen{X \mid R}$ such that $X$ embeds into $G$ (usually $X \sub G$) and the image of $X$ generates $G$. If $H$ is another group, then a map $f : X \ra H$ extends uniquely to a homomorphism $\wt{\psi} : G \ra H$ if and only if
	\[f(x_1)^{\ve_1} \cdots f(x_n)^{\ve_n} = e_H \tag{$\ast$}\]
in $H$ for all $[x_1^{\ve_1} \cdots x_n^{\ve_n}] \in R$.
\end{proposition}

\begin{proof}
First, observe the following diagram:
\begin{tikzcd}
X \arrowvert{dr}{f}
\arrowvert[hookrightarrow,swap]{d}{\iota} \\
F(X) \arrowvert[dotted]{r}{\exists ! \ \psi}
\arrowvert[two heads,swap]{d}{p}
& H \\
X \sub G = F(X)/\gen{R}_{F(X)} \arrowvert[dotted,swap]{ur}{\tn{Want: } \wt{\psi}}
\end{tikzcd}
There exists a (unique) homomorphism $\psi : F(X) \ra H$ satisfying $\psi \circ \iota = f$, and the natural projection $p : F(X) \ra G$ satisfies $p|_X = \id_X$. \\

\tb{Uniqueness: } Assume $\wt{\psi}$ exists, so that $\wt{\psi} \circ p = \psi$. Let $w = [x_1^{\ve_1} \cdots x_n^{\ve_n}] \in R$. Because $p([w]) = e_G$, it follows that
	\begin{align*}
	e_H &= \wt{\psi}(e_G) \\
	&= (\wt{\psi} \circ p)(x_1^{\ve_1} \cdots x_n^{\ve_n}) \\
	&= (\wt{\psi} \circ p)(x_1)^{\ve_1} \cdots (\wt{\psi}\circ p)(x_n)^{\ve_n} \\
	&= \psi(x_1)^{\ve_1} \cdots \psi (x_n)^{\ve_n} \\
	&= f(x_1)^{\ve_1} \cdots f(x_n)^{\ve_n}
	\end{align*}
We have shown that $(\ast)$ is necessary for the existence of $\wt{\psi}$. $\wt{\psi}$ is clearly unique, since $\psi|_X = f$, and $X$ generates $G$.

Conversely, assume that $(\ast)$ is satisfied by $f : X \ra H$. Then $R \sub \ker \psi \implies \gen{R}_{F(X)} \sub \ker \psi$, thus
	\begin{align*}
	f(x_1)^{\ve_1} \cdots f(x_n)^{\ve_n} &= e_H \\
	\iff \psi([x_1^{\ve_1} \cdots x_n^{\ve_n}]) &= e_H \quad \forall \ [x_1^{\ve_1} \cdots x_n^{\ve_n}] \in R
	\end{align*}
For $g \in G$, there exists $z \in F(X)$ with $p(z) = g$, so we define $\wt{\psi}(g) = \psi(z)$. Then $\wt{\psi} : G \ra H$ is a well-defined (check) group homomorphism satisfying $\wt{\psi}|_X = f$.
\end{proof}

\vs

\begin{example}
\
\begin{enumerate}
\item[(a)] $G = \z_m \times \z_n \cong \gen{a,b \mid a^m = b^n = aba\inv b\inv = 1}$. We want to define a homomorphism $\phi : G \ra G$ by setting $\phi(a) = a^i b^j$, $\phi(b) = a^k b^l$.

Question: When can $f : X \ra G$, $a \mapsto a^i b^j, b \mapsto a^k b^l$, be extended to a homomorphism $\phi : G \ra G$? If $G$ is abelian, then $f(a) f(b) = f(b) f(a)$ is automatically satisfied. We need $f(a)^m = 1 \iff a^{mi} b^{mj} = 1 \iff b^{mj} = 1 \iff n \mid mj \iff \frac{n}{(n,m)} \mid j$. Similarly, $f(b)^n = 1 \iff \frac{m}{(n,m)} \mid k$.

Special case: $m \mid n$, say $n = md$. Then the only condition needed is $d \mid j$. Additionally, $\phi \in \aut(G) \iff \gen{a^ib^j, a^kb^l} = G$.

\item[(b)] $G = \gen{a,b \mid a^k = 1, a^2 = b^2, aba\inv = b\inv}$. Then the previous proposition yields a group homomorphism $\phi : Q \ra G$, $i \mapsto a, j \mapsto b$, because $Q$ has a presentation $\gen{i,j \mid i^2 = j^2, iji\inv = j\inv}$. Since $|Q| = |G| = 8$, $\phi$ is bijective. One can show that $|\aut(Q)| = 24$. In fact, $\aut(Q) \cong S_4$.
\end{enumerate}
\end{example}



\part{Commutative Rings}

\chapter{ \null Basic Definitions and Concepts}

\begin{defn*}
A \tb{ring} is an ordered triple $R,+,\cdot)$, where $R$ is a set and $+,\cdot : R \times R \ra R$ are binary operations such that
\begin{enumerate}
\item[(1)] $(R,+)$ is an abelian group with identity $0$.
\item[(2)] Multiplication is associative.
\item[(3)] Multiplication distributes over addition: $r \cdot (s+t) = r \cdot s + r \cdot t$ and $(r+s) \cdot t = r \cdot t + s \cdot t$ for all $r,s,t \in R$.
\item[(3)] (For us) There exists a multiplicative identity $1 \in R$ such that $1 \cdot x = x \cdot 1 = x$ for all $x \in R$.
\end{enumerate}

$R$ is called \tb{commutative} if multiplication is commutative. Note that $R = \{0\} \iff 1 = 0$.

\begin{example*}
$\z, \z_n, \Q, \R, \C, \z[i] = \{a+bi \mid a,b \in \z\}$ are all commutative rings. Examples of non-commutative rings include $M_n(F), n \geq 2$ for any field $F$, or more generally $M_n(R)$ for $R$ a ring. Another example is $\h = \R + i \R + j \R + k \R$, the quaternions, which is also an example of a \emph{skew field}. Formally a ring $R$ is a \tb{skew field} (i.e., \tb{division ring}) if $R \ne \{0\}$ and $R\x = R \bs \{0\}$, i.e., $(R\bs\{0\},\cdot)$ is a group. $R$ is a \tb{field} iff $R$ is a commutative skew field.
\end{example*}
\end{defn*}

\vs

\begin{defn*}[Direct Product of Rings]
Let $R,S$ be rings. Define the ring $R \times S = \{(r,s) \mid r \in R, s \in S\}$ with componentwise addition and multiplication, i.e.,
	\begin{align*}
	(r_1,s_1) + (r_2,s_2) &= (r_1 + r_2, s_1 + s_2) \\
	(r_1,s_1) \cdot (r_2,s_2) &= (r_1 \cdot r_2, s_1 \cdot s_2)
	\end{align*}
\end{defn*}

\vs

\begin{defn*}
A \tb{ring homomorphism} $\phi : R \ra S$ is a map satisfying
\begin{enumerate}
\item[(1)] $\phi(x+y) = \phi(x) + \phi(y)$, and
\item[(2)] $\phi(xy) = \phi(x) \phi(y)$ for all $x,y \in R$.
\item[(3)*] $\phi(1_R) = 1_S$ (independent of the previous two conditions).
\end{enumerate}
\end{defn*}

\vs

\begin{example*}
For $R \ne \{0\}$, $\phi : R \ra R \times R$, $r \mapsto (r,0)$, is NOT a ring homomorphism from our definition because $\phi(1_R) = (r_R,0) \ne (1_R,1_R) = 1_{R \times R}$.
\end{example*}

\vs

\begin{defn*}
A nonempty subset $S \sub R$ is called a \tb{subring} of $R$ if it is closed under addition and multiplication and $1_R \in S$.

\begin{example*}
$R \times \{0\}$ is not a subring of $R \times R$ for $R \ne \{0\}$.
\end{example*}

An element $x \in R$ is called a \tb{zero divisor} if there exists $0 \ne y \in R$ such that $xy = 0$ or $yx = 0$. Note that $0$ is always a zero divisor for $R \ne \{0\}$, since $1 \cdot 0 = 0$.
\end{defn*}

\vs

\begin{defn*}
A ring $R$ is called an \tb{integral domain} if $R \ne \{0\}$ is a commutative ring and $0$ is the only zero divisor of $R$.

\begin{example*}
$\z_n$ is an integral domain if and only if $n$ is prime. An exercise is to show that if $R$ is a finite integral domain, then $R$ is a field.
\end{example*}
\end{defn*}

\vs

\begin{defn*}
An element $x \in R$ is called a \tb{unit} if there exists $y \in R$ such that $xy = yx = 1$ (both equalities are necessary). Put $R\x = \{x \in R \mid x \tn{ is a unit}\}$, a group under multiplication with identity 1.

\begin{example*}
$\z\x = \{1,-1\}$, $\z_n\x = \{\ov{a} \mid (a,n) = 1\}$.
\end{example*}

If $\phi : R \ra S$ is a ring homomorphism, then $\phi(R\x) \sub S\x$. However, even if $\phi$ is surjective, $\phi(R\x)$ needn't be $S\x$.

\begin{example*}
$p : \z \ra \z_n, a \mapsto \ov{a}$. $|\z_n\x| = \phi(n) > 2 = |\z\x|$ for $n \geq 7$, where $\phi : \N \ra \N$ is the Euler totient function.
\end{example*}
\end{defn*}

\vs

From now on, all rings $R$ are commutative with 1.

\vs

\begin{defn*}[Ideals]
A nonempty subset $I \sub R$ is called an \tb{ideal} of $R$, written $I \nsg R$, if
\begin{enumerate}
\item[(i)] $x,y \in I \implies x+y \in I$
\item[(ii)] $x \in I$ and $r \in R \implies rx \in I$
\end{enumerate}
Note that $I = R \iff 1 \in I \iff I \cap R\x \ne \vn$. Note also that $x \in I \implies -x = (-1)x \in I$. Using $(i)$, $0 = x-x \in I$, so $I$ always contains $0$. Hence, $(I,+)$ is a subgroup of $(R,+)$.
\end{defn*}

\vs

\begin{remark}
If $\phi : R \ra S$ is a ring homomorphism, and $I \nsg R, J \nsg S$, then $\phi(I) \nsg S$ if $\phi$ is surjective, and $\phi\inv(J) \nsg R$ (always).
\end{remark}

\vs




%Below should be Proposition 2.1.6

\begin{proposition}
Let $I$ be an ideal of $R$ (commutative with 1).
\begin{enumerate}
\item[(a)] $I$ is a prime ideal if and only if $R/I$ is an integral domain.
\item[(b)] $I$ is a maximal ideal if and only if $R/I$ is a field.
\end{enumerate}
\end{proposition}

\begin{proof}\
\begin{enumerate}
\item[(a)] For $x,y \in R$, the following are equivalent:
	\begin{enumerate}
	\item[(i)] $xy \in I \implies x \in I$ or $y \in I$.
	\item[(ii)] $(x+I)(y+I) = I$ in $R/I$ $\implies x+I = I$ or $y+I = I$.
	\end{enumerate}
Also note that $I \ne R \iff R/I \ne \{0+I\}$. The former is part of the definition of prime ideal, and the latter is part of the definition of integral domain.

\item[(b)] Suppose $I \nsg R$ is maximal. Let $r + I \in R/I$ for some $r \in R\bs I$. By the correspondence theorem for rings, we must have that $0 + I \ne (r+I) \nsg R/I \implies (r+I) = R/I \implies r+I \in (R/I)\x$. Now, suppose $R/I \ne \{\ov{0}\}$ is a field, so that $R/I$ has no proper nontrivial ideals. By the correspondence theorem, this is equivalent to saying that there exist no ideals $J$ of $R$ with $I \subsetneq J \subsetneq R$. That is, $I$ is a maximal ideal.
\end{enumerate}
\end{proof}

\vs

\begin{corollary}
If $I \nsg R$ is maximal, then $I$ is prime.
\end{corollary}

\vs

\begin{remark}
Let $I \tl R$ be any proper ideal of $R$. Put $\ov{R} = R/I$, and let $\pi : R \ra \ov{R}, r \mapsto r+I$ be the canonical projection onto $\ov{R}$. By the correspondence theorem for rings, there is a one-to-one correspondence theorem between the set of ideals of $R$ containing $I$ and the set of ideals of $\ov{R}$ given by $J \mapsto \ov{J} = J/I \nsg \ov{R}$ and $L \mapsto \pi\inv(L)$ for $I \nsg J \nsg R$ and $L \nsg \ov{R}$. \\

This same map between sets of ideals can be restricted to a one-to-one correspondence between the set of \emph{prime ideals} of $R$ containing $I$ and the set of \emph{prime ideals} of $\otimes{R}$ or a one-to-one correspondence between the \emph{maximal ideals} of $R$ containing $I$ and the \emph{maximal ideals} of $\ov{R}$. The justification of this is fact is left as an exercise.
\end{remark}

\vs

\begin{notation*}
Let $R$ be a commutative ring with 1, as always. For $S \sub R$, we let $(S) \myeq \bigcap_{S \sub I \nsg R} I$ denote the ideal of $R$ generated by $S$. If $S \ne \vn$, then
	\[(S) = \{r_1x_1 + \dots + r_n s_n \mid n \in \N, r_i \in R, s_i \in S\} \]
A special case of this is when $S = \{x\}$ is a singleton, in which we write $(S) = (x) = Rx = \{rx \mid r \in R\}$, called the \textbf{principal ideal generated by x}.
\end{notation*}

\vs

\begin{defn}
An integral domain $R$ is called a \tb{principal ideal domain (PID)} if every ideal of $R$ is principal.
\begin{example*}
$\z,F,F[X]$ for $F$ a field.
\end{example*}
Let $a,b \in R$. We say that $a$ \tb{divides} $b$ or $b$ is a \tb{multiple} of $a$ if there exists $c \in R$ such that $ac = b$. We write $a \mid b$ for "$a$ divides $b$".
\end{defn}

\vs

\begin{remark}
Let $a \in R$.
\begin{enumerate}
\item[(a)] $(a) = \{b \in R \mid a \mid b\}$.
\item[(b)] $a \mid 0 \ \forall \ a \in R$ (includes $a = 0$).
\item[(c)] If $u \in R\x$, then $u \mid a$ since $u(u\inv a) = a$.
\item[(d)] $(a) = R \iff a \in R\x$ ($R$ is commutative).
\item[(e)] About fractions: An equation of the form $ac = b$ ($a,b,c \in R$) should \underline{not} be written as $c = \frac{a}{b}$, since in general $\frac{a}{b}$ is \emph{ambiguous}.
\begin{example*}
Let $R = \z_6$. Then $\ov{2} \cdot \ov{2} = \ov{4} = \ov{2} \cdot \ov{5}$. What is $\frac{\ov{4}}{\ov{2}}$ ? Is it $\ov{2}$ or $\ov{5}$?
\end{example*}
\end{enumerate}
\end{remark}

\vs

\begin{defn}
Two ideals $I,J \nsg R$ are called \tb{comaximal} if $I+J = R$. Equivalently, $I$ and $J$ are comaximal if there exist $a \in I, b \in J$ such that $a+b = 1$.
\end{defn}

\vs

\begin{example*}
$(m), (n) \nsg \z$ are comaximal if and only if $\gcd(m,n) = 1$.
\end{example*}

\vs

\begin{lemma}
	If $I_1, \dots, I_n \nsg R$ are pairwise comaximal, then also
	\begin{enumerate}
		\item $I_1 \cdots I_{n-1}$ and $I_n$ are comaximal
		\item \(\bigcap_{j=1}^n I_j = I_1 \cdots I_n \)
	\end{enumerate}
\end{lemma}

\begin{proof}
	\begin{enumerate}
		\item We have that $I_j$ and $I_n$ are comaximal for all $1 \leq j \leq n-1$. Thus, for each $1 \leq j \leq n-1$, there exist $a_j \in I_j, b_j \in I_n$ such that $a_j + b_j = 1 \implies 1 = 1 \cdot 1 \cdots 1 = \prod_{j=1}^{n-1} (a_j + b_j) \in a_1 \cdots a_{n-1} + I_n \sub I_1 \cdots I_{n-1} + I_n$, so $I_1 \cdots I_{n-1}$ and $I_n$ are comaximal.

		\item We proceed by induction on $n$.
		\begin{enumerate}
			\item Base Case ($n = 1$): Clearly.

			\item Base Case ($n = 2$): We need only to show that $I_1 \cap I_2 \sub I_1 I_2$. Since $I_1$ and $I_2$ are comaximal, there exist $a \in I_1$ and $b \in I_2$ such that $a+b = 1$. For every $x \in I_1 \cap I_2$, we have $x = x(a+b) = xa + xb \in I_1 I_2$.

			\item Inductive Step ($n \geq 3$): Observe that
			\begin{align*}
				&\quad\, \bigcap_{j=1}^n I_j \\
				&= \left(\bigcap_{j=1}^{n-1} I_j\right) \cap I_n \\
				&= I_1\cdots I_{n-1} \cap I_n &\text{by the Inductive Hypothesis} \\
				&= I_1\cdots I_{n-1}I_n &\text{by part 1 and Base Case $n=2$} \\
				&= I_1\cdots I_{n}
			\end{align*}
		\end{enumerate}
	\end{enumerate}
\end{proof}

\vs

\begin{proposition}[Chinese Remainder Theorem]
Let $I_1, \dots, I_n \nsg R$ (commutative with 1) and $\phi R \ra \times_{j=1}^n R/I_j$ be $r \mapsto (r/I_1, \dots, r/I_n)$.
\begin{enumerate}
\item[(a)] $\phi$ is a ring homomorphism with $\ker \phi = \bigcap_{j=1}^n I_j$.
\item[(b)] If $I_1,\dots,I_n$ are pairwise comaximal, then of course $\ker \phi = \prod_{J=1}^n I_j$, $\phi$ is surjective, and thus
	\[R / \prod_{j=1}^n I_j \cong \bigtimes_{j=1}^n R/I_j \]
\end{enumerate}
\end{proposition}

\begin{proof}\
\begin{enumerate}
\item[(a)] Trivial (check).
\item[(b)] $\ker \phi = \prod_{j=1}^n I_j$ follows from (a) and the previous lemma. We need to show that $\phi$ is surjective. By 2.1.12, we get for all $1 \leq l \leq n$ tuat the ideals $\widehat{I_l} = \prod_{1 \leq j \ne l \leq n} I_j$ and $I_l$ are comaximal $\implies \exists \ x_l \in \widehat{I_l}$ and $y_l \in I_l$ such that $x_l + y_l = 1$. That is, $x_l \cong 0 \pmod{I_j}$ for all $j \ne l$ and $x_l \cong 1 \pmod{I_l}$. Given $(a_j + I_j)_1^n \in \bigtimes_{j=1}^n R/I_j$, ($a_j \in R$), we set $x \myeq \sum_{l=1}^n a_l x_l \in R$, so that $x+I_j = a_jx_j + I_j = a_j + I_j$ for all $1 \leq j \leq n$. That is,
	\[\phi(x) = (x+I_j)_1^n = (a_j + I_j)_1^n, \]
so $\phi$ is surjective.
\end{enumerate}
\end{proof}

\vs

\begin{remark*}
A special case of the CRT is when $R = \z$.
\begin{enumerate}
\item[(a)] $n = 2$. Let $I_1 = (m), I_2 = (n)$ with $\gcd(m,n) = 1$. Then $\z_{mn} = \z/(m)(n) \cong \z_m \times \z_n$, which is 1.2.3.
\item[(b)] Let $n \in \N$ be arbitrary, $p_1, \dots, p_n$ distinct primes, and $e_1,\dots,e_n \in \N$. Put $I_j = (p_j^{e_j})$ for $1 \leq j \leq n$. Then
	\[\z/(p_1^{e_1} \cdots p_n^{e_n}) \cong \z/(p_1^{e_1}) \times \dots \times \z/(p_n^{e_n}) \]
This means that there exists a solution to the system of congruences $x \cong a_j \pmod{p_j^{e_j}}$, $1 \leq j \leq n$, which is unique modulo $p_1^{e_1} \cdots p_n^{e_n}$.
\end{enumerate}
\end{remark*}



\chapter{ \null Rings of Fractions}

% Matbe talk abou Q and Z as the prototype for rings of fractions

\tb{Question:} If $R$ is a commutative ring with 1, is it possible to embed $R$ as a subring in some field $F$?

\tb{Answer:} Not always. A necessary and sufficient condition is that $R$ is an integral domain.

\vs

\tb{Main motivations for introducing rings of fractions:}
\begin{enumerate}
\item[(1)] To canonically embed an integral domain $R$ into its "field of fractions".
\item[(2)] "Localization"
\end{enumerate}

\vs

\begin{defn}
A subset $D$ of $R$ is called \tb{multiplicatively closed} if
\begin{enumerate}
\item[(a)\ \ ] $d,d' \in D \implies dd' \in D$
\item[(b)*] $1 \in D$
\end{enumerate}
\end{defn}

Note that one might call this a multiplicative monoid, or a multiplicative submonoid of $R$.

\vs

\begin{example*}
\
\begin{enumerate}
\item[(a)] If $R$ is an integral domain, $D = R \bs \{0\}$ is multiplicatively closed.
\item[(b)] If $R$ is any ring and $P \nsg R$ is a prime ideal, then $D = R \bs P$ is multiplicatively closed.
\item[(c)] For $x \in R$, $D = \{x^n \mid n \in \N_0\}$ is multiplicatively closed.
\end{enumerate}
\end{example*}

Note that (a) is just a special case of $(b)$, as $\{0\}$ is a prime ideal of $R$ $\iff R$ is an integral domain.

\vs

\section*{Constructing $\mb{D\inv R}$}

\vs

Our motivation for constructing the ring of fractions $D\inv R$ is to \emph{invert} the elements of $D$. We construct $D\inv R$ as follows. Start with the cartesian product $R \times D$. Define an equivalence relation $\sim$ on $R \times D$ by $(r,d) \sim (r',d') \iff $ there exists $e \in D$ such that $e(rd' - r'd) = 0$. The only thing to check is that $\sim$ is transitive. Suppose $(r,d) \sim (r',d') \sim (r'',d'')$, so that there exist $d,f \in D$ such that
\begin{alignat*}{3}
e(rd'-r'd) &= 0 = f(r'd''-r''d) && && \\
\implies erd' &= er'd, \qquad && fr'd'' &&= fr''d'' \\
\implies erd'fd'' &= er'dfd'', \qquad && er'dfd'' &&= edfr''d \tag{multiply by $fd''$} \\
\implies efd'(rd''-r''d) &= 0 && &&
\end{alignat*}
Now $erd' \in D$ as $D$ is multiplicatively closed, hence $(r,d) \sim (r'',d'')$.

\vs

\begin{notation*}
For $r \in R$, $d \in D$, $\frac{r}{d}$ denotes the equivalence class of $(r,d)$ with respect to $\sim$. We then define the set
	\[D\inv R = \left\{ \frac{r}{d} \mid r \in R, d \in D \right\} \]
\end{notation*}

\vs

We now introduce the binary operations $+$ and $\cdot$ on $D\inv R$ as follows:
	\[\frac{r}{d} + \frac{s}{e} \myeq \frac{re+sd}{de}, \quad \frac{r}{d}\frac{s}{e} \myeq \frac{rs}{de} \]
We must check that $+$ and $\cdot$ are well-defined. To show that $+$ is well-defined, suppose $\frac{r}{d} = \frac{r'}{d'}, \frac{s}{e} = \frac{s'}{e'}$ in $D\inv R$. Then there exist $x,y \in D$ such that $x(rd'-r'd) = 0 = y(se'-s'e)$. We need to show that
	\[\frac{r'e'+s'd'}{d'e'} = \frac{re+sd}{de} \]
Indeed, we have
\begin{align*}
xy[(re+sd)d'e' - (r'e'+s'd')de] &= xyred'e' + xysdd'e' - xyr'e'de - xys'd'de \\
&= (xr'd)yee' + xysdd'e' - xys'd'de - (xrd')ye'e \\
&= xyee'(r'd-rd') + xydd'(s'e-se') \\
&= 0,
\end{align*}
where the second equality used our assumption that $xrd' = xr'd$ and $yse' = ys'e$. It is more easily shown that multiplication is well-defined, and we omit those arguments. \\

By using representatives for addition and multiplication and the fact that $R$ is already a commutative ring with 1, it follows easily now that $D\inv R$ is a commutative ring with unity $\frac{1}{1}$. We also get that $\left( \frac{d}{1} \right)\inv = \frac{1}{d} \implies \frac{1}{d} \in (D\inv R)\x$ for all $d \in D$.

\vs

\begin{remark*}
$D\inv R = \{0\} \iff 0 \in D$. Indeed, $D\inv R = \{0\} \iff \frac{0}{1} = \frac{1}{1} \iff \exists \ d \in D$ such that $0 = d(1-0) = d$. One usually requires that $0$ is excluded from $D$, but this is not necessary for $D\inv R$ to be a ring-you just end up with the (boring) zero ring.

If $D$ contains any nilpotent elements $x$, say $x^n = 0$, then $D\inv R = \{0\}$. However, $D$ may contain zero divisors and still $D\inv R \ne \{0\}$ (e.g., $D = R \bs P$ for $P \nsg R$ prime).
\end{remark*}

\vs

There is a canonical ring homomorphism $j : R \ra D\inv R$, $r \mapsto \frac{r}{1}$. Clearly, $\ker j = \{r \in R \mid \exists \ d \in D \tn{ such that } dr = 0\}$. Thus, $j$ is injective if and only if $D$ contains no zero divisors. In particular, if $R$ is an integral domain and $0 \notin D$, then $j$ is injective. Our discussion is summarized in the following proposition/definition.

\vs

\begin{proposition}[And Definition]
\
\begin{enumerate}
\item[(a)] For any multiplicatively closed subset $D \sub R$, $(D\inv R, +, \cdot)$ as defined above is a commutative ring with 1, called the \tb{ring of fractions} of $R$ with respect to $D$.
\item[(b)] $D\inv R = \{0\} \iff 0 \in D$.
\item[(c)] The canonical map $j : R \ra D\inv R, r \mapsto \frac{r}{1}$, is a ring homomorphism with $j(D) \sub (D\inv R)\x$.
\item[(d)] $j$ is injective if and only if $D$ contains no zero divisors.
\item[(e)] If $R$ is an integral domain and $D = R\bs\{0\}$, then $D\inv R$ is a field containing $j(R) \cong R$ as a subring, called the \tb{ring of fractions of $\mb{R}$}. In this case, we write $\tn{Frac}(R) = D\inv R$.
\end{enumerate}
\end{proposition}

\vs

\begin{proposition}[Universal Property of Rings of Fractions]
For any commutative ring $S$ with unity and any ring homomorphism $\phi : R \ra S$ with $\phi(D) \sub S\x$, there exists a unique ring homomorphism $\wt{\phi} : D\inv R \ra S$ such that $\wt{\phi} \circ j = \phi$. That is, the following diagram commutes:
\begin{tikzcd}
R \arrow{r}{\phi}
\arrow[swap]{d}{j} & S \\
D\inv R \arrow[dotted,swap]{ur}{\exists ! \ \wt{\phi}}
\end{tikzcd}
\end{proposition}

\begin{proof}
Note that if $\wt{\phi}$ exists, then $\phi(d) = \wt{\phi}(\frac{d}{1}) \in S\x$ since $\frac{d}{1} \in (D\inv R)\x$. Thus
\begin{align*}
1_S = \wt{\phi} \left( \frac{1}{1} \right) &= \wt{\phi} \left( \frac{d}{1} \cdot \frac{1}{d} \right) = \wt{\phi} \left( \frac{d}{1} \right) \wt{\phi} \left(\frac{1}{d} \right) = \phi(d) \wt{\phi} \left(\frac{1}{d} \right) \\
\implies \wt{\phi} \left(\frac{1}{d} \right) &= \phi(d)\inv
\end{align*}
Thus, for all $r \in R$, $d \in D$, $\wt{\phi}(\frac{r}{d}) = \wt{\phi}(\frac{r}{1}) \wt{\phi}(\frac{1}{d}) = \phi(r) \phi(d)\inv$, so $\wt{\phi}$ is uniquely determined by $\phi$. \\

We now show that $\wt{\phi}$ exists. Define $\wt{\phi} : D\inv R \ra S$ by $\wt{\phi}(\frac{r}{d}) = \phi(r) \phi(d)\inv$.

$\wt{\phi}$ is \underline{well-defined}. Suppose $\frac{r}{d} = \frac{r'}{d'}$ in $D\inv R$, so that there exists $e \in D$ with $e(rd'-r'd) = 0_R$. Then
\begin{align*}
\phi(e) (\phi(r)\phi(d') - \phi(r')\phi(d)) &= 0_S \\
\iff \phi(r)\phi(d') &= \phi(r') \phi(d) \\
\implies \phi(r)\phi(d)\inv &= \phi(r')\phi(d')\inv \\
\iff \wt{\phi} \left( \frac{r}{d} \right) &= \wt{\phi} \left( \frac{r'}{d'} \right)
\end{align*}
It immediately follows that $\wt{\phi}$ satisfies the remaining properties of a ring homomorphism because $\phi$ is a ring homomorphism.
\end{proof}

\vs

\begin{corollary}
Let $R$ be an integral domain with $F \myeq \tn{Frac}(R)$. If $\phi : R \ra K$ is an injective ring homomorphism for some field $K$, then $K$ contains an isomorphic copy of $F$. Thus, $F = \tn{Frac}(R)$ is the smallest field containing $R$.
\end{corollary}

\begin{proof}
Because $\phi$ is injective, $\phi(R \bs \{0\}) \sub K \bs\{0\} = K\x$. We apply 2.2.3 to get that there exists a unique ring homomorphism $\wt{\phi} : F \ra K$ with $\wt{\phi} \circ j = \phi$. In particular, $\wt{\phi} \not\equiv 0$, so $\ker \wt{\phi} = \{0\}$, since $\{0\}$ and $F$ are the only ideals of the field $F$. Thus, $\wt{\phi}$ is injective $\implies F \cong \wt{\phi}(F) \sub K$.
\end{proof}

\vs

\section*{Ideals in $\mb{D\inv R}$}

In this section, $R$ is always a nonzero commutative ring with 1 and $D$ is a multiplicatively closed subset of $R$ which does not contain 0.

\vs

\begin{remark}
\
\begin{enumerate}
\item[(a)] If $I \nsg R$, then $D\inv I \myeq \{\frac{a}{d} \mid a \in I, d \in D\} \nsg D\inv R$.
\item[(b)] $D\inv I = D\inv R \iff \frac{1}{1} \in D\inv I \iff D \cap I \ne \vn$.
\end{enumerate}
\end{remark}

\vs

\begin{proposition}
\
\begin{enumerate}
\item[(a)] The map $\{I \tl R \mid I \cap D = \vn \} \ra \{\tn{proper ideals of } D\inv R\}$, $I \mapsto D\inv I$, is surjective.
\item[(b)] If $P \tl R$ is a prime ideal with $P \cap D = \vn$, then $j\inv(D\inv P) = P$.
\item[(c)] The map $\phi : \{P \tl R \mid P \tn{ prime}\} \ra \{\tn{prime ideals of } D\inv R\}, P \mapsto D\inv P$, is bijective.
\end{enumerate}
\end{proposition}

\begin{proof}\
\begin{enumerate}
\item[(a)]Let $J \tl D\inv R$. Put $I = j\inv(J) \nsg R$. We claim that $D\inv I = J$. Clearly, $D\inv I \sub J$. To show $J \sub D\inv I$, let $\frac{a}{d} \in J$. Then
	\[j(a) = \frac{a}{1} = \left( \frac{a}{d} \right) \left( \frac{d}{1} \right) \in J \iff a \in j\inv \left(\frac{a}{1} \right) \implies \frac{a}{d} \in D\inv J = I \]
We have also shown that $I \ne R$ and $I \cap D = \vn$, since $D\inv I = J \ne D\inv R$.

\item[(b)] Let $P \tl R$ be a prime ideal with $P \cap D = \vn$. It is clear that $P \sub j\inv(D\inv P)$. Let $a \in j\inv(D\inv P)$, so that $j(a) = \frac{a}{1} \in D\inv P$. Then there is $b \in P$ and $d \in D$ such that $\frac{a}{1} = \frac{b}{d} \iff ade = be \in P$ for some $e \in D$. Now $P$ is a prime ideal and $d,e \notin P$ as $P \cap D = \vn$, thus $de \notin P$, and $a(de) \in P$, so it follows that $a \in P$, as desired.

\item[(c)] We first show that $\im \phi \sub \mc{P} \myeq \{\tn{prime ideals of } D\inv R \}$. If $P \tl R$ is a prime ideal with $P \cap D = \vn$, then we claim that $\phi(P) = D\inv P \nsg D\inv R$ is prime. Indeed, $D\inv P \ne D\inv R$ by $A$. Suppose that $\frac{r}{d} \cdot \frac{s}{e} \in D\inv P$ for some $\frac{r}{d}, \frac{s}{e} \in D\inv R$. Then there exists $a \in P$ and $f \in D$ such that $\frac{rs}{de} = \frac{a}{f}$, hence there exists $g \in D$ with $g(rsf-ade) = 0 \iff (rs)(fg) = adeg \in P$. Now, $f,g \in D \implies f,g \notin P \implies fg \notin P$ as $P$ is prime, so it follows that $fs \in P \implies$ $r \in P$ or $s \in P \implies$ $\frac{r}{d} \in D\inv P$ or $\frac{s}{e} \in D\inv P$. So $D\inv P$ is prime. \\

$\phi$ is injective by (b). To show surjectivity, let $P' \tl D\inv R$ be a prime ideal, and consider $P \myeq j\inv(P') \nsg R$. $P$ is prime in $R$ as a consequence of Exercise 4 on Homework 9, and in the proof of part (a) we verified that $j(P) = P'$.
\end{enumerate}
\end{proof}

\vs

\begin{remark*}
Note that the map in part (a) above is \emph{order-preserving}. That is, if $I\sub J$ are proper ideals of $R$ with $I \cap D = J\cap D = \vn$, then $D\inv I \sub D\inv J$.
\end{remark*}

\vs

\begin{defn}[\& Corollary]
If $P \tl R$ is a prime ideal, then $D \myeq R \bs P$ is multiplicatively closed. The ring $D\inv R$, denoted by $R_P$, is called the \tb{localization of $\mb{R}$ at $\mb{P}$}. By 2.2.6, we obtain the bijective map
\begin{align*}
\{Q \tl R \mid Q \tn{ is prime and } Q \sub P\} &\ra \{\tn{prime ideals of } R_P\} \\
Q &\mapsto D\inv Q
\end{align*}
As a consequence, $PR_P \myeq D\inv P$ is the \emph{unique} maximal ideal of $R_P$. A (commutative) ring is called a \tb{local ring} if it has a unique maximal idea.
\end{defn}

\vs

\begin{example}
Let $R = \z$, $P = (p)$ for $p \in \z$ a prime. Then
\[\z_{(p)} = \{ \frac{a}{b} \mid a,b \in \z, p \nmid b\} \sub \Q = \tn{Frac}(\z) = \z_{(0)} \]
The prime ideals of $\z_{(p)}$ are $\{0\}$ and $p\z_{(p)} = \{\frac{a}{b} \mid a,b \in \z, p\nmid b, p \mid a\}$.
\end{example}

\vs

\chapter{ \null Polynomial Rings}

As usual, $R$ always denotes a commutative ring with $1 \ne 0$. \\

Let us first consider a bad way to formally construct the polynomial ring $R[x]$. We could set
	\[R[x] \myeq \{f : R \ra R \mid \exists \ a_i \in R \tn{ such that } f(x) = \sum_{i=0}^n a_i x^i \ \ \forall \ x \in R\} \]
If $R = \mbb{F}_p$ for $p \in \z$ a prime, then $x,x^p \in \mbb{F}_p[x]$ define the same function from $\mbb{F}_p \ra \mbb{F}_p$. But if $x$ is to truly be an \emph{indeterminant} variable, then we want $x \ne x^p$.

\vs

\begin{defn}
The \tb{polynomial ring} (in one variable over $R$) is the set
	\[R[x] \myeq \{(a_i)_{i \in \N_0} \mid a_i \in R, a_i=0 \tn{ for all but finitely many }i \} \]
together with two binary operations $+, \cdot : R[x] \times R[x] \ra R[x]$ defined as
\begin{align*}
(a_i)_{i \in \N_0} + (b_i)_{i \in \N_0} &\myeq (a_i + b_i)_{i \in \N_0} \\
(a_i)_{i \in \N_0} \cdot (b_i)_{i \in \N_0} &\myeq \left(\sum_{i=0}^k a_i b_{k-i}\right)_{k \in \N_0}
\end{align*}
One should verify that $(R[x],+,\cdot)$ is a commutative ring with unity element $(1,0,0,\dots)$. The only part of this verification that takes some work is the associativity of multiplication. If $a = (a_i),b=(b_i),c=(c_i) \in R[x]$, then
	\[((ab)c)_m = \sum_{i+j+k=0}^m a_i b_j c_k \]
The elements of $R[x]$ are called \tb{polynomials}. There is a natural embedding $R \hra R[x], r \mapsto (r,0,0\dots)$. There is also a distinguished element $x \myeq (0,1,0,0,\dots) \in R[x]$. For $a = (a_i) \in R[x]$, one checks that
\begin{enumerate}
\item[(i)] $(xa)_0 = 0$, and
\item[(ii)] $(xa)_{i+1} = a_i$ for all $i \in \N_0$.
\end{enumerate}
It should also be verified (inductively) that $x^n = (\delta_{in})_{i \in \N_0}$ for all $n \in \N_0$, where
	\[\delta_{ij} = \begin{cases}
	1, \qquad & i = j \\
	0, & i \ne j
	\end{cases} \]
\end{defn}

\vs

\begin{lemma}[\& Definition]
Any nonzero element $a \in R[x]$ can be uniquely written in the form \[\sum_{i=0}^d a_i x^i \tag{$\ast$} \]
with $a_i \in R$, $a_d \ne 0$, and $a_i = 0$ for all $i < d$. $d$ is called the \tb{degree} if the polynomial $a$, denoted by $d = \deg a$. Note that $\deg a = 0 \iff a \in R \bs \{0\}$. $a_d$ is called the \tb{leading coefficient} of $a$, denoted by $a_d = \ell(a)$. $a$ is called \tb{monic} if $\ell(a) = 1$. By convention, we set $\ell(a)$ and $\deg a = -\infty$ if $a = 0$. \\

With elements in $R[x]$ represented as in ($\ast$), we get the usual definition of addition and multiplication of polynomials, namely
\begin{align*}
\sum_{i=0}^d a_i x^i + \sum_{j=0}^e b_j &= \sum_{i=0}^{\max(d,e)} (a_i + b_i)x^i, \\
\left( \sum_{i=0}^d a_i x^i\right) \left( \sum_{j=0}^e b_j\right) &= \sum_{k=0}^{d+e} \left(\sum_{k=0}^k a_i b_{k-i}\right) x^k
\end{align*}
\end{lemma}

\begin{proof}
Let $ 0 \ne a = (a_0,a_1,\dots,a_d,0,0,\dots) \in R[x]$, with $d \in \N_0$ such that $a_d \ne 0$ and $a_i = 0$ for all $i > d$. Such $d$ exists by taking $d = \max \{i\in \N_0 \mid a_i \ne 0\}$. Since $x^i = (0,0,\dots,1,0,\dots)$ ($(i+1)$-th slot), we get
\begin{align*}
\sum_{i=0}^d a_i x^i &= (a_0,0,0,\dots) + \dots + (0,0,\dots,a_d,0,\dots) \\
&= (a_0,a_1,\dots,a_d,0,0,\dots) \\
&= a
\end{align*}
The elements $a_i$ of the sequence $(a_i)_{i \in \N_0}$ uniquely determine the coefficients in $\sum_{i=0}^d a_i x^i$, so the representation ($\ast$) is unique. The expressions for addition and multiplication are obvious.
\end{proof}

\vs

\begin{remark}
In a similar way, one can define the \tb{power series ring} $R[[x]]$ as the set
	\[R[[x]] \myeq \{(a_i)_{i \in \N_0} \mid a_i \in R \} \]
with the same addition and multiplication as for polynomials:
\begin{align*}
(a_i) + (b_i) = (a_i + b_i) (a_i)(b_i) &= \left(\sum_{i=0}^k a_i b_{k-i} \right)_{k \in \N_0}
\end{align*}
Again, we define $x \myeq (0,1,0,0,\dots)$.
\end{remark}

\vs

\begin{lemma}
Let $f,g \in R[x]$.
\begin{enumerate}
\item[(a)] $\deg (f+g) \leq \max(\deg f, \deg g)$.
\item[(b)] $\deg(fg) \leq \deg f + \deg g$, with equality holding if $\ell(f)$ is not a zero divisor.
\end{enumerate}

\begin{proof}
Suppose $f = \sum_{i=0}^d a_i x^i, g = \sum_{j=0}^e b_j x^j$, where, without loss of generality, we assume $f \ne 0$ and $g \ne 0$.
\begin{enumerate}
\item[(a)] We get
	\[f + g = \sum_{i=0}^d a_i x^i + \sum_{j=0}^e b_j x^j = \sum_{i=0}^{\max(d,e)} (a_i + b_i)x^i, \]
so that $\deg(f+g) \leq d+e = \deg f + \deg g$, because $a_i = b_i = 0$ for all $i > \max(d,e)$.

\item[(b)] We have
\[fg = a_0 b_0 + \sum_{k=1}^{d+e-1} \left(\sum_{i=0}^k a_i b_{k-i}\right)x^k + a_d b_ex^{d+e} \]
Clearly, we have $\deg (fg) \leq d+e = \deg f + \deg g$. If $a_d = \ell(f)$ is not a zero divisor, then, because $b_e = \ell(g) \ne 0$, we have $a_db_e \ne 0$, so that $\ell(fg) = a_d b_e$ and $\deg(fg) = d+e = \deg f + \deg g$.
\end{enumerate}
\end{proof}
\end{lemma}

\vs

\begin{proposition}
Let $R$ be an integral domain.
\begin{enumerate}
\item[(a)] If $f,g \in R[x]$, then $\deg(fg) = \deg(f) + \deg(g)$.
\item[(b)] $R[x]$ is an integral domain.
\item[(c)] $(R[x])\x = R\x$.
\end{enumerate}
\end{proposition}

\begin{proof}
\
\begin{enumerate}
\item[(a)] We may assume, without loss of generality, that $f,g \ne 0$, so that $\ell(f)$ and $\ell(g)$ are not zero divisors. Then the statement follows from 2.3.4 (b).
\item[(b)] If $f,g \ne 0$, then it follows from part (a) that $ \deg(Fg) = \deg f + \deg g \geq 0 \ne -\infty = \deg 0$, so $fg \ne 0$.
\item[(c)] As $R$ injects into $R[x]$, $R\x \sub (R[x])\x$. Let $f \in R[x]\bs R$, so that $\deg f \geq 1$. For all $g \in R[x] \bs \{0\}$, $\deg(fg) = \deg f + \deg g \geq 1 \implies fg \ne 1$, hence $f \notin (R[x])\x$.
\end{enumerate}
\end{proof}

\vs

\begin{remark}
Let $R$ be a commutative ring with 1.
\begin{enumerate}
\item[(a)] One can show that $a_0 + a_1x + \dots + a_dx^d \in (R[x])\x \iff a_0 \in R\x$ and $a_1, \dots,a_d$ are nilpotent.
\item[(b)] $(R[[x]])\x = \{\sum_{i=0}^\infty a_i x^i \mid a_i \in R, a_0 \in R\x\}$ (Exercise).
\end{enumerate}
\end{remark}

\vs

\begin{proposition}[Universal Property of \tn{$R[x]$}]
For any commutative ring $S$ (with 1), any ring homomorphism $\phi : R \ra S$, and any $a \in S$, there exists a unique ring homomorphism $\wt{\phi}_a : R[x] \ra S$ such that $\wt{\phi}_a(x) = a$ and $\wt{\phi}_a|_R = \phi$.
\begin{tikzcd}
R \arrow{r}{\phi}
\arrow[hookrightarrow]{d} & S \ni a \\
R[x] \arrow[swap]{ur}{\wt{\phi}_a}
\end{tikzcd}
\end{proposition}

\begin{proof}
Set $\psi = \wt{\phi}_a$. If $f = \sum_{i=0}^d r_i x^i \in R[x]$, and if $\psi$ exists, then $\psi$ must satisfy
\begin{align*}
\psi(f) = \psi\left( \sum_{i=0}^d r_i x^i \right) &= \sum_{i=0}^d \psi(r_i) \psi(x)^i = \sum_{i=0}^d \phi(r_i)a^i, \tag{$\ast$}
\end{align*}
so $\psi$ is uniquely determined by $\phi$ and $a$. To show that $\psi$ exists, define $\psi : R[x] \ra S$ by ($\ast$). Clearly, $\psi|_R = \phi$, and $\psi(x) = a$. For $f = \sum_{i=0}^d r_i x^i, g = \sum_{i=0}^e s_ix^i \in R[x]$,
\begin{align*}
\psi(f+g) = \sum_{i=0}^{\max(d,e)}\phi(r_i+s_i)a^i &= \sum_{i=0}^{\max(d,e)} [\phi(r_i)+\phi(s_i)]a^i \\
&= \sum_{i=0}^d \phi(r_i) a^i + \sum_{i=0}^e \phi(s_i)a^i \\
&= \psi(f) + \psi(g)
\end{align*}
Similarly,
\begin{align*}
\psi(fg) = \psi \left(\sum_{k=0}^{d+e} \left(\sum_{i=0}^k r_is_{k-i}\right)x^k\right) &= \sum_{k=0}^{d+e}\left(\sum_{i=0}^k \phi(r_is_{k-i})\right)a^k \\
&= \sum_{k=0}^{d+e} \left(\sum_{i=0}^k \phi(r_i)\phi(s_{k-i})\right)a^k \\
&= \left[\sum_{k=0}^d \left(\sum_{i=0}^k \phi(r_i)\right)x^k\right] \left[\sum_{k=0}^e \left(\sum_{i=0}^k \phi(s_i)\right)x^k\right] \\
&= \psi(f) \psi(g)
\end{align*}
\end{proof}

\vs

\begin{remark*}[\& Definition]
Because $\wt{\phi}_a$ is a ring homomorphism, $\wt{\phi}_a(R[x])$ is a subring of $S$. A special case of the situation above is when $R$ is a subring of $S$. Then we define
	\[R[a] \myeq \wt{\phi}_a(R[x]) = \left\{\sum_{i=0}^d r_i a^i \mid n \in \N_0, r_i \in R\right\}, \]
which is called the ring obtained from $R$ by \tb{adjoining} $a$, and it is thought of as the smallest subring of $S$ containing $R$ and $a$.
\begin{example*}
\
\begin{enumerate}
\item[(a)] Consider $\z \sub \C$; $\z[i] = \{a+bi \mid a,b \in \z\}$ is called the ring of \tb{Gaussian integers.}
\item[(b)] Consider $\z \sub \Q$; $\z\left[\frac{1}{2}\right] = \{\frac{m}{2^n} \mid m \in \z, n \in \N_0\}$.
\item[(c)] If we think of $\z$ as a subring of $\R$, then $\z[\sqrt[3]{2}] = \{a + b \sqrt[3]{2} + c \sqrt[3]{4} \mid a,b,c \in \z\}$.
\end{enumerate}
\end{example*}
\end{remark*}

\vs

\begin{remark}[\& Definition]
Another special case of 2.3.7 is when $S=R$, $\phi = \id_R$. Then $a \in R$, so we define $\tn{ev}_a = \wt{\phi}_a : R[x] \ra \R$, $\sum_{i=0}^d r_i x^i \mapsto \sum_{i=0}^d r_i a^i$, called the \tb{evaluation homomorphism at a.} For $f \in R[x]$, we define $f(a) \myeq \tn{ev}_a(f) \in R$. It is now easy to see that $(f+g)(a) = f(a)+g(a)$ and $(fg)(a) = f(a)g(a)$ for all $f,g \in R[x]$. \\

Moreover, noting that $R^R = \{h : R \ra R\}$ is a commutative ring with 1 under componentwise addition and multiplication, we have a ring homomorphism $\psi : R[x] \ra R^R$, $f \mapsto \psi(f) : R \ra R$, $\psi(f)(a) = f(a)$. If this map $\psi$ is injective, then we may consider $R[x]$ as a ring of functions $R \ra R$, i.e., a subring of $R^R$.
\end{remark}

\vs

\begin{proposition}[Division Algorithm for Polynomial Rings]
Assume $0 \ne f \in R[x]$, with $\ell(f) \in R\x$. Then for any $g \in R[x]$, there exist uniquely determined $q,r \in R[x]$ such that $g = fq +r$, where $\deg r < \deg f$ or $r=0$.
\end{proposition}

\begin{proof}
We show uniqueness first. Suppose $g = fh+r = fh'+r'$, where $\deg r,\deg r' < \deg f$. Then $f(h-h') = r'-r$. If $h-h' \ne 0$, then, because $\ell(f) \in R\x$, we have
	\[\deg(r'-r) = \deg(f(h-h')) \geq \deg f > \max\{\deg r,\deg r'\} \geq \deg (r'-r) \]
Therefore, we must have $h-h' = 0 \implies r'-r = 0$. That is, $h=h'$ and $r=r'$. \\

We now show existence. We can assume, without loss of generality, that $g \ne 0$, otherwise we can simply take $0 = f \cdot 0 + 0$. We proceed by induction on $e := \deg g$. If $e = 0$, then $\deg g \leq \deg f$, so we may take $g = f\cdot 0 + g$. \\

For the inductive step, we may assume, without loss of generality, that $e \geq d := \deg f$, for otherwise we can simply take $g = f\cdot 0 + g$. We can write $g = \ell(g) x^e + g_1$, where $\deg g_1 < e$, and $f = ux^d + f_1$, where $u \in R\x$, $\deg f_1 < d$. Note that $g' := g - \ell(g)u\inv x^{e-d}f$ has degree at most $e-1 < e$. By the inductive hypothesis, there exist $h',r' \in R[x]$ with $g' = fh' + r'$ and $\deg r' < d$. Then
\[g = g' + \ell(g)u\inv x^{e-d}f = (h'+\ell(g)u\inv x^{e-d})f + r',\]
so taking $h = h'+\ell(g)u\inv x^{e-d}$ and $r=r'$ yields the result.
\end{proof}

\vs

\begin{corollary}
If $F$ is a field and $0 \ne f \in F[x]$, then for all $g \in F[x]$, there exist uniquely determined $h,r \in F[x]$ such that $g = fh + r$ with $\deg r < \deg f$ or $r=0$.

Note that $F[x]$ is therefore a Euclidean Domain.
\end{corollary}

\vs

\begin{example}
\
\begin{enumerate}
\item[(a)] Let $R = \z_6$, $f = -x +2$, $g = 2x^3+3$. Using polynomial long division, we see that $2x^3 = (-x+2)(-2x-4)-1$.
\item[(b)] Let $R = \z, f = 2x+1, g=x^2-1$. Note that $\ell(f) = 2 \notin \z\x$. We will see from (c) below that there do not exist $h,r \in \z[x]$ with $g=fh+r$ and $\deg r < 1$, as such $h$ and $r$ must be unique.
\item[(c)] Let $R=\Q, f=2x+1,g=x^2-1$. Polynomial long division yields $x^2-1 = (2x+1)(\frac{1}{2}x-\frac{1}{4})-\frac{3}{4}$.
\end{enumerate}
\end{example}

\vs

\begin{lemma}
For $g \in R[x]$ and $a \in R$ we have $g(a) = 0 \iff (x-a) \mid g$ in $R[x]$.
\end{lemma}

\begin{proof}
$\ell(x-a) = 1 \in R\x$, so by the division algorithm there exist $h,r \in R[x]$ with $g = (x-a)h + r$ and $\deg r < 1$. But $\deg r < 1 \iff r \in R$. Thus $g(a) = 0 \iff r=0 \iff g = (x-a)h$.
\end{proof}

\vs

\begin{proposition}
If $R$ is an integral domain, then any nonzero polynomial $f \in R[x]$ has at most $\deg f$ many roots in $R$.
\end{proposition}

\begin{proof}
By induction on $n = \deg f$. If $d = 0$, then $f \in R\bs\{0\}$, so $f$ has $0 \leq d$ roots in $R$. For the inductive step, we have $d \geq 1$. Assume $f$ has at least one root, say $\alpha \in R$ (otherwise, we are done). Then $f(\alpha) = 0$, so by 2.3.12, there exists $h \in R[x]$ with $f=(x-\alpha)h$. Now $\deg h = d-1 < d$, so by the induction hypothesis $h$ has at most $d-1$ rots in $R$. Because $R$ is an integral domain, we have that $\beta \in R$ is a root of $f$ if and only if $0 = f(\beta) = (\beta - \alpha)h(\beta) \iff \beta = \alpha$ or $\beta$ is a root of $h$. Thus, $f$ has at most $(d-1)+1 = d$ roots in $R$.
\end{proof}

\vs

\begin{remark*}
One really needs $R$ to be an integral domain for 2.1.13.
\begin{example*}
\
\begin{enumerate}
\item[(a)] $R = \z_8$, $f = x^2-1$ has 4 roots in $\z_8$: 1,3,5,7.
\item[(b)] If $R$ is a Boolean ring, then every $r \in R$ is a root of $x^2-x \in R[x]$.
\item[(c)] If $R = \mathbb{H}$ (quaternions), then $x^2+1$ has infinitely many roots, including all points on the "unit sphere" $\{a\i+b\j+c\k \in \mathbb{H} \mid a^2+b^2+c^2=1\}$.
\end{enumerate}
\end{example*}
\end{remark*}

\vs

\begin{theorem}
Let $R$ be an integral domain. Then any finite subgroup of $R\x$ is cyclic.
\end{theorem}

\begin{proof}
We combine 2.3.13 with 1.2.9 (b). Let $A \leq R\x$ be a subgroup of $R\x$ with order $n \in \N$. Note, of course, that $A$ is abelian. Fix $m \in \N$. Then for all $a \in R$, $a^m = 1 \iff a^m-1=0 \iff a$ is a root of $f := x^m-1\in R[x]$. By 2.1.13,
	\[|\{a \in A \mid a^m=1\}| = |\{a \in R \mid f(a)=0\}| \leq m, \]
so $A$ is cyclic by 1.2.9 (b).
\end{proof}

\vs

\begin{remark*}
2.3.14 is not true for arbitrary rings $R$. For example, consider:
\begin{enumerate}
\item[(a)] $R = \z_8 \implies R\x = \{1,3,5,7\} \cong \z_2 \times \z_2$.
\item[(b)] $R = \H$, $Q \leq \H\x$ is not cyclic.
\end{enumerate}
\end{remark*}

\vs

\begin{corollary}
If $F$ is a finite field, then $F\x$ is cyclic.
\end{corollary}

\vs

\begin{remark*}
Notice the interplay between group theory and ring theory that we have demonstrated: 1.2.9 (b), which is a purely group-theoretic result, gave us the statement 2.3.14 about rings. Conversely, 2.3.14 tells us that $\aut(\z_p) \cong \z_p\x \cong \z_{p-1}$ is cyclic, to complete the proof of 1.2.5.
\end{remark*}

\vs

\begin{remark}[Generalizations of Polynomial Rings]
There are at least two ways to construct the ring $R[x_1,\dots,x_n]$.
\begin{enumerate}
\item[(a)] Inductive approach for finitely many variables: $R[x_1,\dots,x_{n+1}] := R[x_1,\dots,x_n][x_{n+1}]$ for all $n \in \N_0$.
\item[(b)] Direct definition, generalizing 2.3.1. The idea to keep in mind is that
	\[R[x_1,\dots,x_n] := \left\{\sum_{\tn{finite}}a_{i_1,\dots,i_n}x^{i_1}\cdots x^{i_n} \mid i_j \in \N_0, a_{i_j} \in R \right\} \]
\end{enumerate}
The rigorous definition goes as follows: Fix $n \in \N$. Denote a sequence $(i_1,\dots,i_n) \in \N_0^n$ by $\vec{i}$. We define
	\[R[x_1,\dots,x_n] := \left\{f : \N_0^n \ra R \mid f(\vec{i})=0 \tn{ for all but finitely many } \vec{i}\in \N_0^n \right\} \]
We think of the elements $f(i_1,\dots,i_n) \in R$ as the coefficients $a_{i_1,\dots,i_n}$ (only finitely many are nonzero). We define $+$ and $\cdot$ on $R[x_1,\dots,x_n]$ by
\begin{align*}
(f+g)(\vec{i}) &= f(\vec{i}) + g(\vec{i}), \\
(f\cdot g)(\vec{k}) &= \sum_{\vec{i}+\vec{j} = \vec{k}} f(\vec{i})g(\vec{j}) \tag{$\ast$}
\end{align*}
With this definition, it is not hard to show that $R[x_1,\dots,x_n]$ is a commutative ring with 1 which contains $R$ as a subring. For $1 \leq j \leq n$, we distinguish the element $x_j \in R[x_1,\dots,x_n]$, where $x_j : \N_0^n \ra R$ is the function
	\[x_j(\vec{i}) = \begin{cases}
						1, \qquad & \tn{if } \vec{i} = (0,0,\dots,1,0,\dots,0)\ (j\tn{th position}) \in \N_0^n  \\
						0, & \tn{otherwise}
	\end{cases} \]
Further generalization to arbitrarily many variables can also be made. If $\mc{J}$ is any index set, then we can define
\[R[x_j \mid j \in \mc{J}] := \left\{f : \bigoplus_{\mc{J}}\N_0 \ra R \mid f(\vec{i})=0 \tn{ for all but finitely many } \vec{i} \in \bigoplus_{\mc{J}}\N_0 \right\}, \]
with the same addition and multiplication as in ($\ast$). We again distinguish, for each $j \in \mc{J}$, the element $x_j \in R[x_j \mid j \in \mc{J}]$ defined by
	\[x_j (\vec{i}) = \vec{i}_j, \]
where $\vec{i}_j = \vec{i}(j) \in \N_0$. It is not difficult to see that every $f \in R[x_j \mid j \in \mc{J}]$ can be written as a finite linear combination of the $x_j$'s.
\end{remark}



\chapter{ \null Euclidean Domains}



\begin{defn}
A \tb{Euclidean ring} is a ring $R$ which admits a (degree, norm) function $d : R\bs\{0\} \ra \N_0$ such that for all $a,b \in R$, $a \ne 0$, there exist $q,r \in R$ such that $b = qa + r$ with $r = 0$ or $d(r) < d(a)$. \\

Note that we do not require uniqueness of $q$ and $r$, $d(0)$ is not defined, and it does not have to be the case that $d(a) \leq d(ab)$ for all $a,b \in R \bs \{0\}$.

Note that if $R$ is an integral domain, then it is a Euclidean domain.
\end{defn}

\vs

\begin{example}
\
\begin{enumerate}
\item[(a)] $d : \z\bs\{0\} \ra \N_0$, $d(z) = |z|$.
\item[(b)] $d : F[x]\bs\{0\} \ra \N_0$, $d(f) = \deg f$.
\item[(c)] $d : F\x \ra \N_0$ can be any function, since $b = (ba\inv)a + 0$ for all $a,b \in F$, $a \ne 0$.
\item[(d)] $d : \z[i]\bs\{0\} \ra \N_0$, $d(a+bi) = a^2+b^2 = (a+bi)(a-bi)$.
\end{enumerate}
\end{example}

\vs

\begin{theorem}
Any Euclidean ring is a principal ideal ring.
\end{theorem}

\begin{proof}
Let $R$ be Euclidean. Let $I \nsg R$. Without loss of generality, take $I \ne \{0\} = (0)$. Put $m = \min\{d(x)\in \N_0 \mid x \in I\bs\{0\}\} \in \N_0$. Choose $a \in I$ such that $d(a) = m$. Note that, because $d(a)$ is defined, $a \ne 0$. We claim that $I = (a)$.  Let $b \in I$, and choose $q,r \in R$ such that $b = aq + r$ with $r = 0$ or $d(r) < a$. Since $r = b - aq \in I$, we cannot have $d(r) < d(a)$, hence $r = 0 \implies b = aq \in (a)$. Thus, $I = (a)$ is principal.
\end{proof}

\vs

\begin{defn}
Let $R$ be a commutative ring (with 1), $a,b \in R$. An element $c \in R$ is called a \tb{common divisor} of $a$ and $b$ if $c | a$ and $c | b$. A common divisor $d$ of $a$ and $b$ is called a \tb{greatest common divisor}, denoted $d = \gcd(a,b)$, if $c \mid d$ for all common divisors $c$ of $a$ and $b$.
\end{defn}

\vs

\begin{remark}
\
\begin{enumerate}
\item[(a)] $0 = \gcd(0,0)$.
\item[(b)] If $c = \gcd(a,b)$, then $uc = \gcd(a,b)$ for any $u \in R\x$.
\item[(c)] If $R$ is an integral domain, $a,b \in R$, and $c,c' = \gcd(a,b)$, then there exists $u \in R\x$ such that $c' = uc$. Indeed, we have $c = c'd$ and $c' = ce$ for some $d,e \in R$, so that $c = c'd = cde \implies 0 = c-cde = c(1-de)$. If $c = 0$, then it must be the case that $a=b=0 \implies c'=0 \implies c'=1 \cdot c$,  and we are done. Otherwise $c\ne 0$, and $R$ is a domain, so $c(1-de) = 0 \implies 1-de = 0 \implies d,e \in R\x$.
\item[(d)] In general, $\gcd(a,b)$ need not exist. For example, take $R = \z[\sqrt{-5}] = \{a+b\sqrt{-5} \mid a,b \in \z\}$, so $R\x = \{\pm 1\}$. Let
	\[a = 2(1+\sqrt{-5}), \qquad b = 6 = 2\cdot 3 = (1+\sqrt{-5})(1-\sqrt{-5}) \]
The divisors of $a$ are $\pm 1, \pm 2, \pm(1\pm \sqrt{-5}), \pm a$, so the common divisors of $a$ and $b$ are $\pm 1, \pm 2, \pm(1\pm \sqrt{-5})$. Since none of these common divisors is divisible by all of the others, $\gcd(a,b)$ does not exist.
\end{enumerate}
\end{remark}

\vs

\begin{lemma}
Let $R$ be a commutative ring with 1, and let $a,b,c \in R$.
\begin{enumerate}
\item[(a)] $c \mid a$ and $c \mid b \iff (a,b) \sub (c)$.
\item[(b)] $(a,b) = (c) \implies c = \gcd(a,b)$.
\end{enumerate}
\end{lemma}

\vs

\begin{remark}
\
\begin{enumerate}
\item[(a)] If $R$ is a PID, then for all $a,b \in R$, $\gcd(a,b)$ exists, namely as a generator of $(a,b)$. Note that, for such $R$, there exist $x,y \in R$ with $\gcd(a,b) = xa + yb$, since $\gcd(a,b) \in (a,b)$. Such $x,y$ can be computed by the Euclidean algorithm.
\item[(b)] The converse of 2.4.6 (b) is in general not true. For example, take $R = \z[x]$. Then $\gcd(2,x) = 1$ (check), but $(2,x) \ne R$ as $1 \notin (2,x)$.
\end{enumerate}
\end{remark}

\vs

\begin{proposition}[Euclidean Algorithm]
Let $R$ be a Euclidean domain with $a,b \in R, a \ne 0$. Then there exists a finite chain of equations of the form
\begin{alignat*}{2}
b &= q_0a + r_1, \qquad \qquad && d(r_1) < d(a), \\
a &= q_1 r_1 + r_2,     && d(r_2) < d(r_1), \\
\vdots & &&\vdots \\
r_{i-1} &= q_i r_i + r_{i+1},   && d(r_{i+1})<d(r_i), \\
\vdots & && \vdots \\
r_{n-1} &= q_n r_n + r_{n+1}, && r_{n+1} = 0
\end{alignat*}
Then
\begin{enumerate}
\item[(a)] $d(r_i) \leq d(a)-i$, so that $n \leq d(a)$.
\item[(b)] $(a,b) = (r_{i-1},r_1) = (r_n) \implies \gcd(a,b) = r_n$.
\item[(c)] If we define inductively $\alpha_{-1} = 0 = \beta_0, \beta_{-1} = 1 = \alpha_0$,
	\[\alpha_i = \alpha_{i-2} - q_{i-1}\alpha_{i-1}, \quad \beta_i = \beta_{i-2} - q_{i-1}\beta_{i-1}, \]
for $1 \leq i \leq n$, then $r_i = \alpha_ia + \beta_i b$ for all $-1 \leq i \leq n$, and hence $\gcd(a,b) = r_n = \alpha_n a + \beta_n b$.
\end{enumerate}
\end{proposition}

\begin{proof}
\
\begin{enumerate}
\item[(b)] We set $r_{-1} = b, r_0 = a$, so that $r_{-1} = q_0 r_0 + r_1$. Then $r_{i-1} = q_i r_i + r_{i+1} \implies (r_{i-1},r_i) \sub (r_i,r_{i+1})$, and $r_{i+1} = r_{i-1} - q_i r_i \implies (r_i,r_{i+1}) \sub (r_{i-1},r_i)$, and hence $(r_i,r_{i+1}) = (r_{i-1},r_i)$ for all $-1 \leq i \leq n$. Therefore,
	\[(a,b) = (r_{-1},r_0) = (r_{n+1},r_n) = (r_n) \]
\item[(c)] We have $b = r_{-1} = \alpha_{-1}a + \beta_{-1}b$, and $a = r_0 =\alpha_0 a + \beta_0 b$. We proceed now by induction on $i \geq 1$. We have
\begin{align*}
r_{i-2} &= q_{i-1}r_{i-1} + r_i \\
\iff r_i &= r_{i-2} - q_{i-1}r_{i-1} \\
&= (\alpha_{i-2} a + \beta_{i-2}b) - q_{i-1}(\alpha_{i-1}a + \beta_{i-1}b) \tag{by induction hypothesis} \\
&= \alpha_i a + \beta_i b
\end{align*}
\end{enumerate}
\end{proof}

\vs

\begin{defn}
	$m \in Z$ is \tb{square free} if $m \neq 0$, $m \neq 1$, and the square of any prime number does not divide $m$.
\end{defn}


\begin{remark}[Quadratic Number Fields]
Fix $m \in \z\bs\{0,1\}$ \tb{square-free}, so that either $m = -1$ or $m = p_1 \cdots p_r$ with each $p_i$ a distinct prime. Then $\sqrt{m} \notin \Q$. Indeed, if $m < 0$, then $\sqrt{m} \in \C \bs \R$. If $m > 0$, then $mb^2 \notin \z^2 \forall b \in \z\bs\{0\}$ (because of the unique prime factorization of $m$). Thus if $\sqrt{m} = \frac{a}{b}, a,b \in \N$, then $mb^2 = a^2 \in \z^2$. Contradiction.

Therefore, we have that $s + t\sqrt{m} = 0 \iff s=t=0$ for all $s,t \in \Q$, i.e., $\{1,\sqrt{m}\}$ is a basis for $K_m := \Q[\sqrt{m}] = \{s + t\sqrt{m} \mid s,t \in \Q\}$ as a vector space over $\Q$. $K_m$ is also closed under addition and multiplication, so $K_m$ is a subring of $\C$.

Define $\phi : K_m \ra K_m$ by $s+t\sqrt{m} \mapsto s-t\sqrt{m}$. $\phi$ is a ring homomorphism (check) with $\phi|_{\Q} = \id_{\Q}$ and $\phi^2 = \id_{K_m}$. Define the \tb{norm} $N : K_m \ra \Q$ by $N(z) = z \phi(z)$ for $z \in K_m$.
\begin{enumerate}
\item[(i)] $N(z) \in \Q$. Indeed, if $z = s+t\sqrt{m}$ for $s,t \in \Q$, then $N(z) = (s+t\sqrt{m})(s-t\sqrt{m}) = s^2-mt^2 \in \Q$.
\item[(ii)] $N(z) =0 \iff z=0$, since $s^2-t^2m = 0 \implies s=t=0$.
\item[(iii)] If $z,w \in K_m$, then
	\[N(zw) = (zw)(\phi(z)\phi(w)) = (z\phi(z))(w\phi(w)) = N(z)N(w) \]
\end{enumerate}
It is now easy to see that
\begin{enumerate}
\item[(1)] $K_m$ is a field. If $0 \ne z \in K_m$, then $N(z) \ne 0$, so $\frac{\phi(z)}{N(z)} \in K_m$, and hence
	\[z \left(\frac{\phi(z)}{N(z)}\right) = \frac{z \phi(z)}{N(z)} = 1 \]
\item[(2)] $N|_{K_m\x} : K_m\x \ra \Q\x$ is a group homomorphism.
\end{enumerate}
\end{remark}

\vs

\begin{theorem}
	$N(ab) = N(a)N(b)$
\end{theorem}

\begin{proof}
	We know that $\phi$ is a ring homomorphism.  It follows that

	$$N(ab) = ab \phi(ab) = ab \phi(a) \phi(b) = (a\phi(a))(b\phi(b)) = N(a)N(b)$$
\end{proof}

\vs

\begin{theorem}
	If $a|b$, then $N(a) | N(b)$ in $\Q$.
\end{theorem}

\vs

\begin{lemma}
Let $R$ be a subring of $K = K_m$ satisfying $N(R) \sub \z$. Then
\begin{enumerate}
\item[(a)] If $a \mid b$ in $R$, then $N(a) \mid N(b)$ in $\z$.
\item[(b)] If $\phi(R) \sub R$, then $R\x = \{a \in R \mid N(a) = \pm1\}$.
\item[(c)] If for all $z \in K$ there exists $q \in R$ such that $|N(z-q)| < 1$, then $R$ is Euclidean with $d(a) := |N(a)|$ for $a \in R$.
\end{enumerate}
\end{lemma}
\begin{proof}
\
\begin{enumerate}
\item[(c)] Given $a,b \in R$, $a \ne 0$, set $z = \frac{b}{a} \in K$. By assumption, there exists $q \in R$ with $|N(z-q)| < 1 \implies |N(z)| > |N(a)||N(z-q)| = |N(b-qa)| \iff d(b-qa) < d(a)$. Putting $r = b-qa$, we have $b = qa + r$ where $d(r) < d(a)$.
\end{enumerate}
\end{proof}

\vs

\begin{example*}[Applications]
\
\begin{enumerate}
\item[(a)] Let $m=-1$, so that $\sqrt{m} = i$ (principal branch) and $K = K_{-1} = \Q[i]$. Put $R = \z[i] = \{k+\ell i \mid k,\ell \in \z\}$. Note that the norm on $K$ is always nonnegative, so we don't need absolute value bars. We have
	\[R\x = \{k + \ell i \in R \mid k^2+\ell^2 = 1\} = \{\pm 1,\pm i\} =\gen{i}\]
To see that $R$ has the Euclidean property, let $z = s + ti \in K$. We can always find integers $k,\ell$ such that $|s-k|,|t-\ell| \leq \frac{1}{2}$. Thus
	\[N(z-(k+\ell i)) = (s-k)^2+(t-\ell)^2 \leq \frac{1}{2} < 1 \]
Since $N(w) \in \z$ for all $w \in R$, we have by 2.4.10 that $\z[i]$ is Euclidean. We can even carry out the Euclidean algorithm on $R$.
\begin{example*}
Let $b = 7+5i,a=3-4i \in \z[i]$. We want $b = aq+r$ with $N(r) < N(a) = 25$. Put
	\[z := \frac{b}{a} = \frac{7+5i}{3-4i} = \frac{1}{25} (1+43i), \]
so we take $q = 2i \implies r = -1-i$, so that
	\[7+5i = 2i(3-4i) + (-1-i) \]
\end{example*}

\item[(b)]Let $R = \z[\sqrt{2}] \sub K_2 \implies R\x = \{\pm(1+\sqrt{2})^n \mid n \in \z\}$ (not easy to show). Let $z = s + t\sqrt{2} \in K_2$. Choose $k,\ell \in \z$ such that $|s-k|,|t-\ell|\leq \frac{1}{2}$, so that
\[|N(z-(k+\ell i)) = |(s-k)^2-2(t-\ell)^2| \leq \frac{1}{4} + \frac{1}{2} < 1, \]
and thus $\z[\sqrt{2}]$ is Euclidean.
\end{enumerate}
\end{example*}

\vs

\begin{remark}[Quadratic Number Rings]
Let $m \in \z\bs\{0\}$ be square-free, $K_m = \Q[\sqrt{m}] = \Q(\sqrt{m})$. Define
	\[\omega_m = \begin{cases}
					\sqrt{m}, \qquad & m \notequiv 1 \pmod{4} \\
					\frac{1+\sqrt{m}}{2},   & m \equiv 1 \pmod{4}
	\end{cases},\]
and consider $\mc{O} = \mc{O}(m) := \z[\omega_m]$. If $m \equiv 1 \pmod{4}$, then $2\omega_m-1 = \sqrt{m} \implies \z[\sqrt{m}] \sub \mc{O}$, but it is easy to see that equality does not hold by the definition of $\omega_m$. If $m \equiv 1 \pmod{4}$, then $\omega_m$ is a root of the \emph{monic} quadratic polynomial with integer coefficients
\[\left(x-\omega_m\right)(x-\phi(\omega_m)) = x^2 - (\omega_m + \phi(\omega_m))x + \omega_m\phi(\omega_m) \in \z[x], \]
since $\omega_m + \phi(\omega_m) = \frac{1}{2} [(1+\sqrt{m})+(1-\sqrt{m})] = 1 \in \z$ and $\omega_m \phi(\omega_m) = \frac{1}{4}(1-m) \in \z$ as $m \equiv 1 \pmod{4}$.  Relating this to the above lemma, we get that $\phi(\mc{O}(m)) \sub \mc{O}(m)$ since $\phi(\omega_m) \in \mc{O}(m)$. Also, $N(\mc{O}(m)) \sub \z$. If $m \not\equiv 1 \mod 4$, this is clear. Suppose $m \equiv 1 \mod{4}$, so that $\omega_m = \frac{1}{2}(1+\sqrt{m})$. Then for $a,b \in \z$, we have
\begin{align*}
N(a+b\omega_m) = (a+b\omega_m) (a+b\phi(\omega_m)) &= a^2 + (\omega_m + \phi(\omega_m))ab + \omega_m \phi(\omega_m) b^2 \\
&= a^2 + ab + \left(\frac{1-m}{4}\right) b^2 \in \z
\end{align*}
\end{remark}

\vs

\begin{remark}
If $m \equiv 1 \mod{4}$, then $R := \z[\sqrt{m}]$ is a proper subring of $\mc{O}(m)$, and $\phi(R) \sub R$ and $N(R) \sub \z$ are still true. In this case, $R$ is not "integrally closed", so $R$ is not a PID.
\end{remark}

\vs

\begin{example}
\
\begin{enumerate}
\item[(a)] $R = \mc{O}(-1) = \z[i]$ satisfies 2.4.10 (a)-(c) and is Euclidean.
\item[(b)] $R = \mc{O}(2)$ is also Euclidean by 2.4.10. In fact, $R\x = \{\pm (1+\sqrt{2})^n \mid n \in \z\}$.
\item[(c)] $R = \z[\sqrt{-3}]$. Let $z = s + t \sqrt{-3} \in K_{-3}$, $s,t \in \Q$. Choose $a,b \in \z$ such that $|s-a|,|t-b| \leq 1/2$, and put $q = a + b \sqrt{-3} \in R$. Then
	\[N(z-q) \leq \frac{1}{4} + \frac{3}{4} = 1, \]
so 2.4.10 (c) is not applicable. In fact, $R$ is not a PID by 2.4.12.

\item[(d)] $R = \mc{O}(-3) = \z \left[\frac{1+\sqrt{-3}}{2}\right] = \z[\zeta_6]$, where $\zeta_6 = \omega_{-3}$ is a primitive $6$th root of unity. Then
	\[R\x = \gen{\zeta_6} = \{\frac{k+l\sqrt{-3}}{2} \mid k,l \in \z, k \equiv l \mod{3}\} \]
Given $s+t\sqrt{-3} \in K_{-3}$, $s,t \in \z$, choose $l \in \z$ with $|t-l/2| \leq 142$. Then choose $k \in \z$ with $k \equiv l \mod{2}$ and $|s-k/2| \leq 1/2$. Put $q = \frac{k+l\sqrt{-3}}{2} \in R$, so that
\[N(z-q) \leq \frac{1}{4} + \frac{3}{16} = \frac{7}{16} < 1 \]
By 2.4.10 (c), $R$ is Euclidean.

\item[(e)] $\mc{O}(-19) = \z \left[\frac{1+\sqrt{-19}}{2}\right]$ is a PID but not Euclidean (see [DF], last examples in 8.1 and 8.2).

\item[(f)] $\mc{O}(-5) = \z[\sqrt{-5}]$ is not a PID, nor is $\mc{O}(-15)$. There are only 9 PID's among the rings $\mc{O}(m)$ with $m < 0$.
\end{enumerate}
\end{example}





\chapter{ \ \ Noetherian Rings, PIDs, and UFDs}

\begin{defn}
A commutative ring (integral domain) $R$ with 1 is called \tb{Noetherian} (Noetherian domain) if every ideal of $R$ is finitely generated.
\end{defn}

\vs

\begin{remark} \
\begin{enumerate}
\item[(a)] If $R$ is a principal ideal ring, then $R$ is Noetherian.
\item[(b)] For a commutative ring $R$ with 1, the following are equivalent:
	\begin{enumerate}
	\item[(i)] $R$ is Noetherian.
	\item[(ii)] (ACC) $R$ satisfies the \tb{ascending chain condition} on ideals: Every ascending chain $I_1 \sub I_2 \sub \dots$ of ideals of $R$ becomes stationary, i.e., there exists $N \in \N$ such that $I_n = I_{n+1}$ for all $n \geq N$.
	\item[(iii)] Every nonempty set of ideals of $R$ has a maximal element.
	\end{enumerate}
\begin{proof}
$[(i) \implies (ii)]$. Let $I_1 \sub I_2 \sub \dots$ be an ascending chain of ideals in $R$. Put $I = \bigcup_{n \in \N} I_n$. Then $I \nsg R$, and $R$ is Noetherian, so $I = (a_1,a_2,\dots,a_m)$ for some $a_1,\dots,a_m \in R$. Now each $a_i \in I_{N_i}$ for some $N_i \in \N$, and putting $N = \max(N_1,\dots,N_m) \in \N$, we have that $I_n = I_{n+1} = I$ for all $n \geq N$, and so $R$ satisfies ACC. \\

$[(ii) \implies (iii)]$ Put $\mc{S}$ be a nonempty set of ideals of $R$. $\mc{S}$ is partially ordered by inclusion, and by our assumption of $(ii)$, every chain in $\mc{S}$ has an upper bound, so by Zorn's Lemma $\mc{S}$ contains a maximal element. \\

$[(iii) \implies (i)]$ Let $I \nsg R$. Define the set $\mc{A} = \{J \nsg R \mid J \sub I \tn{ and $J$ is finitely generated}\}$. Certainly $\mc{A}$ is nonempty and partially ordered by inclusion. By our assumption of $(iii)$, every chain in $\mc{A}$ has an upper bound, so by Zorn's Lemma, $\mc{A}$ contains a maximal element $J = (a_1,a_2,\dots,a_n) \nsg R$ for some $a_1,\dots,a_n \in R$. We claim that $I = J$. If not, then $J \subsetneq I$, and so there exists $a \in I \bs J$. Put $J' = (a_1,a_2,\dots,a_n,a) \in \mc{A}$. Then $J \subsetneq J'$, contradicting the maximality of $J$. Therefore, $I$ is finitely generated, and so $R$ is Noetherian.
\end{proof}

\item[(c)] The homomorphic image of a Noetherian ring is Noetherian. Indeed, let $R$ be Noetherian and $\phi : R \ra S$ a (WLOG surjective) ring homomorphism. Given any $I \nsg S$, $\phi\inv(I) \nsg R$. Now $R$ is Noetherian, so $I = (a_1,\dots,a_n)$ for some $a_1,\dots,a_n \in R$, and thus, because $\phi$ is surjective, $I = \phi(\phi\inv(I)) = \phi(a_1,\dots,a_n) = (\phi(a_1),\dots,\phi(a_n))$ to show that $I$ is finitely generated. In particular, we have shown that $S$ is a PID if $R$ is a PID.
\end{enumerate}
\end{remark}

\vs

\begin{theorem}[Hilbert's Basis Theorem, 1888]
If $R$ is a Noetherian ring, then so is $R[x]$.
\end{theorem}

\begin{proof}
Let $I \ne (0)$ be an ideal in $R[x]$. We want to show that $I$ is finitely generated.
\begin{enumerate}
\item[\tb{Step 1.}] Define $L := \{a \in R \mid a = \ell(f) \tn{ for some } f \in I\}$. Note that $0 \in L$ as $\ell(0) = 0$. It is straightforward to check that if $a,b \in L$ and $r \in R$, then $a+br \in L$. Hence $L \nsg R$, and $R$ is Noetherian, so there exist $a_1,\dots,a_n \in R$ with $L = (a_1,\dots,a_n)$. For each $1 \leq i \leq n$, choose $f_i \in I$ with $\ell(f_i) = a_i$, $\deg (f_i) = e_i \in \N_0$, and set $N = \max(e_1,\dots,e_n)$.

\item[\tb{Setp 2.}] For $0 \leq d \leq N-1$, consider the set
	\[L_d := \{a \in R \mid \exists \ f \in I \tn{ such that } \deg f = d, \ell(f) = a \} \cup \{0\} \]
$L_d \nsg R$ (check). We therefore have $b_{d,1},\dots,b_{d,n_d} \in R$ such that $L_d = (b_{d,1},\dots,b_{d,n_d})$. For $0 \leq d \leq N-1$, choose $f_{d,j} \in I$ with $\deg(f_{d,j}) = d$ and $\ell(f_{d,j}) = b_{d,j}$ for $1 \leq j \leq n_d$. We claim that
	\[I = I' := (\{f_j \mid 1 \leq j \leq n\} \cup \{f_{d,j} \mid 0 \leq d \leq N-1, 1 \leq j \leq n_d\})\]
$I' \sub I$ by our choice of $f_j, f_{d,j}$. We need to show that if $0 \ne f \in I$, then $f \in I'$. We show this by induction on $d = \deg f \geq 0$. \\

If $d = 0$, then $f \in L_0 = (f_{0,1},\dots,f_{0,d_0}) \sub I'$. \\

For the first case of the inductive step, we take $1 \leq d \leq N-1$. Let $a = \ell(f) \in L_d = (b_{d,1},\dots,v_{d,n_d})$, so that there exist $r_j \in R$, $1 \leq j \leq n_d$, such that
	\[a = \sum_{j=1}^{n_d} r_j b_{d,j} = \ell\left(\sum_{j=1}^{n_d} r_j f_{d,j} \right) \in I' \]
Now $f - \sum_{j=1}^{n_d} r_j f_{d,j} \in I$, and $\deg \left( f - \sum_{j=1}^{n_d} r_j f_{d,j}\right) < d$, so by the inductive hypothesis $f - \sum_{j=1}^{n_d} r_j f_{d,j} \in I'$. Therefore $f \in I'$, since $\sum_{j=1}^{n_d} r_j f_{d,j} \in I'$. \\

For the other case of the inductive step, we take $d \geq N$, and let $a = \ell(f) \in L = (a_1,\dots,a_n)$ so that there exist $r_ \in R$ such that $a = \sum_{j=1}^n r_j a_j$. Now $a_j = \ell(f_j)$ where $\deg f_j = e_j \leq N$, thus
	\[\ell \left(\sum_{j=1}^n r_j x^{d-e_j} f_j\right) = \sum_{j=1}^n r_j a_j = a\]
But $\deg \left(f-\sum_{j=1}^n r_j x^{d-e_j} f_j\right) < d$, and $\sum_{j=1}^n r_j x^{d-e_j} f_j \in I'$, so by the inductive hypothesis $f - \sum_{j=1}^n r_j x^{d-e_j} f_j \in I' \implies f \in I'$.
\end{enumerate}
\end{proof}

\vs

\begin{remark*}
What Hilbert actually proved in 1888 was that if $f_1,f_2,\dots,$ is a sequence of polynomials in $F[x_1,\dots,x_n]$ for $F$ a field, then there exists $m \in \N$ such that for all $j \in \N$, there are $g_{i_j} \in F[x_1,\dots,x_n]$ with $f_j = g_{1_j} f_1 + \dots + g_{m_j} f_m$.

From a modern point of view, one would consider a finite set $f_1,\dots,f_m$ of generators a "basis" of the ideal $(f_j \mid j \in \N) \nsg F[x_1,\dots,x_n]$. There is a narrative which says that the famous invariant theorist Paul Gordan, who gave explicit basis constructions for the case $n = 2$, remarked on Hilbert's proof of the basis theorem by saying, "This is not mathematics; this is theology."
\end{remark*}

\vs

\begin{corollary*}
If $R$ is Noetherian, then so is $R[x_1,\dots,x_n]$.
\end{corollary*}

\vs

\begin{corollary}
If $R$ is a Noetherian ring and $S$ is a finitely generated ring extension of $R$, then $S$ is Noetherian.
\end{corollary}

\begin{proof}
Put $S = R[a_1,\dots,a_n]$ for some $a_1,\dots,a_n \in S$. Since $R$ is Noetherian, $R[x_1,\dots,x_n]$ is Noetherian. By application of the universal property (2.3.7) of $R[x]$ $n$ times, we get that there exists a ring homomorphism $\phi : R[x_1,\dots,x_n] \ra S$ such that $\phi|_R = \id_R$ and $\phi(x_i) = a_i$ for $1 \leq i \leq n$.

$\phi$ is surjective, since by assumption $S = R[a_1,\dots,a_n] = \phi(R[x_1,\dots,x_n])$. Hence $S$ is Noetherian by 2.5.2 (b).
\end{proof}

\vs

\begin{example*}
For all $m \in \z\bs\{0,1\}$ square-free, $\mc{O}(m)$ is Noetherian.
\end{example*}

\vs

From now on, we take $R$ to be a domain.

\vs

\begin{defn}
Elements $a,b \in R$ are called \tb{associates} if $a = bu$ for some $ \in R\x \iff (a) = (b)$. We write $a \sim b$ for "$a$ is an associate of $b$," and it is easy to see that the relation $\sim$ $R$ is an equivalence relation.
\end{defn}

\vs

\begin{defn}
Set $R' = R \bs(R\x \cup \{0\})$. An element $b \in R'$ is called \tb{irreducible} if for all divisors $a$ of $b$ in $R$, either $a \in R\x$ or $a \sim b$. It is straightforward to check that for all $a,a' \in R$, if $a \sim a'$, then $a$ is irreducible if and only if $a'$ is irreducible.
\end{defn}

\vs

\begin{proposition}
If $R$ is a Noetherian domain, then any $a \in R'$ is a (finite) product of irreducibles.
\end{proposition}

\begin{proof}
Let $\mc{S} \sub R'$ be the set of all $a \in R'$ which are not finite products of irreducibles. We want to show that $\mc{S} = \vn$. Suppose $a \in \mc{S}$, so that there exist $a_1,b_1 \in R'$ such that $a = a_1b_1$ (otherwise, $a$ would be irreducible). Again, because $a$ is not a product of irreducibles, either $a_1 \in \mc{S}$ or $b_1 \in \mc{S}$. WLOG, say $a_1 \in \mc{S}$. Now $a_1 \not\sim a$ and $a_1 \mid a$, so $(a_1) \subsetneq (a)$. Applying the same argument inductively, we get a strictly ascending chain $(a) \subsetneq (a_1) \subsetneq (a_2) \subsetneq \dots$ of ideals in $R$, which contradicts $R$ being Noetherian. Therefore, $\mc{S} = \vn$.
\end{proof}

\vs

\begin{defn}
An element $p \in R'$ is called \tb{prime} if it satisfies for all $a,b \in R$, if $p \mid ab$, then $p \mid a$ or $p \mid b$.
\end{defn}

\vs

\begin{example*}
The prime numbers are prime elements of $\z$, and $x$ is a prime element of $R[x]$.
\end{example*}

\vs

\begin{lemma}
Let $p \in R'$. Then
\begin{enumerate}
\item[(a)] $p$ is irreducible if and only if $(p)$ is a maximal ideal among the (proper) principal ideals of $R$.
\item[(b)] $p$ is prime if and only if $(p)$ is a prime ideal.
\item[(c)] If $p$ is prime, then $p$ is irreducible.
\end{enumerate}
\end{lemma}

\begin{proof}
\begin{enumerate}
\item[(a)] Assume that $p$ is irreducible and $(p) \sub (a) \ne R$ for some $a \in R$. Then $p \in (a)$, so $p = ab$ for some $b \in R$. Because $p$ is irreducible, either $a \in R\x$ or $b \in R\x$. But $a$ is not a unit since $(a) \ne R$, so $b \in R\x \implies (p) = (a)$.

Now assume that $(p)$ is maximal among the principal ideals of $R$. Suppose that $p = ab$ for some $a,b \in R$. If neither $a$ nor $b$ is a unit, then $(p) \subsetneq (a) \ne R$, a contradiction.

\item[(b)] Exercise in using definitions.

\item[(c)] Assume $p \in R'$ is prime, and suppose that $a \mid p$, say $ab = p$ for some $b \in R$. Then $p \mid ab$, and $p$ is prime, so $p \mid a$ or $p \mid b$. WLOG, say $a = pc$ for some $c \in R$. Then $p = ab = pbc \implies p(1-bc) = 0 \implies b \in R\x$, and therefore $p$ is irreducible.
\end{enumerate}
\end{proof}

\vs

\begin{example}[Irreducibles Needn't Be Prime]
Consider $\mc{O}(-5) = \z[\sqrt{-5}]$. $N(a+b\sqrt{-5}) = a^2 + 5b^2$ for all $a,b \in \z$. There are no elements of $R$ of norm $2$ or $3$. Also, $R\x = \{1,-1\}$. Now consider the element $6 = 2 \cdot 3 = (1+\sqrt{-5})(1-\sqrt{-5})$ in $R$. If $p \in \{2,3,1\pm \sqrt{-5}\}$, then $p$ is irreducible but not prime. Indeed, if $a \mid p$, then $N(a) \mid N(p) \in \{4,6,9\}$, in which case $N(a) = 1$ and $a$ is a unit or $N(a) = N(p)$ and $a \sim p$, since $a \mid p$.

However, for example, $2$ is not prime as $2 \mid 6 = (1+\sqrt{-5})(1-\sqrt{-5})$, but $2 \nmid 1\pm \sqrt{-5}$.
\end{example}

\vs

\begin{proposition}
For a PID $R$ and an element $a \in R'$, the following are equivalent:
\begin{enumerate}
\item[(i)] $p$ is prime.
\item[(ii)] $p$ is irreducible.
\item[(iii)] $(p)$ is maximal ideal of $R$.
\item[(iv)] $(p)$ is a prime ideal of $R$.
\end{enumerate}
\end{proposition}

\begin{proof}
Exercise.
\end{proof}

\vs

\begin{remark*}
An explicit argument for $(ii) \implies (i)$ above goes as follows. Assume $p \mid ab$. Suppose that $p \nmid a$ so that $(p) \subsetneq (p,a) = R \implies$ there exist $r,s \in R$ such that $pr + as = 1 \implies prb + abs = b \implies p \mid b$, since $p \mid ab$.
\end{remark*}

\vs

\begin{lemma}
Let $a \in R'$ and $a = p_1 \cdots p_m = q_1 \cdots q_n$ with $primes p_i$ and $q_j$. Then $m = n$ and, after renumbering, $p_i \sim q_i$ for all $1 \leq i \leq m = n$.
\end{lemma}

\begin{proof}
By induction on $m$. If $m = 1$, then $a = p_1$ is prime and hence irreducible, and this implies that $n = 1$ as each $q_i$ is irreducible. Thus, $p_1 = a = q_1$. \\

For the inductive step, $m \geq 2$. We have $p_m \mid a = q_1 \cdots q_n$, so there exists $1 \leq j \leq n$ such that $p_m \mid q_j$. After renumbering, we may take $j = n$. Now $p_m \mid q_n$, and both $p_m$ and $q_n$ are irreducible, so $p_m \sim q_n$, i.e., there exists $u \in R\x$ such that $p_m = u q_n \implies u p_1 \cdots p_{m-1} q_n = q_1 \cdots q_{n-1} q_n$, hence $(up_1) p_2 \cdots p_{m-1} = q_1 \cdots q_{n-1}$. We apply the inductive hypothesis to get that $m-1 = n-1 \iff m = n$ and, after renumbering, $p_i \sim q_i$ for each $1 \leq i \leq n-1$.
\end{proof}

\vs

\begin{defn}
An integral domain $R$ is called a \tb{unique factorization domain} (UFD) if every element $a \in R'$ can be written as a finite product of primes. By 2.5.12, this factorization is unique up to associates and reordering of the factors.
\end{defn}

\vs

\begin{theorem}
If $R$ is a Noetherian domain in which each irreducible element is prime, then $R$ is a UFD. In particular, any PID is a UFD.
\end{theorem}

\begin{proof}
Let $a \in R'$. Because $R$ is a Noetherian domain, we have by 2.5.7 that $a$ is a product of irreducibles, which by assumption are prime. If $R$ is a PID, then $R$ is Noetherian and every irreducible is prime by 2.5.11.
\end{proof}

\vs

\begin{remark}[Hierarchy of Certain Integral Domains] \ \\

\begin{center}
%\begin{align*}
\[ \left. \begin{array}{c}
\begin{tikzcd}
& & & \tn{Noetherian domain} \\
\tn{Field} \arrow[r,Rightarrow] & \tn{Euclidean domain}
\arrow[Rightarrow]{r}{} & \tn{PID} \arrow[Rightarrow]{ur}{}
\arrow[Rightarrow]{dr}{} \\
& & & \tn{irreducibles are prime}
\end{tikzcd}
\end{array} \right\} \Longrightarrow \tn{UFD}
%\Bigg\} \longrightarrow UFD
\]
%\end{align*}
\end{center}
\end{remark}

\vs

\begin{example}
Let $R$ be the Euclidean domain $R = \z[\sqrt{2}]$. Consider the curious factorization of 7 in $R$:
	\[7 = (3+\sqrt{2})(3-\sqrt{2}) = (5+4\sqrt{2})(-5+4\sqrt{2}) \tag{All factors are irreducible!}\]
While 7 does not seem to factor uniquely into irreducibles, contradicting Theorem 2.5.14, we notice that $5 + 4\sqrt{2} = (1+\sqrt{2})(3+\sqrt{3})$ and $-5+4\sqrt{2} = (-1+\sqrt{2})(3-\sqrt{2})$, where $\pm 1 + \sqrt{3} \in R\x$. The moral of the story is \emph{pay attention to units!}
\end{example}

\vs

\noindent \tb{Fact} Let $R = \mc{O}(m)$ or $R = \z[\sqrt{m}]$, where $m \equiv 1 \mod{4}$ square-free. If $\pi \in R$ satisfies $N(\pi) = p$ for some prime number $p$, then $\pi$ is irreducible in $R$. In fact, $\pi$ is also prime, but that is more difficult to show.

\vs

\begin{remark}
If $R$ is a UFD and $q \in R'$ is irreducible, then $q$ is prime. Since $R$ is a UFD, $q = p_1 \cdots p_n$ for primes $p_i \in R$. Since $q$ is irreducible, we must have $n = 1 \implies q = p_1$ is prime. As a matter of fact, for an integral domain $R$, the following are equivalent:
\begin{enumerate}
\item[(i)] $R$ is a UFD (according to our definition of UFD).
\item[(ii)] Each element in $R'$ factors into irreducibles and all irreducibles are prime.
\item[(iii)] Each element in $R'$ factors \emph{uniquely} (up to associates and order of factors) into a product of irreducibles.
\end{enumerate}
\end{remark}

\vs

\begin{remark}
Let $(C_i)_{i \in I}$ be the family of classes of associate elements which are primes in the UFD $R$. For each $i \in I$, fix $p_i \in C_i$. Hence $p \in C_i \iff p \sim p_i$. We can reformulate the uniqueness of 2.5.12 as follows: for each $a \in R'$, there exists a unique factorization of the form $a = u \prod_{i \in I} p_i^{e_i}$ (really a finite product) with $u \in R\x$, $e_i \in \N_0$, and almost $e_i = 0$.

Then it is easy to show (exercise) that if for $a,b \in R\bs\{0\}$ with $a = u \prod_{i \in I} p_i^{e_i}, b = u_2\prod_{i \in I} p_i^{f_i}$, then
\begin{enumerate}
\item[(a)] $a \mid b \iff e_i \leq f_i \ \forall \ i \in I$
\item[(b)] $\gcd(a,b) = \prod_{i \in I} p_i^{\min(e_i,f_i)} $
\item[(c)] $\lcm(a,b) = \prod_{i \in I} p_i^{\max(e_i,f_i)}$
\item[(d)] If $R$ is a PID, then $(a,b) = (1) = R \iff \min(e_i,f_i) = 0$ for all $i \in I$. This provides us with a reformulation of the Chinese Remainder Theorem (2.1.13) for a PID $R$: If $p_1,\dots,p_n$ are pairwise non-associate primes in $R$ and $e_i \in \N$ for $1 \leq i \leq n$, then
	\[\frac{R}{(p_1^{e_1}\cdots p_n^{e_n})} \cong  \bigtimes_{i=1}^n \frac{R}{(p_i^{e_i})}, \]
since $(p_i^{e_i},p_j^{e_j}) = (p_i^{e_i}) + (p_j^{e_j}) = (1)$ for all $1 \leq i \ne j \leq n$.
\end{enumerate}
\end{remark}

\vs

\begin{proposition}
Let $F$ be a field and $f \in F[x]$ an irreducible. Then $F[x]/(f)$ is a field (extension of $F$).
\end{proposition}

\begin{proof}
$F[x]$ is Euclidean by 2.3.10 which implies that $F[x]$ is a PID. Therefore, $(f) \nsg F[x]$ is a maximal ideal by 2.5.11, hence $F[x]/(f)$ is a field by 2.1.6 (b).
\end{proof}

\vs

\begin{remark}
Let $F$ be a field and $f \in F[x]$. $f$ is irreducible if and only if for all $g \in F[x]$, $g \mid f$ implies that $\deg g = 0$ or $\deg g = \deg f$. In particular, if $\deg f = 1$, then $f$ is irreducible, and if $\deg f = 2,3$, then $f$ is irreducible if and only if $f$ has no roots in $F$.
\end{remark}

\vs

\begin{example}
\
\begin{enumerate}
\item[(a)] $x^2 + 1 \in \R[x]$ is irreducible $\implies \R[x]/(x^2+1) \cong \C$ is a field. But 	\[\frac{\C[x]}{(x^2+1)} \cong \frac{\C[x]}{(x+i)} \times \frac{\C[x]}{(x-i)} \cong \C \times \C \]
is not a field.
\item[(b)] Let $m \in \z\bs\{0,1\}$ be square-free. Then $\sqrt{m} \notin \z \implies x^2-m \in \Q[x]$ is irreducible $\implies \Q[x]/(x^2-m) \cong K_m$ is a field.
\item[(c)] $p_1 = x^2 + x + 1, p_2 = x^3 + x + 1, p_3 = x^3 + x^2 + 1 \in \F_2[x]$ are each irreducible, so
	\[\frac{\F_2[x]}{(p_1)} \ (\cong \F_4), \quad \frac{\F_2[x]}{(p_2)} \cong \frac{\F_2[x]}{(p_3)} \ (\cong \F_8) \]
are all fields. Also, $x^2 + 1 \in \F_3[x]$ is irreducible, so $\F_3[x]/(x^2+1) \ (\cong \F_9)$ is a field.
\end{enumerate}
\end{example}

\end{document}
